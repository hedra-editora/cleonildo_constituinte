\chapter{A Constituição Federal das garantias dos direitos sociais e o
golpe político desconstitutivo do trabalho no Brasil}

\begin{flushright}
\emph{Marcio Pochmann}\footnote{Professor do Instituto de Economia e
  pesquisador do Centro de Estudos Sindicais e de Economia do Trabalho,
  ambos da Universidade Estadual de Campinas.}
\end{flushright}

A longa e gradual jornada de efetivação da regulação do mundo do
trabalho no Brasil encontrou o seu descenso com a interdição do governo
democraticamente eleito em 2014. Com o impedimento da presidenta Dilma
em 2016, uma série de projetos liberalizantes da legislação social e
trabalhista que se encontrava represada desde a ascensão da nova
Constituição Federal, em 1988, passou a ser a descortinada.

Com isso, o Brasil passou a conviver com uma quarta onda de
flexibilização do seu sistema de proteção social e trabalhista
instituído a partir da Revolução de 1930, quando passou a se consolidar
a transição da velha sociedade agrária para a urbana e industrial. Isso
porque a constituição de mercado nacional de trabalho resultou de uma
lenta transição de 80 anos, iniciada em 1850, com o fim do tráfico de
escravos e a implantação da lei de terras, a finalizada em 1930, com a
superação da condição de mercados regionais de trabalho.

Mesmo diante da passagem do Império para a República em 1889, a
regulação do mercado de trabalho terminou sendo postergada frente à
prevalência da situação de ``liberdade do trabalho'' definida pela
primeira constituição republicana, em 1891. Nem mesmo a aprovação, em
1926, da emenda constitucional 29, que possibilitou ao Congresso
Nacional legislar sobre o tema do trabalho, alterou a perspectiva
liberal de manter o Estado fora da regulação social e trabalhista.

A partir da Revolução de 1930, contudo, a regulação do trabalho passou a
ser uma novidade, difundida fragmentadamente, segundo pressão localizada
em categorias mais fortes. Após uma década de embates, com avanços
pontuais na implementação de leis dispersas de regulação do emergente
emprego assalariado, foi implementada a Consolidação das Leis do
Trabalho (CLT) no ano de 1942, em pleno regime político autoritário do
Estado Novo (1937 -- 1945).

Mesmo assim, a maior parte dos trabalhadores esteve excluída do código
do trabalho frente à oposição liberal conservadora dos proprietários
rurais, antiga força dominante na República Velha (1889 -- 1930). Até o
ano de 1963, com a aprovação do Estatuto do Trabalhador Rural, que abriu
a possibilidade de incorporação lenta e gradual do trabalho rural, a CLT
voltava-se tão somente às relações de trabalho urbanas.

Pela Constituição Federal de 1988, ou seja, 45 anos após a implementação
da CLT, que os trabalhadores rurais passaram a ter direitos equivalentes
aos empregados urbanos, embora ainda hoje tenham segmentos dos ocupados
sem acesso à regulação social e trabalhista. Na década de 1940, por
exemplo, a CLT atingia mesmo de 10\% dos trabalhadores, superando
atualmente aos 2/3 dos ocupados.

Diante disso, destaca-se uma primeira onda de flexibilização da
legislação social e trabalhista transcorrida a partir da segunda metade
da década de 1960, com a ascensão da Ditadura Militar (1964 -- 1985). Na
oportunidade, a implantação do Fundo de Garantia por Tempo de Serviço
(FGTS), por exemplo, não apenas interrompeu a trajetória de estabilidade
no emprego, como inaugurou enorme rotatividade na contração e demissão
da mão de obra no Brasil.

A taxa de rotatividade que atingia a cerca de 15\% da força de trabalho
ao ano na década de 1960 rapidamente foi acelerada, aproximando-se da
metade do empregos formais do País. Com isso, a generalização do
procedimento patronal de substituir empregados de maior salário por
trabalhadores de menor remuneração.

Na política salarial vigente entre 1964 e 1994, o resultado foi, em
geral, a perda de poder compra do rendimento dos trabalhadores,
sobretudo no valor real do salário mínimo, que atende a base da pirâmide
distributiva do País. Diante da significativa expansão da produtividade
do trabalho, os salários perderam para a corrida para a inflação, o que
contribuiu ainda mais para o agravamento da desigualdade de renda no
Brasil.

Esta segunda onda de flexibilização se caracterizou por deslocar a
evolução dos rendimentos do trabalho do comportamento acelerado da
produtividade, trazendo, por consequência, a prevalência de uma economia
industrial de baixos salários. Ao mesmo tempo, uma enorme desigualdade
tanto intrarenda do trabalho entre altas e baixas remunerações como
entre o rendimento do trabalho e as demais formas de renda da
propriedade (juros, lucros, alugueis e outras).

A terceira onda de flexibilização das relações de trabalho pode ser
constatada na década de 1990, com a dominação de governos com orientação
neoliberal. Dessa forma, assistiu-se à generalização de medidas de
liberalização da contratação de trabalhadores por modalidades abaixo da
orientação estabelecida pela CLT. Entre elas, a emergência da
terceirização dos contratos, em plena massificação do desemprego e
precarização das relações de trabalho.

A partir da metade da década de 2010, todavia desencadeou-se uma quarta
onda de flexibilização das leis sociais e trabalhistas. Com a recente e
parcial derrota dos trabalhadores imposta pela Câmara dos Deputados pela
aprovação da legislação para terceirização, a septuagenária CLT passou a
ser rebaixada como antes jamais havia sido identificada.

A atualidade da lei da terceirização encontra-se inserida na lógica da
desconstituição do trabalho tal como se conhece, pois integra o sistema
da UBER\emph{ização} do trabalho do início do século 21. Isso porque o
modo UBER de organizar e remunerar a força de trabalho distancia-se
crescentemente da regularidade do assalariamento formal, acompanhado
geralmente pela garantia dos direitos sociais e trabalhistas.

Como os direitos sociais e trabalhistas passam crescentemente a ser
tratados pelos empregadores e suas máquinas de agitação e propaganda
enquanto fundamentalmente custos, a contratação direta, sem direitos
sociais e trabalhistas libera a competição individual maior entre os
próprios trabalhadores em favor dos patrões. Os sindicatos ficam de fora
da negociação, contribuindo ainda mais para esvaziamento do grau de
organização em sua própria base social.

Ao depender cada vez mais do rendimento diretamente recebido, sem mais a
presença do histórico salário indireto (férias, feriado, previdência,
etc.), os fundos públicos voltados ao financiamento do sistema de
seguridade social enfraquecem, quando não contribuem para a prevalência
da sistemática do rentismo. A contenção da terceirização, em função
disso, poderia estancar a trajetória difusora do modo Uber de
precarização das contratações de trabalho.

\section{A desestruturação da sociedade salarial}

A confirmação da interrupção do governo Dilma concedeu inédita força ao
retorno da era da desregulação e flexibilização das políticas sociais e
trabalhistas, conforme a Constituição Federal de 1988 terminou
apontando. Com a decadência do padrão de industrialização e regulação
fordista, o Brasil dá sequência ao movimento maior da desestruturação da
sociedade salarial, especialmente aquela conformada pela maior
proximidade entre a base e o cume da estrutura social. Assiste-se,
assim, à transição das tradicionais classes médias assalariadas e de
trabalhadores industriais para um novo e extensivo precariado, com
importante polarização social (Standing, 2013; Beck, 2000; Pochmann,
2012).

O vazio proporcionado pela desindustrialização vem sendo ocupado pela
chamada sociedade de serviço, que constitui, neste sentido, uma nova
perspectiva de mudança estrutural do mundo do trabalho. Mudança esta que
torna cada vez maior o padrão de exploração do trabalho frente ao
esvaziamento da regulação social e trabalhista e às promessas de
modernidade pelo receituário neoliberal que não se realizam.

Embalados certamente por certo determinismo tecnológico e saltos
imaginados na produtividade do trabalho imaterial, uma nova gama de
promessas foi forjada em direção à almejada sociedade do tempo livre
estendida pelo avanço do ócio criativo, da educação em integral e da
contenção do trabalho heterônomo (apenas pela sobrevivência). Penetrados
cada vez mais pela cultura midiática do individualismo e pela ideologia
da competição, o neoliberalismo seguiu ampliando apoiadores no mundo.

Com isso, surgiu a perspectiva de que as mudanças nas relações sociais
repercutiriam inexoravelmente sobre o funcionamento do mercado de
trabalho. Com a transição demográfica, novas expectativas foram sendo
apresentadas. A propaganda de elevação da expectativa de vida para
próximo de 100 anos de idade, como exemplo, deveria abrir inédita
perspectiva à postergação do ingresso no mercado de trabalho para a
juventude completar o ensino superior, estudar a vida toda e trabalhar
com jornadas semanais de até 12 horas.

A nova sociedade pós-industrial, assim, estaria a oferecer um padrão
civilizatório jamais alcançado pelo modo capitalista de produção e
distribuição (Masi, 1999; Reich, 2002; Santos e Gama, 2008). E sob este
manto de promessas de maior libertação do homem do trabalho pela luta da
sobrevivência (trabalho heterônomo) por meio da postergação da idade de
ingresso no mercado de trabalho para somente depois do cumprimento do
ensino superior, bem como da oferta educacional ao longo da vida, que o
racionalismo neoliberal se constituiu.

De certa forma, trouxe o entendimento de que o esvaziamento do peso
relativo da economia nacional proveniente dos setores primário
(agropecuária) e secundário (indústria e construção civil) consagraria
expansão superior do setor terciário (serviços e comércio) (Aron, 1981;
Bell, 1973). Enfim, estaria a surgir uma sociedade pós-industrial
protagonista de conquistas superiores aos marcos do possibilitado desde
a década de 1930.

Estas promessas, contudo, não resultaram efetivas e tão pouco aguardadas
pela modernização neoliberal de realização. Em pleno curso da transição
para a sociedade de serviços, a inserção no mercado de trabalho precisa
ser gradualmente postergada, possivelmente para o ingresso na atividade
laboral somente após a conclusão do ensino superior, com idade acima dos
22 anos, e saída sincronizada do mercado de trabalho para o avanço da
inatividade. Tudo isso acompanhado por jornada de trabalho reduzida, o
que permite observar que o trabalho heterônomo deva corresponder a não
mais do que 25\% do tempo da vida humana.

Nesse sentido que se apresenta a perspectiva do trabalho humano.
Destaca-se que na antiga sociedade agrária, o começo do trabalho ocorria
a partir dos 5 a 6 anos de idade para se prolongar até praticamente a
morte, com jornadas de trabalho extremamente longas (14 a 16 horas por
dia) e sem períodos de descanso, como férias e inatividade remunerada
(aposentadorias e pensões). Para alguém que conseguisse chegar aos 40
anos de idade, tendo iniciado o trabalho aos 6 anos, por exemplo, o
tempo comprometido somente com as atividades laborais absorvia cerca de
70\% de toda a sua vida.

Na sociedade industrial, o ingresso no mercado laboral foi postergado
para os 16 anos de idade, garantindo aos ocupados, a partir daí, o
acesso a descanso semanal, férias, pensões e aposentadorias provenientes
da regulação pública do trabalho. Com isso, alguém que ingressasse no
mercado de trabalho depois dos 15 anos de idade e permanecesse ativo por
mais 50 anos teria, possivelmente, mais alguns anos de inatividade
remunerada (aposentadoria e pensão).

Assim, cerca de 50\% do tempo de toda a vida estariam comprometidos com
o exercício do trabalho heterônomo. A parte restante do ciclo da vida,
não comprometida pelo trabalho e pela sobrevivência, deveria estar
associada à reconstrução da sociabilidade, estudo e formação, cada vez
mais exigidos pela nova organização da produção e distribuição
internacionalizada.

Isso porque, diante dos elevados e constantes ganhos de produtividade,
torna-se possível a redução do tempo semanal de trabalho de algo ao
redor das 40 horas para não mais que 20 horas. De certa forma, a
transição entre as sociedades urbano-industrial e pós-industrial tende a
não mais separar nítida e rigidamente o tempo do trabalho do não
trabalho, podendo gerar maior mescla entre os dois, com maior
intensidade e risco da longevidade ampliada da jornada laboral para além
do tradicional local de exercício efetivo do trabalho.

Dentro deste contexto que se recoloca em novas bases a relação do tempo
de trabalho heterônomo e a vida. Em geral, o funcionamento do mercado de
trabalho relaciona, ao longo do tempo, uma variedade de formas típicas e
atípicas de uso e remuneração da mão de obra com excedente de força de
trabalho derivado dos movimentos migratórios internos e externos sem
controles.

\section{Considerações finais}

Após sete décadas de construção de uma sociedade superior,
consolidaram-se, com o golpe político de 2016, ingredientes inegáveis da
regressão no interior da sociedade do capital em avanço no Brasil. Do
progresso registrado em torno da construção de uma estrutura social
medianizada por politicas sociais e trabalhistas desde a década de 1930
e sistematizadas pela Constituição federal de 1988, constata-se, neste
início do século XXI, o retorno da forte polarização social no Brasil.

Por uma parte, a degradação da estrutura social herdada da
industrialização fordista tem desconstituído a antiga classe
trabalhadora da manufatura e ampla parcela da classe média, fortalecendo
expansão do novo precariado. Por outra, a concentração de ganhos
significativos de riqueza e renda em segmento minoritário da população
gera contexto social inimaginável, onde somente parcela contida dos
brasileiros detém parcelas crescentes da riqueza.

Desde 2016, o sentido da construção de padrão civilizatório superior
encontra-se desfeito. O avanço possível concentra-se em poucos, enquanto
o retrocesso observado serve a muitos.

\textbf{Referencias bibliográficas}

AGLIETTÀ, M. \textbf{Regulación y crisis del capitalismo}. México: Siglo
XXI, 1979.

ALIER, J. \textbf{El ecologismo de los pobres}: conflictos ambientales y
lenguajes de valoración. Barcelona: Icaria Editorial, 2005.

ALTVATER, E. \textbf{O preço da riqueza}. Pilhagem ambiental e a nova
(des)ordem mundial. São Paulo: Ed. UNESP, 1995.

ANDERSON, C. \textbf{Makers:} a nova revolução industrial. Coimbra:
Actual, 2013.

ARON, R. \textbf{Dezoito lições sobre a sociedade industrial}. Brasília:
UNB/MF, 1981.

BECK, U. \textbf{Un nuevo mundo feliz}: la precariedad del trabajo em la
era de la globalización. Buenos Aires: Paidós, 2000.

BEINSTEIN, J. \textbf{Capitalismo senil}. Rio de Janeiro: Record, 2001.

BELL, D. \textbf{O advento da sociedade pós-industrial}. São Paulo:
Cultrix, 1973.

BOLTANSKI, L.; CHIAPELLO, E. \textbf{O novo espírito do capitalismo.}
Rio de Janeiro: Martins Fontes, 2009.

COATES, D. \textbf{Models of capitalism}. Oxford: Polity Press, 2000.

DAVIS, S. \emph{et al.} \textbf{The new capitalists}. Boston: HBSP,
2008.

DREIFUSS, R\textbf{. Transformações}: matizes do século XXI. Petrópolis:
Vozes, 2004.

ELLSBERG, M. \textbf{The education of milionaires}. New York: Penguin,
2011.

FREIDEN, J. \textbf{Capitalismo global}. Madrid: M. Crítica, 2007.

GLATTFELDER, J. \textbf{Decoding complexity}: uncovering patterns in
economic networks. Switzerland: Springer, 2013.

KUMAR, K. \textbf{Da sociedade pós-industrial à pós-moderna}: novas
teorias sobre o mundo contemporâneo. 2. ed. Rio de Janeiro: Zahar, 1997.

LOJIKINE, J. \textbf{Adieu à la classe moyenne}. Paris: La Dispute,
2005.

MARX, K. \textbf{Grundrisse}. São Paulo: Boitempo, 2011.

MASI, D. \textbf{O futuro do trabalho}: fadiga e ócio na sociedade
pós-industrial. Brasília: UNB/JOE, 1999.

MAZOYER, M.; ROUDART, L. \textbf{História das agriculturas no mundo}.
São Paulo: Editora Unesp, 2009.

MELMAN, E. \textbf{Depois do capitalismo}. São Paulo: Futura, 2002.

MILBERG, W.; WINKLER, D. \textbf{Outsourcing economics}: global value
chains in capitalist development. Cambridge: CUP, 2013.

NARODOWSKI, P.; LENICOV, M. \textbf{Geografia económica mundial}: um
enfoque centro-periferia. Moreno: UNM, 2012.

O'CONNOR, M. Is capitalism sustainable? \emph{In}: \textbf{Political
economy and the politics of ecology}. New Cork: Guilfort, 1994.

OCDE -- ORGANIZAÇÃO DE COOPERAÇÃO E DE DESENVOLVIMENTO ECONÓMICO.
\textbf{Perspectives du développement mondial}. Paris: OCDE, 2010.

POCHMANN, M. \textbf{Classes do trabalho em mutação}. Rio de Janeiro:
Revan, 2012.

POCHMANN, M. \textbf{O emprego na globalização}. São Paulo: Boitempo,
2001.

REICH, R. \textbf{O futuro do sucesso}: o equilíbrio entre o trabalho e
qualidade de vida. Barueri: Manole, 2002.

ROTHKOPF, D. \textbf{Superclass}: the global power elite and the world
they are making. London: L. B, 2008.

SANTOS, N.; GAMA, A. \textbf{Lazer}: da conquista do tempo à conquista
das práticas. Coimbra: IUC, 2008.

STANDING, G. \textbf{O precariado}: a nova classe perigosa. Belo
Horizonte: Autêntica, 2013.

REICH, R. \textbf{Supercapitalismo}. Rio de Janeiro: Campus, 2007.
