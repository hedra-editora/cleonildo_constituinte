\chapter*{Discurso de Dilma Rousseff em julgamento do impeachment no Senado, em 29
de agosto de 2016}

\addcontentsline{toc}{chapter}{Dilma Rousseff, 29 de agosto de 2016}

Excelentíssimo senhor presidente do Supremo Tribunal Federal Ricardo
Lewandowski, excelentíssimo senhor presidente do Senado Federal Renan
Calheiros, excelentíssimas senhoras senadoras e excelentíssimos senhores
senadores, cidadãs e cidadãos de meu amado Brasil, no dia 1º de janeiro
de 2015 assumi meu segundo mandato à Presidência da República Federativa
do Brasil. Fui eleita por mais 54 milhões de votos. Na minha posse,
assumi o compromisso de manter, defender e cumprir a Constituição, bem
como o de observar as leis, promover o bem geral do povo brasileiro,
sustentar a união, a integridade e a independência do Brasil. Ao exercer
a Presidência da República respeitei fielmente o compromisso que assumi
perante a nação e aos que me elegeram. E me orgulho disso.

Sempre
acreditei na democracia e no Estado de Direito, e sempre vi na
Constituição de 1988 uma das grandes conquistas do nosso povo. Jamais
atentaria contra o que acredito ou praticaria atos contrários aos
interesses daqueles que me elegeram. Nesta jornada para me defender do
impeachment me aproximei mais do povo, tive oportunidade de ouvir
seu reconhecimento, de receber seu carinho. Ouvi também críticas duras
ao meu governo, a erros que foram cometidos e a medidas e políticas que
não foram adotadas. Acolho essas críticas com humildade. Até porque,
como todos, tenho defeitos e cometo erros. Entre os meus defeitos não
está a deslealdade e a covardia. Não traio os compromissos que assumo,
os princípios que defendo ou os que lutam ao meu lado. Na luta contra a
ditadura, recebi no meu corpo as marcas da tortura. Amarguei por anos o
sofrimento da prisão. Vi companheiros e companheiras sendo violentados,
e até assassinados. Na época, eu era muito jovem. Tinha muito a esperar
da vida. Tinha medo da morte, das sequelas da tortura no meu corpo e na
minha alma. Mas não cedi. Resisti. Resisti à tempestade de terror que
começava a me engolir, na escuridão dos tempos amargos em que o país
vivia. Não mudei de lado. Apesar de receber o peso da injustiça nos meus
ombros, continuei lutando pela democracia. Dediquei todos esses anos da
minha vida à luta por uma sociedade sem ódios e intolerância. Lutei por
uma sociedade livre de preconceitos e de discriminações. Lutei por uma
sociedade onde não houvesse miséria ou excluídos. Lutei por um Brasil
soberano, mais igual e onde houvesse justiça. Disso tenho orgulho. Quem
acredita, luta.

Aos quase setenta anos de idade, não seria agora, após ser mãe e avó,
que abdicaria dos princípios que sempre me guiaram. Exercendo a
Presidência da República tenho honrado o compromisso com o meu país, com
a Democracia, com o Estado de Direito. Tenho sido intransigente na
defesa da honestidade na gestão da coisa pública. Por isso, diante das
acusações que contra mim são dirigidas neste processo, não posso deixar
de sentir, na boca, novamente, o gosto áspero e amargo da injustiça e do
arbítrio. E por isso, como no passado, resisto. Não esperem de mim o
obsequioso silêncio dos covardes. No passado, com as armas, e hoje, com
a retórica jurídica, pretendem novamente atentar contra a democracia e
contra o Estado do Direito.

Se alguns rasgam o seu passado e negociam as
benesses do presente, que respondam perante a sua consciência e perante
a história pelos atos que praticam. A mim cabe lamentar pelo que foram e
pelo que se tornaram. E resistir. Resistir sempre. Resistir para acordar
as consciências ainda adormecidas para que, juntos, finquemos o pé no
terreno que está do lado certo da história, mesmo que o chão trema e
ameace de novo nos engolir. Não luto pelo meu mandato por vaidade ou por
apego ao poder, como é próprio dos que não tem caráter, princípios ou
utopias a conquistar. Luto pela democracia, pela verdade e pela justiça.
Luto pelo povo do meu país, pelo seu bem"-estar.

Muitos hoje me perguntam
de onde vem a minha energia para prosseguir. Vem do que acredito. Posso
olhar para trás e ver tudo o que fizemos. Olhar para a frente e ver tudo
o que ainda precisamos e podemos fazer. O mais importante é que posso
olhar para mim mesma e ver a face de alguém que, mesmo marcada pelo
tempo, tem forças para defender suas ideias e seus direitos. Sei que, em
breve, e mais uma vez na vida, serei julgada. E é por ter a minha
consciência absolutamente tranquila em relação ao que fiz no exercício
da Presidência da República que venho pessoalmente à presença dos que me
julgarão. Venho para olhar diretamente nos olhos de Vossas Excelências,
e dizer, com a serenidade dos que nada tem a esconder, que não cometi
nenhum crime de responsabilidade. Não cometi os crimes dos quais sou
acusada injusta e arbitrariamente.

Hoje, o Brasil, o mundo e a história
nos observam e aguardam o desfecho deste processo de impeachment. No
passado da América Latina e do Brasil, sempre que interesses de setores
da elite econômica e política foram feridos pelas urnas, e não existiam
razões jurídicas para uma destituição legítima, conspirações eram
tramadas resultando em golpes de estado. O Presidente Getúlio Vargas,
que nos legou a \versal{CLT} e a defesa do patrimônio nacional, sofreu uma
implacável perseguição; a hedionda trama orquestrada pela chamada
``República do Galeão'', que o levou ao suicídio. O Presidente Juscelino
Kubitscheck, que contruiu essa cidade, foi vítima de constantes e
fracassadas tentativas de golpe, como ocorreu no episódio de Aragarças.
O presidente João Goulart, defensor da democracia, dos direitos dos
trabalhadores e das Reformas de Base, superou o golpe do parlamentarismo
mas foi deposto e instaurou"-se a ditadura militar, em 1964. Durante 20
anos, vivemos o silêncio imposto pelo arbítrio e a democracia foi
varrida de nosso País. Milhões de brasileiros lutaram e reconquistaram o
direito a eleições diretas. Hoje, mais uma vez, ao serem contrariados e
feridos nas urnas os interesses de setores da elite econômica e política
nos vemos diante do risco de uma ruptura democrática. Os padrões
políticos dominantes no mundo repelem a violência explícita. Agora, a
ruptura democrática se dá por meio da violência moral e de pretextos
constitucionais para que se empreste aparência de legitimidade ao
governo que assume sem o amparo das urnas. Invoca"-se a Constituição para
que o mundo das aparências encubra hipocritamente o mundo dos fatos.

As provas produzidas deixam claro e inconteste que as acusações contra
mim dirigidas são meros pretextos, embasados por uma frágil retórica
jurídica. Nos últimos dias, novos fatos evidenciaram outro aspecto da
trama que caracteriza este processo de impeachment. O autor da
representação junto ao Tribunal de Contas da União que motivou as
acusações discutidas nesse processo foi reconhecido como suspeito pelo
Presidente do Supremo Tribunal Federal. Soube"-se ainda, pelo depoimento
do auditor responsável pelo parecer técnico, que ele havia ajudado a
elaborar a própria representação que auditou. Fica claro o vício da
parcialidade, a trama, na construção das teses por eles defendidas.

São
pretextos, apenas pretextos, para derrubar, por meio de um processo de
impeachment sem crime de responsabilidade, um governo legítimo,
escolhido em eleição direta com a participação de 110 milhões de
brasileiros e brasileiras. O governo de uma mulher que ousou ganhar duas
eleições presidenciais consecutivas. São pretextos para viabilizar um
golpe na Constituição. Um golpe que, se consumado, resultará na eleição
indireta de um governo usurpador. A eleição indireta de um governo que,
já na sua interinidade, não tem mulheres comandando seus ministérios,
quando o povo, nas urnas, escolheu uma mulher para comandar o país. Um
governo que dispensa os negros na sua composição ministerial e já
revelou um profundo desprezo pelo programa escolhido pelo povo em 2014.

Fui eleita presidenta por 54 milhões e meio de votos para cumprir um
programa cuja síntese está gravada nas palavras ``nenhum direito a
menos''. O que está em jogo no processo de impeachment não é apenas o
meu mandato. O que está em jogo é o respeito às urnas, à vontade
soberana do povo brasileiro e à Constituição. O que está em jogo são as
conquistas dos últimos 13 anos: os ganhos da população, das pessoas mais
pobres e da classe média; a proteção às crianças; os jovens chegando às
universidades e às escolas técnicas; a valorização do salário mínimo; os
médicos atendendo a população; a realização do sonho da casa própria. O
que está em jogo é o investimento em obras para garantir a convivência
com a seca no semiárido, é a conclusão do sonhado e esperado projeto de
integração do São Francisco. O que está em jogo é, também, a grande
descoberta do Brasil, o pré"-sal. O que está em jogo é a inserção
soberana de nosso país no cenário internacional, pautada pela ética e
pela busca de interesses comuns. O que está em jogo é a autoestima dos
brasileiros e brasileiras, que resistiram aos ataques dos pessimistas de
plantão à capacidade do País de realizar, com sucesso, a Copa do Mundo e
as Olimpíadas e Paralimpíadas. O que está em jogo é a conquista da
estabilidade, que busca o equilíbrio fiscal mas não abre mão de
programas sociais para a nossa população. O que está em jogo é o futuro
do País, a oportunidade e a esperança de avançar sempre mais.

Senhoras e
senhores senadores, no presidencialismo previsto em nossa Constituição,
não basta a eventual perda de maioria parlamentar para afastar um
Presidente. Há que se configurar crime de responsabilidade. E está claro
que não houve tal crime. Não é legítimo, como querem os meus acusadores,
afastar o chefe de Estado e de governo pelo ``conjunto da obra''. Quem
afasta o Presidente pelo ``conjunto da obra'' é o povo, e só o povo, nas
eleições. E nas eleições o programa de governo vencedor não foi este
agora ensaiado e desenhado pelo governo interino e defendido pelos meus
acusadores. O que pretende o governo interino, se transmudado em
efetivo, é um verdadeiro ataque às conquistas dos últimos anos.
Desvincular o piso das aposentadorias e pensões do salário mínimo será a
destruição do maior instrumento de distribuição de renda do país, que é
a Previdência Social. O resultado será mais pobreza, mais mortalidade
infantil e a decadência dos pequenos municípios. A revisão dos direitos
e garantias sociais previstos na \versal{CLT} e a proibição do saque do \versal{FGTS} na
demissão do trabalhador são ameaças que pairam sobre a população
brasileira caso prospere o impeachment sem crime de responsabilidade.
Conquistas importantes para as mulheres, os negros e as populações \versal{LGBT}
estarão comprometidas pela submissão a princípios ultraconservadores. O
nosso patrimônio estará em questão, com os recursos do pré"-sal, as
riquezas naturais e minerárias sendo privatizadas. A ameaça mais
assustadora desse processo de impeachment sem crime de responsabilidade
é congelar por inacreditáveis 20 anos todas as despesas com saúde,
educação, saneamento, habitação. É impedir que, por 20 anos, mais
crianças e jovens tenham acesso às escolas; que, por 20 anos, as pessoas
possam ter melhor atendimento à saúde; que, por 20 anos, as famílias
possam sonhar com casa própria.

Senhor Presidente Ricardo Lewandowski,
Sras. e Srs. Senadores, a verdade é que o resultado eleitoral de 2014
foi um rude golpe em setores da elite conservadora brasileira. Desde a
proclamação dos resultados eleitorais, os partidos que apoiavam o
candidato derrotado nas eleições fizeram de tudo para impedir a minha
posse e a estabilidade do meu governo. Disseram que as eleições haviam
sido fraudadas, pediram auditoria nas urnas, impugnaram minhas contas
eleitorais, e após a minha posse, buscaram de forma desmedida quaisquer
fatos que pudessem justificar retoricamente um processo de impeachment.
Como é próprio das elites conservadoras e autoritárias, não viam na
vontade do povo o elemento legitimador de um governo. Queriam o poder a
qualquer preço. Tudo fizeram para desestabilizar a mim e ao meu governo.

Só é possível compreender a gravidade da crise que assola o Brasil desde
2015 levando"-se em consideração a instabilidade política aguda que,
desde a minha reeleição, tem caracterizado o ambiente em que ocorrem o
investimento e a produção de bens e serviços. Não se procurou discutir e
aprovar uma melhor proposta para o País. O que se pretendeu
permanentemente foi a afirmação do ``quanto pior melhor'', na busca
obsessiva de se desgastar o governo, pouco importando os resultados
danosos desta questionável ação política para toda a população. A
possibilidade de impeachment tornou"-se assunto central da pauta política
e jornalística apenas dois meses após minha reeleição, apesar da
evidente improcedência dos motivos para justificar esse movimento
radical. Nesse ambiente de turbulências e incertezas, o risco político
permanente provocado pelo ativismo de parcela considerável da oposição
acabou sendo um elemento central para a retração do investimento e para
o aprofundamento da crise econômica. Deve ser também ressaltado que a
busca do reequilíbrio fiscal, desde 2015, encontrou uma forte
resistência na Câmara dos Deputados, à época presidida pelo deputado
Eduardo Cunha. Os projetos enviados pelo governo foram rejeitados,
parcial ou integralmente. Pautas bombas foram apresentadas e algumas
aprovadas. As comissões permanentes da Câmara, em 2016, só funcionaram a
partir do dia 5 de maio, ou seja, uma semana antes da aceitação do
processo de impeachment pela Comissão do Senado Federal. Os srs. e as
sras. senadores sabem que o funcionamento dessas Comissões era e é
absolutamente indispensável para a aprovação de matérias que interferem
no cenário fiscal e encaminhar a saída da crise. Foi criado assim o
desejado ambiente de instabilidade política, propício à abertura do
processo de impeachment sem crime de responsabilidade. Sem essas ações,
o Brasil certamente estaria hoje em outra situação política, econômica e
fiscal.

Muitos articularam e votaram contra propostas que durante toda a vida
defenderam, sem pensar nas consequências que seus gestos trariam para o
país e para o povo brasileiro. Queriam aproveitar a crise econômica,
porque sabiam que assim que o meu governo viesse a superá"-la, sua
aspiração de acesso ao poder haveria de ficar sepultada por mais um
longo período. Mas, a bem da verdade, as forças oposicionistas somente
conseguiram levar adiante o seu intento quando outra poderosa força
política a elas se agregou: a força política dos que queriam evitar a
continuidade da ``sangria'' de setores da classe política brasileira,
motivada pelas investigações sobre a corrupção e o desvio de dinheiro
público.

É notório que durante o meu governo e o do Presidente Lula foram dadas
todas as condições para que estas investigações fossem realizadas.
Propusemos importantes leis que dotaram os órgãos competentes de
condições para investigar e punir os culpados. Assegurei a autonomia do
Ministério Público, nomeando como Procurador Geral da República o
primeiro nome da lista indicado pelos próprios membros da instituição.
Não permiti qualquer interferência política na atuação da Polícia
Federal. Contrariei, com essa minha postura, muitos interesses. Por
isso, paguei e pago um elevado preço pessoal pela postura que tive.

Arquitetaram a minha destituição, independentemente da existência de
quaisquer fatos que pudessem justificá"-la perante a nossa Constituição.
Encontraram, na pessoa do ex"-Presidente da Câmara dos Deputados, Eduardo
Cunha, o vértice da sua aliança golpista. Articularam e viabilizaram a
perda da maioria parlamentar do governo. Situações foram criadas, com
apoio escancarado de setores da mídia, para construir o clima político
necessário para a desconstituição do resultado eleitoral de 2014. Todos
sabem que este processo de impeachment foi aberto por uma ``chantagem
explícita'' do ex"-presidente da Câmara, Eduardo Cunha, como chegou a
reconhecer em declarações à imprensa um dos próprios denunciantes.
Exigia aquele parlamentar que eu intercedesse para que deputados do meu
partido não votassem pela abertura do seu processo de cassação. Nunca
aceitei na minha vida ameaças ou chantagens. Se não o fiz antes, não o
faria na condição de Presidenta da República. É fato, porém, que não ter
me curvado a esta chantagem motivou o recebimento da denúncia por crime
de responsabilidade e a abertura deste processo, sob o aplauso dos
derrotados em 2014 e dos temerosos pelas investigações.

Se eu tivesse me
acumpliciado com a improbidade e com o que há de pior na política
brasileira, como muitos até hoje parecem não ter o menor pudor em
fazê"-lo, eu não correria o risco de ser condenada injustamente. Quem se
acumplicia ao imoral e ao ilícito não tem respeitabilidade para
governar o Brasil. Quem age para poupar ou adiar o julgamento de uma
pessoa que é acusada de enriquecer às custas do Estado brasileiro e do
povo que paga impostos, cedo ou tarde, acabará pagando perante a
sociedade e a história o preço do seu descompromisso com a ética. Todos
sabem que não enriqueci no exercício de cargos públicos, que não desviei
dinheiro público em meu proveito próprio, nem de meus familiares, e que
não possuo contas ou imóveis no exterior. Sempre agi com absoluta
probidade nos cargos públicos que ocupei ao longo da minha vida.
Curiosamente, serei julgada, por crimes que não cometi, antes do
julgamento do ex"-presidente da Câmara, acusado de ter praticado
gravíssimos atos ilícitos e que liderou as tramas e os ardis que
alavancaram as ações voltadas à minha destituição. Ironia da história?
Não, de forma nenhuma. Trata"-se de uma ação deliberada que conta com o
silêncio cúmplice de setores da grande mídia brasileira. Viola"-se a
democracia e pune"-se uma inocente. Este é o pano de fundo que marca o
julgamento que será realizado pela vontade dos que lançam contra mim
pretextos acusatórios infundados.

Estamos a um passo da consumação de uma grave ruptura institucional.
Estamos a um passo da concretização de um verdadeiro golpe de Estado.
Senhoras e senhores senadores, vamos aos autos deste processo. Do que
sou acusada? Quais foram os atentados à Constituição que cometi? Quais
foram os crimes hediondos que pratiquei? A primeira acusação refere"-se à
edição de três decretos de crédito suplementar sem autorização
legislativa. Ao longo de todo o processo, mostramos que a edição desses
decretos seguiu todas as regras legais. Respeitamos a previsão contida
na Constituição, a meta definida na \versal{LDO} e as autorizações estabelecidas
no artigo 4° da Lei Orçamentária de 2015, aprovadas pelo Congresso
Nacional. Todas essas previsões legais foram respeitadas em relação aos
3 decretos. Eles apenas ofereceram alternativas para alocação dos mesmos
limites, de empenho e financeiro, estabelecidos pelo decreto de
contingenciamento, que não foram alterados. Por isso, não afetaram em
nada a meta fiscal.

Ademais, desde 2014, por iniciativa do Executivo, o
Congresso aprovou a inclusão, na \versal{LDO}, da obrigatoriedade que qualquer
crédito aberto deve ter sua execução subordinada ao decreto de
contingenciamento, editado segundo as normas estabelecidas pela Lei de
Responsabilidade Fiscal. E isso foi precisamente respeitado. Não sei se
por incompreensão ou por estratégia, as acusações feitas neste processo
buscam atribuir a esses decretos nossos problemas fiscais. Ignoram ou
escondem que os resultados fiscais negativos são consequência da
desaceleração econômica e não a sua causa. Escondem que, em 2015, com o
agravamento da crise, tivemos uma expressiva queda da receita ao longo
do ano --- foram R\$ 180 bilhões a menos que o previsto na Lei
Orçamentária. Fazem questão de ignorar que realizamos, em 2015, o maior
contingenciamento de nossa história. Cobram que, quando enviei ao
Congresso Nacional, em julho de 2015, o pedido de autorização para
reduzir a meta fiscal, deveria ter imediatamente realizado um novo
contingenciamento. Não o fiz porque segui o procedimento que não foi
questionado pelo Tribunal de Contas da União ou pelo Congresso Nacional
na análise das contas de 2009.

Além disso, a responsabilidade com a
população justifica também nossa decisão. Se aplicássemos, em julho, o
contingenciamento proposto pelos nossos acusadores cortaríamos 96\% do
total de recursos disponíveis para as despesas da União. Isto
representaria um corte radical em todas as dotações orçamentárias dos
órgãos federais. Ministérios seriam paralisados, universidades fechariam
suas portas, o Mais Médicos seria interrompido, a compra de medicamentos
seria prejudicada, as agências reguladoras deixariam de funcionar. Na
verdade, o ano de 2015 teria, orçamentariamente, acabado em julho. Volto
a dizer: ao editar estes decretos de crédito suplementar, agi em
conformidade plena com a legislação vigente. Em nenhum desses atos, o
Congresso Nacional foi desrespeitado. Aliás, este foi o comportamento
que adotei em meus dois mandatos. Somente depois que assinei estes
decretos é que o Tribunal de Contas da União mudou a posição que sempre
teve a respeito da matéria. É importante que a população brasileira seja
esclarecida sobre este ponto: os decretos foram editados em julho e
agosto de 2015 e somente em outubro de 2015 o \versal{TCU} aprovou a nova
interpretação. O \versal{TCU} recomendou a aprovação das contas de todos os
presidentes que editaram decretos idênticos aos que editei. Nunca
levantaram qualquer problema técnico ou apresentaram a interpretação que
passaram a ter depois que assinei estes atos. Querem me condenar por ter
assinado decretos que atendiam a demandas de diversos órgãos, inclusive
do próprio Poder Judiciário, com base no mesmo procedimento adotado
desde a entrada em vigor da Lei de Responsabilidade Fiscal, em 2001? Por
ter assinado decretos que somados, não implicaram, como provado nos
autos, em nenhum centavo de gastos a mais para prejudicar a meta fiscal?

A segunda denúncia dirigida contra mim neste processo também é injusta e
frágil. Afirma"-se que o alegado atraso nos pagamentos das subvenções
econômicas devidas ao Banco do Brasil, no âmbito da execução do programa
de crédito rural Plano Safra, equivale a uma ``operação de crédito'', o
que estaria vedado pela Lei de Responsabilidade Fiscal. Como minha
defesa e várias testemunhas já relataram, a execução do Plano Safra é
regida por uma lei de 1992, que atribui ao Ministério da Fazenda a
competência de sua normatização, inclusive em relação à atuação do Banco
do Brasil. A Presidenta da República não pratica nenhum ato em relação à
execução do Plano Safra. Parece óbvio, além de juridicamente justo, que
eu não seja acusada por um ato inexistente. A controvérsia quanto a
existência de operação de crédito surgiu de uma mudança de interpretação
do \versal{TCU}, cuja decisão definitiva foi emitida em dezembro de 2015.
Novamente, há uma tentativa de dizer que cometi um crime antes da
definição da tese de que haveria um crime. Uma tese que nunca havia
surgido antes e que, como todas as senhoras e senhores senadores
souberam em dias recentes, foi urdida especialmente para esta ocasião.

Lembro ainda a decisão recente do Ministério Público Federal, que
arquivou inquérito exatamente sobre esta questão. Afirmou não caber
falar em ofensa à lei de responsabilidade fiscal porque eventuais
atrasos de pagamento em contratos de prestação de serviços entre a União
e instituições financeiras públicas não são operações de crédito.
Insisto, senhoras senadoras e senhores senadores: não sou eu nem
tampouco minha defesa que fazemos estas alegações. É o Ministério
Público Federal que se recusou a dar sequência ao processo, pela
inexistência de crime. Sobre a mudança de interpretação do \versal{TCU}, lembro
que, ainda antes da decisão final, agi de forma preventiva. Solicitei ao
Congresso Nacional a autorização para pagamento dos passivos e defini em
decreto prazos de pagamento para as subvenções devidas. Em dezembro de
2015, após a decisão definitiva do \versal{TCU} e com a autorização do Congresso,
saldamos todos os débitos existentes. Não é possível que não se veja
aqui também o arbítrio deste processo e a injustiça também desta
acusação. Este processo de impeachment não é legítimo. Eu não atentei,
em nada, em absolutamente nada contra qualquer dos dispositivos da
Constituição que, como Presidenta da República, jurei cumprir. Não
pratiquei ato ilícito. Está provado que não agi dolosamente em nada. Os
atos praticados estavam inteiramente voltados aos interesses da
sociedade. Nenhuma lesão trouxeram ao erário ou ao patrimônio público.
Volto a afirmar, como o fez a minha defesa durante todo o tempo, que
este processo está marcado, do início ao fim, por um clamoroso desvio de
poder. É isto que explica a absoluta fragilidade das acusações que
contra mim são dirigidas. Tem"-se afirmado que este processo de
impeachment seria legítimo porque os ritos e prazos teriam sido
respeitados. No entanto, para que seja feita justiça e a democracia se
imponha, a forma só não basta. É necessário que o conteúdo de uma
sentença também seja justo. E no caso, jamais haverá justiça na minha
condenação. Ouso dizer que em vários momentos este processo se desviou,
clamorosamente, daquilo que a Constituição e os juristas denominam de
``devido processo legal''. Não há respeito ao devido processo legal
quando a opinião condenatória de grande parte dos julgadores é divulgada
e registrada pela grande imprensa, antes do exercício final do direito
de defesa.

Não há respeito ao devido processo legal quando julgadores afirmam que a
condenação não passa de uma questão de tempo, porque votarão contra mim
de qualquer jeito. Nesse caso, o direito de defesa será exercido apenas
formalmente, mas não será apreciado substantivamente nos seus argumentos
e nas suas provas. A forma existirá apenas para dar aparência de
legitimidade ao que é ilegítimo na essência. Senhoras e senhores
senadores, Nesses meses, me perguntaram inúmeras vezes porque eu não
renunciava, para encurtar este capítulo tão difícil de minha vida.
Jamais o faria porque tenho compromisso inarredável com o Estado
Democrático de Direito. Jamais o faria porque nunca renuncio à luta.

Confesso a Vossas Excelências, no entanto, que a traição, as agressões
verbais e a violência do preconceito me assombraram e, em alguns
momentos, até me magoaram. Mas foram sempre superados, em muito, pela
solidariedade, pelo apoio e pela disposição de luta de milhões de
brasileiras e brasileiros pelo país afora. Por meio de manifestações de
rua, reuniões, seminários, livros, shows, mobilizações na internet,
nosso povo esbanjou criatividade e disposição para a luta contra o
golpe. As mulheres brasileiras têm sido, neste período, um esteio
fundamental para minha resistência. Me cobriram de flores e me
protegeram com sua solidariedade. Parceiras incansáveis de uma batalha
em que a misoginia e o preconceito mostraram suas garras, as brasileiras
expressaram, neste combate pela democracia e pelos direitos, sua força e
resiliência. Bravas mulheres brasileiras, que tenho a honra e o dever de
representar como primeira mulher Presidenta do Brasil.

Chego à última
etapa desse processo comprometida com a realização de uma demanda da
maioria dos brasileiros: convocá"-los a decidir, nas urnas, sobre o
futuro de nosso País. Diálogo, participação e voto direto e livre são as
melhores armas que temos para a preservação da democracia. Confio que as
senhoras senadoras e os senhores senadores farão justiça. Tenho a
consciência tranquila. Não pratiquei nenhum crime de responsabilidade.
As acusações dirigidas contra mim são injustas e descabidas. Cassar em
definitivo meu mandato é como me submeter a uma pena de morte política.
Este é o segundo julgamento a que sou submetida em que a democracia tem
assento, junto comigo, no banco dos réus. Na primeira vez, fui condenada
por um tribunal de exceção. Daquela época, além das marcas dolorosas da
tortura, ficou o registro, em uma foto, da minha presença diante de meus
algozes, num momento em que eu os olhava de cabeça erguida enquanto eles
escondiam os rostos, com medo de serem reconhecidos e julgados pela
história.

Hoje, quatro décadas depois, não há prisão ilegal, não há
tortura, meus julgadores chegaram aqui pelo mesmo voto popular que me
conduziu à Presidência. Tenho por todos o maior respeito, mas continuo
de cabeça erguida, olhando nos olhos dos meus julgadores. Apesar das
diferenças, sofro de novo com o sentimento de injustiça e o receio de
que, mais uma vez, a democracia seja condenada junto comigo. E não tenho
dúvida que, também desta vez, todos nós seremos julgados pela história.
Por duas vezes vi de perto a face da morte: quando fui torturada por
dias seguidos, submetida a sevícias que nos fazem duvidar da humanidade
e do próprio sentido da vida; e quando uma doença grave e extremamente
dolorosa poderia ter abreviado minha existência. Hoje eu só temo a morte
da democracia, pela qual muitos de nós, aqui neste plenário, lutamos com
o melhor dos nossos esforços. Reitero: respeito os meus julgadores. Não
nutro rancor por aqueles que votarão pela minha destituição.

Respeito e tenho especial apreço por aqueles que têm lutado bravamente
pela minha absolvição, aos quais serei eternamente grata. Neste momento,
quero me dirigir aos senadores que, mesmo sendo de oposição a mim e ao
meu governo, estão indecisos. Lembrem"-se que, no regime presidencialista
e sob a égide da nossa Constituição, uma condenação política exige
obrigatoriamente a ocorrência de um crime de responsabilidade, cometido
dolosamente e comprovado de forma cabal. Lembrem"-se do terrível
precedente que a decisão pode abrir para outros presidentes,
governadores e prefeitos. Condenar sem provas substantivas. Condenar um
inocente. Faço um apelo final a todos os senadores: não aceitem um golpe
que, em vez de solucionar, agravará a crise brasileira. Peço que façam
justiça a uma presidenta honesta, que jamais cometeu qualquer ato
ilegal, na vida pessoal ou nas funções públicas que exerceu. Votem sem
ressentimento. O que cada senador sente por mim e o que nós sentimos uns
pelos outros importa menos, neste momento, do que aquilo que todos
sentimos pelo país e pelo povo brasileiro. Peço: votem contra o
impeachment. Votem pela democracia. Muito obrigada.

%\emph{Da Assessoria da Presidente da República afastada}
