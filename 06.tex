\chapter*{A Constituinte, as mulheres e o Golpe}

\addcontentsline{toc}{chapter}{A Constituinte, as mulheres e o Golpe,\\
\scriptsize{por Vivian Farias}}
\hedramarkboth{A Constituinte, as mulheres e o Golpe}{}

\begin{flushright}
\emph{Vivian Farias}\footnote{Assistente social pela \versal{UFPE}, pós"-graduada em
  Gestão de Políticas Públicas pela \versal{FESP}"-\versal{SP}, mestranda em Estado,
  Governo e Políticas Públicas pela \versal{FLACSO} -- Brasil, Dirigente Nacional
  do Partido dos Trabalhadores, Feminista e Militante na pauta das
  Mulheres.}
\end{flushright}

A Constituição Federal de 1988 é a oitava formulação constitucional
brasileira, considerada um marco na construção do Estado democrático de
direito. As Constituições outorgadas, ou seja, as de 1824, 1937, 1967
foram formuladas respectivamente por D. Pedro \versal{I}, por Getúlio Vargas e
pela junta militar composta pelos ministros da Marinha, do Exército e da
Aeronáutica. A constituição do regime militar, apesar de ter sido
promulgada pelo Congresso Nacional em 1967, foi formulada exclusivamente
pelos militares, ou seja, na prática foi outorgada pelos que haviam
usurpado o poder político no país.

As edições de 1891, 1934, 1946 e 1988 contaram com diferentes graus de
participação popular, sendo a última a que mais avançou no sentido da
democratização do país, contou com eleição específica para tal tarefa,
embora a constituinte não fosse exclusiva\footnote{Numa constituinte
  exclusiva, os eleitos para a Assembleia teriam a função exclusiva de
  formular a constituição, sendo dissolvida após a promulgação. Em 1987,
  os parlamentares legislaram de maneira ordinária e cumpriram o
  restante dos mandatos, com direito à reeleição.}, elegendo 559
constituintes (487 deputados e 72 senadores), com representação de todos
os 23 Estados e o Distrito Federal\footnote{Posteriormente seriam
  formados mais 3 estados, o que afere a atual composição de 26 estados
  e o Distrito Federal.}. A população participou diretamente com a
formulação de 122 emendas ao texto, ocorreram intensos debates com as
mais diversas pautas, gerando uma grande efervescência política.

A Constituição promulgada em 1988 é a que contou com maior presença de
mulheres na sua elaboração. Se durante o período anterior a presença das
mulheres era de cerca de 0,6\%, na Assembleia Constituinte o salto de
presença de mulheres foi para 5,3\%.

Para além do avanço numérico, a união em bloco, como posicionamento
político das 26 deputadas constituintes\footnote{Não foi eleita nenhuma
  mulher na postulação ao Senado Federal na Assembleia Constituinte.}
contra uma herança histórica de subordinação foi fundamental para a
formação de uma bancada feminina que, apesar de sua heterogeneidade no
campo ideológico e político, se uniram na luta para o empoderamento das
mulheres. Com isso, foi possível a vitória de várias propostas deste
bloco na referida constituição\footnote{Avanços como
  ``licença"-maternidade de 120 dias, a criação de uma
  licença"-paternidade, benefícios sociais e direitos trabalhistas para
  empregadas domésticas, direito ao divórcio, além de artigos garantindo
  a igualdade entre mulheres e homens independentes de cor/raça''
  (\versal{SENKEVICS}, A., 2013).}.

Um ponto relevante neste processo foi o debate acerca da igualdade entre
homens e mulheres, compreendidos como cidadãs e cidadãos, garantindo o
acesso a todos aos direitos fundamentais. Este marco formal foi um passo
importante no fomento dos direitos e emancipação das mulheres, que só
foi possível dada a luta incessante de mulheres como Benedita da Silva e
Lidice da Mata.

As pautas das mulheres foram apresentadas na Carta das Mulheres
Brasileiras aos Constituintes de 1987\footnote{(Conselho Nacional dos
  Direitos da Mulher, 1986).}, síntese de muito trabalho coletivo
das constituintes com o movimento feminista. A Carta Magna garante que
``Homens e mulheres são iguais em direitos e obrigações, nos
termos desta Constituição'' e ``Os direitos e deveres referentes à
sociedade conjugal são exercidos pelo homem e pela mulher'', sendo um
marco nas relações jurídico, político e social na árdua caminhada da
garantia de direitos e condições de equidade de gênero, dando às
mulheres a proteção do direitos humanos pela primeira vez na República
Brasileira\footnote{(Conselho Nacional dos Direitos da Mulher, 2014).}.

O esforço das mulheres -- e da Câmara dos Deputados -- para criar
condições objetivas favoráveis a sua emancipação fez parte, portanto,
das bases legais e institucionais que deram forma ao Estado Democrático
de Direito instituído pós 1988.

Desde então a aliança das mulheres parlamentares tem crescido, o mesmo
fenômeno que ocorre na representação parlamentar: estudo recente versa que a
participação de mulheres no Parlamento federal brasileiro cresceu 87\%
no intervalo de janeiro de 1990 até dezembro de 2016, saltando de 5,3\%
para 9,9\%. Todavia, os nossos índices de participação de mulheres são
muito baixos, cerca de 10\%, em comparação com a média mundial, que
passou de 12,7\% em 1990 para 23\% em 2016\footnote{Alana Gandra,
  Agencia Brasil. Disponível em: 

  \textless{}\emph{https://bit.ly/2nkFJot}\textgreater{}.}.
Já o Brasil se assemelha com os índices do Oriente Médio e norte da África.

Apesar dos vertiginosos avanços contidos na Constituição Cidadã\footnote{\versal{GUIMARÃES},
  Ulysses. ``A Constituição cidadã.'' \emph{Discurso pronunciado pelo
  Presidente da Assembleia Nacional Constituinte, Deputado Ulysses
  Guimarães, na Sessão} 27 (1988).} de 1988, os governos
eleitos\footnote{Após o impeachment (renúncia) de Fernando Collor de Mello
  (1990--1992) assume em 29 de dezembro de 1992 Itamar Franco até o
  dia 1º de janeiro de 1995.}
subsequentes adotaram uma agenda neoliberal\footnote{Conceito a partir de:
  Teixeira, Francisco José Soares. ``O neoliberalismo em
      debate.'' \emph{Neoliberalismo e reestruturação produtiva: as novas
  determinações do mundo do trabalho} 2 (1996): 195-252.}. Collor e
Fernando Henrique Cardoso não priorizaram a execução das políticas
sociais contidas na Constituição: pelo contrário, privatizações,
flexibilização das leis trabalhistas, terceirizações, desmonte do papel
do Estado seguiram norteando o poder executivo. Na política para as
Mulheres, assim como nas temáticas sociais, ocorreram algumas melhorias
pontuais.

Mas é no governo Lula que se dá o encontro entre a Carta Magna e o poder
executivo. Já no dia 1º de janeiro de 2003, em um dos primeiros atos como
presidente da República, Luiz Inácio Lula da Silva criou
a Secretaria Especial de Políticas para as
Mulheres da Presidência da República (\versal{SPM}/\versal{PR}) com status de ministério,
dando um salto à formulação, coordenação e articulação de políticas que
promovam a igualdade entre mulheres e homens\footnote{``Com Lula e Dilma, igualdade de gênero vira política de Estado'', \emph{por Luana Spinillo}, da Agência \versal{PT} de
  Notícias.}, consolidou ações como a Lei Maria da Penha
e o Disque 180, lei que criminaliza o feminicídio. Ele rompe com a
agenda neoliberal com a execução de programas como Bolsa Família, Fome
Zero, Minha Casa Minha Vida. Evidencia"-se no período o Estado como
indutor de desenvolvimento, através de políticas públicas voltadas para
a inclusão social e distribuição de renda.

Em 31 de outubro de 2010 Dilma Rousseff é eleita presidenta do Brasil,
sendo a 36ª presidente da República e a primeira mulher a ocupar o cargo
na história do Brasil, dando continuidade ao papel fundamental do
Estado no combate às desigualdades sociais e de gênero. Desenvolvendo
políticas públicas voltadas para a inserção e a permanência das mulheres
no mundo do trabalho e a ampliação dos seus direitos sociais. Ela
implementou o programa Brasil Carinhoso, contribuiu para adoção de leis
que ampliam os direitos das trabalhadoras domésticas, das licenças
maternidade e paternidade\footnote{Dilma sancionou, sem vetos, a lei que
  cria a Política Nacional Integrada para a Primeira Infância e que
  permite, entre outros pontos, que as empresas possam ampliar de 5 para
  20 dias a duração da licença"-paternidade.}; destacou a importância da
agenda do trabalho decente e da ampliação da oferta de vagas em creches.
Ressaltamos que estas são algumas das medidas que reforçam a autonomia
econômica das mulheres e promovem a igualdade no mundo do trabalho. Assim, a
Presidenta Dilma reforça a política de seu antecessor que coloca as
mulheres como principais beneficiárias e titulares dos programas
sociais.

O crescente empoderamento das mulheres, dos direitos reprodutivos, da
liberdade sexual, não seriam possíveis sem estes governos, que deram ao
povo brasileiro um Estado e políticas públicas comprometidas com os que
mais precisam. É preciso reconhecer que passos largos foram dados na
transversalidade de gênero nas políticas interministeriais, de forma a
confirmar o protagonismo das mulheres na construção de um projeto de
sociedade mais justa, igualitária e democrática.

Entretanto, o impacto cultural de termos uma mulher ocupando o mais alto
cargo do poder executivo não foi pequeno. Se por um lado essa
representação deu o sonho às mulheres de ideias de equidade, de
empoderamento, de divisão igualitária dos afazeres domésticos, de
construção de uma nova ordem societária sem discriminações de gênero;
aos misóginos e machistas o fato de termos uma mulher presidenta os dava
passe livre para bradar insultos, piadas e adjetivos vexatórios, como se
o fato de ser mulher permitisse uma desqualificação irreparável.

A nossa sociedade é patriarcal, as mulheres são vistas como
inferiores, e as próprias mulheres muitas vezes incorporam e reproduzem
essa ideologia, enfraquecendo umas às outras, permitindo que os homens e
as macro estruturas culturais e sociais reiterem a presença hegemônica
dos homens nos espaços de poder. A mulher ainda é vista como pertencente
ao ambiente privado, que deve ficar em casa, sendo a esfera pública um
espaço do homem, seguindo preceitos de funcionamento masculinos.

É neste cenário de disputa de valores que se dá o Golpe midiático,
parlamentar e jurídico encabeçado pelo então vice"-presidente da
República Michel Temer. Para além do retrocesso da própria
democracia brasileira, esta afronta também atingiu as mulheres da nossa
nação, pois retiraram da presidência uma mulher honesta, com capacidade
reconhecida de trabalho e de luta por um Brasil melhor.

O governo golpista tratou de retomar a agenda neoliberal, com desmontes e
retrocessos na pauta das mulheres e em todas as áreas sociais, jurídicas
e econômicas.

No entanto, ainda hoje, 30 anos depois da promulgação da Constituição,
há uma grande distância entre o que a lei e o Estado garantem e o que a
realidade impõe.

Apesar da garantia legal para as mulheres, dos avanços em torno dos
temas de violência e feminicído, por exemplo, ainda há dificuldades em
torno da superação de condições estruturais adversas ao exercício de
seus direitos. O machismo estrutural da sociedade brasileira, além do
avanço das pautas conservadoras dentro das instituições representativas,
e o golpe dado em 2016, também de caráter machista contra a primeira
mulher a presidir a república federativa do Brasil, impõe barreiras e
desafios para a luta das mulheres.

O golpe dado na Presidenta Dilma foi um golpe em todas as mulheres, já
que, para além do peso político de se retirar do poder uma mulher
honesta e legitimamente eleita, as políticas para as mulheres foram
enfraquecidas e escamoteadas.

O desmonte do Estado, motivação principal do golpe, significa na prática
o corte de recursos para política social, afetando programas que têm
como principais beneficiários as mulheres, como o Programa Bolsa
Família. Não obstante, a política econômica excludente afeta diretamente
mulheres, que compõem parte significativa da massa desempregada e cada
vez mais empobrecida\footnote{De acordo com dados da \versal{PNAC"-C} referentes
  ao quarto trimestre de 2016, a taxa de desemprego entre mulheres era
  3,1\% maior do que entre os homens (\versal{IBGE}, 2017).}. Quando considerada
a interseccionalidade entre raça e gênero, os números são historicamente
maiores.

O quesito raça/cor é ponto relevante quando se trata de violência
contra as mulheres no Brasil. Violências psicológicas, morais, verbais,
físicas e sexuais acometem as mulheres de todas as classes sociais, de
todas as etnias, mas é na mulher negra que esta realidade se impõe com
maior veemência. Segundo diagnóstico do Ministério da Justiça (2015), as
mulheres negras têm o dobro de chances de serem assassinadas do que as
mulheres brancas, o que é um grave exemplo de como a sociabilidade atual
ainda ratifica as opressões e vulnerabiliza ainda mais as mulheres não
brancas.

Motivos não nos faltam para lutar, o que está em jogo é qual o país que
queremos. Excludente ou inclusivo? Opressor, preconceituoso, ou plural e
libertário? Um Estado público ou privado? Comprometido com os que mais
precisam ou com o grande capital?

Para retomarmos a estagnação do neoliberalismo e do conservadorismo a
palavra de ordem é resistência. O Brasil ainda detém números vergonhosos
no campo da violência contra as mulheres; com tripla jornada, elas ainda
têm salários menores quando ocupam os mesmos cargos e funções que os
homens, com pequena presença nos espaços decisórios de empresas e nas
esferas da intervenção pública.

A luta contra o patriarcado é coletiva, de homens e mulheres que têm o
ideal de construção de uma nova cultura política sem opressões,
explorações e desigualdades.

Mais que belas e aguerridas palavras, a Constituição de 1988 ainda é uma
meta a ser alcançada, e para as mulheres, ela foi fortemente atacada e
golpeada quando tiraram a Presidenta Dilma do poder. A melhor resposta é
organização coletiva, e muita luta.

Sigamos nas trincheiras contra o machismo até que todas sejamos livres!

\section{Referências bibliográficas}

\begin{Parskip}
\versal{ABADIA}, Maria de Lourdes. Apesar dos avanços, há discriminação. Jornal
da Constituinte, Brasília, nº 38, 7 a 13 de março de 1988, p. 4.

Conselho Nacional dos Direitos da Mulher. Carta das Mulheres Brasileiras
aos Constituintes de 1987. \emph{Assembleia Nacional Constituinte}.
Congresso Nacional, Brasília, 1986. Disponível em:
\textless{}\emph{https://bit.ly/2eAHMox}\textgreater{}.

Conselho Nacional dos Direitos da Mulher. Constituição de 1988
é marco na proteção às mulheres. \emph{Secretaria de Políticas para as
Mulheres -- \versal{SPM}}, Presidência da República. Brasília, 2014.
Disponível em:
\textless{}\emph{https://bit.ly/2NpjFGn}\textgreater{}.

Instituto Brasileiro de Geografia e Estatística (\versal{IBGE}). Pesquisa
Nacional por Amostra de Domicílios Contínua -- 4º trimestre de 2016.
Brasília, 2017.

\versal{SANTOS}, Eurico A.G.C. dos; \versal{BRANDÃO}, Paulo H.; \versal{AGUIAR}, Marcos M. de. Um
toque feminino: recepção e formas de tratamento das proposições sobre
questões femininas no Parlamento Brasileiro, 1826-2004. In: \versal{SENADO}
\versal{FEDERAL}. Proposições legislativas sobre questões femininas no Parlamento
Brasileiro, 1826-2004. Brasília: Senado Federal, Comissão Temporária do
Ano da Mulher/Subsecretaria de Arquivo, 2004.

\versal{SENKEVICS}, Adriano. Mulheres e feminismo no Brasil: um panorama da
ditadura à atualidade. \emph{Ensaios de Gênero,} 2013\emph{.} Disponível
em: \textless{}\emph{https://bit.ly/1PuJNOU}\textgreater{}.

\versal{SOUZA}, Marcius F. B. de. A Participação das mulheres na elaboração da
Constituição de 1988. In: Constituição de 1988 : O Brasil 20 anos
depois. Os Alicerces da Redemocratização. v.1 Senado Federal, Brasília,
2008. Diponível em: \textless{}\emph{https://bit.ly/2y2sYHh}\textgreater{}.
\end{Parskip}