\chapter*{A classe trabalhadora e a luta em defesa intransigente da
Constituição Brasileira}

\addcontentsline{toc}{chapter}{A classe trabalhadora e a luta em defesa intransigente da
Constituição Brasileira, \scriptsize{por Carlos Veras}}
\hedramarkboth{A classe trabalhadora e a luta em defesa\ldots{}}{}

\begin{flushright}
\emph{Carlos Veras}\footnote{Presidente da \versal{CUT}"-\versal{PE}}
\end{flushright}

Parto do princípio de que o trabalho é um elemento definidor do próprio
ser humano ou de sua dimensão ontológica como, destacadamente, defende
Kalr Marx. Nessa perspectiva, o trabalho seria determinante do ser,
considerando que este produz as condições reais de sua existência. Ou
seja, o trabalho mediaria a relação \emph{entre o sujeito e o objeto do
seu carecimento}. Tal conceito tem grande relevância por não se
aprisionar dentro da naturalidade do ser, já que se transformam ao longo
da história os objetos da necessidade humana, bem como os modos de
satisfação destes, conforme argumenta o citado autor. Nessa ótica, o
trabalho seria o bem mais importante do ser humano e aliená"-lo, isto é,
transferir o direito de proveito dos frutos desse trabalho para outra
pessoa, seria o mesmo que alienar o direito à própria vida. Foi por essa
razão que o sociólogo definiu a força de trabalho como o bem
``inalienável'' do ser humano.

Ao nos debruçarmos sobre os períodos históricos anteriores -- na Idade
Média, por exemplo --, constataremos que o trabalho rural era a
principal forma de labor da época. A produção de alimentos ou de outros
bens de consumo estava diretamente ligada à necessidade daqueles que o
produziam. Assim, os(as) trabalhadores(as) do campo não produziam em
função de lucro ou da moeda corrente, mas para consumo próprio. O
comércio estava circunscrito a formas elementares de troca de produtos
produzidos por outros trabalhadores(as), de modo que o trabalhador(a)
mantinha contato direto com o que produzia. Tratava"-se de uma relação
próxima entre produto, produção e consumo.

Com o advento da Revolução Industrial (1760-1840), houve uma mudança
significativa nas relações sociais e nas relações de trabalho. O
surgimento das cidades deslocou o indivíduo que dependia da terra para a
sua sobrevivência para os centros urbanos no movimento acachapante de
êxodo rural. Segundo Marx, esse novo indivíduo urbano perdeu seu acesso
à terra a partir do momento no qual surge uma classe de trabalhadores
que deveria vender sua força de trabalho.
Para ele, marca"-se então uma diferença
histórica entre as relações de produção capitalistas e as relações de
produção pré"-capitalistas. O modo de produção capitalista caracteriza"-se
pela impessoalidade do(a) trabalhador(a) com o que produz, isto é, ele
não possui nenhuma ligação pessoal com o que está produzindo, uma vez
que não encadeia o processo produtivo. Já nas relações de produção
pré"-capitalistas, o produto do trabalho estava intrinsecamente
relacionado ao(à) trabalhador(a), que era o(a) protagonista de toda a
cadeia produtiva. Segundo o filósofo, essa característica rege as
relações de trabalho em uma sociedade capitalista, na qual o(a)
trabalhador(a) que não possui os meios de produção para produzir o que
necessita para sobreviver, passa a vender a única ``mercadoria'' que
detém: sua força de trabalho. É essa nova forma de se relacionar com o
trabalho que transforma as relações sociais em todos os aspectos. O
sujeito, antes integrado ao seu labor, agora, encontra"-se apartado do
que produz, nunca colhendo os frutos de seu trabalho. Esse trabalho, por
sua vez, passa a ser comprado por um salário, que, na maior parte das
vezes, é insuficiente para uma vida digna.

É nesse contexto histórico que também estava inserida a classe
trabalhadora na segunda metade do século \versal{XX} no Brasil. A construção de
uma Constituição Cidadã para os(as) trabalhadores(as) brasileiros(as)
configurou"-se nesse período como uma frente estratégica de uma luta
iniciada no século \versal{XVI}, com uma colonização marcada pela resistência dos
povos negros e dos povos indígenas, passando pelo Império. E, já na
República Brasileira, o embate dos(as) trabalhadores(as) rurais e
urbanos entram em cena. A Carta Magna não foi uma benesse do tempo e nem
muito menos uma concessão da Casa Grande, mas resultado de uma febril
luta de bravos(as) trabalhadores(as) brasileiros(as) do quilombo, da
aldeia, do campo e da cidade. São milhões de marias e de joãos que
fizeram greves e marchas camponesas, colocando suas vidas em risco para
afirmação dos direitos da classe trabalhadora tendo em vista a criação
de um ambiente político favorável à construção da democracia no País.

Debatida e produzida no fim de uma década marcada por acontecimentos
políticos e econômicos que mudariam o Brasil -- encerramento do regime
militar, fundação do Partido dos Trabalhadores (\versal{PT}), criação da Central
Única dos Trabalhadores (\versal{CUT}), realização da primeira eleição direta
para presidente da República-, a Constituição de 1988 é considerada por
especialistas, entre todas as sete, a mais avançada da história
brasileira em relação aos direitos sociais e às garantias e individuais.
Quando promulgada em outubro daquele ano, acreditávamos ter garantido
avanços sociais sem precedentes para a classe trabalhadora e de ter sido
resultado de uma Assembleia Constituinte que promoveu um rearranjo
institucional do País em substituição à Carta Magna imposta pelo regime
militar em 1967. Sabíamos que ainda seria necessária muita luta por
regulamentações e emendas que ratificassem seus avanços em favor da
classe trabalhadora e lutamos para garantir que seus artigos fossem de
fato implementados no campo e no chão da fábrica.

No calor da Assembleia Constituinte instalada em 1985 e que duraria 18
meses até a promulgação, ao tempo em que ocorriam os debates sobre as
mais de 70 mil emendas populares aditivas ao texto dos parlamentares,
explodiam grandes greves, como a dos eletricitários, que atingiu sete
Estados; petroleiros, em mais oito, e a dos servidores federais, que
mobilizou 400 mil trabalhadores em todo o País.

Indiscutivelmente, a Constituição de 1988 registrou progressos
importantes, contudo, poderia ter avançado mais, particularmente quanto
à ampliação da participação popular e do controle da cidadania sobre os
Poderes, como forma de avançar na democracia participativa, assim como
poderia ter sido mais ampla no capítulo relativo às relações de trabalho
e direitos dos(as) trabalhadores(as).

Hoje, 30 anos depois da promulgação da Carta Magna, a classe
trabalhadora brasileira trava desafiantes batalhas contra inimagináveis
retrocessos impostos a partir do golpe de 2016, representado pelo
ilegítimo, corrupto e majoritariamente rejeitado Michel Temer. Seu
programa ultraliberal e entreguista desvirtua mais de cem artigos
constitucionais, o que nos remete ao século passado, não apenas no campo
das relações de trabalho, mas nos direitos sociais de modo geral.

Centenas de milhões de pessoas que começaram a sonhar com uma vida
digna, hoje se encontram diante de uma década perdida e destroçada por
uma recessão econômica gerada a partir de uma política econômica austera
com o povo brasileiro e benevolente com a elite capitalista nacional e
internacional.

Em verdade, as tais medidas ditas austeras atendem pelo nome de Estado
Mínimo que reduz drasticamente investimentos em saúde, educação e
seguridade social, além de entregar a autonomia econômica do País nas
mãos do mercado financeiro. Isto significa o desmonte do Estado de
Bem"-Estar Social que vinha sendo construído no Brasil há pouco mais de
uma década. Aliás, a instalação do projeto ultraliberal em curso é a
contrapartida de Michel Temer e seus apaniguados para a decrépita e
descontente Casa Grande e para o capital internacional que financiaram a
deposição de um governo legitimamente eleito pelo povo para impor o
projeto liberal que vinha sendo derrotado nas urnas pela quarta vez
consecutiva, sem dar sinais de êxito pelos próximos dez anos, no mínimo.
A raivosa e medíocre elite brasileira já não mais suportava assistir à
ascensão da classe trabalhadora em redutos tipicamente dominados por
homens brancos e ricos, enquanto que o capital internacional ambicionava
fincar novamente suas garras afiadas nas riquezas nacionais,
especialmente na área de petróleo que crescia exponencialmente com a
exploração da camada do pré"-sal.

O chefe Michel Temer não decepcionou seus fiadores. Logo tratou de
aprovar a Proposta de Emenda à Constituição do Teto dos Gastos Públicos
que congela por 20 anos os investimentos sociais com cortes letais em
setores fundamentais para os(as) brasileiros(as), especialmente
populações com baixa renda. A sanha ultraliberal logo fez aprovar a Lei
da terceirização e a reforma trabalhista do jeito que desejavam os
patrões. Em curso, um leilão para a venda de mais de 50 empresas
estatais. Os bancos públicos. também estão sob ameaça de privatização
com a implementação de processos de reestruturação que esvaziam seus
papéis sociais, fatiam setores e desligam empregados. Já a reforma do
Ensino Médio pretende restringir o acesso dos(das) filhos(as) dos(as)
trabalhadores(as) às universidades públicas e formatar mão de obra
barata para servir às grandes corporações empresariais. Para bater o
Teto dos Gastos Públicos, o governo golpista ainda novamente colocará em
pauta a reforma da Previdência sob o argumento de que o setor queda"-se
deficitário.

As tragédias social e econômica trazidas por este governo golpista não
poupam nem mesmo a Amazônia já prometida às mineradoras internacionais.
O fato causou repercussões e polêmicas entre vários setores da
sociedade, inclusive dos ambientalistas em níveis nacional e
internacional. Pois bem: houve recuo. O congelamento do plano de
exploração privada em reserva mineral respondeu a críticas. Agora, eles
querem promover "um amplo" debate com a sociedade sobre o tema, por 120
dias. E sabemos que, se depender do Congresso mais reacionário e
corrupto da história do Brasil, nosso maior patrimônio ambiental será
rifado.

Os ventos privatistas sopram nas janelas do Palácio do Governo do Estado
de Pernambuco. A Companhia Pernambucana de Gás (Copergás) está na mira
do setor privado. É uma empresa que vem gerando lucros cada vez maiores
a cada ano, mesmo com a situação de crise econômica no País. A empresa é
importante para indústrias, residências, automóveis, comércios e até
para a Refinaria de Abreu e Lima (\versal{PE}). Embora a estatal tenha acumulado
um lucro de R\$ 70,9 milhões em 2016, o Executivo estadual revela
interesse de vendê"-la.

Tal pacote de austeridade vem trazendo consequências nefastas aos(às)
trabalhadores(as) brasileiros e vultosos lucros para o mercado
financeiro. Não é à toa que mesmo envolvido em grandes escândalos de
corrupção, o (des)governo Temer ainda usufrui da confiança do setor de
capitais, cuja economia vai muito bem, enquanto o povo pena para
sobreviver.

Os cortes na área social têm reduzido drasticamente os investimentos em
áreas vitais para a população brasileira, especialmente, para o imenso
segmento com baixa. A navalha neoliberal vem mutilando políticas
públicas indispensáveis como os programas de distribuição de renda,
habitação popular, acesso ao ensino superior, obtenção de microcrédito e
desenvolvimento da agricultura familiar.

Com as medidas, mais de 50 milhões de brasileiros(as) que eram
beneficiados(as) pelas políticas sociais podem voltar à situação de
pobreza extrema. Atualmente, mais de 14 milhões de pessoas estão
desempregados ou veem diminuir sua renda, vivendo na incerteza em
relação ao trabalho e à proteção social. Em razão da regulamentação da
terceirização e da reforma trabalhista, os ainda empregados já sofrem
ameaça de demissão para contratação de profissionais terceirizados
precarizados com redução salarial de cerca de 30\%, sem definição de
salário"-mínimo, sem direito a férias, horas extras, 13º salário,
delimitação de carga horária, seguro"-desemprego etc. Em caso de admissão
ou demissão, não há mais a exigibilidade de o sindicato intermediar a
negociação, já que o acordado prevalecerá sobre o legislado, ou seja, o
acordo será fechado entre patrão e empregado numa correlação de força
desigual onde o poder estará todo concentrado nas mãos do empregador.
Quantos aos(às) trabalhadores(as) rurais, esses vão ficar no campo,
exercendo suas atividades de produção, sem ter direito à aposentadoria,
sendo permitido apenas que eles ganhem moradia e alimentação, remontando
à época da escravidão. Sem condições de trabalho, sem direitos a
créditos financeiros, inúmeros chefes de famílias virão para as grandes
cidades que não comportarão um número excessivo de trabalhadores(as),
aumentando desta forma o desemprego, o trabalho infantil, a miséria e a
violência. O impacto será desastroso, considerando que a agricultura
familiar se destaca como um dos setores da economia que mais cresce na
produção de alimentos, geração de riquezas e em distribuição de renda,
contribuindo para o desenvolvimento rural e sustentável. A atividade é
uma das principais geradoras de trabalho e renda na América Latina e
Caribe, segundo o relatório ``Perspectivas da Agricultura e do
Desenvolvimento Rural nas Américas 2014: uma visão para a América Latina
e Caribe'' produzido pela Organização das Nações Unidas para Alimentação
e Agricultura (\versal{FAO}) e pelo Instituto Interamericano de Cooperação para a
Agricultura (\versal{IICA}). As reformas supracitadas pretendem desidratar o
setor e lançar mais investimentos no agronegócio que cria poucos
empregos e apenas gera \emph{``Royalties''} que não são servidos na mesa
dos(as) trabalhadores(as) brasileiros(as) como o arroz e o feijão.

Após arruinadas as políticas sociais e extirpados os direitos da classe
trabalhadora, é chegada a hora de entregar as riquezas nacionais ao
capital estrangeiro. O programa que deve ser finalizado até 2018 inclui
58 estatais, entre elas, aeroportos, portos, áreas de exploração de
petróleo, Eletrobras - Centrais Elétricas Brasileiras S.A, Companhia
Hidroelétrica do São Francisco (Chesf) e até a Casa da Moeda. Conforme o
histórico das privatizações no Brasil, as contas públicas pioraram
substancialmente, enquanto que o desempenho das empresas sob a
responsabilidade do setor privado avançou consideravelmente. Em apenas
um decênio, por exemplo, a quantidade de empregados nas empresas
privatizadas caiu 70,5\% (de 95 mil, em 1995, para 28 mil, em 2005),
enquanto a lucratividade foi multiplicada por 10 vezes (de R\$ 11
bilhões, em 1995, para R\$ 110 bilhões, em 2005). Ademais da ação
voltada para a redução de custos, como a demissão em massa, as empresas
privadas elevaram radicalmente o lucro por meio do significativo
crescimento dos seus preços acima da inflação. No setor elétrico, por
exemplo, o preço médio da energia elétrica ao consumidor subiu próximo
de 120\% acima da inflação entre 1995 e 2015, ou seja, 4\% em média de
aumento real ao ano. Em síntese, a privatização tornou"-se um mito
neoliberal. Não contribui na melhora da contabilidade pública, mas eleva
o custo de produção com preços de bens e serviços de empresas
privatizadas crescendo acima da inflação. Ou seja, lucros de países
ricos combinados com preços e qualidade dos bens e serviços de país
pobre, sem tocar no crescimento econômico, nem na melhora das contas
públicas.

A onda privatista também ameaça os bancos públicos que têm como missão
primordial fomentar o desenvolvimento social, além de funcionarem como
órgãos reguladores da economia. O desmonte tendo em vista a venda já
começou com os processos de reestruturação, que na prática significam
fechar agências e setores e desligar milhares de empregados, com já
fizeram com a Caixa, o Banco do Brasil e o Banco do Nordeste do Brasil.
Essa desestruturação não apenas eleva o número de desempregados, como
piora o atendimento ao público. Mas, a real motivação é conceder ao
mercado rentista a parte lucrativa desses bancos, como por exemplo, a
administração do Fundo de Garantia do Tempo de Serviço (\versal{FGTS}) que somam
mais de R\$ 500 bilhões. Até então, esse fundo vem sendo investido em
programas como Minha Casa, Minha Vida, Financiamento do Ensino Superior
(Fies), Bolsa Família, em infraestrutura das cidades e tantas outras
áreas imprescindíveis ao desenvolvimento social e econômico do País. Já
o Banco do Brasil, por exemplo, é o principal fomentador do Programa
Nacional de Agricultura Familiar (Pronaf), que é responsável pela
produção de 70\% dos alimentos que chegam à mesa do(a) brasileiro(a).
Enquanto que o Banco do Nordeste do Brasil (\versal{BNB}) é o maior banco de
desenvolvimento regional da América Latina e é responsável por fomentar
o desenvolvimento social e econômico da Região Nordeste. O operador do
Fundo Constitucional de Financiamento do Nordeste (\versal{FNE}), beneficia a
economia de cerca de 2 mil municípios com linhas de financiamento para
micro e pequenas empresas e microempreendedores individuais. É certo que
esses ativos nas mãos do mercado especulativo serão utilizados para a
prática de agiotagem com juros elevadíssimos, tendo em vista, meramente,
a obtenção do lucro. Sua concretização trará prejuízos irreparáveis
aos(às) trabalhadores(as) com impactos nefastos sobre os programas de
habitação popular, de distribuição de renda, de segurança alimentar, de
acesso ao ensino superior, de obtenção do microcrédito e para a
autonomia econômica do País.

É inegável que a \versal{CUT} se constituiu um ator social importante, não apenas
na defesa dos direitos da classe trabalhadora, mas, ativista
intransigente na luta pelas liberdades democráticas. Assim atuou contra
o Regime Militar, contra a flexibilização nas Leis Trabalhistas nos anos
2000. É chegada a hora de novamente recobrarmos o espírito libertário, a
força e a coragem que explodiram na década de 1980 para desta vez
defendermos com unhas e dentes todas as nossas conquistas
constitucionais. Temer e seus comparsas não têm história política e nem
legitimidade popular para alterar uma linha sequer do maior patrimônio
da nação brasileira: a Constituição de 1988.

Imbuída dessa vital missão, a classe trabalhadora brasileira novamente
ocupa as ruas desse País. Registramos importantes momentos de
resistência que certamente constarão nos capítulos de nossa história
contemporânea. Iniciaremos pela página 28 de abril, cuja greve geral
tomou uma dimensão nunca vista ao se enraizar por todo território
nacional, com centenas de categorias cruzando os braços nos 26 Estados e
no Distrito Federal, dispostas a barrar as temerosas reformas. A
mobilização prosseguiu vibrante nos dias 8 e 15 de março de 2017, para
denunciar e repudiar a reforma da Previdência, que pretende acabar com a
seguridade social no Brasil. Saímos novamente às ruas para denunciar e
repudiar a reforma trabalhista, que rasga a Consolidação das Leis
Trabalhistas (\versal{CLT}) e precariza as relações de trabalho. Fomos para as
ruas repudiar o Projeto de Lei (\versal{PL}) 4.302, aprovado na Câmara dos
Deputados numa manobra espúria do presidente da casa, Rodrigo Maia(\versal{DEM}),
que fragiliza a organização sindical e permite a terceirização da
atividade"-fim, condenando os(as) trabalhadores(as) a ``viverem de
bico'', sem nenhuma garantia dos direitos básicos, como férias, 13º
salário, jornada de trabalho delimitada, descanso remunerado, pagamento
de horas extras, entre outros direitos fundamentais conquistados após
décadas de lutas.

Desde as manifestações do Grito dos Excluídos, em dia 7 de setembro de 2017, a \versal{CUT} esteve nas ruas de todo o País com a Campanha pela Anulação da Reforma Trabalhista, que coletou 1,3 milhão de assinaturas para um Projeto de Lei de Iniciativa Popular que propõe a revogação da reforma trabalhista, que entrou em vigor no dia 11 de novembro de 2017. 

Após o recolhimento das assinaturas, o projeto foi entregue à Câmara dos Deputados. O objetivo do Projeto de Lei de Iniciativa Popular é fazer com que essa medida se some a outras 11 leis revogadas por meio desse instrumento. 

A campanha pela anulação da reforma trabalhista foi aprovada pelas confederações, federações e sindicatos da \versal{CUT}, durante o recente Congresso Extraordinário e prevê também a criação de comitês por essas entidades.

No dia 11 de novembro de 2017, dia em que entrou em vigor a reforma trabalhista, os movimentos sindical e social protestaram em Brasília (\versal{DF}). Na ocasião, a Central apresentou o projeto pela revogação do ataque aos direitos da classe trabalhadora. A luta é pela revogação não só da reforma trabalhista, mas de todas as decisões tomadas durante esse governo golpista, nocivas aos trabalhadores e à soberania nacional. A \versal{CUT} tem feito enfrentamento extraordinário contra o golpe e pelo restabelecimento democrático do País. Contra essa política de Estado Mínimo, a \versal{CUT} propõe a anulação dos atos do governo ilegítimo de Michel Temer, a restauração da democracia e a retomada de um projeto soberano e sustentável de crescimento do País. 

Em Pernambuco, juntos com a Frente Brasil Popular (\versal{FBP}), faremos a segunda etapa da Caravana Popular em Defesa da Democracia e dos Direitos e contra as privatizações. Todas as medidas tomadas a partir do golpe de 2016 são inaceitáveis. Vamos à luta! A \versal{CUT} e os movimentos sociais seguem firmes. Estamos articulando grandes ações e estratégias para construir uma nova Greve Geral. Nós vamos parar novamente e quantas vezes forem necessárias este País, sem ódio e sem medo. A classe trabalhadora não vai baixar a cabeça e permitir novos ataques promovidos por este (des)governo e, sobretudo, por este Congresso golpista, ilegítimo e corrupto que está surrupiando nossos direitos. 

Nosso compromisso, nossa obrigação: defender cada vez mais os(as) trabalhadores(as) pela liberdade e autonomia sindical, para que nenhuma mulher ganhe menos numa mesma função que os homens, para que os índios sejam respeitados, os negros sejam valorizados e que cada vez mais tenhamos uma central sindical pujante com grande capacidade de mobilização e com propósito de melhorar a vida de milhões de brasileiros. 

Nesse contexto do golpe concatenado entre os Poderes, o capital e a mídia corporativa, sabemos o quanto é desafiante reverter os retrocessos, pois nosso movimento não se alimenta de fábulas, mas de lutas históricas cujos frutos, muitas vezes, são colhidos por gerações futuras. Mas, a disputa do projeto de sociedade se faz com o verbo lutar no presente do indicativo, conjugado em todas as pessoas. É hora de atualizarmos coletivamente a nossa análise de conjuntura, as estratégias, as forças e o plano de luta construindo a unidade com os movimentos sociais para enfrentarmos e lutarmos ainda mais contra os retrocessos, em defesa da democracia e por um País mais igualitário e justo. O grande desafio da \versal{CUT} é manter o foco de luta e fortalecer a democracia, juntamente a todas as forças autênticas, democráticas e resistentes. 

Fora Temer! Nenhum direito a menos! 

No esteio de toda nossa luta está a defesa intransigente da Constituição Brasileira como premissa para a afirmação dos nossos direitos humanos, políticos, econômicos, culturais, ambientais, trabalhistas e de outros campos que ainda precisamos construir e avançar. É amparada no escopo constitucional que a classe trabalhadora busca resgatar o trabalho, hoje alienado do trabalhador, como o bem mais importante do ser humano intrinsecamente ligado ao direito à própria vida no sentido mais amplo e mais profundo da existência humana. É esse sonho que nos inspira fé e coragem para seguirmos firmes na luta. 

%SUBSTITUIDO POR CLEONILDO
%Desde as manifestações do Grito dos Excluídos, em dia 7 de setembro de
%2017, a \versal{CUT} está nas ruas de todo o País com a Campanha pela Anulação da
%Reforma Trabalhista, que vai coletar 1,3 milhão de assinaturas para um
%Projeto de Lei de Iniciativa Popular que propõe a revogação da reforma
%trabalhista, prevista para entrar em vigor no próximo dia 11 de novembro
%de 2017.
%
%Após o recolhimento das assinaturas, o projeto será entregue à Câmara
%dos Deputados, com o lançamento de uma nova etapa da campanha, para
%exigir a votação da proposta. O objetivo do Projeto de Lei de Iniciativa
%Popular é fazer com que essa medida se some a outras 11 leis revogadas
%por meio desse instrumento.
%
%A campanha pela anulação da reforma trabalhista foi aprovada pelas
%confederações, federações e sindicatos da \versal{CUT}, durante o recente
%Congresso Extraordinário e prevê também a criação de comitês por essas
%entidades, para coleta de assinatura.
%
%Para a semana de 11 de novembro de 2017, dia em que pode entrar em vigor
%a reforma trabalhista, os movimentos sindical e social preparam uma
%manifestação em Brasília(\versal{DF}). Na ocasião, a Central pretende já ter um
%número suficiente de assinaturas para apresentar o projeto pela
%revogação do ataque aos direitos da classe trabalhadora. A luta é pela
%revogação não só da reforma trabalhista, mas de todas as decisões
%tomadas durante esse governo golpista, nocivas aos trabalhadores e à
%soberania nacional. A \versal{CUT} tem feito enfrentamento extraordinário contra
%o golpe e pelo restabelecimento democrática no País. Contra essa
%política de Estado Mínimo, a \versal{CUT} propõe a anulação dos atos do governo
%ilegítimo de Michel Temer, a restauração da democracia e a retomada de
%um projeto soberano e sustentável de crescimento do País.

%Em Pernambuco, juntos com a Frente Brasil Popular (\versal{FBP}), faremos a
%segunda etapa da Caravana Popular em Defesa da Democracia e dos Direitos
%e contra as privatizações. Todas as medidas tomadas a partir do golpe de
%2016 são inaceitáveis. Vamos à luta! A \versal{CUT} e os movimentos sociais
%seguem firmes. Estamos articulando grandes ações e estratégias para
%construir uma nova Greve Geral. Nós vamos parar novamente e quantas
%vezes forem necessárias este País, sem ódio e sem medo. A classe
%trabalhadora não vai baixar a cabeça e permitir novos ataques promovidos
%por este (des)governo e, sobretudo, por este Congresso golpista,
%ilegítimo e corrupto que está surrupiando nossos direitos.
%
%Nosso compromisso, nossa obrigação: defender cada vez mais os(as)
%trabalhadores(as) pela liberdade e autonomia sindical, para que nenhuma
%mulher ganhe menos numa mesma função que os homens, para que os índios
%sejam respeitados, os negros sejam valorizados e que cada vez mais
%tenhamos uma central sindical pujante com grande capacidade de
%mobilização e com propósito de melhorar a vida de milhões de
%brasileiros.

%Nesse contexto do golpe concatenado entre os Poderes, o capital e a
%mídia corporativa, sabemos o quanto é desafiante reverter os
%retrocessos, pois nosso movimento não se alimenta de fábulas, mas de
%lutas históricas cujos frutos, muitas vezes, são colhidos por gerações
%futuras. Mas, a disputa do projeto de sociedade se faz com o verbo lutar
%no presente do indicativo, conjugado em todas as pessoas. É hora de
%atualizarmos coletivamente a nossa análise de conjuntura, as
%estratégias, as forças e o plano de lutas construindo a unidade com os
%movimentos sociais para enfrentarmos e lutarmos ainda mais contra os
%retrocessos, em defesa da democracia e por um País mais igualitário e
%justo. O grande desafio da \versal{CUT} é manter o foco de luta e fortalecer a
%democracia, juntamente a todas as forças autênticas, democráticas e
%resistentes.

%Fora Temer! Nenhum direito a menos!

%No esteio de toda nossa luta está a defesa intransigente da Constituição
%Brasileira como premissa para afirmação dos nossos direitos humanos,
%políticos, econômicos, culturais, ambientais, trabalhistas e de outros
%campos que ainda precisamos construir e avançar. É amparada no escopo
%constitucional que a classe trabalhadora busca resgatar o trabalho, hoje
%alienado do trabalhador, como o bem mais importante do ser humano
%intrinsecamente ligado ao direito à própria vida no sentido mais amplo e
%mais profundo da existência humana. È esse sonho que nos inspira fé e
%coragem para seguirmos firmes na luta.
