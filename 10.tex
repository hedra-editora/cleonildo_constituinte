\chapter{A defesa desvirtuada -- 30 anos da Contituição Cidadã}

\begin{flushright}
\emph{Ademar Rigueira Neto}
\end{flushright}

Há trinta anos deixei a Casa de Tobias Barreto, a inesquecível Faculdade
de Direito do Recife, para investir num mercado de trabalho estimulado
pela recente vigência da carta cidadã. Vivíamos tempos de euforia, o
Estado Democrático de Direito se consolidava com a promulgação da
Constituição Federal. O exercício pleno da cidadania restava-se, para
nós, garantido e assegurado com o elenco irremovível dos direitos
individuais previstos no art. 5º da Lei Maior. Finalmente a luta, o
denodo, o sacrifício pessoal de tantos, na intransigência necessária
para se combater o arbítrio, parecia-nos compensada.

As crises antigas, no entanto, não foram remediadas, combatidas,
tampouco extintas pela nova ordem constitucional. A vigência dos
preceitos não impulsionou a normal e indispensável aplicabilidade dos
institutos.

Ganhou corpo no Judiciário brasileiro a corrente daqueles que defendiam,
de forma equivocada, à proteção desmedida do interesse público em
detrimento das garantias individuais. A busca da verdade ``real'' passou
a não ter limites, não se contrapondo ao necessário respeito às
formalidades dos atos processuais. A prova descoberta passou a ser o
único fim almejado, quase sempre na desconsideração cínica dos meios
utilizados. O processo penal, antes empregado na salvaguarda dos
interesses dos acusados em geral, transformou-se em rito de passagem
para condenações que espelham unicamente o sentimento da justiça
subjetiva do julgador.

O réu deixou de ser cidadão de direitos e esse conceito não vem apenas
da toga, mas é forjado por uma ``moralidade média'' que autoriza e
contribui com o sacrifício dos direitos mais elementares da pessoa
humana.

As pessoas e seus conceitos pré-estabelecidos impõem ao Judiciário
``moderno'' alternativas arrimadas em conceitos ressurgidos da França
inquisitorial, retratadas com indignação por Fernando da Costa Tourinho
Filho de que ``se o réu era inocente, não necessitava de defesa'',
entretanto, ``se era culpado, era indigno de defesa''.

Sabemos que só pode haver justiça dentro da lei, garantindo-se ao
acusado o direito de defesa. Justiça fora da lei não é justiça; é
vingança, acerto de contas com as próprias mãos, arbítrio, violência. O
povo brasileiro, que em grande parte vive o tormento de um estado social
de fato e não de direito, acaba muitas vezes aprovando transgressões
legais em nome do que imagina ser justiça. Acostumou-se a viver sem
proteção legal.

Malgrado todo um cabedal de direitos e proteções individuais, que
disciplinam e protegem os acusados em processos criminais, criou-se uma
mentalidade média que justifica, ou tenta justificar, que o direito
coletivo deve se sobrepor às garantias individuais, elevando-se quase
sempre provas ilícitas, ou obtidas de forma ilícitas, àquelas
necessárias, ou mesmo indispensáveis, às investigações policiais.

Uma febre de prisões cautelares veio à tona no cenário do judiciário
brasileiro, sempre enaltecida e acompanhada pela mídia nacional,
reverberando discursos prontos, nos quais se alardeia a eficácia do
Estado no combate à criminalidade.

Tudo é feito para desmoralizar -- operações com títulos chamativos;
algemas desnecessárias; exposição da imagem dos acusados; fornecimento
clandestino aos órgãos de imprensa de provas obtidas mediante o sigilo
processual na intenção de antecipar um julgamento, antes mesmo do início
do processo judicial competente.

Não obstante os Tribunais tenham pacificado o entendimento de que as
prisões de natureza cautelar -- temporária e preventiva -- só devam ser
aplicadas em casos excepcionais, a exceção virou regra. O que se vê, são
prisões antecipadas fundadas em meras conjecturas acerca da
possibilidade de tornar o acusado a delinquir ou ainda na probabilidade
de frustrar a colheita de provas e a aplicação da lei penal.

Agora a situação se agrava, e o próprio Supremo Tribunal Federal
legitima prisões antecipadas, abrindo o leque autorizador das execuções
provisórias das penas.

O retrocesso desta feita não veio em conta-gotas. O salto, não só
aniquilou o princípio da presunção da presunção de inocência - art. 5º,
inc. LVII da novel Constituição -, mas invocou os tempos do Estado Novo
-- 1941 - quando se permitia a execução provisória da pena (CPP, art.
669, I). Como disse Alberto Zacharias Toron no prefácio do Livro ``Do
Alto da Tribuna''\footnote{DO ALTO DA TRIBUNA -- Ademar Rigueira Neto --
  Ed. Lumen Juris -- 1ª ed.}, ``\emph{pode até ser que o sistema
estivesse disfuncional, como disseram alguns ministros do STF e que em
outros países seja diferente. Todavia, temos lei e Constituição
regulando a matéria. Não por acaso o antigo Subprocurador Geral da
República, \textsc{Guilherme Magaldi}, redesenhou a pirâmide kelseniana
para, acima da Constituição, colocar a Suprema Corte}\footnote{Publicado
  no JOTA (www.jota.com.br) em 7/10/2016.}\emph{. Não se trata,
obviamente, de uma homenagem, mas do reconhecimento a um autoritarismo
voluntarista dos que pensam que assim vão fazer o sistema funcionar
melhor; talvez tenham razão, mas o respeito às leis e à Constituição
deveria vir antes. Do contrário, temos os juízes da Suprema Corte, como
os militares outrora, fazendo o que bem entendem...''.}

O Supremo chegou ao ponto de afirmar que são os Tribunais de Apelação
que fazem o exame sobre os fatos e provas da causa. É ali, segundo a
novel decisão, que se exaure essa possibilidade, não se prestando o
Supremo, tampouco o Superior Tribunal de Justiça ao debate da matéria
fática probatória, não se justificando o impedimento antes imposto à
execução provisória da pena. Não esclareceu a decisão que, não obstante
os Recursos Especiais e Extraordinários não admitam reexame de provas, a
valoração delas é plenamente admitida, principalmente na via do recursal
especial.\\
A prova produzida e examinada pelos Tribunais de Apelação, que foi capaz
de condenar e impulsionar a execução imediata da pena segundo o Supremo,
poderá, por lógico, ser valorada pelos Tribunais Superiores,
especialmente quando houver a necessidade de se atribuir um novo e
devido valor jurídico a fato incontroverso. Aos Tribunais Superiores
recaem em última análise a obrigação de zelar pela correta adequação das
provas ao ordenamento jurídico, ou seja, em dizer se os fatos apurados
podem ser considerados como uma conduta criminosa.~\\
Não há de ser razoável uma justificativa -- seja ela qual for -- que
imponha prematura pena privativa de liberdade a um acusado, quando ainda
cabível recurso que possa retificar decisão anterior de Tribunal que
condenou por error in judicando (equívoco do juízo na valoração das
provas) ou com error in procedendo (erro no proceder, cometido pelo
juiz).\\
Não há de ser razoável iniciar o cumprimento de uma pena quando,
estatisticamente, segundo dados obtidos no voto do ministro Celso de
Melo, 25\% dos recursos interpostos nos Tribunais superiores são
providos, modificando-se substancialmente os decretos condenatórios.\\
Para coroar o quadro, tivemos ainda uma recente tentativa, absurda e
desconexa, de fuzilar as garantias constitucionais. Num denominado
``pacote anticorrupção'' apresentado pelo Ministério Público Federal
(MPF) tentou-se nada menos do que restringir os efeitos e o cabimento do
\emph{habeas corpus}.

O remédio heroico, como deve ser efetivamente chamado o instituto, por
ser uma ação autônoma de impugnação, tem lugar ainda que em
concomitância com outro recurso ou mesmo no lugar deste. Deve ser
utilizado quando houver ameaça, ``ainda que remota, ao direito de ir e
vir garantindo a totalidade dos direitos do acusado relacionados com sua
liberdade de locomoção, ainda que este, na simples condição de
direito-meio, possa ser afetado apenas de modo reflexo, indireto ou
oblíquo\footnote{MELLO FILHO, José Celso. \textbf{Constituição Federal
  Anotada.} 2 ed. São Paulo: Saraiva, 1986.}.

Pois bem, a situação esteve posta e nela tentava o Ministério Público
castrar o \emph{writ}, transformando-o numa espécie de remédio apenas
para dirimir questões diretamente ligadas à liberdade de ir e vir. Um
absurdo inigualável, repudiado pelo Poder legislativo.

De outra parte, potencializava-se sensivelmente a \emph{investigação
secreta}, realizada, como método de ação, pela Polícia Federal e, não
raro, pelo Ministério Público, entregando-se os dois segmentos a
perquirições que alcançam, seguidamente, parâmetros constitucionalmente
inaceitáveis (chegam informações da existência de células oficiosas de
escuta telefônica, devassamento e captação de dados, como estratégia de
prospecção geral de delitos, tudo ao largo do controle jurisdicional).
Tais procedimentos ofendem o ordenamento jurídico brasileiro,
violentando o direito constitucional de intimidade e privacidade. Em
suma, constituem hipóteses concretas de infrações penais.

Mas não é só. Trinta anos se passaram da Constituição e as investigações
policiais continuam violando sistematicamente garantias individuais de
âmbito constitucional -- quebra do sigilo bancário, fiscal e telefônico,
publicidade opressiva, além da utilização desmedida de delação premiada
como forma de coação --, antes mesmo de uma análise mais criteriosa
acerca do conteúdo da denúncia, antes mesmo até da busca de indícios
justificadores da autoria, tentando-se obter, a qualquer custo, uma
prova suficiente, capaz de dar azo a uma investigação criminal. E o que
é pior, esses desmandos que violam os princípios da razoabilidade,
proporcionalidade e da dignidade da pessoa humana são cotidianamente
utilizados pela polícia judiciária do nosso país, sob o manto
ratificador de decisões judiciais desfundamentadas.

Na verdade, no que ouso chamar de processo penal contemporâneo, as
autoridades abusam da utilização de medidas cautelares com o fito de
produzir provas com a mitigação de garantias individuais protegidas
constitucionalmente. As violações quase sempre são justificadas pelo
reconhecimento da presença do interesse público/coletivo na
investigação.

Este interesse público fartamente utilizado como fundamento, também
possui previsão constitucional ao estabelecer limites às proteções
constitucionais. O art. 5º, inciso XII, da Constituição Federal
determina ao final as suas próprias exceções: \emph{é inviolável o
sigilo da correspondência e das comunicações telegráficas, de dados e
das comunicações telefônicas, salvo, no último caso, por ordem judicial,
nas hipóteses e na forma que a lei estabelecer para fins de investigação
criminal ou instrução processual penal};\emph{~~}

Trata-se de verdadeiro exercício de ponderação de princípios: afastam-se
garantias constitucionais sobre direitos individuais à intimidade,
privacidade, sigilo das correspondências e comunicações em prol do
interesse público inerente à investigação de crimes, considerando-se o
monopólio estatal do poder punitivo.

Malgrado se trate de exceção, aplicada portanto em última \emph{ratio},
quando ao caso se verifique realmente que o interesse público deva se
sobrepor aos outros preceitos constitucionais, massificaram,
banalizando, o exercício da ponderação dos princípios, com o intuito
único de produzir provas em processo penal contra cidadãos que há muito
deixaram de ser detentores de direitos.

E não é só! Por ser banal, a produção de provas mediante a mitigação de
direitos constitucionais passou a ser utilizada pelas autoridades
investigativas como forma de pressão midiática em desfavor do cidadão
investigado.

Se o interesse coletivo é fundamento para afastar garantias individuais
com vistas à produção de provas no processo penal, também deve-se
observar que a finalidade desta coleta de provas à revelia de direitos
fundamentais é bastante restrita: instruir a investigação ou o processo
penal no bojo do qual foi autorizada. Tanto que, repita-se, o inciso XII
é claro em determinar a finalidade específica da mitigação de direitos
fundamentais mediante autorização judicial ``\emph{para fins de
investigação criminal ou instrução processual penal}''.

A autorização para produzir provas mediante a quebra de garantias
constitucionais, portanto, não inclui a divulgação livre ou utilização
destas provas para outras finalidades, mas apenas para instruir a
investigação ou o processo penal.

Não há dúvida que também são direitos e garantias fundamentais amparados
na Constituição Federal: a livre manifestação do pensamento e das
comunicações (liberdade de imprensa), bem como a garantia de publicidade
dos atos processuais.

Todavia, com o vertiginoso aumento de investigações que contam com o
deferimento de medidas cautelares assecuratórias e de produção de prova,
uma nova questão se coloca no âmbito da ponderação destes princípios
constitucionais (liberdade de imprensa e publicidade dos atos
processuais x direitos fundamentais individuais).

Havendo nos autos de investigação provas produzidas mediante a
flexibilização de direitos fundamentais do investigado, não se tem
dúvidas de que, uma vez concluídas as diligências, o sigilo para a
partes deve ser levantado, franqueando-se à defesa acesso a todo o seu
conteúdo. Todavia, poderia este mesmo processo estar livremente à
disposição do público e, em especial, poderiam as autoridades
responsáveis por conduzir a investigação prestar declarações,
entrevistas e, de qualquer forma, expor livremente estas provas
encartadas aos autos, como se tem feito reiteradamente nos dias atuais?

Por ora, não há dúvidas de que os atos processuais, em geral, são
públicos, em especial no interesse do acusado, conforme anteriormente
esposado. Todavia, as provas cautelarmente produzidas diferem
infinitamente das provas angariadas através de investigação comum,
justamente por terem sido autorizadas judicialmente, na proteção do
interesse público e mediante o afastamento de direitos fundamentais dos
investigados (ponderação de princípios constitucionais).

A própria Constituição oferece um norte para o tema, quando define, no
art. 5º, inciso LX, que a lei só poderá restringir a publicidade dos
atos processuais quando a defesa da intimidade ou o interesse social o
exigirem.

Não se pode olvidar que a preservação do \emph{interesse social} também
invoca a ideia de \emph{justiça social}, e o direito coletivo de acesso
à justiça e, por conseguinte, a julgamentos justos. A sociedade, como um
todo, também precisa confiar no judiciário e na justiça como instrumento
democrático e republicano, consubstanciado no acesso a um julgamento
justo, mediante a obediência às regras processuais penais (ótica formal)
mas também o mais protegido possível de influências externas (ótica
material).

Um dos grandes desafios atualmente colocados ao Judiciário é justamente
o esforço de não ser permeabilizado pela influência de agentes externos,
notadamente a mídia, através da chamada publicidade opressiva. A
doutrina atual tem se debruçado sobre esta potencial influência da mídia
e dos vazamentos na imprensa de provas inseridas nos processos no
resultado final do julgamento.

Neste contexto, a norma Constitucional também apresenta um norte: em
contraponto ao interesse social, também tutela o \emph{interesse
individual}, através do limite à publicidade dos atos processuais com
vistas a salvaguardar a intimidade do investigado, como direito
fundamental que é.

Indo ainda mais além, o Pacto de São José da Costa Rica, do qual o
Brasil é signatário, no art. 8º, n. 5, prevê a possibilidade de
restringir a publicidade dos atos no processo penal para preservar os
interesses da justiça.

Apresenta-se, portanto, o desafio de ponderar entre o direito
fundamental do investigado à intimidade e vida privada (já mitigado para
a produção da prova cautelar) e a publicidade dos atos processuais junto
com a liberdade de imprensa constitucionalmente garantida.

Quando se chega a uma decisão judicial que autoriza a produção da prova
cautelar de interceptação telefônica e telemática, evidentemente foram
mitigados direitos fundamentais na consideração do interesse público na
investigação, mediante uma ponderação legitima e também prevista na CF.

Todavia, ao se admitir eventual plublicização da prova amealhada,
disponibilizando-a aos órgãos de imprensa, nova ponderação de princípios
é realizada, desta feita em flagrante colisão com a finalidade estrita
estabelecida na primeira ponderação, qual seja, instruir a investigação.

Ora, se foi necessário (I) obter autorização judicial para a coleta das
provas, por força de determinação constitucional; (II) se tais elementos
ainda deverão ser apresentados submetidos ao contraditório ao longo da
instrução processual e (III) ainda serão submetidos também ao crivo de
legalidade \emph{a posteriori}, tendo em vista a possibilidade concreta
de ter havido procedimentos de mácula à legalidade daquelas provas, não
há absolutamente nenhuma razão para se admitir que, de forma antecipada
e sem qualquer critério, as autoridades investigativas estejam
autorizadas a veicular na imprensa seu conteúdo (!).

No ponto, cumpre mencionar o paradigmático caso relacionado ao então
Governador do Rio de Janeiro Anthony Garotinho. O STF confirmou o
posicionamento do Tribunal de Justiça do Rio Grande do Sul no sentido de
proibir a veiculação pela imprensa do conteúdo de interceptações
telefônicas reconhecidamente ilegais, aduzindo não haver qualquer
inconstitucionalidade na decisão do Tribunal \emph{a quo} que proibiu
esta divulgação pela mídia (STF: Medida Cautelar na petição nº 2702-7).

À época, a imprensa argumentou que a ``população teria o direito de
informação'' acerca do tema, todavia, o STF, cumprindo devidamente seu
papel de instância contramajoritária de proteção aos direitos
fundamentais, manteve firme a decisão e não cedeu à pressão pública.

Por outro lado, observam-se as nefastas consequências do paradigmático
caso do vazamento ilegal de trechos de interceptação telefônica entre os
ex-presidentes Lula e Dilma. Observe-se: a prova foi vazada quando ainda
detinha presunção de licitude, todavia, foi declarada ilícita \emph{a
posteriori} pelo STF, mas era impossível restaurar o \emph{status quo}
anterior ao vazamento, qual seja, de preservação da privacidade dos
interlocutores.

Realizando uma análise conjunta dos dois casos, se o STF já consignou a
impossibilidade de se divulgar uma prova ilícita, não se pode admitir a
divulgação de uma prova (mesmo que ainda detenha presunção de licitude)
antes de manifestação judicial definitiva sobre seu conteúdo, após o
devido contraditório e ampla defesa no processo, bem como sem controle
posterior de legalidade.

Afinal, a análise da ilegalidade de uma prova cautelar não se limita a
avaliar os fundamentos da decisão que a decretou, mas também envolve
avaliar, dentre vários aspectos, (I) a forma de execução/coleta da prova
em seus aspectos formais e materiais e (II) a preservação da cadeia de
custódia desta prova nos autos. Desta forma, a prova pode ser
considerada nula \emph{a posteriori} mesmo que se reconheça a legalidade
da decisão autorizativa.

Além disso, não se olvide que vem se tornando praxe no país não apenas a
veiculação das provas produzidas mediante flexibilização de direitos
fundamentais, mas também a concessão de entrevistas coletivas pelas
autoridades investigativas, sempre acompanhadas de avaliação
hermenêutica daquelas provas.

Em pronunciamento acerca dos habituais vazamentos de informação à
imprensa ocorridos ao longo da Operação Lava Jato, em violação ao dever
funcional de sigilo (tipo penal do art. 325 do CP), Gilmar Mendes
pontuou que ``\emph{\textbf{\emph{Investigações devem ter por objetivo
produzir provas, não entreter a opinião pública ou demonstrar
autoridade}}}''.

A veiculação de provas produzidas em violação aos direitos fundamentais
do acusado representa grave risco ao princípio constitucional da
paridade de armas no processo penal, pois a tese veiculada na imprensa
cinge-se às conclusões das autoridades investigativas e acusatórias, sem
direito ao devido contraditório; ao direito a um julgamento justo, tendo
em vista que a pressão midiática e a opinião pública representam fatores
de pressão contra o judiciário.

O sigilo se faz necessário por si mesmo, em face da realização de uma
\emph{segunda} ponderação de princípios (publicização da prova x
privacidade, intimidade e direito ao julgamento justo), sem prejuízo da
ponderação anterior, realizada quando da própria autorização da prova
(interesse público x privacidade e intimidade). Desta forma, estar-se-ia
evitando eventual publicidade opressiva em detrimento do acusado, bem
como a interferência de fatores externos no julgamento.

Observa-se claramente, por tudo que já foi exposto, que a visão
utilitarista na produção probatória nos processos criminais
contemporâneos vem sendo continuadamente renovada. No mais das vezes, as
novidades legislativas são limítrofes ao desrespeito e ao sacrifício de
preceitos estabelecidos com a Constituição de 1988.

A Lei n.12.850/13 trouxe nova normatização às organizações criminosas,
dispondo, além de sua própria definição, métodos de investigação
criminal, meios de obtenção de prova e procedimento criminal. Criou,
assim, um novo crime, e, em relação a este, dispôs sobre variada ordem
de coisas. Entre elas, sobre o assunto que tomou de assalto as
discussões do país: a noção de colaboração premiada.

Observe-se assim, que dentre os meios de obtenção de prova, a lei das
organizações criminosas reiterou a permissão da utilização da
colaboração premiada, matéria que, é verdade, já prevista em legislações
anteriores, embora tratassem elas, do instituto apenas no aspecto
material, fornecendo benefícios variados a quem, de qualquer forma,
tivesse colaborado de maneira efetiva e voluntária com a investigação.

Assim, acurou o legislador na lei n. 12.850/13 em detalhar a forma da
colaboração, traçando o procedimento para se alcançar o benefício.
Inovou, ainda, quanto aos benefícios a serem ofertados ao colaborador,
mencionando, portanto, modalidades de não denúncia, e outras, como o
surgimento da possibilidade de se obter a substituição da pena privativa
de liberdade pela pena restritiva de direitos, benefício que não estava
disciplinado nas construções normativas anteriores, que apenas admitiam
a possibilidade da redução da pena e do perdão judicial. Criou,
textualmente, a oportunidade do Ministério Público deixar de oferecer
denúncia quando preenchidos alguns requisitos atinentes à condição
pessoal do colaborador. Sobre tais favores legais, estabeleceu-se certa
falácia, a qual merece nova leitura.

Não há dúvida, que a utilização da colaboração premiada pode gerar um
benefício ao acusado, mesmo que interfira apenas e diretamente na
redução das consequências negativas do crime. Independente da
classificação que se dê à norma que instituiu a colaboração premiada na
Lei n. 12.850/13 -- norma de natureza processual, penal ou processual
com conteúdo penal -, é certo que, por trazer benefício de ordem penal,
a ela deve ser aplicado os atributos da retroatividade. No entanto,
existem certos limites.

As normas atinentes à colaboração premiada, malgrado tenham um conteúdo
essencialmente processual, também dizem respeito a medidas
despenalizadoras, atingindo frontalmente o poder punitivo, o que lhe
encaminha ao tratamento empregado nos casos de leis penais benéficas,
impondo a sua adequação aos casos em apuração, mesmo que regidos no
tempo por legislação anterior.

O grave problema na utilização da colaboração premiada prevista na Lei
n. 12.850/13 em relação a fatos pretéritos, não se finca na
possibilidade de sua retroatividade. Esse, fato indiscutível.

A colaboração instituída pela Lei, malgrado possa trazer inúmeros
resultados positivos, principalmente naqueles casos excepcionais que
resultam da gravidade dos crimes e da complexidade das investigações,
deve estar subordinada à reserva da lei, curvando-se, indeclinavelmente,
ao princípio da legalidade, principalmente porque só a lei pode prever
um crime e sua pena. Se a colaboração atenua ou exclui penas, já
previamente definidas em lei, deve seguir o estrito caminho ditado pela
vontade do legislador, especificamente no que dispõe a Lei 12.850/13.

O importante a se destacar é que o recriado instituto de colaboração
premiada, na forma apresentada, claramente está voltado a disciplinar um
dos meios de prova para a apuração do crime de \emph{organização
criminosa} e infrações penais correlatas. Ou seja, o legislador teve
como intenção redefinir o procedimento e os benefícios da colaboração
visando, exclusivamente, à apuração do crime de \emph{organização
criminosa}, definido e, portanto, tipificado, naquela mesmo diploma
legal.

O problema é que a aplicação desmedida do instituto, com a vulgarização
e o descontrole nos procedimentos adotados pelo Ministério Público e
autoridades policiais põem em risco o princípio da legalidade instituído
no nosso sistema constitucional. É inadmissível que o Ministério
Público, e até mesmo o Poder Judiciário, possam se arvorar na condição
de legislador para aplicar o instituto da colaboração, além dos limites
estatuídos pela Lei 12.850/13.

Não há de se trilhar maiores caminhos para se obter, na própria lei, a
intenção do legislador. O art. 4º ao estabelecer as condições para se
alcançar os benefícios previstos no \emph{caput,} condicionou tal
\emph{benesse} à obtenção de diversos resultados, \emph{in} verbis:

\emph{I - a identificação dos demais coautores e partícipes \textbf{da
organização criminosa} e das infrações penais por eles praticadas};

\emph{II - a revelação da estrutura hierárquica e da divisão de tarefas
\textbf{da} \textbf{organização criminosa};}

\emph{III - a prevenção de infrações penais decorrentes das atividades
\textbf{da organização criminosa};}

\emph{IV - a recuperação total ou parcial do produto ou do proveito das
infrações penais praticadas \textbf{pela organização criminosa};}

E mais: quando o legislador no §4º, trata acerca dos casos em que o
Ministério Público pode deixar de oferecer denúncia, estabelece no
inciso I que o colaborador poderá receber o benefício desde que não seja
o líder \emph{da organização criminosa}.

Não resta dúvida, por conseguinte, que o benefício na forma como foi
criado o instituto só pode ser aplicado aos casos em que o agente tenha
promovido, constituído, financiado ou integrado, pessoalmente ou por
interposta pessoa, \emph{organização criminosa}.

J. J. Gomes Canotilho e Nuno Brandão, em artigo publicado recentemente
na Revista de Legislação e de Jurisprudência, intitulado ``Colaboração
premiada e auxílio judiciário em matéria penal: a ordem pública como
obstáculo à cooperação com a operação Lava Jato'', são incisivos ao
afirmar que a colaboração premiada da lei 12.850/13 tem o seu cerne na
figura da organização criminosa. Há de se ver:

\emph{``Considerando esses fundamentos político-criminais e atento o
dever estatal de estrita observância do princípio da proporcionalidade
em sentido amplo, temos para nós que o regime legal da Lei n. 12.850 tem
um âmbito normativo bem delimitado, circunscrevendo-se ao delito de
organização criminosa e aos crimes a ele ligados, isto é, aos crimes da
organização (as infrações penais correlatas a que se refere o art. 1º da
Lei n. 12.850/13).}

\emph{Crimes externos à organização criminosa caem fora da alçada da Lei
n. 12.850/13 e não podem de objecto de perseguição criminal com recurso
aos meios de obtenção de prova nela consagrados e definidos,
designadamente, à colaboração premiada. Pois não foi para esses
fenômenos criminais que tais meios foram especificadamente pensados e
postos à disposição da investigação criminal pelo legislador federal. A
não ser assim, ficaria aberto caminho para que meios de investigação
excepcionais pudessem banalizar-se e ser usados para a repressão de
crimes ou contextos criminais cuja gravidade de modo algum justificaria
intromissões tão severas na esfera dos direitos de liberdade dos
cidadãos como as que são inerentes aos meios de obtenção de prova
enunciados no art. 3º da Lei n. 12.850/13.''}

Imaginando-se que a norma dispõe de benefícios, mas também de situação
mais gravosa, vale dizer, da previsão de novo crime de organização
criminosa, parece inconteste afirmar-se sua irretroatividade.

Os atributos da retroatividade no caso concreto, inerentes às normas
penais benéficas, só poderiam ser aplicáveis aos fatos em apuração,
consumados antes da vigência da lei, acaso já houvesse, no tempo da
conduta praticada, a definição do crime de organização criminosa, já que
o benefício só se aplica como meio de obtenção de prova com finalidade
exclusiva: facilitar a apuração desse crime específico e as suas
infrações correlatas.

Como de fato não havia definição anterior, e a própria Lei 12.850/13 foi
a responsável por normatizar as condutas típicas previstas como crime de
organização criminosa, não há que se falar na aplicação do instituto, de
suas regras e procedimentos, benefícios e medidas despenalizadoras em
favor de investigados, quando incidirem na prática de outras condutas
criminosas, previstas em outras legislações ou no Código Penal, como de
fato vem acontecendo.

O argumento irretorquível, não impossibilita, genericamente, a aplicação
do instituto da colaboração premiada nesses outros casos, desde que o
agente tenha praticado outros delitos, distintos ou que não sejam
correlatos à organização criminosa, antes da vigência da Lei n.
12.850/13. O certo é que os benefícios e a forma de sua aquisição devem
se submeter a outras legislações (anteriores à Lei n. 12.850/13),
vigentes à época dos fatos criminosos praticados.

Não se deve perder de vista que as provas obtidas mediante colaboração
premiada serão utilizadas contra terceiros delatados, possuidores de
garantias fundamentais, contra os quais se exige a existência de um
processo válido, no qual só se admite como prova lícita aquela que
advenha da adequação lei-processo. E estes terceiros poderão ser partes
legítimas para pleitear \emph{a posteriori} a nulidade da delação e dos
respectivos benefícios concedidos ao delator.

Não se olvida que, para além da Lei n. 12.850/13, no nosso ordenamento,
o instituto da delação já possuía previsão na Lei n. 8.072/90 (crimes
hediondos); Lei n. 7.492/86 (crime contra o sistema financeiro
nacional); Lei n. 8.137/90 (crimes contra a ordem tributária e relações
de consumo); Lei n. 9.613/98 (lavagem de capitais); Lei n. 9.807/99 (lei
de proteção a vítimas e testemunhas); Lei n. 11.343/2006 (lei
antitóxicos) e no próprio Código Penal. Contudo, os regimes não eram
equivalentes e só podem ser aplicados em cada âmbito material típico.
Defeso, pois, a cumulação destes, salvo quando se trata de acusações de
delitos coincidentes.

Destarte, nas condutas criminosas não cingidas à Lei n. 12.850/13, os
investigadores só poderiam se utilizar da colaboração premiada, como
obtenção de meio de prova, em conformidade com a lei vigente à época dos
fatos, no limite do regime aplicável a cada tipo, respeitando-se o
procedimento válido para sua utilização.

É certo que tanto o Código Penal quanto as leis anteriores à Lei n.
12.850/13, por tratarem o instituto exclusivamente no aspecto material,
descuraram de seu procedimento, impondo que os benefícios auferidos com
a colaboração só e, unicamente, sejam analisados pelo julgador quando da
sentença condenatória. Ou seja, independente do acordo firmado com os
investigadores, o delator só poderá se beneficiar com a redução da pena
ou o perdão judicial, após o transcurso normal do processo, no momento
da sentença condenatória, ao critério e nuto do julgador, acaso
preenchidos os requisitos previstos em lei.

Como conclusão lógica, em termos contemporâneos, não cabe aos crimes
distintos da organização criminosa, quando da utilização da colaboração
premiada como meio de obtenção de prova, a aplicação do §2º (os
investigadores a qualquer tempo, poderão requerer ao juiz pela concessão
do perdão judicial ao colaborador), §3º (o prazo para o oferecimento da
denúncia poderá ser suspenso por até 6 meses), §4º (o Ministério Público
poderá deixar de oferecer denúncia se o colaborador: I e II), §5º (se a
colaboração for posterior a sentença, a pena poderá ...), §7º (realizado
o acordo, a colaboração será remetido ao Juiz para homologação...) e §10
(as partes podem retratar-se da proposta ...), todos procedimentos
atinentes à apuração e obtenção de meio de prova nos crimes de
organização criminosa, posto que trazidos exclusivamente no bojo desse
diploma legal.

Portanto, a utilização desses benefícios em procedimentos para apuração
de crimes praticados antes da vigência da Lei n. 12.850/13, ou quando
não se tenha a presença de uma organização criminosa, invalida o acordo
de colaboração firmado tornando a prova ilícita, por ferir o princípio
da reserva legal, haja vista a impossibilidade de o poder judiciário
assumir a função do legislador para alargar e estender a períodos
pretéritos a aplicação de regras e procedimentos que foram disciplinados
apenas para apuração do delito de organização criminosa.

Mas esses exageros pontuais, não esgotam o elenco das irregularidades
trazidas com a delação premiada. Na verdade, tudo faz parte de um
sistema adredemente montado -- prisão provisória sem fundamentação,
coação, delação e condenação - com o intuito de se produzir prova em
desfavor de acusados num flagrante desrespeito aos preceitos
constitucionais. Retomaram-se, nesta quadra histórica da nossa
democracia consolidada, métodos utilizados pelo autoritarismo dos anos
70 contra os que eram reputados inimigos do regime militar. Os métodos
de tortura antes praticados, agora recebem contornos de legalidade e
oficialidade. Se prende arbitrariamente num processo nítido de
desmoralização, afastam o preso processual para outra unidade da
Federação (normalmente num flagrante descompasso com o princípio do Juiz
Natural) com o propósito de lhe ``quebrar o moral''. Pelo isolamento
absoluto em local desconhecido, leva-se o recluso à fragilidade
psíquica, logrando-se fazê-lo colaborar acerca do que talvez não
colaborasse em condições de plena integridade psicológica.

Os conteúdos obtidos com as delações premiadas são acrescidos de acordo
com a vontade dos gestores envolvidos. Agora, se voltam os delatores,
manipulados ou não, contra seus próprios defensores, numa tentativa de
se criminalizar a própria advocacia criminal. Nunca se viu tantos
advogados citados em investigações como cúmplices ou réus.

No desenvolvimento de atividades investigatórias, os advogados
militantes passaram a ser vistos como obstáculo à produção utilitarista
da prova, e alguns setores das referidas instituições -- Polícia, MP -,
munidos de autorizações judiciais, concedidas sem maior critério,
cuidado e prudência, têm invadido escritórios de advogados,
violando-lhes os arquivos e o sigilo profissional, com a utilização de
interceptações epistolares, telefônicas, de dados e telemáticas, na
busca de possíveis indícios ou provas de atos de terceiros,
transformando o exercício da defesa técnica da liberdade humana em
atividade de alto risco. Desnecessário pontuar que tais ações, anômalas,
sempre cercadas de grande estrépito junto à opinião pública, levam ao
desmerecimento os profissionais visados, aviltando-os perante a
comunidade profissional e o meio social.

A comunicação reservada do defensor com o cliente é burlada por escutas,
oficiais e clandestinas, até nos parlatórios das casas de custódia, onde
ela tem lugar em gaiolas envidraçadas equipadas com interfones,
``grampeados''...

Mesmo com todos esses absurdos, a reação de juristas e de diminuta
parcela da população não consegue reverberar essas manifestações em prol
de uma declaração de vigência da Constituição, malferida pela
recalcitrância do arbítrio cometido contra cidadãos em geral.

Chegou a hora e o momento de apontarmos nos 30 anos da Constituinte, que
a nossa Constituição Federal foi vilipendiada, que essa agressão precisa
ser remediada não só para garantir os direitos de um ou de outro
acusado, mas, sobremaneira, para se garantir a normalidade do próprio
Estado Democrático de Direito, impedindo que novas ofensas sejam
cometidas.

A nossa Constituição não pode mais ser interpretada ao critério do
leitor, na distorção de seus próprios interesses. As cláusulas do devido
processo penal, a ampla defesa, o contraditório, a inadmissibilidade das
provas ilícitas, a necessidade de fundamentação concreta de todas as
decisões judiciais, valores constitucionais incidentes no processo
penal, não carecem de interpretação, são princípios que se impõem à
conclusão de que nessa área os fins não justificam os meios.

Em síntese, o texto apresentado é um testemunho de fé, no qual a palavra
chave continua a ser cidadania. É a demonstração do bom combate em
defesa da Constituição.

Nós advogados, militantes na área do direito criminal, protagonistas
fundamentais da cena judiciária, temos não apenas o direito, mas o dever
de sustentar essas bandeiras, ainda que ao custo eventual de
incompreensões e retaliações por parte dos diversos poderes, da opinião
pública incauta ou defensores do punitivismo estatal que teimam em não
nos compreender. Nosso compromisso não é com eles, mas a com a sociedade
-- e a história.
