\chapter*{A Constituinte e o capitalismo brasileiro}

\addcontentsline{toc}{chapter}{A Constituinte e o capitalismo brasileiro,\\
\scriptsize{por Luiz Gonzaga Beluzzo}}

\begin{flushright}
\emph{Luiz Gonzaga Belluzzo}\footnote{Economista, professor do Instituto de Economia da Universidade de Estadual Campinas}
\end{flushright}

Às vésperas do 30° aniversário da Constituição"-Cidadã, não posso negar
ao improvável leitor as palavras de Ulysses Guimarães na sessão de
promulgação da Carta Magna:

``A sociedade foi Rubens Paiva e não os facínoras que o mataram. Foi a
sociedade, mobilizada nos colossais comícios das Diretas"-Já, que, pela
transição e pela mudança, derrotou o Estado usurpador. Termino com as
palavras com que comecei esta fala: a Nação quer mudar. A Nação deve
mudar. A Nação vai mudar. A Constituição pretende ser a voz, a letra, a
vontade política da sociedade rumo à mudança. Que a promulgação seja
nosso grito: --- Mudar para vencer! Muda Brasil''

\section{Esperança e Mudança}

Há quem diga que o Brasil, ao promulgar a Constituição de 1988, entrou
tardia e timidamente no clube dos países que apostaram na ampliação dos
direitos e deveres da cidadania moderna. Submetidos ao longo de mais de
quatro séculos à dialética do obscurecimento que regia as relações de
poder numa sociedade marcada pelo vezo colonial"-escravocrata e, depois
da Independência, pelo coronelato primário"-exportador, os brasileiros
subalternos deram na Constituinte passos importantes para alcançar os
direitos do indivíduo moderno.

Em sua edição de outubro/novembro de 1982, a Revista do \versal{PMDB} publicou o
documento \emph{Esperança e Mudança}. Comandado por Ulysses Guimarães, o
programa foi elaborado com a contribuição de muitos e bons brasileiros.
Esperança e Mudança é um prólogo à Constituinte de 1988.

Por isso, vou aqui reproduzir trechos da Introdução que exprimem o
espírito que guiou a batalha de Ulysses.

A Introdução proclama: 

\begin{quote}
O \versal{PMDB} sabe que a crise nacional não encontrará solução sem
mudanças profundas. Mudanças que só poderão ter início com o fim do
arbítrio e da exceção. Mudanças que haverão de nascer do reencontro do
povo com o poder político. A sociedade brasileira anseia pela Democracia, luta por ela, sonha com
ela. A sociedade repele o arbítrio através de todas as suas formas de
representação de interesses e de organização social: partidos políticos,
movimentos sociais, organizações comunitárias, igrejas, sindicatos,
organizações patronais, profissionais, movimentos setoriais e culturais.

Democracia é Estado de Direito, é liberdade de pensamento e de
organização popular, é respeito à autonomia dos movimentos sociais e
repousa na existência de partidos políticos sólidos. Democracia
significa voto direto e livre, significa restauração da dignidade e das
prerrogativas do Congresso e do Poder Judiciário, significa liberdade e
autonomia sindical, significa liberdade de informação e acesso
democrático aos meios de comunicação de massa. Democracia implica em
democratização das estruturas do Estado, implica em resgatar a soberania
nacional, implica em redistribuição da renda, criação de empregos e em
bem"-estar social crescente. A \emph{Assembleia Nacional
Constituinte} haverá de ser o berço de tudo
isso --- o berço da Democracia --- o berço pacífico e representativo dos
anseios do povo.

Democracia é ruptura com o longo passado autoritário e
elitista, é participação autônoma dos movimentos sociais nas decisões
nacionais através da representação legítima, de meios modernos de
consulta e informação e, da definição dos rumos de nosso desenvolvimento
através do planejamento democrático. As maiorias oprimidas da população --- as mulheres, os jovens, os negros
--- as minorias discriminadas --- os índios, grupos étnico"-culturais
--- não podem continuar sendo tuteladas.
Tampouco podem permanecer os Sindicatos sob o tacão retrógrado do
corporativismo. Numa sociedade com uma estrutura social complexa,
heterogênea, regionalmente diferenciada, o \versal{PMDB} alinha"-se como um
partido amplo --- centrado nos interesses do conjunto dos trabalhadores,
da cidade e do campo, de todos os setores da produção, dos serviços e do
setor público. Um partido que almeja soldar os interesses desse conjunto com os de
outros segmentos sociais --- as classes médias, os autônomos, o
empresariado nacional. O \versal{PMDB} respeita a autonomia da sociedade civil e
reconhece a sua complexidade. O \versal{PMDB} é, e deseja ser, cada vez mais, um
canal de condensação de interesses sociais e, para isso, oferece à
sociedade um projeto global coerente. Um projeto que almeja a
transformação democrática da vida nacional.

O \versal{PMDB} propõe o planejamento democrático como forma de garantir que o
conjunto de políticas públicas obedeça a prioridades fixadas
democraticamente --- prioridades que busquem um novo estilo de
desenvolvimento social, cuja diretriz maior deve ser a redistribuição da
renda e da riqueza social. O Planejamento democrático implica na
elaboração de um Plano, sob controle e sob a influência das
instituições democráticas. Plano fixado através de lei, supervisionado eficazmente
pelo Congresso com a interação e auxilio das organizações populares.

O Brasil é um pais rico --- com povo pobre! É a sétima economia
industrial do bloco das economias de mercado, entretanto é, também, um
dos campeões mundiais de concentração da renda e da riqueza. Persistem
as desigualdades sociais e regionais, persistem os enormes bolsões de
pobreza absoluta. O \versal{PMDB} considera que este estado de coisas e uma
vergonha nacional. Compromisso fundamental do \versal{PMDB} é a extinção do
analfabetismo, é o fim da desnutrição e da mortalidade infantil, é a
erradicação das endemias, é o fim da promiscuidade habitacional, da
insegurança, da falta de transportes. O \versal{PMDB} quer acabar com o estado de
indigência forcada em que vivem pelo menos 25 milhões de brasileiros.
Quer e sabe como fazê"-lo.

O \versal{PMDB} tem planos e propostas sérias,
possíveis, viáveis. Propostas que certamente exigem determinação,
imaginação, competência. O \versal{PMDB} as tem! Propostas em aberto que são
oferecidas ao crivo do debate democrático nacional para o seu contínuo
aperfeiçoamento.

Redistribuição da renda e criação de empregos não
constituem metas simplistas. São processos complexos que requerem um
amplo conjunto de reformas sociais e de políticas públicas
compativelmente articuladas.
\end{quote}

\section{As Eleições Diretas e a Constituinte}

Ocorreu"-me relembrar que a vitória na Constituinte não conseguiu
eliminar as consequências da derrota na campanha pelas diretas. A busca
açodada pelo voto indireto no Colégio Eleitoral não prescindiu da
cumplicidade de muitos que estavam na oposição, mas temiam a
``radicalidade'' de um governo eleito pelo povo. Constrangidos a
participar dos comícios, tais ``oposicionistas'' acenavam com a mão
esquerda para os cidadãos aglomerados nas praças, mas cuidavam de livrar
a direita para montar os arranjos da eleição indireta. Por isso, os
náufragos do regime militar conseguiram chegar à praia, acolhidos pelo
bote salva"-vidas capitaneado pela turma do deixa"-disso.

A campanha pelas diretas promoveu uma forte mobilização popular, mas não
teve forças para derrubar as casamatas do poder real que, desde sempre,
comandam nos bastidores a política brasileira. Essa turma não tem o
hábito de dar refresco ao inimigo. Em suas fileiras abrigam"-se os
liberais que apoiam golpes de Estado, as camadas endinheiradas e
remediadas que mal toleram a soberania popular e as gentes midiáticas
que abominam a opinião divergente. O Senhor"-Diretas superou na
Constituinte as amarguras que compartilhou com os amigos reunidos em sua
casa na posteridade da derrota das Diretas"-Já. No emocionante discurso
que proferiu, Ulysses fustigou a já mencionada cumplicidade de muitos
que estavam na oposição.

A democracia dos modernos, seus direitos e contradições, são conquistas
muito recentes. Digo contradições porque o sufrágio universal foi
conseguido com sacrifício entre final do século \versal{XIX} e o começo do século
\versal{XX}. Mas, já em 1910, Robert Michels cuidava de denunciar a deformação da
representação popular promovida pelo surgimento de oligarquias
partidárias, fenômeno que nasce e se desenvolve no ``interior'' dos
sistemas democráticos.

Os direitos econômicos e sociais nasceram da luta política das classes
subalternas. Entre o final dos anos 30 do século passado e o desfecho da
Segunda Guerra Mundial a presença das massas assalariadas e
urbanas no cenário político impôs importantes transformações no papel do
Estado. A função de garantir o cumprimento dos contratos, de assegurar
as liberdades civis e os direitos políticos, apanágio do Estado Liberal,
é enriquecida pelo surgimento de novos encargos e obrigações: tratava"-se
de proteger o cidadão não"-proprietário dos mecanismos cegos do livre"-mercado,
sobretudo dos azares do ciclo econômico.

Em 1992 os caras"-pintadas acorreram às ruas para pedir o impeachment do
então presidente Fernando Collor de Mello. Antes de morrer, Ulysses
compreendeu que a campanha popular pelas eleições diretas e a
Constituição ainda sofriam o assédio insidioso, persistente e renovado
do velho arranjo oligárquico que pretende controlar a vida dos
brasileiros.

Em longa conversa em meu gabinete, na presença do jornalista Roberto
Muller Filho, Ulysses Guimarães desfiou temores e preocupações diante do
iminente impeachment do presidente eleito pelo voto popular.

Os receios do Senhor"-Diretas concentravam"-se no ``vício
antidemocrático'' dos donos do poder, habituados a manejar os cordéis do
arbítrio a seu talante e ao sabor de seus interesses. A cavalgada do
mandonismo pode ocorrer no lombo dos fardados ou nos ombros dos
bacharéis habilitados a chicanas e firulas de variado sabor doutrinário.

Reafirmo, em seguida, o que disse em colunas anteriores: nas almas dos
\emph{impichadores} brasileiros de hoje estão entrelaçadas as
brutalidades do atraso oligárquico e a hipermodernidade da barbárie
``internética'' que intoxica o ambiente social com sua nuvem de
ignorâncias.

As baixarias revelam, sobretudo, indigência cultural e desprezo absoluto
pelos valores do liberalismo político, o que nos coloca na rabeira do
processo civilizador, ou se quiseram, na vanguarda do movimento de
retorno à idade da pedra lascada. O Estado Democrático de Direito não
``pegou'' na terra de Santa Cruz. Seus princípios jazem inertes nos
compêndios. As garantias individuais ainda não saíram dos códigos para
ganhar vida nos ambientes sociais frequentados pelos abusos dos
senhoritos da ``ordem'' e seus sequazes. O Datafolha informa que 76\%
dos que exibiam sua ignorância nas manifestações pró"-impeachment de
Dilma têm nível superior. A cifra é uma delação não premiada, com o
indicador apontado para a impotência da educação em conter a degradação
dos indivíduos na sociedade capitalista de massas.

Os brasileiros --- alguns hoje se manifestam nas ruas --- foram submetidos
a um processo de ``esquecimento coletivo'' promovido \emph{cum ira et sine
studio} por uma conspiração de silêncio. A conspirata envolve não só os
conhecidos esbirros do conservadorismo, os senhores da mídia e seus
lacaios nas redações, mas também o sistema educacional --- do ensino
básico ao superior --- empenhado em formar analfabetos funcionais ou, na
melhor das hipóteses, ``especialistas'' incapazes de compreender o mundo
em que vivem. A turma do andar de cima exalta as virtudes da educação,
mas promove com esmero e persistência as crueldades da \emph{Pátria
Deseducadora.}

A estrutura de classes no Brasil é muito original: na cúspide, os
predadores que disputam os despojos da riqueza velha; no meio, os
trouxas e os espertalhões ideológicos das camadas falantes
semi"-ilustradas; lá embaixo, os ``ferrados'' que tentam desesperadamente
emergir da miséria.

Se não restringisse suas fontes aos idiotas funcionais do cosmopolitismo
caboclo, os editores da revista \emph{The Economist} nas duas matérias de
capa que trataram do Brasil teriam a oportunidade de escapar dos
extremos ridículos: na primeira capa, a exaltação precipitada; na
segunda o besteirol fecundado nas ideologias que levaram a economia
mundial ao desastre financeiro. Perceberiam que as lideranças das
classes dominantes brasileiras e seus porta"-vozes na mídia estão sempre
alinhados com o que há de mais expressivo no caquético capitalismo
brasileiro.

O arranjo social do atraso preconiza uma sociedade submissa ao rentismo,
refém da estagnação, prisioneira da defesa da riqueza estéril alimentada
pelo fluxos de \emph{hot dollars}. Imobilizados nos pântanos do
parasitismo, os bacanas e sabichões acovardam"-se diante dos azares da
incerteza, avesso aos riscos de construção da nova riqueza. Aí está
desvelado, em sua perversidade essencial, o ``segredo'' das
reivindicações antissociais dos vassalos do enriquecimento sem esforço
cevado por taxas de juros absurdas. Clamam pelo aumento do desemprego.
Este é o alto preço que o presente agrilhoado ao passado cobra do
futuro.
