\chapter*{Discurso do ex"-presidente Luiz Inácio Lula da Silva, em frente ao
Sindicato dos Metalúrgicos do \versal{ABC}, em São Bernardo do Campo, em São
Paulo, 7 de abril de 2018, antes de se entregar para a Polícia Federal\footnote{Frente Brasil Popular}}

\addcontentsline{toc}{chapter}{Luiz Inácio Lula da Silva, 7 de abril de 2018}

Queridas companheiras e queridos companheiros\ldots{}

Eu não sei se esse som aumenta um pouquinho mais, porque isso
facilitaria minha voz já rouca.

Querido companheiro Vagner [Freitas], presidente da \versal{CUT},
querido companheiro Aloízio Mercadante, ex"-senador, ex"-deputado federal,
ex"-ministro da Ciência e Tecnologia, ex"-ministro da Educação,
ex"-ministro da Casa Civil da presidenta Dilma\ldots{} porra, se eu tivesse
tantos títulos assim, eu seria presidente da República.

Companheiro Guilherme Boulos, nosso companheiro que está iniciando
uma jornada sendo candidato a presidente da República pelo \versal{PSOL}, mas é
um companheiro da mais alta qualidade, vocês têm que levar em conta
a seriedade desse menino.

Eu digo `menino' porque ele só tem 35 anos de idade, e, quando eu
fiz a greve de 78, eu tinha 33 anos de idade e consegui, através da
greve, chegar a criar um partido e virar presidente. Você tem futuro,
meu irmão, é só não desistir nunca.

Quero cumprimentar essa garota, essa garota bonita, garota
militante do \versal{PC}do\versal{B}, que também está fazendo a sua primeira experiência
como candidata a presidenta da República pelo \versal{PC}do\versal{B} --- e que eu acho
um motivo de orgulho e uma perspectiva de esperança para esse país ter
gente nova se dispondo a enfrentar a negação da política, assumindo a
política e dizendo: ``nós queremos ser presidente da República para mudar
a história do país''.

Quero agradecer a companheira dessa mulher, possivelmente a mais
injustiçada das mulheres que um dia ousaram fazer política nesse país. A
injustiçada pelo jeito de governar, acusada de não saber conversar,
acusada de não saber fazer política\ldots{} Mas eu quero ser testemunha de
vocês: a Dilma foi a pessoa que me deu a tranquilidade de fazer quase
tudo o que eu consegui fazer na Presidência da República pela confiança,
pela seriedade e pela qualidade e competência técnica.

Eu sou grato, grato de coração, porque não teria sido o que foi se
não fosse a companheira Dilma. Portanto, Dilma, você sabe que
profundamente, para o resto da vida, repartirei o meu sucesso na
Presidência com Vossa Excelência, independentemente do que aconteça
nesse mundo.

Quero cumprimentar o meu querido companheiro Fernando Haddad. Ele
viveu o melhor período de investimento na educação brasileira nesse
país.

Quero cumprimentar o meu companheiro Celso Amorim, o companheiro
que certamente foi mais importante ministro das Relações Exteriores que
esse país já teve, que colocou o Brasil como protagonista mundial
durante todo o nosso governo.

Quero parabenizar o nosso companheiro Ivan Valente, deputado pelo
\versal{PSOL}, que está aqui. Quero cumprimentar o nosso valoroso,
extraordinário João Pedro Stédile, presidente coordenador do Movimento
Sem Terra.

Quero cumprimentar o Juliano [Medeiros], jovem presidente do \versal{PSOL}.

Quero cumprimentar o nosso querido escritor Fernando Morais,
que está escrevendo a biografia do meu governo --- que nunca termina,
porra! Eu estou quase para morrer e ele não termina a minha biografia.

Quero cumprimentar o nosso querido companheiro Paulo Pimenta,
líder do \versal{PT}, o homem que tem o blog dos deputados mais importantes de
Brasília e o cidadão que melhor tem enfrentado o Moro e a operação Lava
Jato naquilo que são os defeitos dela. Parabéns, companheiro Pimenta.

Quero cumprimentar o índio mais esperto do Brasil, o presidente do
Piauí, o governador do Piauí --- o companheiro Wellington [Dias]
está cumprindo o terceiro mandato e, pelo andar das pesquisas, ele está
a caminho do quarto mandato como governador do estado do
Piauí.

Quero aqui cumprimentar o companheiro Emídio [de Souza],
tesoureiro do \versal{PT}, ex"-prefeito de Osasco, que tem trabalhado
incansavelmente pra gente recuperar o papel do \versal{PT} na história deste
país.

Quero cumprimentar o companheiro Orlando Silva, presidente, ou
melhor, deputado do \versal{PC}do\versal{B}.

Quero cumprimentar o nosso companheiro [Edson Carneiro] Índio,
que é da Intersindical --- é um companheiro de muita qualidade.

Quero cumprimentar o presidente da \versal{CTB} [Central de Trabalhadores
e Trabalhadoras do Brasil], que está aqui, o companheiro Adílson
[Araújo], que é um companheiro também muito importante no movimento
sindical.

Quero cumprimentar a nossa companheira Gleisi Hoffmann, a nossa
querida presidenta do nosso partido.

Quero cumprimentar o companheiro Luiz Marinho, presidente do \versal{PT},
ministro do Trabalho, ministro da Previdência. Eu vou contar duas coisas
do Marinho. O Marinho foi catador de algodão, catador de café e catador
de amendoim em Santa Fé. O Marinho foi pintor na Volkswagen. O Marinho
foi presidente deste sindicato, o Marinho foi presidente da \versal{CUT}. O
Marinho foi certamente o mais importante ministro de Trabalho do meu
governo e foi melhor ministro da Previdência, que foi ministro que
acabou com a fila na Previdência. E o Marinho foi o melhor prefeito que
São Bernardo teve. E agora é o nosso presidente estadual.

Quero cumprimentar o nosso senador, nosso querido Lindbergh
[Farias] --- grande Lindbergh, que eu conheci ainda na campanha
para derrubar o Collor. Tentei tirá"-lo do \versal{PC}do\versal{B} para levar para o \versal{PT},
mas a minha relação de amizade com o João Amazonas era tão forte que eu
não tive coragem de conversar com ele.

Quero cumprimentar, aqui --- gente, eu não tenho nome de todo
mundo --- o Wagner [Santana], presidente do
Sindicato dos Metalúrgicos de São Bernardo e o companheiro Moisés
[Selerges].

Ah, é que está ali atrás e eu não estou vendo: o nosso companheiro
senador da República --- não, vereador, mas futuro senador --- Eduardo
Suplicy. Olha eu não posso falar que ele teve uma tontura, porque isso
não é recomendável para quem está sendo o candidato, viu? Eu vou dizer
que você estava ali sentado conversando com eleitores, está bem?

Eu pedi para vir aqui o companheiro de Sergipe, que é o
companheiro vice"-presidente do \versal{PT} que tem a incumbência de coordenar as
caravanas da cidadania por todo o território nacional e vocês têm
acompanhado pela internet, o companheiro Márcio [Macedo].

Eu pedi para vir aqui dois sindicalistas porque eu nasci nesse
sindicato. Quando eu cheguei aqui, esse sindicato era um barraco. Esse
prédio foi construído já na nossa diretoria. Aqui, para vocês saberem,
eu fui diretor de uma escola de madureza que tinha 1.800 alunos. Vocês
pensam que eu sou só torneiro mecânico? Pode dizer: ``diretor de escola
com 1.800 alunos também''.

E a minha relação com esse sindicato\ldots{} aqui está o Paulão, que é
vice"-presidente do sindicato e é presidente da Confederação Nacional dos
Metalúrgicos e é da secretaria do movimento sindical do \versal{PT}. Eu não tenho
nome de nada, estou chutando de improviso o que eu estou vendo.

Mas eu queria aproveitar, Wagner, a tua presença aqui, para que
esse pessoal soubesse que, na minha consciência, parte das conquistas da
democracia brasileira a gente deve a este sindicato dos metalúrgicos a
partir de 1978.

Aqui foi a minha escola, aqui eu aprendi sociologia, aprendi
economia, aprendi física, química e aprendi a fazer muita política
porque, no tempo que eu era presidente deste sindicato, as fábricas
tinham 140 mil professores que me ensinavam como fazer as coisas.

Toda vez que eu tinha dúvida, eu ia na porta da fábrica perguntar
para a peãozada como fazer as coisas nesse país. Na dúvida, não erre. Na
dúvida, pergunte. E se você perguntar, a chance de você acertar é muito
maior.

E o Wagner é o companheiro que está cedendo este prédio pra gente
fazer toda a nossa campanha. E quero agradecer ao Moisés. O Moisés é o
companheiro do Wagner, é o diretor financeiro do sindicato e é um
companheiro que nunca se negou a contribuir com o movimento social, a
contribuir com outras tarefas da democracia --- não para partido
político, mas para o movimento social. Este sindicato nunca negou
absolutamente nada. Então eu quero uma salva de palmas para esses
companheiros que são um sustentáculo da nossa luta.

Este sindicato, diferente de outros sindicatos, tem quase 283
diretores. Para ser diretor deste sindicato, as pessoas têm que ser
eleitas pelo chão da fábrica, pra um comitê. Se não tiver no chão da
fábrica, não é eleito. E depois de eleito no membro do comitê, se
escolhe os que vão ser diretores do sindicato. E tem a diretoria
executiva, mas tem 283 pessoas que são diretores e que são conselheiros.
Se a gente fizesse isso em todo sindicato do Brasil, certamente a gente
teria muito menos pelego no movimento sindical brasileiro.

Eu fiz questão de citar eles porque às vezes o cara compra o
alimento, lava o alimento, cozinha o alimento, leva pra gente comer e a
gente sai sem saber quem nutre o alimento. Então foram esses guerreiros
aqui que deram essa possibilidade extraordinária de a gente estar aqui
fazendo isso.

A segunda coisa é que eu confesso que vivi os meus melhores
momentos políticos nesse sindicato. Eu nunca esqueci a minha matrícula
do sindicato: é 25986, de outubro, de setembro de
1968.

E de lá pra cá, eu mantenho uma relação com este sindicato que, eu
acho, é a relação mais forte, porque qualquer presidente tem aqui ---
Vicentinho já foi presidente, Meneguelli já foi presidente, Guiba já foi
presidente, Zé Nobre já foi presidente, Feijó já foi presidente, quem
mais? O Guiba já falei, agora o Wagnão. E por todos eles eu sou tratado
como se ainda fosse presidente deste sindicato pela relação que nós
ficamos. Mas aqui\ldots{} o Rafael, foi o penúltimo presidente aqui.

Eu queria dizer pra vocês que eu estou contando isso para tentar
chegar ao que eu quero dizer pra vocês. Em 1979, este sindicato fez uma
das greves mais extraordinárias. E nós conseguimos fazer um acordo com a
indústria automobilística que foi talvez o melhor.

E eu tinha uma comissão de fábrica com 300 trabalhadores, o acordo
era bom e eu resolvi levar o acordo para a assembleia. E resolvi pedir
para a comissão de fábrica ir mais cedo para conversar com a peãozada. E
eu fazia assembleia de manhã pra evitar que o pessoal bebesse um
pouquinho à tarde. Porque quando a gente bebe um pouquinho, a gente fica
mais ousado. Mesmo assim, não evitava porque o cara levava litro de
conhaque dentro da mala e eu ainda passava e tomava uma dosezinha pra
garganta ficar melhor, coisa que não aconteceu hoje.

Pois bem, nós começamos a colocar o acordo em votação e 100 mil
pessoas no estádio da Vila Euclides não aceitavam o acordo. Era o melhor
possível: a gente não perdia dia de férias, a gente não perdia 13º
salário e tinha 15\% de aumento, mas a peãozada estava tão radicalizada
que queria 83 ou nada --- e nós conseguimos.

Passamos um ano sendo chamados de pelegos pelos trabalhadores, a
gente, Guilherme, ia na porta de fábrica, a peãozada\ldots{}

Oh, Jorge Viana, está aqui o meu querido senador do Acre, que eu
não vi --- ele é baixinho. Nosso querido companheiro, foi governador,
prefeito, agora é senador do Acre. Obrigada pela presença.

Olha, para falar em nome dos artistas daqui, para citar todos, eu
queria que o nosso Osmar Prado viesse aqui. Ele é o decano.

Olha, tem muita gente aqui. E tem a mulherada do Pará, a mulherada
do Pará também está aqui.

Mas eu citei Osmar Prado porque o Osmar Prado é um artista de uma
qualidade irrepreensível. O que Deus não deu de tamanho para ele, deu de
inteligência e de capacidade artística. E ele já fez papéis
extraordinários, mas tem um que eu nunca esqueço, que ele era motorista
e era tratado como se fosse chamado de Tabaco, e o Tabaquinho marcou a
minha vida. E eu fico mais feliz porque ele tem uma posição política
extraordinária.

E eu acho que esse aqui tem lado, esse tem lado e é com essa essa
gente que a gente vai construir a nova política deste país.

[Osmar Prado pede o microfone]

\emph{Posso dizer uma coisinha sobre o Tabaco? Olha, o Tabaco tinha três
mulheres --- e ainda o que tinha por fora, aí eu pedi ao autor que desse
o final do Tabaco, ele sendo traído --- porque todo traidor um dia é
traído. E aí aparece a mulher do Tabaco, com uma penca de filho, grávida
--- e eu digo: ``mulher como é que você está grávida? Faz mais de um ano
que eu não vou lá''.}

[Lula continua]

Isso é vingança das mulheres, vingança. Porque o homem pensa que
só ele é esperto, mas as mulheres também são espertas.

Então, companheiros e companheiras, nós conseguimos\ldots{}

[Lula interrompe para pedir que socorram uma pessoa que passa
mal no chão]

Mas eu ia dizendo pra vocês que nós não conseguimos aprovar a
proposta que eu considerava boa, e o pessoal então passou a desrespeitar
a diretoria do sindicato. E eu ia na porta da fábrica e ninguém parava,
e a imprensa escrevia: ``Lula fala para os ouvidos moucos dos
trabalhadores''.

Nós levamos um ano para recuperar o nosso prestígio na categoria e
eu fiquei pensando com ar de vingança: os trabalhadores dizem que podem
fazer 100 dias de greve, 400 dias de greve, que eles vão até o fim, pois
eu vou testá"-los em 1980.

E fizemos a maior greve da nossa história: a maior greve, 41 dias
de greve. Com 17 dias de greve, eu fui preso, e os trabalhadores
começaram, depois de alguns dias, a furar a greve. E nós então\ldots{} eu sei
que o Tuma, eu sei que o doutor Almir, eu sei que doutor Vilela iam
dentro da cadeia e falavam para mim: ``você tem que acabar com a greve'',
e eu dizia ``eu não vou acabar com a greve; os trabalhadores vão decidir
por conta própria''.

O dado concreto é que ninguém aguentou 41 dias, porque, na
prática, o companheiro tinha que pagar leite, tinha que pagar conta de
luz, tinha que pagar gás. A mulher passou a cobrar dele o dinheiro do
pão, ele então começou a sofrer pressão, não aguentou.

Mas é engraçado porque, na derrota, a gente ganhou muito mais, sem
ganhar economicamente, do que quando a gente ganhou economicamente.

Significa que não é dinheiro que resolve o problema de uma greve.
Não é 5\%, não é 10\%, é o que está embutido de teoria política, de
conhecimento político e de tese política numa greve.

Agora, nós estamos quase que na mesma situação, eu estou sendo
processado e tenho dito claramente: eu
sou o único ser humano processado por um apartamento que não é
meu.

E ele sabem que O Globo mentiu quando disse que era meu. A Polícia
Federal da Lava Jato, quando fez o inquérito, mentiu que era meu. O
Ministério Público, quando fez a acusação, mentiu dizendo que era meu. E
eu pensei que o Moro ia resolver, e ele mentiu dizendo que era meu. E me
condenou a nove anos de cadeia.

É por isso que eu sou um cidadão indignado. Porque eu já fiz muita
coisa nos meus 72 anos, mas eu não os perdoo por terem passado para a
sociedade a ideia de que eu sou um ladrão.

Deram a primazia dos bandidos fazerem um Pixuleco pelo Brasil
inteiro. Deram a primazia dos bandidos chamarem a gente de petralha.
Deram a primazia de criar quase que um clima de guerra, negando a
política nesse país.

Eu digo todo dia: nenhum deles tem coragem ou dorme com a
consciência tranquila da honestidade, da inocência, que eu durmo. Nem um
deles.

Eu não estou acima da Justiça. Se eu não acreditasse na Justiça,
eu não tinha feito um partido político. Eu tinha proposto uma revolução
nesse país.

Mas eu acredito na Justiça, numa Justiça justa, numa Justiça que
vota um processo baseado nos autos do processo, baseado nas informações
das acusações, das defesas, na prova concreta que tem a arma do crime.

O que eu não posso admitir é um procurador que fez um PowerPoint e
foi para a televisão dizer que o \versal{PT} é uma organização criminosa que
nasceu para roubar o Brasil e que o Lula, por ser a figura mais
importante desse partido, o Lula é o chefe. E, portanto, se o Lula é o
chefe, diz o procurador: ``Eu não preciso de provas, eu tenho
convicção''.

Eu quero que ele guarde a convicção dele para os comparsas deles.
Para os asseclas deles, e não para mim. Não para mim. Certamente um
ladrão não estaria exigindo provas. Estaria de rabo preso, com a boca
fechada, torcendo para a imprensa não falar o nome dele.

Eu tenho mais de 70 horas de Jornal Nacional me triturando. Eu
tenho mais de 70 capas de revistas me atacando. Eu tenho
milhares de páginas de jornais e matérias me atacando. Eu tenho mais a
Record me atacando. Eu tenho mais a Bandeirantes me atacando. Eu tenho
mais a rádio do interior, a rádio do [inaudível]. E o que eles não
se dão conta é que quanto mais eles me atacam, mais cresce a minha
relação com o povo brasileiro.

Eu não tenho medo deles. Até já falei que gostaria de fazer um
debate com o Moro sobre a denúncia que ele fez contra mim. Eu gostaria
que ele me mostrasse alguma coisa de prova. Eu já desafiei os juízes do
\versal{TRF}"-4. Que ele fosse para um debate na universidade que ele quiser, no
público que ele quiser, provar qual é o crime que eu cometi nesse país.

E às vezes tenho a impressão, e tenho porque sou um construtor
de sonho\ldots{} Eu, há muito tempo atrás, sonhei que era possível
governar esse país envolvendo milhões e milhões de pessoas pobres na
economia, envolvendo milhões de pessoas nas universidades, criando
milhões e milhões de empregos nesse país.

Eu sonhei, eu sonhei que era possível um metalúrgico sem diploma
de universidade cuidar mais da educação do que os diplomados e
concursados que governaram esse país.

Eu sonhei que era possível a gente diminuir a mortalidade infantil
levando leite, feijão e arroz para que as crianças pudessem comer todo
dia. Eu sonhei que era possível pegar os estudantes da periferia e
colocar nas melhores universidades desse país. Para que a gente não
tenha juiz e procurador só da elite.

Daqui a pouco nós vamos ter juízes e procuradores nascidos na
favela de Heliópolis, nascidos em Itaquera, nascidos na periferia. Vamos
ter muita gente dos Sem Terra, do \versal{MTST}, da \versal{CUT} formado. Esse crime eu
cometi.

Cometi esse crime que eles não querem que eu cometa mais. É por
conta desse crime que já tem uns dez processos contra mim. E se for por
esses crimes, de colocar pobre na universidade, negro na universidade,
pobre comer carne, pobre comprar carro, pobre viajar de avião, pobre
fazer sua pequena agricultura, ser microempreendedor, ter sua casa
própria, se esse é o crime que eu cometi, eu quero dizer eu vou
continuar sendo criminoso nesse país porque vou fazer muito mais. Vou
fazer muito mais.

Companheiros e companheiras, eu, em 1990, em 1986, fui o
deputado constituinte mais votado na história do país. E nós ficamos
descobrindo que dentro do \versal{PT}, Manuela, companheiros --- o Ivan era do \versal{PT} na
época --- havia uma desconfiança que só tinha poder no \versal{PT} quem tinha
mandato.

Quem não tivesse mandato era tido\ldots{} Eu não citei o senador
Humberto Costa que eu vi aqui, Humberto Costa, senador de Pernambuco, eu
esqueci de citar para vocês. Ninguém me deu nominata. A Fátima
[Bezerra] é do Rio Grande do Norte, ela será a futura governadora do
Rio Grande do Norte. Esse aqui, junto com Paulo Pimenta, é o companheiro
que mais briga e mais denuncia a Lava Jato. O [Miguel] Rossetto foi
ministro do Trabalho e da Previdência e será talvez o governador do Rio
Grande do Sul nessas eleições agora.

Está aqui nossa companheira Jandira Feghali que é uma companheira
extraordinariamente combativa, tá? O Glauber Rocha\ldots{} É Braga, é Braga.
Alguém prepara uma nominata para mim que eu vou citando as pessoas.

Então, companheiros, quando eu percebi que o povo desconfiava que
só tinha valor no \versal{PT} quem era deputado, Manuela e Guilherme, sabe o que
eu fiz? Deixei de ser deputado. Porque eu queria provar ao \versal{PT} que eu ia
continuar sendo a figura mais importante do \versal{PT} sem ter mandato. Porque
se alguém quiser ganhar de mim no \versal{PT}, só tem um jeito: é trabalhar mais
do que eu e gostar do povo mais do que eu. Porque se não gostar, não vai
ganhar.

Pois bem, nós agora estamos num trabalho delicado. Eu talvez viva
o momento de maior indignação que um ser humano vive. Não é fácil o que
sofre a minha família. Não é fácil o que sofrem os meus filhos. Não é
fácil o que sofreu a Marisa.

E quero dizer que a antecipação da morte da Marisa foi a
safadeza e a sacanagem que a imprensa e o Ministério Público fizeram
contra ela. Tenho certeza. Porque essa gente eu acho que não tem filho,
eu acho que não tem alma e não tem noção do que sente uma mãe e um pai
quando vê um filho massacrado, quando vê um filho sendo atacado. E eu,
então, companheiros, resolvi levantar a cabeça.

Não pensem que eu sou contra a Lava Jato não. A Lava Jato, se pegar
bandido, tem que pegar bandido mesmo, que roubou, e prender. Todos nós
queremos isso. Todos nós a vida inteira dizíamos, só prende pobre, não
prende rico. Todos nós dizíamos. E eu quero que continue prendendo rico.
Eu quero.

Agora, qual é o problema? É que você não pode fazer julgamento
subordinado à imprensa. Porque no fundo, no fundo, você destrói as
pessoas na sociedade, na imagem das pessoas, e depois os juízes vão
julgar e falam ``Eu não posso ir contra a opinião pública porque a
opinião pública está pedindo pra cassar''.

Quem quiser votar com base na opinião pública, largue a toga e vá
ser candidato a deputado. Escolha um partido político e vá ser
candidato. Ora, a toga é um emprego vitalício. O cidadão tem que votar
apenas com base nos autos do processo. Aliás, eu acho que ministro da
Suprema Corte não deveria dar declaração de como vai votar. Nos Estados
Unidos, termina a votação e você não sabe o que o cidadão votou
exatamente para que ele não seja vítima de pressão.

Imagina um cara ser acusado de homicídio e não tenha sido ele o
assassino. O que que a família do morto quer? Que ele seja morto, que
ele seja condenado. Então o juiz tem que ter, diferentemente de nós, a
cabeça mais fria. Mais responsabilidade de fazer acusação ou de
condenar.

O Ministério Público é uma instituição muito forte, por isso esses
meninos, que entram muito novos, fazem um curso de direito, depois fazem
três anos de concurso, porque o pai pode pagar, esses meninos precisavam
conhecer um pouco da vida, conhecer um pouco de política para fazer o
que eles fazem na sociedade brasileira. Ter uma coisa chamada
responsabilidade.

E não pensem que, quando eu falo assim, eu sou contra. Eu fui
presidente e indiquei quatro procuradores. E fiz discurso em todas as
posses. E eu dizia: quanto mais forte for a instituição, mais
responsáveis os seus membros têm que ser. Você não pode condenar a
pessoa pela imprensa para depois julgá"-la. Vocês estão lembrados
que quando eu fui prestar depoimento lá em Curitiba eu disse pro Moro:
você não tem condições de me absolver porque a Globo está exigindo que
você me condene e você vai me condenar.

Pois bem, eu acho que tanto o \versal{TFR}"-4 quanto o Moro, a Lava Jato e a
Globo têm um sonho de consumo. O sonho de consumo é que, primeiro,
o golpe não terminou com a Dilma. O golpe só vai concluir quando eles
conseguirem convencer que o Lula não pode ser candidato a presidente da
República em 2018.

Eles não querem, não é porque eu vou ser eleito, eles não querem
que eu participe apenas porque tem a possibilidade de cada um de nós se
eleger. Eles não querem o Lula, eles não podem [inaudível] que pobre
na cabeça deles [inaudível]. Pobre não pode andar de avião, pobre
não pode fazer universidade, pobre nasceu, segundo a lógica deles, pra
comer e ter coisa de segunda categoria.

O sonho de consumo deles é a fotografia do Lula preso. Ah, eu fico
imaginando o tesão da Veja colocando a capa minha preso. Eu fico
imaginando o tesão da Globo colocando a fotografia minha preso. Eles vão
ter orgasmos múltiplos.

Eles decretaram a minha prisão. E deixa eu contar uma coisa pra
vocês. Eu vou atender o mandado deles. E vou atender porque eu quero
fazer a transferência de responsabilidade. Eles acham que tudo o que
acontece nesse país, acontece por minha causa. Eu já fui condenado a
três anos de cadeia. [Corte no vídeo] chegando a hora de a onça
beber água e os camponeses mataram o fazendeiro e eles acham que essa
frase minha era a senha.

O que eu quero transferir de responsabilidade? Eles já tentaram me
prender por obstrução de justiça, não deu certo. Eles agora querem me
pegar numa prisão preventiva, que é uma coisa mais grave, porque não tem
habeas corpus. O Vaccari já está preso há três anos, o Marcelo Odebrecht
já gastou R\$ 400 milhões e não teve habeas corpus. Eu não vou gastar um
tostão.

Mas eu vou lá com a seguinte crença: eles vão descobrir pela
primeira vez o que eu tenho dito todo dia, eles não sabem que o problema
desse país não se chama Lula. O problema desse país chama"-se vocês, a
consciência do povo, o Partido dos Trabalhadores, o \versal{PC}do\versal{B}, o \versal{MST}, o
\versal{MTST}\ldots{} Eles sabem que tem muita gente.

E aquilo que nossa pastora diz, e eu tenho dito todo discurso: não
adianta tentar evitar que eu ande por esse país porque tem milhões e
milhões de Lulas, de Boulos, de Manuelas, de Dilmas Rousseff para andar
por mim. Não adianta tentar acabar com as minhas ideias, elas já estão
pairando no ar e não tem como prendê"-las. Não adianta tentar parar os
meus sonhos porque quando eu parar de sonhar, eu sonharei pela cabeça de
vocês.

Não adianta achar que tudo vai parar no dia que o Lula tiver
infarto. É bobagem porque o meu coração baterá pelo coração de vocês e
são milhões de corações.

Não adianta eles acharem que vão fazer com que eu pare, eu não
pararei porque eu não sou mais um ser humano. Eu sou uma ideia. Uma
ideia misturada com a ideia de vocês.

E eu tenho certeza que companheiros como os Sem Terra, \versal{MTST}, os
companheiros da \versal{CUT}, do movimento sindical [corte]. E essa é uma
prova. Eu vou cumprir o mandado e vocês vão ter que se transformar, cada
um de vocês, vocês não vão mais chamar Chiquinha, Joãozinho, Zezinho,
Robertinho. Todos vocês, daqui pra frente, vão virar Lula e vão andar
por esse país.

Vamos fazer definitivamente uma regulação dos meios de comunicação
para que o povo não seja vítima das mentiras todo santo dia. Eles têm
que saber, que vocês, quem sabe, são até mais inteligentes do que eu, e
poderão queimar os pneus que tanto queimam, fazer as passeatas que tanto
vocês [inaudível], fazer as ocupações no campo e na cidade\ldots{}
Parecia difícil a ocupação de São Bernardo e amanhã vocês vão receber a
notícia de que ganharam o terreno que vocês invadiram.

Portanto, companheiros, eu tive chance agora, eu estava no Uruguai,
entre Livramento e Rivera. E as pessoas diziam assim pra mim: Lula, você
dá uma voltinha ali, é só atravessar a rua, finge que você vai comprar
um uisquezinho, você está no Uruguai junto com Pepe Mujica e vai
embora e não volta mais e pede asilo político. Ô Lula, você pode ir na
embaixada da Bolívia, pode ir na embaixada do Uruguai. Ô Lula, vai na
embaixada da Rússia, vai na embaixada e de lá você pode ficar falando. E
eu falei que não tenho mais idade.

A minha idade é enfrentá"-los, olho no olho, e eu vou enfrentá"-los
aceitando cumprir o mandado. Eu quero saber quantos dias eles vão pensar
que estão me prendendo. E quanto mais dias eles me deixarem lá, mais
Lulas vão nascer nesse país e mais gente vai querer brigar nesse país
porque a democracia não tem limite, não tem hora pra gente brigar.

Por isso eu estou fazendo uma coisa muito consciente, mas muito
consciente. Eu falei para os companheiros, se dependesse da minha
vontade eu não iria, mas eu vou. Eu vou porque eles vão dizer a partir
de amanhã que o Lula está foragido, que o Lula está escondido. Não, eu
não estou escondido. Eu vou lá na barba deles, para eles saberem que eu
não tenho medo, para eles saberem que eu não vou correr e para eles
saberem que eu vou provar a minha inocência. Eles têm que saber disso,
tá?

E façam o que quiserem, eu vou terminar com uma frase que eu
peguei em 1982, com uma menina de dez anos em Catanduva, que eu não sei
quem é. E essa frase não tem autor. A frase dizia: ``Os poderosos podem
matar uma, duas ou três rosas, mas jamais conseguirão deter a chegada da
primavera''.

Porque nós queremos mais casa, nós queremos mais escola, nós
queremos menos mortalidade. Nós não queremos impedir a barbaridade que
fizeram com a Marielle no Rio de Janeiro? Nós não queremos impedir a
barbaridade que fazem com meninos negros na periferia desse país? Não
queremos mais que volte a desnutrição, a mortalidade por desnutrição
nesse país. Nós não queremos mais que um jovem não tenha esperança de
entrar na universidade. Porque esse país é tão cretino que foi o último
do mundo a ter uma universidade. O último. Todos os países mais pobres
tiveram. Porque eles não queriam que a juventude brasileira estudasse e
falaram que custava muito fazer escola. E se perguntar quanto custou não
fazer há 50 anos atrás\ldots{}

Então eu quero que vocês saibam que eu tenho orgulho, profundo
orgulho, de ter sido o único presidente da república sem ter um diploma
universitário, mas sou o presidente da república que mais fiz
universidades na história desse país para mostrar para essa gente que
não confunda inteligência com quantidade de anos na escolaridade.

Isso não é inteligência, é conhecimento. Inteligência é quando
você sabe tomar decisão. Inteligência é quando você tem lado. Quando
você não tem medo de descobrir com os companheiros aquilo que é
prioridade. E a prioridade é garantir que esse país volte a
ter cidadania.

Não vão vender a Petrobras. Vamos fazer uma nova Constituinte,
vamos revogar a lei do petróleo que eles estão fazendo. Não vamos deixar
vender o \versal{BNDES}, não vamos deixar vender a Caixa Econômica, não vamos
deixar destruir o Banco do Brasil, e vamos fortalecer a agricultura
familiar que é responsável por 70\% do alimento que comemos nesse país.

É com essa crença, companheiros, de cabeça erguida, como eu estou
falando com vocês, que eu quero chegar lá e falar para o delegado: estou
à sua disposição. E a história, a história, daqui a alguns dias, vai
provar que quem cometeu crime foi o delegado que me acusou, foi o juiz
que me julgou e foi o Ministério Público que foi leviano comigo.

Por isso companheiros, eu não tenho lugar no meu coração para todo
mundo. Mas eu quero que vocês saibam, se tem uma coisa que eu aprendi a
gostar é da minha relação com o povo. Quando eu pego na mão de um de
vocês, quando eu abraço um de vocês, quando eu beijo --- porque agora
eu beijo homem e mulher igualzinho --- quando eu beijo um de vocês,
eu não estou beijando com segundas intenções. Eu estou beijando porque
quando eu era presidente, eu dizia, eu vou voltar para onde eu vim e eu
sei quem são meus amigos eternos e quem são os amigos eventuais.

Os de gravatinha, que iam atrás de mim, agora desapareceram. Quem
estão comigo são aqueles companheiros que eram meus amigos antes de eu
ser presidente da República. São aqueles que comiam rabada aqui no
Zelão, que comiam frango com polenta no Demarchi, aqueles que tomavam
caldo de mocotó no Zelão. Esses continuam sendo nossos amigos.

Aqueles que têm coragem de invadir um terreno para fazer casa.
Aqueles que têm coragem de fazer uma greve contra a Previdência, aqueles
que têm coragem de ocupar um campo para fazer uma fazenda produtiva.
Aqueles que, na verdade, precisam do Estado.

Então companheiros, eu vou dizer uma coisa para vocês, vocês vão
perceber que eu sairei dessa maior, mais forte, mais verdadeiro e
inocente porque eu quero provar que eles é que cometeram o crime. Um
crime político, de perseguir um homem que tem 50 anos de história
política. E por isso eu sou muito grato.

Eu não tenho como pagar a gratidão, o carinho e o respeito que
vocês têm dedicado a mim nesses tantos anos. E quero dizer a você,
Guilherme, e à Manuela que, para mim, é motivo de orgulho pertencer a uma
geração que está no final dela vendo nascer dois jovens disputando o
direito de ser presidente da república desse país.

Por isso companheiros, um grande abraço.

Pode ficar certo, esse pescoço aqui não abaixa, a minha mãe já fez
um pescoço curto para ele não abaixar e não vai abaixar porque eu vou de
cabeça erguida e vou sair de peito estufado de lá porque vou provar a
minha inocência.

Um abraço companheiros, obrigado, mas muito obrigado a todos vocês
pelo que vocês me ajudaram. Um beijo querido, muito obrigado.
