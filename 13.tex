\chapter*{Réquiem para a Constituição de 1988}

\addcontentsline{toc}{chapter}{Réquiem para a Constituição de 1988,
\scriptsize{por Rafael Valim}}

\begin{flushright}
\emph{Rafael Valim}\footnote{Doutor e Mestre em Direito Administrativo
pela \versal{PUC}/\versal{SP}. Professor de Direito Administrativo da Faculdade de Direito
da \versal{PUC}/\versal{SP}. Advogado.}
\end{flushright}

O projeto de democracia no Brasil, a exemplo dos demais países
latino"-americanos, é constantemente interrompido por golpes de Estado.
Após mais de vinte anos de ditadura militar (1974 a 1985), as
brasileiras e os brasileiros viveram mais um curto período de
\emph{governo} eleito por vias democráticas, cujo término se deu em 31
de agosto de 2016, data em que se afastou definitivamente do cargo a
Presidenta eleita Dilma Rousseff.

Nas lições de Guillermo O'Donnell, no Brasil já se instalaram
\emph{governos} democraticamente eleitos, mas ainda não se ultrapassou a
``segunda transição'', mais complexa e demorada, para um \emph{regime}
verdadeiramente democrático, em que compareça uma sólida sociedade
democrática.\footnote{``Democracia delegativa?'' \emph{Novos estudos,}
  São Paulo: Cebrap, n.~31, pp. 25-40, out. 1991, n. 31, p. 26.}
Persiste uma sociedade profundamente autoritária, hostil aos mais
elementares avanços em termos de direitos humanos, o que, naturalmente,
\emph{explica a facilidade com que a exceção não só é assimilada, como
também dissimulada em seu seio}. Nas palavras de Paulo Sérgio Pinheiro,
``o autoritarismo é tão socialmente implantado que o regime de exceção
tem condições de gozar, durante certos períodos, de larga capacidade de
dissimulação e de ocultação de grande parte dos seus feitos, mantendo"-se
quase que totalmente imune à efetiva autodefesa dos
cidadãos''.\footnote{\versal{PINHEIRO}, Paulo Sergio. ``Estado e Terror''.
  \emph{In}: \versal{NOVAES}, Adauto (coord.). \emph{Ética}. São Paulo: Companhia
  das Letras, 2007, p. 114.}

Desta vez a democracia não foi abatida por um golpe militar, com tanques
e fuzis, mas sim pelo que vem sendo chamado de um ``golpe
institucional'', gestado e levado a efeito sob uma aparência de
legalidade. Instaurou"-se um processo, ouviram"-se as partes e as
testemunhas, elaboraram"-se relatórios, mas tudo não passava de uma
grande farsa, um simulacro de devido processo legal encenado por
parlamentares toscos e venais, sob o impulso decisivo da mídia nativa.

Apesar de nos parecer sumamente interessante, não cabe nos propósitos do
presente trabalho a pormenorização da conjuntura que levou à queda da
Presidenta Dilma Rousseff, tampouco os eventos que sucederam ao golpe de
Estado. Limitar"-nos"-emos a narrar os fatos que demonstram, de maneira
irretorquível, a proliferação do estado de exceção no Brasil atual.

De qualquer modo, é fundamental desde já compreender que o golpe de
estado de 2016 é tão só \emph{um} exemplo das múltiplas exceções que, se
já não sepultaram por completo o combalido Estado de Direito brasileiro,
estão em vias de fazê"-lo. Na realidade, como restará claro, o principal
e mais perigoso agente da exceção no Brasil é o Poder Judiciário.

Com efeito, a partir de novembro de 2014, com o início da chamada
``Operação Lava jato'', uma série de prisões cautelares de empresários e
de agentes públicos, revestidas de grande espetacularização, somadas aos
chamados ``vazamentos seletivos'' de informações, em absoluta
orquestração com grandes veículos de comunicação social, criaram as
condições sociais e políticas para a instauração do processo de
\emph{impeachment} e a posterior destituição da Presidenta eleita.

Além da evidente ilegalidade das prisões cautelares, fundadas, no mais
das vezes, em conceitos indeterminados como ``defesa da ordem pública'',
pouco antes da instauração do processo de \emph{impeachment} chegou"-se
ao cúmulo de uma conversa da Presidenta da República ser interceptada
por um juiz de primeira instância -- manifestamente incompetente no caso
-- e, este mesmo juiz, não satisfeito com a gravíssima ilegalidade que
acabara de cometer, ordenar a \emph{divulgação} do diálogo, em
claríssima violação do art. 8\sout{º} da Lei n. 9.296/96, cujos termos
seja"-nos permitido transcrever: ``a interceptação de comunicação
telefônica, de qualquer natureza, ocorrerá em autos apartados, apensados
aos autos do inquérito policial ou do processo criminal, preservando"-se
o \emph{sigilo} das diligências, gravações e transcrições
respectivas''.\footnote{O art. 9\sout{º} da mesma lei ainda estabelece
  que ``a gravação que não interessar à prova será inutilizada por
  decisão judicial, durante o inquérito, a instrução processual ou após
  esta, em virtude de requerimento do Ministério Público ou da parte
  interessada''.} Para agravar este quadro tétrico, o Supremo Tribunal
Federal reconheceu posteriormente a ilegalidade da conduta do aludido
magistrado\footnote{Medida Cautelar na Reclamação n. 23.457 -- Paraná,
  sob relatoria do Min. Teori Zavascki. Decisão prolatada no dia 22 de
  março de 2016.} -- ou seja, restou configurado o cometimento de
\emph{crime}, à luz do art. 10 da mencionada Lei n. 9.296/96 --, mas
nenhuma providência de ordem criminal ou disciplinar foi tomada contra
ele até o presente momento.

Deveras, não só se deixou de punir o magistrado pelo evidente crime que
praticou, senão que o Tribunal Regional Federal da 4\sout{ª} Região, sob
a relatoria do Desembargador Federal Rômulo Puzzollatti, consagrou
explicitamente \emph{um estado de exceção jurisdicional}, para o
escárnio universal do Judiciário brasileiro\footnote{P.A. N.
  0003021-32.2016.4.04.8000/\versal{RS} -- Corte Especial. Neste caso, não se
  pode deixar de saudar, sob pena de grave injustiça, o eminente
  Desembargador Federal Rogério Favreto, único membro da Corte Especial
  do Tribunal Regional Federal da 4\sout{ª} Região que votou pela
  abertura de processo disciplinar contra o Juiz Federal Sérgio Moro.}:

Ora, é sabido que os processos e investigações criminais decorrentes da
chamada ``Operação Lava"-Jato'', sob a direção do magistrado
representado, constituem caso inédito (único, excepcional) no direito
brasileiro. Em tais condições, neles haverá situações inéditas, que
escaparão ao regramento genérico, destinado aos casos comuns. Assim,
tendo o levantamento do sigilo das comunicações telefônicas de
investigados na referida operação servido para preservá"-la das
sucessivas e notórias tentativas de obstrução, por parte daqueles,
garantindo"-se assim a futura aplicação da lei penal, é correto entender
que o sigilo das comunicações telefônicas (Constituição, art. 5\sout{º},
\versal{XII}) pode, em casos excepcionais, ser suplantado pelo interesse geral na
administração da justiça e na aplicação da lei penal. A ameaça
permanente à continuidade das investigações da Operação Lava"-Jato,
inclusive mediante sugestões de alterações na legislação, constitui, sem
dúvida, uma situação inédita, a merecer um tratamento excepcional.

A propósito, na persecução criminal deflagrada contra o Ex"-Presidente
Lula encontramos uma síntese eloquente das grosseiras e aberrantes
inconstitucionalidades que vêm sendo cometidas em nossa atual quadra
histórica no exercício da função jurisdicional.\footnote{Para um exame
  aprofundado do caso, consultar: \versal{ZANIN} \versal{MARTINS}, Cristiano; \versal{ZANIN}
  \versal{MARTINS}, Valeska Teixeira; \versal{VALIM}, Rafael (coord.). \emph{O Caso Lula:}
  a luta pela afirmação dos direitos fundamentais no Brasil. São Paulo:
  Editora Contracorrente, 2017.} Os princípios do juiz natural, da
imparcialidade e da presunção de inocência vêm sendo solenemente
desconsiderados, sob os olhares cúmplices da mídia e a atenção de uma
turba ignara que, a cada nova arbitrariedade, destila seu ódio nas ruas
e nas redes sociais. A isto se somam as graves violações às
prerrogativas profissionais dos advogados do Ex"-Presidente, também
vítimas -- para ficar com apenas um exemplo -- de interceptações
telefônicas ilegais.\footnote{Todos estes ilícitos levaram o
  Ex"-Presidente Lula a formular um comunicado individual ao Comitê de
  Direitos Humanos da \versal{ONU}.}

Não se imagine, contudo, que o atual estado de exceção no Brasil se
circunscreva a juízes provincianos. Até mesmo a mais alta Corte do país,
o Supremo Tribunal Federal, por ação ou omissão, curvou"-se à exceção,
conforme comprova, de maneira irrefutável, a decisão emitida no dia 17
de fevereiro de 2016, no bojo do \emph{habeas corpus} n. 126.292, na
qual se admitiu, em claríssimo contraste com o art. 5\sout{º}, inc.
\versal{LVII}, da Constituição Federal -- segundo a qual ninguém será considerado
culpado \emph{até o trânsito em julgado de sentença penal condenatória}
--, a possibilidade de início da execução de sentença penal condenatória
após a sua confirmação em segundo grau. Em outras palavras, o Supremo
Tribunal Federal, a título de aplicar a Constituição, violou"-a às
escâncaras, na medida em que extraiu do texto constitucional um sentido
nele não comportado.

Em outro dizer, a Constituição foi \emph{sequestrada} pelo Supremo
Tribunal Federal, o que nos faz lembrar um trecho do famoso discurso de
Franklin Delano Roosevelt, ao apresentar um projeto de reforma da
Suprema Corte estadunidense: ``We have, therefore, reached the point as
a nation where we must take action to save the Constitution from the
Court and the Court from itself (\ldots{}). We want a Supreme Court which
will do justice under the Constitution and not over it''.

A degradação do Poder Judiciário é tão grave que um juiz do 3\sout{º}
Tribunal do Júri do Estado do Rio de Janeiro, para o fim de conceder a
liberdade de dois policiais militares presos em flagrante por conta do
brutal homicídio de dois suspeitos feridos, invocou explicitamente em
sua decisão a ``voz das ruas''.\footnote{Autos n.
  0076306-12.2017.8.19.0001. Juiz Alexandre Abrahão Dias Teixeira.} Ou
seja, não é mais a voz do povo, plasmada na Constituição Federal e nas
leis, senão que uma insondável ``voz das ruas'', cujo conteúdo é
determinado, arbitrariamente, pelos espíritos ``iluminados'' de
determinados juízes. A propósito, não é demais recordar a advertência do
Ministro Eros Grau ao Ministro Carlos Britto quando este, também afeito
ao ``clamor das ruas'', pretendeu deslocar o julgamento de um
\emph{habeas corpus} ao Pleno do Supremo Tribunal Federal\footnote{Questão
  de ordem em~habeas corpus~85.298-0 -- São Paulo.}: ``(\ldots{}) embora seja
novo no Tribunal, para mim todos os casos têm repercussão idêntica.
Porque o meu compromisso é aplicar o direito. O fato de a imprensa tocar
ou não no assunto, a mim não incomoda. Já estou imune ao clamor público.
Para mim, o que importa é o clamor da Constituição. Isso em primeiro
lugar''.

Ocioso observar que todas estas demonstrações de desfaçatez do Poder
Judiciário são um convite ao desrespeito à ordem jurídica. Nesse
sentido, testemunha"-se uma aluvião de cenas explícitas de violência de
agentes de segurança pública contra jornalistas, grupos vulneráveis e
movimentos sociais, de que é um triste exemplo a~absurda e truculenta
invasão~da Escola Nacional Florestan Fernandes, mantida pelo Movimento
dos Trabalhos Sem Terra (\versal{MST}), pela Polícia Civil do Estado de São
Paulo.

É neste ambiente de completa arbitrariedade que se insere o golpe de
estado de 2016.

Os motivos invocados para a deflagração do processo de impedimento foram
as chamadas ``pedaladas fiscais'' -- apelido atribuído à sistemática
mora do Tesouro Nacional nos repasses de recursos ao Banco do Brasil e à
Caixa Econômica Federal para que estes paguem benefícios sociais como o
``Bolsa Família'' e ``Minha Casa, Minha Vida'' -- e a abertura de
créditos suplementares sem autorização legal. Ambas as condutas, a teor
do que dispõe a legislação brasileira, \emph{jamais} poderiam ser
consideradas \emph{crime de responsabilidade} e, portanto, seriam de
todos imprestáveis a justificar o \emph{impeachment} do Chefe do Poder
Executivo Federal.

Apesar disso, a Câmara dos Deputados admitiu a acusação contra a
Presidenta da República e, em 12 de maio de 2016, o Senado, por 55 votos
a 22, determinou a instauração do processo, com o consequente
afastamento da Presidenta de suas funções, à luz do art. 86, §
1\sout{º}, inc. \versal{II}, da Constituição Federal.

A partir deste momento, assumiu, interinamente\footnote{Sobre o período
  de interinidade, consultar, por todos: \versal{SALGADO}, Eneida Desirée.
  \emph{Um diário do governo interino}. Curitiba: Íthala, 2016.}, o
então Vice"-Presidente Michel Temer, quem, de imediato, não só compôs um
novo governo, mediante a substituição de Ministros e outras autoridades,
como também promoveu uma aberta e despudorada campanha junto ao Senado
em favor da condenação da Presidenta afastada. É dizer: a norma
constitucional que determina o afastamento do Presidente da República,
cujo evidente objetivo é evitar a interferência daquele no desfecho do
processo, prestou"-se à interferência explícita do Vice"-Presidente
\emph{em prol} do impedimento.

Finalmente, em 31 de agosto de 2016, após outras tantas
inconstitucionalidades e demonstrações de misoginia, consumou"-se a
destituição da Presidenta Dilma Rousseff.

A partir daí, o governo ilegítimo, em aliança com o Parlamento, inicia
uma avassaladora estratégia de desfiguração do modelo de Estado Social
de Direito consagrado na Constituição de 1988, diante de um povo
domesticado pelos grandes veículos de comunicação social, cujas verbas
publicitárias cresceram exponencialmente desde a chegada dos golpistas
ao poder.

Tal estratégia inclui a adoção, por meio de Emenda Constitucional
(Emenda Constitucional n. 95/2016), de um programa de austeridade
\emph{seletivo}, com duração de vinte anos, em que se sacrificam as
despesas sociais e se preservam as despesas com o setor financeiro; a
alteração da Lei n. 13.365/2016, para o fim de extinguir a exclusividade
da Petrobras como operadora do Pré"-sal; a aprovação de uma Reforma
Trabalhista que promove escandalosos retrocessos sociais; a formulação
de uma reforma da Previdência Social que, se aprovada, sacrificará, uma
vez mais, os trabalhadores; a proposta de facilitação de venda de terras
a estrangeiros, com sérios riscos à soberania social.

Esta breve narração histórica nos permite identificar, com chocante
clareza, os três elementos centrais do estado de exceção: o
\emph{soberano}, o \emph{inimigo} e a \emph{superação da normatividade}.

A agenda neoliberal imposta pelo governo ilegítimo -- cujos contornos se
amoldam perfeitamente à \emph{doutrina do shock} exposta por Naomi
Klein\footnote{Afirma Naomi Klein: ``(\ldots{}) particularmente en países en
  los que la clase dirigente ha perdido su credibilidade ante el
  público, se dice que sólo un shock político enorme y decidido puede
  lograr `enseñar' al público esta dura lección'' (\versal{KLEIN}, Naomi.
  \emph{La doctrina del shock: el auge del capitalismo del desastre}.
  Barcelona: Paidós, 2007, p. 118).} -- somada à devastação da indústria
nacional operada pela Operação Lava Jato, apontam, univocamente, para o
verdadeiro \emph{soberano} no Brasil: o \emph{mercado}, encarnado em uma
elite que, apenas em 2015, apropriou"-se, através de pagamento de juros e
amortizações da dívida pública, de novecentos e sessenta e dois bilhões
de reais do povo brasileiro, ou seja, quarenta e dois por cento do
orçamento da União.

Já o \emph{inimigo} está plasmado na figura do \emph{corrupto}, a quem
são negadas as mais óbvias garantias processuais enfeixadas no princípio
do devido processo legal, em uma guerra que desconhece limites. Nesse
contexto, o enfrentamento da corrupção, enquanto desafio fundamental das
democracias contemporâneas, passa a constituir um \emph{cavalo de troia}
dentro do Estado de Direito, sendo usado em favor de interesses
inconfessáveis.\footnote{\versal{VALIM}, Rafael; \versal{COLANTUONO}, Pablo Ángel
  Gutiérrez. O enfrentamento da corrupção nos limites do Estado de
  Direito. \emph{In}: \versal{ZANIN} \versal{MARTINS}, Cristiano; \versal{ZANIN} \versal{MARTINS}, Valeska
  Teixeira; \versal{VALIM}, Rafael (coord.). \emph{O Caso Lula: a luta pela
  afirmação dos direitos fundamentais no Brasil}. São Paulo: Editora
  Contracorrente, 2017, pp. 74.}

Na lição de Jessé Souza,

Como em toda a história republicana brasileira, o mote da corrupção é
sempre usado como arma letal para o inimigo de classe da elite e de seus
aliados. Isso sempre ocorre quando existem políticas que envolvam
inclusão dos setores marginalizados -- que implicam menor participação
no orçamento dos endinheirados e aumento do salário relativo dos
trabalhadores, o que também não os interessa -- ou condução pelo Estado
de políticas de desenvolvimento de longo prazo.\footnote{\versal{SOUZA}, Jessé.
  \emph{A radiografia do golpe}. São Paulo: LeYa, 2016, p. 112.}

Em outra passagem, Jessé Souza revela, com agudeza, a razão da
configuração do corrupto como inimigo: ``Como o combate à desigualdade é
um valor universal, que não se pode atacar em público sem causar forte
reação, tem"-se que combater essa bandeira inatacável com outra bandeira
inatacável''.\footnote{\versal{SOUZA}, Jessé. \emph{A radiografia do golpe}. São
  Paulo: LeYa, 2016, p. 112.}

Por fim, assiste"-se a um fenômeno de maciça \emph{superação da
normatividade}, especialmente por parte do Poder Judiciário, o que, sem
sombra de dúvida, confere maior gravidade ao estado de exceção
brasileiro, porquanto se origina, fundamentalmente, do órgão que, em
tese, seria a última fronteira de defesa da ordem constitucional. Todo o
catálogo de direitos fundamentais é atingido -- individuais, sociais e
políticos --, em um acelerado \emph{processo desconstituinte.}\footnote{\versal{FERRAJOLI},
  Luigi. \emph{A democracia através dos direitos:} o constitucionalismo
  garantista como modelo teórico e como projeto político. São Paulo:
  Revista dos Tribunais, 2015, p. 162.}

A esta altura, cumpre"-nos perguntar se há alguma saída para a crise
estrutural\footnote{Merecem transcrição as palavras dos Professores Luiz
  Gonzaga Belluzzo e Gabriel Galípolo: ``Desconfiamos que o mundo não
  padeça apenas sofrimentos de uma crise periódica do capitalismo, mas,
  sim, as dores de um desarranjo nas práticas e princípios que sustentam
  a vida civilizada'' (\versal{BELLUZZO}, Luiz Gonzaga; \versal{GALÍPOLO}, Gabriel.
  \emph{Manda quem pode, obedece quem tem prejuízo}. São Paulo: Editora
  Contracorrente, 2017, p. 206).} que atravessa a sociedade brasileira.
Apesar do desalentador quadro atual e dos falaciosos discursos
deterministas que pregam o ``fim da história'', é imperioso construir um
projeto de resistência à racionalidade neoliberal.\footnote{\versal{SANTOS},
  Milton. \emph{Por uma outra globalização:} do pensamento único à
  consciência universal. 15\sout{ª} ed. São Paulo: Record, 2008, p. 159;
  \versal{AVELÃS NUNES}, António José. \emph{A crise atual do capitalismo:}
  capital financeiro, neoliberalismo, globalização. São Paulo, Revista
  dos Tribunais, 2012, p. 184.}

Sob o aspecto \emph{político}, impõe"-se recuperar o sentido da
\emph{política} como veículo de assimilação e resolução coletiva da
conflitividade social, em que o outro é visto como um \emph{semelhante}
e não como um \emph{inimigo}. Assim, pois, deve"-se substituir a lógica
da \emph{guerra}, própria da necropolítica neoliberal\footnote{\versal{MBEMBE},
  Achille. ``Necropolitics''. \emph{Public} \emph{Culture}, 2003, vol.
  15, n. 1, pp.~11-40.}, pela lógica da \emph{solidariedade}. No dizer
de Wendy Brown, ``in its barest form, this would be a vision in which
justice would not center on maximizing individual wealth or rights but
on developing and enhancing the capacity of citizens to share power and
hence to collaboratively govern themselves''. \footnote{\versal{BROWN}, Wendy.
  \emph{Edgework:} critical essays on knowledge and politics. Princeton:
  Princeton University Press, 2005, p. 58.}

Isto implica, inelutavelmente, uma radical transformação da relação hoje
existente entre economia e política. Aquela deve ser subalterna a esta,
ou, em outras palavras, a economia deve servir às pessoas e não o
contrário. Daí emergirão as condições para o enfrentamento da criminosa
desigualdade social que, em rigor, inviabiliza qualquer projeto de
sociedade democrática.

Malgrado a racionalidade neoliberal não se esgote na disciplina do
mercado, espraiando"-se para todos os domínios da vida social, parece"-nos
que, para confrontá"-la, é decisiva esta reconquista da economia pela
política.

Sob o ângulo \emph{jurídico}, é fundamental, de um lado,
\emph{descolonizar} o conhecimento jurídico, investindo a Ciência do
Direito, no léxico de Luigi Ferrajoli, de um papel \emph{crítico e
projetual}\footnote{\versal{FERRAJOLI}, Luigi. \emph{A democracia através dos
  direitos:} o constitucionalismo garantista como modelo teórico e como
  projeto político. São Paulo: Revista dos Tribunais, 2015, p. 162.}, em
que a \emph{descrição} do direito positivo seja acompanhada da
\emph{denúncia} dos desvios na aplicação normativa e da \emph{proposição
de estratégias} de colmatação das lacunas que impedem a plena realização
da Constituição.

Com isso, serão criadas as condições para \emph{criar} a confiança no
Direito. O povo, justificadamente, sempre desconfiou das leis, vendo
nelas um instrumento de dominação habilmente manejado pelas elites, por
isso se trata de \emph{criar} e não \emph{recuperar} a confiança no
Direito.\footnote{\versal{ZAFFARONI}, E. Raúl. \emph{El derecho latinoamericano
  en la fase superior del colonialismo.} Buenos Aires: Ediciones Madres
  de la Plaza de Mayo, 2016, p. 91.} É preciso levar o Direito a sério,
o que significa libertá"-lo dos grilhões da exceção e devolvê"-lo ao povo,
único titular da soberania.
