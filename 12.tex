\chapter*{Justiça de transição e usos políticos do Poder Judiciário
no Brasil em 2016: um golpe de Estado institucional?}

\addcontentsline{toc}{chapter}{Justiça de transição e usos políticos do Poder Judiciário
no Brasil em 2016: um golpe de Estado institucional?,\\
\scriptsize{por José Carlos Moreira da Silva Filho}}
\hedramarkboth{Justiça de transição e usos políticos do Poder\ldots{}}{}

\begin{flushright}
\emph{José Carlos Moreira da Silva Filho}\footnote{Professor da Escola
  de Direito da Pontifícia Universidade Católica do Rio Grande do Sul -
  \versal{PUC}/\versal{RS} (Programa de Pós"-Graduação em Ciências Criminais e Graduação em
  Direito); Ex"-Vice"-Presidente da Comissão de Anistia do Brasil;
  Bolsista Produtividade em Pesquisa do \versal{CNP}q.}
\end{flushright}

Em relação aos demais países da América Latina que amargaram ditaduras
civis"-militares de segurança nacional na segunda metade do século \versal{XX}, o
Brasil apresentou uma peculiaridade que acabou por influenciar
sobremaneira as características do regime democrático que se seguiu a
partir de 1988: a redemocratização guiou"-se sob o signo de uma anistia
ambígua, que representou tanto as lutas da sociedade civil pela abertura
do regime, como o empenho dos agentes da ditadura em garantir uma
transição que não os responsabilizasse pelos crimes que praticaram. Este
último aspecto encontrou solo fértil para prosperar, visto que ao longo
de todo o período ditatorial houve um amplo e intenso processo de
judicialização da repressão política, o que certamente cultivou no poder
judiciário brasileiro uma grande resistência em revisar os termos dessa
anistia, mesmo em período democrático. Argumenta"-se neste artigo que o
ambiente criado a partir do caráter ambíguo da anistia, em especial
considerando a atuação do poder judiciário, contribuiu para a ruptura da
democracia ocorrida no Brasil em 2016.

Na primeira seção analisa"-se a ambiguidade do processo de anistia e
redemocratização do Brasil. Na segunda seção o foco é o papel do
judiciário tanto na judicialização da repressão durante a ditadura
quanto no processo de anistia. Na terceira seção é fornecida uma breve
caracterização da ruptura institucional ocorrida no Brasil em 2016 e ao
final, em tom conclusivo identificam"-se relações entre os processos
mencionados.

\section{A Ambiguidade da Anistia no Brasil}

No dia 28 de agosto de 1979, em plena ditadura, foi promulgada a lei de
anistia no Brasil, a Lei \versal{N}º6.683. Esta lei reflete uma acentuada
ambiguidade, e que se transmite ao próprio sentido da palavra ``anistia''
no contexto político brasileiro.

De um lado, a lei foi o resultado de uma ampla mobilização social em
torno da pauta da anistia aos que estavam presos, no exílio ou na
clandestinidade, acusados de terem praticado crimes políticos. A demanda
pela anistia representou a demanda pela redemocratização do
país\footnote{No ano de 1975 é desencadeada a campanha pela Anistia, com
  o lançamento do Manifesto da Mulher Brasileira pelo Movimento Feminino
  pela Anistia (\versal{MFPA}). Este movimento começa forte em São Paulo,
  conduzido por D. Terezinha Zerbini, e, dali, espalha"-se por
  todo o país. Era o ano internacional da mulher e foi principalmente
  pelo protagonismo das mulheres, esposas de maridos desaparecidos,
  presos ou foragidos, irmãs, amigas, militantes, que se deu início a
  uma das mais intensas movimentações políticas da sociedade civil
  brasileira. Sobre a movimentação popular em prol da anistia na segunda
  metade da década de 70 ver o aprofundado e detalhado estudo de Carla
  Rodeghero, Gabriel Dienstmann e Tatiana Trindade: \versal{RODEGHERO}, Carla
  Simone; \versal{DIENSTMANN}, Gabriel; \versal{TRINDADE}, Tatiana. \emph{Anistia ampla,
  geral e irrestrita: história de uma luta inconclusa}. Santa Cruz do
  Sul: \versal{EDUNISC}, 2011. Também importa conferir a tese de Heloísa Greco:
  \versal{GRECO}, Heloísa Amélia. \emph{Dimensões fundacionais da luta pela
  Anistia}. 2003. 456f. {[}Tese de Doutorado{]} -- Curso de
  Pós"-Graduação das Faculdades de Filosofia e Ciências Humanas da
  Universidade Federal de Minas Gerais. Belo Horizonte. 2003.}. O largo
contingente de setores da sociedade que conseguiu mobilizar
(trabalhadores, artistas, intelectuais, políticos, imprensa, igreja,
presos políticos, entre outros) constituiu a base sobre a qual mais
tarde viriam as mobilizações pelas Diretas Já em 1984 e a participação
no processo Constituinte em 1987 e 1988.

Por outro lado, a lei representou uma vitória para o projeto de
transição controlada idealizado pela cúpula do regime
ditatorial\footnote{A anistia fez parte de um projeto cuidadosamente
  delineado por estrategistas do regime, comandados pelo arquiteto
  intelectual da ditadura, o General Golbery do Couto e Silva. Fazia
  parte desse plano o esfacelamento das forças políticas de oposição,
  que àquela altura, apesar de todos os esforços dos governos militares
  em sentido contrário, haviam se agrupado em torno do partido de
  oposição consentida, o \versal{MDB} (\versal{ALVES}, Maria Helena Moreira.
  \emph{Estado e oposição no Brasil (1964-1984)}. 3.ed. Petrópolis:
  Vozes, 1984. p.269-270; \versal{SKIDMORE}, Thomas. \emph{Brasil: de Castelo
  a Tancredo}. 8.ed. Rio de Janeiro: Paz e Terra, 1988. p.427-428).}, já
que conseguiu o feito de anistiar os agentes da ditadura, impedindo
qualquer investigação sobre os seus crimes, sem sequer afirmar que tais
agentes teriam praticado assassinato, tortura, desaparecimento forçado e
outras graves violações de direitos humanos\footnote{No Art. 1º da Lei
  \versal{N}º 6683 de 1979, se afirma que estão anistiados os ``crimes políticos
  ou conexos com estes'', em seguida, o § 1º define que ``consideram"-se
      conexos, para efeito deste artigo, os crimes de qualquer natureza
      relacionados com crimes políticos ou praticados por motivação
      política'', e o § 2º retira os benefícios da anistia para ``os que foram
              condenados pela prática de crimes de terrorismo, assalto, sequestro e
              atentado pessoal''.}. Do mesmo modo, excluiu a anistia para os presos
políticos que estavam condenados por terem tomado parte na resistência
armada. E, por fim, a promulgação da lei foi apresentada como uma
benesse ofertada pelo governo militar sem que se promovesse o
reconhecimento da ampla participação popular neste processo.

A redemocratização do país foi balizada pelo que a lei de anistia
representou. O aspecto emancipatório e popular da luta pela anistia
desaguou na ampla participação da sociedade civil no processo
constituinte nos anos de 1987 e 1988 e na característica avançada da lei
em termos de princípios e reconhecimento de direitos
fundamentais\footnote{Sobre a mobilização dos movimentos sociais em
  torno da Constituinte como um legado do enfrentamento com a ditadura
  ver: \versal{SOUSA} \versal{JUNIOR}, José Geraldo de. ``Soberania e Direitos: processos
  sociais novos?'' \emph{Caderno \versal{CEAC}/UnB}, Ano 1, \versal{N}º 1, 1987. p.9-16.
  {[}Constituinte: temas em análise{]}; \versal{SOUSA} \versal{JUNIOR}, José Geraldo de.
  ``Triste do Poder que não pode''. \emph{Caderno \versal{CEAC}/UnB}, Ano 1, \versal{N}º 1,
  1987. p.25-31. {[}Constituinte: temas em análise{]}; \versal{COELHO}, João
  Gilberto Lucas. ``A Garantia das Instituições''. \emph{Caderno
  \versal{CEAC}/UnB}, Ano 1, \versal{N}º 1, 1987. p.37-45. {[}Constituinte: temas em
  análise{]}; \versal{BARBOSA}, Leonardo Augusto de Andrade. ``O legado do processo
  Constituinte''. In: \versal{SOUSA} \versal{JUNIOR},
  José Geraldo de; \versal{SILVA} \versal{FILHO}, José
  Carlos Moreira da; \versal{PAIXÃO}, Cristiano; \versal{FONSECA}, Lívia Gimenes Dias da;
  \versal{RAMPIN}, Talita Tatiana Dias (Orgs.). \emph{O Direito Achado na Rua:
  Introdução Crítica à Justiça de Transição na América Latina}. Brasília:
  UnB, 2015. p.51-54. (Livro todo disponível em
  \textless{}\emph{https://bit.ly/2N17gtt}\textgreater{})
  (Acesso em 27/10/2017).}. Já o aspecto autoritário e reacionário
da anistia refletiu"-se no esquecimento institucional dos crimes contra a
humanidade praticados e sua necessária responsabilização. Tal bloqueio,
devidamente afirmado pelo Poder Judiciário em todas as tentativas que
foram feitas de investigar e responsabilizar esses crimes\footnote{Especificamente
  sobre este ponto ver: \versal{SILVA} \versal{FILHO}, José Carlos Moreira da.
  \emph{Justiça de Transição -- da ditadura civil"-militar ao debate
  justransicional: Direito à memória e à verdade e os caminhos da
  reparação e da anistia no Brasil}. Porto Alegre: Livraria do Advogado,
  2015. p.237-260.}, também favoreceu a ausência de reformas
institucionais que buscassem esclarecer a participação dos poderes
constituídos no regime ditatorial, bem como de processos de
responsabilização administrativa e judicial sobre os agentes e
funcionários públicos que facilitaram ou praticaram diretamente tais
crimes. Em outras palavras, militares, policiais, juízes, promotores,
políticos e demais funcionários públicos que participaram ativamente do
processo de perseguição política aos opositores do regime ditatorial
continuaram nos seus postos de trabalho como se nada houvesse
acontecido.

A Constituinte foi instalada em 1987 a partir de uma emenda
constitucional produzida na ordem jurídica autoritária, uma emenda à
Constituição outorgada de 1967, a Emenda Constitucional \versal{N}º 26/1985.
Nesta mesma emenda a lei de anistia de 1979 foi reafirmada\footnote{A \versal{EC}
  \versal{N}º 26/1985 repete a lei \versal{N}º 6.683/1979 quanto ao lapso temporal da
  anistia, e repete a anistia a crimes políticos ou conexos, contudo não
  reproduz a definição do que seriam crimes conexos nem a exclusão dos
  chamados ``crimes de sangue'' do alcance da anistia. A emenda ainda
  amplia as situações de recomposição para incluir estudantes,
  dirigentes sindicais, servidores e empregados civis.}, como que para
sugerir que a nova Constituição a ser criada não pudesse rever os seus
termos.

A despeito dessa peculiaridade, o texto da nova Constituição não
reproduz mais a anistia aos crimes conexos. Além disso, em seu Art.~8º
do Ato das Disposições Constitucionais Transitórias, o constituinte
firmou, com clareza inequívoca, que a anistia era devida aos que ``foram
atingidos, em decorrência de motivação exclusivamente política, por atos
de exceção, institucionais ou complementares''. Assim, a anistia aos
agentes da ditadura não foi recebida pelo texto constitucional de 1988.
Por outro lado, também não foi expressamente repudiada. De todo modo, ao
não mencionar o tema e ao assinalar o forte repúdio à tortura,
considerada crime inafiançável e insuscetível de graça ou
anistia\footnote{No Art. 5º, \versal{XLIII} a Constituição brasileira estabelece
  esta condição, complementada pela Lei 9.455/97. Importa mencionar,
  além disso, o Art. 5º, §4º que reconhece a submissão do Brasil ao
  Tribunal Penal Internacional. O Tratado de Roma penetra a ordem
  jurídica interna brasileira por força do Decreto Legislativo \versal{N}º
  4.388/2002, estabelecendo explicitamente que a tortura praticada de
  forma sistemática a parcelas da população civil, ou seja, como prática
  de um crime contra a humanidade, é imprescritível. Por fim, a
  Constituição demarca no Art. 5º, \versal{XLIV} que ``constitui crime
      inafiançável e imprescritível a ação de grupos armados, civis ou
      militares, contra a ordem constitucional e o Estado Democrático''. Como
  se sabe, foi exatamente isto que fizeram os militares e civis
  golpistas em 1964.}, a partir dos seus princípios e direitos
fundamentais, a Constituição revela"-se um local muito pouco confortável
para abrigar a anistia aos crimes conexos entendida como a anistia aos
crimes dos agentes da ditadura. Há uma evidente contradição
principiológica e valorativa no argumento de que a Constituição
brasileira de 1988 endossa a anistia a tais crimes.

Além de excluir da sua apreciação a anistia aos crimes da ditadura, o
Artigo 8º do Ato das Disposições Constitucionais Transitórias lançou as
bases de uma verdadeira política de reparação aos ex"-perseguidos
políticos. O termo ``anistia'', mesmo na legislação produzida pela
ditadura sempre trouxe alusão igualmente a algum sentido de reparação e
de restituição do status anterior à perseguição política. Porém, como
era de se esperar naquele ambiente ainda mutilado politicamente,
contaminado pelo esquecimento forçado e seguido de perto pelo
autoritarismo, a lei regulamentadora dessa política de reparação
sinalizada pelo texto constitucional só viria à luz cerca de 13 anos
depois, mais precisamente em 2001, via Medida Provisória depois
convertida na Lei \versal{N}º 10.559/2002.

A nova lei de anistia, além de prever direitos como a declaração de
anistiado político, a reparação econômica, a contagem do tempo e a
continuação de curso superior interrompido ou reconhecimento de diploma
obtido no exterior\footnote{De acordo com o Art.1° da Lei \versal{N}º
  10.559/2002.}, institui a Comissão de Anistia, vinculada ao Ministério
da Justiça, e que fica responsável pela apreciação e julgamento dos
requerimentos de anistia\footnote{De acordo com a lei, os Conselheiros e
  Conselheiras são escolhidos e nomeados pelo Ministro da Justiça, e
  liderados pelo Presidente da Comissão de Anistia, também escolhido
  pelo Ministro. Até 2016 sempre foi prática do Estado brasileiro, para
  fins da escolha dos membros da Comissão de Anistia, a consulta aos
  grupos e movimentos da sociedade civil em torno da pauta de verdade,
  memória e justiça em particular e de direitos humanos em geral. Dos
  membros da Comissão um necessariamente representa o Ministério da
  Defesa e outro representa os anistiandos. Os Conselheiros não recebem
  pagamento pelo seu trabalho, considerado, de acordo com a lei, de
  relevante interesse público. O conselho funciona como um tribunal
  administrativo, mas a responsabilidade final da decisão é do Ministro
  da Justiça, completando"-se o processo de anistia apenas após a
  assinatura e publicação da Portaria Ministerial.}. A Comissão de
Anistia é, na verdade, uma comissão de reparação, mas que carrega
consigo a própria ambiguidade do termo ``anistia'', forjada no processo de
redemocratização do país.

Observando a atuação da Comissão de Anistia, desde a sua criação, e,
especialmente, durante o segundo mandato do Presidente Lula, iniciado em
2007, quando o Ministério da Justiça foi conduzido por Tarso Genro e a
presidência da Comissão de Anistia por Paulo Abrão, percebe"-se uma
radical mudança na concepção da anistia como política de esquecimento.
Em primeiro lugar, ao exigir a verificação e comprovação da perseguição
política sofrida\footnote{Em seu art. 2º, a Lei \versal{N}º 10.559/2002 prevê ao
  todo 17 situações de perseguição por motivação exclusivamente política
  que justificam o reconhecimento da condição de anistiado político e os
  direitos dela decorrentes. Aqui estão prisões, perda de emprego, ser
  compelido ao exílio, ser atingido por atos institucionais, entre
  outras situações.}, a lei de anistia acaba suscitando a apresentação
de documentos e narrativas que trazem de volta do esquecimento os fatos
que haviam sido desprezados pela anistia de 1979. Passa a ser condição
para a anistia a comprovação e detalhamento das violências sofridas
pelos perseguidos políticos, circunstância que por si só associa anistia
à memória.

Nas sessões de julgamento da Comissão de Anistia, os requerentes que
estão presentes são convidados a se manifestarem, proporcionando em
muitos casos importantes testemunhos, que são devidamente registrados.
Os autos dos processos contêm uma narrativa muito diferente daquela que
está registrada nos arquivos oficiais. Os processos da Comissão de
Anistia fornecem a versão daqueles que foram perseguidos políticos pela
ditadura civil"-militar, contrastando com a visão normalmente pejorativa
que sobre eles recai a partir dos documentos produzidos pelos órgãos de
informação do período.

Para além da reparação econômica, a Comissão de Anistia também é
conhecida internacionalmente por ter empreendido de maneira inovadora e
sensível políticas públicas de memória e projetos vanguardistas como as
Caravanas da Anistia\footnote{Até agosto de 2016 foram quase 100
  Caravanas realizadas por todo o Brasil. Nelas, a Comissão se desaloja
  das instalações do Palácio da Justiça em Brasília e percorre os
  diferentes Estados brasileiros para julgar requerimentos de anistia
  emblemáticos nos locais onde as perseguições aconteceram, realizando
  os julgamentos em ambientes educativos como Universidades e espaços
  públicos e comunitários. O momento alto das Caravanas e de todas as
  sessões de apreciação de requerimentos de anistia é o pedido formal de
  desculpas em nome do Estado brasileiro aos que por ele foram
  perseguidos no passado. Longe de tal pedido significar que o Estado já
  fez todo o possível para reparar a perseguição que promoveu no
  passado, ele sinaliza para o necessário aprofundamento célere do
  processo justransicional brasileiro, incluindo"-se aí as medidas de
  responsabilização e de reforma das instituições, que sempre foram duas
  bandeiras da Comissão de Anistia em inúmeras ocasiões (o que ficou
  registrado, por exemplo, em artigo de opinião escrito pelo então
  Vice"-Presidente da Comissão de Anistia e publicado no jornal Zero Hora
  em 16 de abril de 2014. Ver:
  \textless{}\emph{https://bit.ly/2tgQ6iH}\textgreater{}). Para um registro das
  Caravanas da Anistia, ver a obra editada e publicada pela Comissão de
  Anistia intitulada: \versal{COELHO}, Maria José H.; \versal{ROTTA}, Vera (orgs.).
  \emph{Caravanas da Anistia: o Brasil pede perdão}. Brasília:
  Ministério da Justiça; Florianópolis: Comunicação, Estudos e
  Consultoria, 2012. Disponível em:
  \textless{}\emph{https://bit.ly/2N3dklo}\textgreater{}.
  (Acesso em 27/10/2017).}, as Clínicas do Testemunho\footnote{As
  Clínicas do Testemunho são um projeto inédito de assistência
  psicológica e psicanalítica às vítimas da violência do Estado
  ditatorial, e que já possui quase três anos de existência. O projeto
  Clínicas do Testemunho iniciou"-se no ano de 2013 a partir de Edital
  Público publicado em 2012 e vinculado à Comissão de Anistia. Seu
  objetivo é propiciar atendimento psicanalítico às vítimas da repressão
  estatal promovida pela ditadura civil"-militar no Brasil. Em sua
  primeira edição o projeto contemplou duas iniciativas na cidade de São
  Paulo, uma no Rio de Janeiro e outra em Porto Alegre. Na segunda
  edição (2015 a 2017) o projeto foi ampliado graças à participação do
  Fundo Newton (que para aportar recursos tem como condição o aporte no
  mesmo valor por parte do Estado brasileiro), passando a contemplar
  também a cidade de Florianópolis. Entre as diversas ações já
  produzidas, além dos atendimentos, estão eventos e publicações. Eis a
  referência das publicações e os links para as respectivas versões
  digitais: \versal{SIGMUND} \versal{FREUD} \versal{ASSOCIA}ÇÃO \versal{PSICANALÍTICA} (Org.).
  \textbf{Clínicas do Testemunho}: reparação psíquica e construção de
  memórias. Porto Alegre: Criação Humana, 2014; \versal{SILVA} \versal{JR}, Moisés
  Rodrigues da; \versal{MERCADANTE}, Issa. \textbf{Travessia do silêncio,
  testemunho e reparação}. Brasília: Comissão de Anistia; São Paulo:
  Instituto Projetos Terapêuticos, 2015; \versal{CARDOSO}, Cristiane; \versal{FELIPPE},
  Marilia; \versal{BRASIL}, Vera Vital (Orgs.). \textbf{Uma perspectiva
  clínico"-política na reparação simbólica:} Clínica do Testemunho no Rio
  de Janeiro. Brasília: Comissão de Anistia; Rio de Janeiro: Instituto
  Projetos Terapêuticos, 2015; \versal{OCARIZ}, Maria Cristina (Org.).
  \textbf{Violência de estado na ditadura civil"-militar brasileira
  (1964-1985)} -- efeitos psíquicos e testemunhos clínicos. São Paulo:
  Escuta, 2015;
  http://www.justica.gov.br/central-de-conteudo/anistia/anexos/livro-on-line-2.pdf;
  http://www.justica.gov.br/central-de-conteudo/anistia/anexos/travessia\_final.pdf;
  http://www.justica.gov.br/central-de-conteudo/anistia/anexos/sedes-violencia-de-estado-2.pdf;
  http://www.justica.gov.br/central-de-conteudo/anistia/anexos/livro-clinicas-do-testemunho.pdf;
  http://www.justica.gov.br/central-de-conteudo/anistia/arquivos-da-vo-alda.pdf.
  (Acesso em 27/10/2017).}, o Projeto Marcas da Memória\footnote{O
  Marcas da Memória é na verdade um amplo guarda"-chuva no qual se
  abrigam políticas de memória diversas. Em seu bojo inserem"-se todas as
  iniciativas da Comissão de Anistia aqui mencionadas, devendo"-se ainda
  acrescentar o aporte de recursos para sustentar e promover iniciativas
  da sociedade civil em prol da memória política do país. Foram dezenas
  de filmes, publicações, peças de teatro e eventos culturais já
  apoiados . A primeira chamada ocorreu no ano de 2010, e eis aqui o
  link para conhecer alguns dos resultados iniciais:
  http://www.justica.gov.br/seus-direitos/anistia/projetos/marcas-da-memoria-i-2010
  . Para maior detalhamento do Projeto Marcas da Memória ver o artigo de
  Roberta Baggio, intitulado ``Marcas da Memória: a atuação da Comissão
      de Anistia no campo das políticas públicas de transição no Brasil'',
  disponível em:
  file:///Users/josecarlosmoreiradasilvafilho/Downloads/2924-10625-1-PB.pdf
  (Acesso em 27/10/2017).}, e por ter iniciado a construção do Memorial
da Anistia\footnote{O projeto Memorial da Anistia Política é fruto de um
  convênio entre o Ministério da Justiça e a Universidade Federal de
  Minas Gerais. O projeto prevê a construção na cidade de Belo
  Horizonte"-\versal{MG} de um espaço de exposição permanente localizado no antigo
  coleginho da \versal{FAFICH}, local histórico de organização da resistência à
  ditadura, de um parque e de um prédio novo que abrigará o acervo da
  Comissão de Anistia e um centro de pesquisas para o público e de
  produção de pesquisas no campo da memória política brasileira e da
  Justiça de Transição. O projeto prevê ainda a constituição de uma Rede
  Latino"-Americana de pesquisa sobre Justiça de Transição (já criada e
  que vem produzindo relatórios e eventos internacionais -- ver o site da
  rede: http://rlajt.com ). O projeto museológico, já pronto e no
  período em que se escreve este artigo ainda dependente da entrada dos
  recursos faltantes para ser inaugurado, pode ser conhecido neste
  vídeo: https://www.youtube.com/watch?v=65IXBY98ggc . (Acesso em
  27/10/2017).}, realizado eventos e intercâmbios acadêmicos e
culturais, além de inúmeras publicações que aprofundam o sentido da
Justiça de Transição no Brasil e na América Latina\footnote{Grande parte
  das publicações promovidas pela Comissão de Anistia encontra"-se no
  site:
  http://www.justica.gov.br/central-de-conteudo/anistia/anistia-politica-2
  .}. Estes programas e projetos compunham até 2016 o Programa
Brasileiro de Reparação Integral, reconhecido e celebrado
internacionalmente, e faziam parte do rol dos direitos de todos aqueles
que foram atingidos por atos de exceção durante a ditadura civil"-militar
e aos seus familiares.

Ao longo desses anos de existência e atuação da Comissão de Anistia é
possível identificar outros órgãos e comissões de Estado que reforçaram
e seguiram o mesmo sentido de resgate da memória política da ditadura a
partir da visão das vítimas, dentre os quais destacam"-se em especial a
Comissão Especial de Mortos e Desaparecidos Políticos, criada em 1995
ainda no governo de Fernando Henrique Cardoso, e a Comissão Nacional da
Verdade, criada em 2011 e instalada em 2012, em meio ao primeiro mandato
da Presidenta Dilma Roussef.

\section{Judiciário Brasileiro entre o autoritarismo e o ativismo}

Em seu livro ``Ditadura e Repressão'', no qual promove um estudo
comparado sobre a judicialização da repressão na Argentina, no Chile e
no Brasil, Anthony Pereira identifica um curioso paradoxo no caso
brasileiro\footnote{\versal{PEREIRA}, Anthony W. \textbf{Ditadura e repressão}: o
  autoritarismo e o estado de direito no Brasil, no Chile e na
  Argentina. São Paulo: Paz e Terra, 2010.}. De todos os três países, o
Brasil foi aquele que mais se aprofundou na judicialização da repressão
ditatorial e que construiu uma legalidade autoritária mais ampla,
arraigada e vinculada à ordem jurídica anterior. Tal se deve, entre
outros fatores, ao alto grau de coesão entre as elites judiciais e as
forças armadas\footnote{Em sua tese de doutorado, Vanessa Shinke
  evidencia esse alto nível de coesão institucional entre o Poder
  Judiciário e as Forças Armadas. \versal{SCHINKE}, Vanessa Dorneles.
  \textbf{Judiciário e autoritarismo}: regime autoritário (1964-1985),
  democracia e permanências. Rio de Janeiro: Lumen Juris, 2016.}, o que
levou estas últimas à opinião de que o judiciário era ``confiável'', e
que portanto, os tribunais poderiam se prestar ao papel de intermediário
entre a ação repressiva direta dos agentes de segurança pública e
aqueles que eram perseguidos políticos, tidos no contexto da ditadura
como criminosos e terroristas.

Se por um lado os milhares de julgamentos ocorridos na ditadura
brasileira faziam vistas grossas em relação às denúncias de tortura e
compactuavam com leis draconianas, como eram os Atos Institucionais e
seus derivados, contando com juízes que defendiam e incorporavam a
ideologia do regime, por outro, tais julgamentos contavam com um arsenal
razoável de garantias e procedimentos e permitiam em grande parte dos
casos evitar que os opositores políticos fossem simplesmente eliminados.
Em sua pesquisa, Anthony Pereira notou também que no Brasil os advogados
de defesa de presos políticos possuíam uma relativa liberdade e
autonomia para atuar nas cortes políticas e conseguiram, por vezes,
induzir os juízes a interpretarem a legislação autoritária de uma
maneira mais benigna para os seus clientes\footnote{Sobre este ponto
  específico, ver ainda: \versal{SPIELER}, Paula; \versal{QUEIROZ}, Rafael Mafei Rabelo
  (Coords.). \textbf{Advocacia em tempos difíceis} -- ditadura militar
  1964-1985. Curitiba: edição do autor, 2013. Disponível em:
  https://www.justica.gov.br/central-de-conteudo/anistia/anexos/advocacia-em-tempos-dificeis\_baixa-resolucao.pdf
  (Acesso em 27/10/2017).}.

Na Argentina, a ausência de uma coesão entre os militares e a elite
judicial levou os militares a considerarem o judiciário pouco ou de modo
algum ``confiável''. Não havia, portanto, mediadores institucionais
entre a violência direta dos agentes da repressão e os seus alvos. A
estratégia adotada foi claramente a da eliminação e do desaparecimento
em massa dos opositores políticos. Contudo, se a forte coesão
institucional ocorrida na ditadura civil"-militar brasileira e a sua
máscara de legalidade foram um dos fatores responsáveis por uma cifra
menor de mortos e desaparecidos do que em relação à Argentina, elas
contribuíram para manter mais arraigada no Brasil a continuidade da
herança autoritária no período pós"-ditatorial. Após a ditadura
brasileira, nenhum juiz, por mais conivente que fosse com o regime,
nenhum policial, por mais que tenha torturado e assassinado opositores,
nenhum político ou dirigente, por mais que tenha aprovado, ordenado ou
tenha sido conivente com a tortura, foi demitido, exonerado ou
responsabilizado pelos seus atos. Muitos deles simplesmente continuaram
a atuar no Poder Público, transferindo agora o foco da sua impunidade
para os criminosos comuns e os suspeitos de o serem, que continuaram a
ser barbaramente torturados nas delegacias e nos presídios\footnote{Este
  fato, notório e registrado em diferentes estudos e levantamentos, é
  palpável, por exemplo, no Relatório da Anistia Internacional lançado
  em fevereiro de 2017 (ver páginas 82 a 87, que tratam do Brasil -
  disponível em:
  https://anistia.org.br/wp-content/uploads/2017/02/AIR2017\_ONLINE-v.3.pdf)
  (Acesso em 27/10/2017).}.

Com relação ao tema da anistia e da responsabilização dos agentes da
ditadura, o judiciário brasileiro sempre foi reticente. No conhecido
caso das mãos amarradas, no qual o sargento Manoel Raymundo Soares foi
morto por agentes da ditadura por afogamento e encontrado boiando com as
mãos amarradas no Rio Jacuí em 1966\footnote{Para mais informações sobre
  o caso ver, entre outros: \versal{ASSUMPÇÃO}, Eliane Maria Salgado (org.).
  \textbf{O Direito na História:} o caso das mãos amarradas. Porto
  Alegre: \versal{TRF} 4a. Região, 2008; e o próprio relatório da Comissão
  Especial de Mortos e Desaparecidos Políticos: \versal{BRASIL}. Secretaria
  Especial dos Direitos Humanos. Comissão Especial sobre Mortos e
  Desaparecidos Políticos. \textbf{Direito à verdade e à memória}.
  Brasília: Secretaria Especial dos Direitos Humanos, 2007. p.75-77.}, a
provocação ao Poder Judiciário foi vã.

Após a Constituição de 1988, houve a tentativa do Ministério Público de
São Paulo de abrir um inquérito civil para apurar, em 1992, a morte do
jornalista Vladimir Herzog e a tentativa de reabrir a investigação do
caso Riocentro\footnote{O caso Riocentro diz respeito a um atentado a
  bomba frustrado ocorrido na noite do dia 30 de abril de 1981 contra
  uma casa de espetáculos situada na cidade do Rio de Janeiro na qual se
  comemoraria o dia do trabalho. A bomba explodiu no colo de dois
  militares, vitimando um e ferindo o outro, antes que pudesse ser
  instalada no local dos shows. A intenção era atribuir o atentado a
  grupos de luta armada, àquela altura já eliminados, para com isto
  buscar justificar a continuidade do fechamento do regime político.
  Este caso ainda foi judicializado uma vez mais no ano de 2014 por
  iniciativa do Ministério Público Federal. A juíza federal da 6a Vara
  Federal Criminal do Rio de Janeiro aceitou a denúncia, mas os réus
  obtiveram o trancamento da ação no Tribunal Regional Federal da 2a
  Região, com o pretexto da prescrição, não mais da anistia. O \versal{MPF}
  recorreu e a questão segue pendente.}, em 1996, no Superior Tribunal
Militar. Em ambos os casos houve o indeferimento dos pleitos pela mesma
razão: incidência da anistia ``bilateral'' de 1979\footnote{A decisão de
  trancar o inquérito policial do caso Herzog veio da Quarta câmara do
  Tribunal de Justiça de São Paulo (\versal{SÃO} \versal{PAULO}. Tribunal de Justiça.
  \textbf{Habeas Corpus \versal{N}º 131.798-3/2}. Relator Péricles Piza) e foi
  mantida pelo Superior Tribunal de Justiça (\versal{SUPERIOR TRIBUNAL DE
  JUSTIÇA}. \textbf{Recurso Especial \versal{N}º 33.782-7-\versal{SP}}, j.18/08/1993, 5a
  Turma, unânime, Relator Ministro José Dantas). Já a decisão de trancar
  as investigações do caso Riocentro com base na Lei de Anistia foi
  tomada pelo Superior Tribunal Militar em 1988, quando declarou de
  ofício a extinção da punibilidade dos autores (Representação \versal{N}º
  1.067-7/\versal{DF}) e quando negou em 1996 novo pedido de abertura da
  investigação (Representação Criminal \versal{N}º 4-0/\versal{DF}). Maiores detalhes
  sobre ambos os casos podem ser vistos em: \versal{SANTOS}, Roberto Lima; \versal{BREGA}
  \versal{FILHO}, Vladimir. Os reflexos da ``judicialização'' da repressão política
  no Brasil no seu engajamento com os postulados da justiça de
  transição. In: \textbf{Revista Anistia Política e Justiça de
  Transição}. Brasília, \versal{N}º 1, p.152-177, jan./jun. 2009. No momento em
  que se escreve este artigo, o caso Herzog encontra"-se judicializado
  pela Corte Interamericana de Direitos Humanos, já tendo ocorrido a
  audiência junto à Corte no dia 24 de maio de 2017.}. O curioso é que,
no segundo caso, referente ao atentado ocorrido em 1981 no Riocentro,
mesmo reconhecendo indícios de autoria de militares no crime, os
Ministros do \versal{STM} -- agindo em desacordo com a própria Lei \versal{N}º 6.683/1979
-- justificaram o arquivamento do procedimento pela incidência da
anistia a crimes cometidos após 1979. A construção de uma ``anistia para
frente'' representou um verdadeiro estelionato jurídico que contribuiu
para fortalecer a noção de que -- no Brasil -- não haveria
responsabilização dos agentes do estado de exceção: como pensar em punir
os crimes de tortura, sequestro e homicídio ocorridos antes de 1979 se
sobre aqueles que ocorreram depois (como o atentado ao Riocentro) também
incidia -- legitimamente, conforme o poder Judiciário -- a malfadada
causa de extinção da punibilidade?

Assim, seja antes, seja depois do estabelecimento da ordem democrática
pela Constituição de 1988 a tentativa de se construir o pilar da
``responsabilização'' no processo transicional brasileiro sempre esteve
presente como reivindicação dos que sofreram com os atos de exceção. No
entanto, como se constatou, os termos da interpretação dada ao instituto
da anistia impediram qualquer análise de mérito que viabilizasse alguma
providência no sentido da investigação e da responsabilização. Somente
após a virada do século, a partir de 2008, é que houve uma nova
mobilização, por parte de organismos da sociedade civil e de órgãos
vinculados ao Estado, que buscou questionar a validade da interpretação
da anistia como ``acordo bilateral'' perante o Supremo Tribunal
Federal\footnote{No dia 31 de julho de 2008 a Comissão de Anistia
  organizou uma audiência pública no prédio sede do Ministério da
  Justiça em Brasília para discutir as possibilidades jurídicas de
  julgamento dos torturadores que atuaram em prol do governo ditatorial.
  A reação da imprensa foi imediata e incessante, e, apesar da tentativa
  inicial de desqualificar o debate, pautou o tema com elevada
  frequência em jornais, revistas e outros meios de massa. Artigos a
  favor e contra a possibilidade do julgamento eram publicados e não
  paravam de surgir nas páginas dos principais jornais do país. Até
  então este parecia um assunto proibido (sobre a cobertura midiática
  feita sobre o tema ver \versal{SILVA} \versal{FILHO}, José Carlos Moreira da. A Comissão
  de Anistia e a concretização da justiça de transição no Brasil -
  repercussão na mídia impressa brasileira: jornal O Globo, 2001 a
  2010.~\versal{SILVA} \versal{FILHO}, José Carlos Moreira da; \versal{ABRÃO}, Paulo; \versal{TORELLY},
  Marcelo D. {[}orgs{]}. \textbf{Justiça de Transição nas Américas} -
  olhares interdisciplinares, fundamentos e padrões de efetivação. Belo
  Horizonte: Fórum, 2013. p.181-226).

  O então Presidente do Conselho Federal da \versal{OAB}, Cezar Britto,
  compareceu à audiência e meses depois, sob a influência da discussão,
  mobilizou o Conselho e propôs, com a assinatura de Fábio Konder
  Comparato, a Argüição de Descumprimento de Preceito Fundamental \versal{N}º 153
  no \versal{STF}.

  Importante também mencionar a corajosa e importante sentença do juiz
  Gustavo Santini Teodoro, de outubro de 2008, confirmada pelo Tribunal
  de Justiça paulista em agosto de 2012, e que, embora só tenha efeitos
  declarativos, foi a primeira (e única) manifestação judicial que
  reconheceu explicitamente um ex"-agente público brasileiro como
  torturador: o Coronel Carlos Alberto Brilhante Ustra, apontado em
  dezenas de relatos de ex"-perseguidos como torturador e que foi
  comandante da temida Operação Bandeirante em São Paulo na década de
  70.

  Esquentando ainda mais o ambiente para o julgamento da \versal{ADPF} \versal{N}º 153 no
  \versal{STF}, em janeiro de 2010 a Secretaria Especial de Direitos Humanos
  lança o \versal{III} Plano Nacional de Direitos Humanos, inaugurando uma
  Diretriz inexistente nos planos anteriores, aquela que cuida do
  Direito à Memória e à Verdade. Entre outras deliberações, o Plano
  propugnou a instituição de uma Comissão Nacional da Verdade, que veio
  a ser constituída em 2012, e uma série de outras políticas públicas em
  torno da memória, dano espaço para as opiniões desfavoráveis ao
  bloqueio da Lei de anistia quanto à investigação e responsabilização
  dos crimes de lesa humanidade praticados pelos agentes da ditadura.}.

Tal questionamento foi feito através da Ação de Descumprimento de
Preceito Fundamental (\versal{ADPF}) \versal{N}º 153, julgada em abril de 2010 em dois
dias de sessão e cujo resultado foi o de sete a dois pelo indeferimento,
com votos que trouxeram fundamentos bastante questionáveis, inclusive
sob o ponto de vista histórico\footnote{Para a crítica da decisão do \versal{STF}
  e seus fundamentos ver: \versal{MEYER}, Emilio Peluso Neder. \versal{CATTONI}, Marcelo.
  Anistia, história constitucional e direitos humanos: o Brasil entre o
  Supremo Tribunal Federal e a Corte Interamericana de Direitos Humanos.
  In \versal{CATTONI}, Marcelo (org.).\textbf{Constitucionalismo e História do
  Direito}. Belo Horizonte: Pergamum, 2011, p. 249-288. \versal{MEYER}, Emilio
  Peluso Neder.~\textbf{Ditadura e Responsabilização~}- elementos para
  uma justiça de transição no Brasil. Belo Horizonte: Arraes, 2012;
  \versal{TORELLY}, Marcelo D.~\textbf{Justiça de Transição e Estado
  Constitucional de Direito}~- perspectiva teórico"-comparativa e análise
  do caso brasileiro. Belo Horizonte: Fórum, 2012; \versal{SILVA} \versal{FILHO}, José
  Carlos Moreira da. O Julgamento da \versal{ADPF} 153 pelo Supremo Tribunal
  Federal e a Inacabada Transição Democrática Brasileira. In: Wilson
  Ramos Filho. (Org.).~\textbf{Trabalho e Regulação}- as lutas sociais e
  as condições materiais da democracia. Belo Horizonte"-\versal{MG}: Fórum, 2012,
  v. 1, p. 129-177; \versal{SILVA} \versal{FILHO}, José Carlos Moreira da; \versal{CASTRO}, Ricardo
  Silveira.~Justiça de Transição e Poder Judiciário brasileiro -- a
  barreira da Lei de Anistia para a responsabilização dos crimes da
  ditadura civil"-militar no Brasil.~\textbf{Revista de Estudos
  Criminais}, \versal{N}º 53, p.50-87; \versal{VENTURA}, Deisy. A Interpretação judicial da
  Lei de Anistia brasileira e o Direito internacional. In: \versal{PAYNE}, Leigh;
  \versal{ABRÃO}, Paulo; \versal{TORELLY}, Marcelo (orgs.).~\textbf{A Anistia na era da
  responsabilização}: o Brasil em perspectiva internacional e comparada.
  Brasília: Ministério da Justiça, Comissão de Anistia; Oxford: Oxford
  University, Latin American Centre, 2011. p.308-34; \versal{PAIXÃO},
  Cristiano.~\emph{The protection of rights in the Brazilian transition:
  amnesty law, violations of human rights and constitutional form}~(01.
  September 2014), in \textbf{forum historiae
  iuris~}\textless{}http://www.forhistiur.de/en/2014-08-paixao/\textgreater{}.
  (Acesso em 27/10/2017).}, chegando"-se a afirmar, por exemplo, que na
década de 70 a sociedade foi às ruas pedir uma anistia ampla, geral e
irrestrita com o sentido de estendê"-la aos torturadores do regime de
força, quando em verdade o famoso bordão se referia aos presos políticos
condenados pela atuação na resistência armada, e que, no final, acabaram
não sendo mesmo anistiados pela Lei \versal{N}º 6.683/1979\footnote{Acrescente"-se
  ainda o fato de que nas frequentes assembleias realizadas pelos
  diversos Comitês Brasileiros pela Anistia (\versal{CBA}'s) as resoluções finais
  sempre pediam a responsabilização dos crimes da ditadura, conforme
  anota Heloísa Grecco em sua tese (\versal{GRECO}, op.cit.), e também \versal{RODEGHERO}
  \emph{et al.}, op.cit, p.160-162).}.

Um dos argumentos mais tortuosos e que apareceu tanto no voto do
relator, Ministro Eros Grau, como no voto do Ministro Gilmar Mendes, foi
o de que o impedimento formado pela anistia de 1979 à investigação e
responsabilização dos crimes da ditadura vinha de uma imposição de
compromisso da \versal{EC} \versal{N}º26/1985 à Constituinte de 1987, isto é, afirmaram
que uma das bases da ordem democrática de 1988 vinha justamente de uma
Emenda à Constituição autoritária e outorgada de 1967, o que limitava a
soberania da Constituinte.

Talvez esta decisão do \versal{STF} seja um dos pontos de inflexão mais nítidos
em direção à ruptura institucional que se consumou no dia 31 de agosto
de 2016 com a conclusão do processo de impedimento da então Presidenta
Dilma Roussef. Ao reproduzir em pleno regime democrático a mesma
interpretação que a ditadura forjou para a Lei de Anistia de 1979,
pode"-se dizer que o \versal{STF} alojou o ``golpismo'' em seus gabinetes e
decisões.

Após a decisão do \versal{STF} na \versal{ADPF} 153, tomada em abril de 2010, o Brasil
sofreu em novembro de 2010 a condenação na Corte Interamericana de
Direitos Humanos no Caso Gomes Lund e outros, também conhecido como Caso
Guerrilha do Araguaia. A decisão deixa claro que o Supremo Tribunal
Federal não fez o devido controle de convencionalidade e que a sua
decisão na \versal{ADPF} 153 contraria as obrigações internacionais brasileiras,
já que ``as disposições da Lei de Anistia brasileira que impedem a
investigação e sanção de graves violações de direitos humanos carecem de
efeitos jurídicos''\footnote{\versal{CORTE INTERAMERICANA DE DERECHOS HUMANOS}.
  \textbf{Caso Gomes Lund e outros vs. Brasil}. Sentencia de 24 de
  novembre de 2010. § 174. Disponível em:
  http://www.corteidh.or.cr/docs/casos/articulos/seriec\_219\_por.pdf .
  (Acesso em 27/10/2017).}.

A partir da condenação do Brasil na Corte, o Ministério Público Federal
assumiu a orientação interna de levar adiante ações de responsabilização
penal dos crimes da ditadura junto ao Poder Judiciário brasileiro. Foram
dezenas de ações penais iniciadas pelo Ministério Público Federal a
partir da condenação do país no Caso Araguaia, mas o poder judiciário
tem negado sistematicamente o seguimento das ações, ora apoiado no
argumento da anistia, ora no da prescrição\footnote{Para um registro
  recente do volume de ações do \versal{MPF} para responsabilizar os crimes
  praticados pela ditadura e as teses jurídicas adotadas, ver:
  http://www.mpf.mp.br/atuacao-tematica/ccr2/publicacoes/roteiro-atuacoes/005\_17\_crimes\_da\_ditadura\_militar\_digital\_paginas\_unicas.pdf
  (Acesso em 27/10/2017).}, contando inclusive com algumas decisões que
chegam a fazer apologia ao regime ditatorial.

Argumenta"-se neste artigo que, no âmbito do poder judiciário brasileiro,
a reafirmação em tempos democráticos de uma certa tolerância e
complacência, para não dizer, em alguns casos, defesa da tomada do poder
pelos militares em 1964, e o bloqueio a medidas justransicionais de
responsabilização e de pleno repúdio à ditadura civil"-militar,
representaram um claro flanco pelo qual alojou"-se a participação do
poder judiciário em novo processo de ruptura institucional, ocorrido
agora em 2016, há quase 30 anos da promulgação da Constituição
democrática.

\section{A Ruptura Institucional no Brasil em 2016: golpe parlamentar
com apoio judicial?}

O que aconteceu no Brasil no ano de 2016, com a saída de Dilma Roussef
da Presidência da República, pode ser explicado sob diferentes ângulos e
a partir de uma multiplicidade de fatores\footnote{Várias análises já
  foram feitas sobre o contexto e o sentido desse golpe: \versal{CITTADINO},
  Gisele; \versal{PRONER}, Carol; \versal{RAMOS} \versal{FILHO}, Wilson; \versal{TENEMBAUM}, Marcio (Orgs.).
  \textbf{A resistência ao golpe de 2016}. Bauru: Canal 6, 2016; \versal{RAMOS},
  Gustavo Teixeira; \versal{MELO} \versal{FILHO}, Hugo Cavalcanti; \versal{LOGUERCIO}, José
  Eymardt; \versal{RAMOS} \versal{FILHO}, Wilson (Orgs.). \textbf{A classe trabalhadora e
  a resistência ao golpe de 2016}. Bauru: Canal 6, 2016; \versal{PRONER}, Carol;
  \versal{CITTADINO}, Gisele; \versal{NEUENSCHWANDER}, Juliana; \versal{PEIXOTO}, Katarina;
  \versal{GUIMARÃES}, Marilia Carvalho (Orgs.). \textbf{A resistência
  internacional ao golpe de 2016}. Bauru: Canal 6, 2016; \versal{JINKINGS},
  Ivana; \versal{DORIA}, Kim; \versal{CLETO}, Murilo (Orgs.). \textbf{Por que gritamos
  golpe?} para entender o impeachment e a crise política no Brasil. São
  Paulo: Boitempo, 2016; \versal{GENTILI}, Pablo (Ed.). \textbf{Golpe en Brasil}
  -- genealogia de uma farsa. Buenos Aires: \versal{CLACSO}; Octubre Editorial,
  2016.}, mas revela inegavelmente uma grave ruptura institucional que
traz diversos paralelos com aquela ocorrida em 1964 com o golpe
civil"-militar que depôs o Presidente João Goulart\footnote{Para um
  sucinto paralelo entre ambos os processos ver: \versal{SILVA} \versal{FILHO}, José
  Carlos Moreira da. O Jogo dos Sete Erros -- 1964-2016. In: \versal{PRONER},
  Carol; \versal{CITTADINO}, Gisele; \versal{TENEMBAUM}, Marcio; \versal{RAMOS} \versal{FILHO}, Wilson
  (Orgs.). \textbf{A Resistência ao Golpe de 2016}. Bauru: Canal 6,
  2016, v. , p. 196-203.}. Diferentemente de 1964, em 2016 não houve a
deposição pelas armas e a participação das Forças Armadas. Seguiu"-se um
caminho semelhante àquele já percorrido por Honduras e Paraguai.

Em Honduras, no ano de 2009 o Poder Judiciário, provocado pelo
Ministério Público hondurenho, emitiu ordem de prisão ao então
Presidente Manuel Zelaya, que retirado de pijamas da sua casa pelo
Exército foi ilegalmente deportado para a Costa Rica. No Paraguai, no
ano de 2012 o então Presidente Fernando Lugo foi deposto pelo Parlamento
em um processo relâmpago de impeachment no qual teve apenas duas horas
para se defender de acusações vagas e atípicas relativas a um suposto
fraco exercício das suas funções. No Paraguai não foi difícil obter o
impeachment, visto que o Congresso estava dominado pela oposição
conservadora.

O que há de comum entre esses casos recentes, incluindo"-se aí o
brasileiro, é o fato de serem países latino"-americanos, de os governos
atingidos serem considerados de esquerda, com políticas populares
voltadas ao combate das desigualdades sociais, e de terem sido
utilizadas as instituições estatais para ao mesmo tempo retirar tais
governantes do poder e ostentar uma aparência de legalidade e
normalidade institucional. Em todos esses casos, igualmente, tratou"-se
de implantar uma agenda de reformas de cunho neoliberal, com fortes
restrições de direitos sociais conquistados nas últimas décadas.

Em obra recente, Anibal Pérez"-Liñán identifica na América Latina, após
as transições realizadas com o fim das ditaduras civis"-militares de
segurança nacional, a tendência de interrupção de mandatos presidenciais
por meio de juízos políticos. Tal tendência acentuou"-se a partir dos
anos 90 e indica um novo modo de instabilidade política na
região\footnote{\versal{PÉREZ"-LIÑÁN}, Anibal. \textbf{Juício político al
  presidente y nueva inestabilidad política en América Latina}. Buenos
  Aires: Fondo de Cultura Economica, 2009. p.282.}. Entre a derrubada do
Presidente brasileiro Fernando Collor em 1992 e o ano de 2004,
Pérez"-Liñán catalogou a deposição de dez presidentes latino"-americanos.
Em seu estudo comparativo, Pérez"-Liñán identifica a confluência de
quatro fatores desse novo processo: a ausência de participação das
forças armadas, a existência de protestos sociais de grande expressão em
face de denúncias de corrupção ou diante de crises econômicas, a
presença da mídia como uma espécie de vigilante moral público da
sociedade e um baixo nível de apoio parlamentar ao presidente eleito,
além da participação decisiva do parlamento na deposição do Presidente
na moldura constitucional.

A pergunta que fica aqui indicada é se esta nova modalidade pode ser
considerada em alguns casos um golpe de Estado. Carlos Barbé assinala
que nos anos 70 do século \versal{XX} a forma mais frequente de golpe de estado
foi a que envolveu a participação de militares\footnote{\versal{BARBÉ}, Carlos.
  Golpe de Estado. In: \versal{BOBBIO}, Norberto; \versal{MATTEUCCI}, Nicola; \versal{PASQUINO},
  Gianfranco (orgs.). \textbf{Dicionário de Política.} 5.ed. Tradução de
  Carmen C. Varrialle\ldots{} {[}et al{]}. Brasília: Universidade de
  Brasília, 1993.p.545-547.}, do que pode se deduzir, em acordo com a
definição do autor, que não é um elemento obrigatório e necessário a
ativa participação militar.

Conforme Barbé, na história do conceito de golpe de Estado, que inicia
com a obra de Gabriel Naudé (\emph{Considérations politiques sur le coup
d'État} -- 1639), identifica"-se uma mudança de atores quanto à sua
promoção ativa. Originalmente o conceito apontava para atos de exceção
praticados pelo soberano. Com o advento do constitucionalismo, o
conceito passou a abranger também situações de mudança do governo
ocorridas com a violação da Constituição vigente, e praticada pelos
próprios detentores do poder político, normalmente com violência. E, por
fim, o golpe militar.

Partindo dessa moldura conceitual, é possível identificar a ocorrência
de um golpe de Estado quando ocorre a mudança do governo a partir de uma
violação das regras constitucionais, sendo também importante a
participação de grupos políticos poderosos na sua realização, e ainda
que não ocorra a participação dos militares.

Também é possível delimitar o caráter de golpe para os recentes
processos de deposição de governantes na América Latina recorrendo à
ideia de que em Estados formalmente democráticos, ainda que de
baixíssima intensidade especialmente para as camadas mais periféricas
das sociedades latino"-americanas, podem ser utilizadas de maneira mais
ampla e ``criativa'' medidas de exceção, isto é, medidas autoritárias,
apoiadas no decisionismo, sem amparo legal ou constitucional. Tais
medidas de exceção podem promover a retirada dos governantes eleitos e
deflagrar mudanças bruscas de orientação política no governo, sem que
para isso seja necessária a instauração de um Estado de exceção
declarado e sem que se rompa ostensivamente com os mecanismos de
democracia formal\footnote{Essa é a leitura proposta por Pedro Serrano
  para interpretar os eventos ocorridos em Honduras em 2009, no Paraguai
  em 2012 e no Brasil em 2016. Com apoio no conceito de exceção de Carl
  Scmitt, o autor caracteriza a exceção como o poder soberano que decide
  sobre a suspensão do direito, e que pode ser identificado quando o
  governante ou grupos políticos poderosos utilizam as ferramentas e
  processos democráticos para suspenderem as garantias legais e imporem
  a exceção, não raramente atribuindo ao ato de exceção o caráter de
  aplicação regular e correta da norma democrática. Um claro exemplo
  dessa estratégia segundo o autor seria o \emph{Patriot Act} nos \versal{EUA},
  instituído após a queda das torres gêmeas. No caso dos recentes
  processos latino"-americanos de deposição de governantes, a exceção
  estaria presente pontualmente no apoio judicial aos processos
  irregulares e inconstitucionais de interrupção de mandatos eletivos.
  Afirma textualmente Serrano: ``Em Honduras e no Paraguai, regimes
      democráticos foram inconstitucionalmente interrompidos, golpeando
      presidentes legitimamente eleitos por obra ou com apoio das
      respectivas cortes supremas. Trata"-se da jurisdição funcionando como
      fonte da exceção, e não do direito.'' (\versal{SERRANO}, Pedro Estevam Alves
  Pinto. \textbf{Autoritarismo e golpes na América Latina} -- breve
  ensaio sobre jurisdição e exceção. São Paulo: Alameda, 2016. p.168).},
contrariamente ao que ocorreu nas ditaduras civis"-militares de segurança
nacional.

No caso brasileiro de 2016 nota"-se uma diferença crucial em relação ao
padrão proposto por Pérez"-Liñán, qual seja o papel decisivo do poder
judiciário na ruptura institucional, fazendo às vezes de guardião moral
da sociedade apoiado e reverberado pela mídia hegemônica, e com isso
"justificado" em seus decisionismos violadores de cláusulas
constitucionais.

A partir dos fundamentos expostos, se apresenta razoável e adequada a
utilização da categoria ``golpe'' para tratar do processo de impeachment
sofrido pela Presidenta Dilma Roussef em 2016.

No dia 02 de dezembro de 2015 o Presidente da Câmara dos Deputados
Eduardo Cunha aceitou pedido de impedimento contra a Presidenta Dilma
Rousseff pela prática de crime de responsabilidade contra a lei
orçamentária (hipótese do Art.85, \versal{VI} da Constituição Federal de 1988). A
aceitação do pedido deu"-se em circunstâncias polêmicas, pois ocorreu
momentos depois que os deputados do Partido dos Trabalhadores (\versal{PT}),
partido da Presidenta, declararam que votariam contra o Presidente da
Câmara em causa de cassação do seu mandato em andamento na Comissão de
Ética da casa legislativa. No dia 17 de abril de 2016 ocorreu sessão
plenária de votação do parecer favorável, aprovado pela Comissão
Especial constituída, ao impedimento da Presidenta. O pedido foi
aprovado pela Câmara com 367 votos a favor, 137 contra, 7 abstenções e 2
ausências. Nas manifestações dos parlamentares para justificar o voto
pouco se tratou da acusação da prática de crime de responsabilidade pela
Presidenta. O que a esmagadora maioria dos deputados disse foram
homenagens a membros da família, acusações de corrupção à Presidenta
(não mencionadas no pedido cuja aceitação se votava) e até homenagens a
notórios torturadores da ditadura civil"-militar, espetáculo que chocou a
sociedade, até mesmo aqueles favoráveis à deposição da Presidenta. A
aprovação do pedido na Câmara representou o momento culminante para o
afastamento da Presidenta pelo Senado Federal, tornando"-o praticamente
irreversível sob o ponto de vista político. O placar do impedimento no
Senado foi de 61 votos a favor e 20 contra, em um parlamento com ampla
maioria oposicionista e conservadora, o que confere à ruptura
institucional um inegável caráter parlamentar.

A denúncia que foi apreciada no Parlamento foi oferecida pelos juristas
Hélio Bicudo, Janaína Paschoal e Miguel Reale Jr. Examinando"-se a peça
inicial, bem como as alegações finais e o relatório do Senador Antonio
Anastasia do \versal{PSDB}\footnote{\versal{BRASIL}. \versal{CÂMARA} \versal{DOS} \versal{DEPUTADOS}. Denúncia \versal{N}º1 de
  2016. Autores: Hélio Bicudo, Janaína Conceição Paschoal, Miguel Reale
  Jr. e outros; \versal{BRASIL}. \versal{SENADO} \versal{FEDERAL}. Parecer \versal{N}º, de 2016. Da Comissão
  Especial do Impeachment, referente à admissibilidade da \versal{DEN} \versal{N}º 1, de
  2016. Relatoria do Senador Antonio Anastasia; \versal{BRASIL}. \versal{SENADO} \versal{FEDERAL}.
  Alegações Finais na Denúncia \versal{N}º1 de 2016. Autores: Hélio Bicudo,
  Janaína Conceição Paschoal, Miguel Reale Jr. e outros. Todas as três
  peças referidas podem ser encontradas em:

  http://www12.senado.leg.br/noticias/materias/2016/08/22/veja-os-principais-documentos-do-processo-de-impeachment-de-dilma-rousseff
  (Acesso em 27/10/2017).}, que foi designado no Senado relator do
pedido aprovado na Câmara, vê"-se uma doutrina absolutamente permissiva
do impeachment no Direito brasileiro, que abre espaço a uma indevida
fiscalização~ ordinária dos atos do Presidente eleito e potencializa a
criminalização de atos de gestão e administração, quando deveria ser um
processo excepcionalíssimo e rigoroso, adstrito às hipóteses
constitucionais.

Embora a hipótese do impeachment esteja prevista na Constituição de
1988, a lei que regulamenta o seu rito e detalha as suas hipóteses é uma
lei de 1950, a Lei \versal{N}º 1.079/50. Esta lei teve como um dos seus redatores
e entusiasta o político Raul Pilla, conhecido por seu fervor
parlamentarista, e que havia sido previamente derrotado em sua campanha
para que a Constituição de 1946 adotasse o sistema. A aprovação da Lei
dos crimes de responsabilidade, a Lei \versal{N}º1.079/50, figurou como uma
espécie de prêmio menor ao bloco político parlamentarista, criando"-se
assim uma lei moldada por um viés parlamentarista vigente em um sistema
presidencialista\footnote{\versal{QUEIROZ}, Rafael Mafei Rabelo. Impeachment e
  lei de crimes de responsabilidade -- o cavalo de tróia parlamentarista.
  In: Blog Direito e Sociedade Publicado em 16 de dezembro de 2015.
  Disponível em:

  http://brasil.estadao.com.br/blogs/direito-e-sociedade/impeachment-e-lei-de-crimes-de-responsabilidade-o-cavalo-de-troia-parlamentarista/
  (Acesso em 27/10/17).}. Interessante notar que foi Raul Pilla quem
redigiu a emenda que adotou o sistema parlamentarista pra retirar os
poderes presidenciais de João Goulart em 1961\footnote{A autoria da
  proposta de emenda constitucional que viria a ser aprovada em 1961 e
  revogada em 1963 pode ser comprovada nos registros da Câmara dos
  Deputados. Ver:
  \emph{https://bit.ly/2C9tshA}
  (Acesso em 27/10/2017).} diante da pressão dos inumeráveis grupos
golpistas daquela época, militares e civis.

Vê"-se, portanto, que o espírito que animou a lei do impeachment foi o
parlamentarista. Contudo, o sistema no Brasil é o presidencialista. Se
no primeiro a perda da maioria parlamentar pode destituir o governante,
no segundo a sua destituição legal só pode ocorrer em circunstâncias
excepcionais e restritas, não sendo suficiente a desconfiança da maioria
parlamentar oposicionista. Necessário é que se configure um crime de
responsabilidade. Afrouxar esta condição tornando"-a permissiva para nela
incluir múltiplas hipóteses determinadas por leis infraconstitucionais,
incluindo até mesmo raciocínios extensivos e de analogia, como ocorreu
no caso do impeachment da Presidenta Dilma Roussef, é fragilizar a
cláusula democrática, substituindo o numeroso e expressivo respaldo
popular que sustenta o mandato do Presidente da República pelo
malabarismo hermenêutico de parlamentares com muito menos votos e de
funcionários públicos sem representatividade alguma, como o são juízes e
procuradores.

O Brasil alargou ainda mais o flanco de fragilidade democrática
institucional ao submeter a Constituição de 1988 à lógica
parlamentarista de uma Lei editada em 1950, e mesmo após o sistema
parlamentarista ter sido rejeitado no plebiscito de 1993 por quase 70\%
da população. Na ausência de uma nova lei, que esteja mais adequada
tanto ao sistema presidencialista como ao marco constitucional
instituído a partir de 1988, seria ao menos necessário uma interpretação
judicial que submetesse a legislação ordinária à lógica e à supremacia
constitucional. De todo modo, mesmo considerando a existência da Lei de
1950, o processo de impeachment da Presidenta Dilma Roussef não
conseguiu de modo consistente identificar qualquer crime de
responsabilidade.

No caso das célebres ``pedaladas fiscais''\footnote{Pedaladas Fiscais
  designam a prática de o Tesouro público atrasar o repasse de verbas,
  que foram utilizadas para programas públicos e sociais, a instituições
  financeiras públicas ou privadas, produzindo com isso uma melhor
  situação fiscal no fechamento de um determinado período.}, o inciso \versal{VI}
do Art.85 da \versal{CF} de 1988 afirma que são crimes de responsabilidade atos
que atentem contra a ``lei orçamentária''. As peças da acusação no
processo de impeachment afirmam que nesta expressão dever"-se"-ia incluir
a Lei de Responsabilidade Fiscal (Lei Complementar \versal{N}º 101/2000). No
entanto, a questão fiscal não se confunde com a orçamentária, ainda que
estejam relacionadas, existindo uma lei diferente para cada qual. Querer
incluir uma lei que não é orçamentária em um dispositivo excepcional e
com consequências drásticas para o mandato presidencial é dar uma
amplitude muito questionável e temerária.

Indo além, o Senador Anastasia afirmou em seu parecer de admissibilidade
ao processo de impedimento no Senado que, como a Lei de Responsabilidade
Fiscal diz no seu Art.73 que as infrações a esta lei serão punidas com
base, entre outras leis, na Lei de 1950, violar qualquer dispositivo da
Lei de Responsabilidade Fiscal implica em crime de responsabilidade. A
partir daí o Senador indica que a Presidenta violou o Art.36, que veda a
realização de empréstimo entre o ente da federação e instituição
financeira por ele controlada. No entanto, em nenhum lugar da lei se diz
que a infração a este artigo é um crime de responsabilidade. Mas ainda
que fosse, atrasar o pagamento de recursos aplicados para subvenção de
programas que garantem direitos sociais, como ocorreu no Plano Safra, um
plano público de concessão de crédito para a agricultura familiar, não é
uma operação de crédito, não existindo sequer precedente judicial ou
doutrinário neste sentido.

Com base na falsa premissa anterior, partiu"-se para a identificação do
que seria outro suposto crime de responsabilidade: a edição de decretos
de crédito suplementar fora da meta fiscal, já que se a premissa fosse
verdadeira não haveria superávit a autorizar os créditos, condição
prevista na Lei de Orçamento de 2015. Deixando a falsa premissa de lado,
a edição desses decretos seguiu rigorosamente as condições exigidas em
lei, e é recurso comum utilizado por governos anteriores.

Ademais, todos os atrasos de pagamentos do tesouro às instituições
financeiras federais foram quitados em janeiro de 2016 e o ano de 2015
fechou com a meta compatível aos gastos realizados, tendo a meta sido
alterada em dezembro diante dos efeitos recessivos da crise econômica
mundial\footnote{Caso a meta não houvesse sido alterada dentro do ano
  fiscal aí sim se configuraria o gasto acima da meta sem autorização do
  parlamento. De todo modo, é sintomático que apenas dois dias após a
  consumação do impeachment, o Senado tenha aprovado uma lei que
  flexibiliza a edição de decretos de crédito suplementar sem
  autorização do Congresso, a Lei \versal{N}º 13.332/2016..}. No entanto, isso
parece não ter qualquer relevância para os denunciantes do impeachment e
os que os apoiaram, sob o pretexto de que se a Lei de Responsabilidade
Fiscal é uma lei que protege a precaução, então qualquer ato considerado
temerário vira um crime de responsabilidade, ainda que não tenha havido
prejuízo aos cofres públicos e os passivos tenham sido saldados. É um
"crime formal de mera conduta", conforme está assinalado no parecer do
Senador Anastasia e nas Alegações Finais dos denunciantes. Não interessa
o resultado.

Em homenagem aos princípios mais elementares do Direito Penal e da
cláusula democrática, exige"-se que o crime ensejador da perda do mandato
presidencial popular seja estritamente previsto na Constituição ou a
partir dela, restando vedado qualquer juízo de analogia ou alargamento.
Querer afastar essa condição para que o Parlamento decida o que quiser,
com a desculpa de que se trata de um juízo eminentemente político é
violar a lógica e a Constituição.

Não só o crime identificado foi fruto de um verdadeiro atentado
hermenêutico à Constituição e à legislação financeira como também não se
conseguiu apontar sua autoria com clareza e coerência. A Presidenta
Dilma foi ao mesmo tempo acusada por ato omissivo e comissivo, como se
depreende da denúncia e das alegações finais. Somente restou aos
defensores do impeachment, em suas alegações finais, invocarem a
``personalidade enérgica e controladora'' da Presidenta para afirmar que
ela foi autora dos crimes criados, ou atestarem que a Presidenta era
``íntima'' do Secretário do Tesouro, a ponto de não se saber ``onde
começava um e terminava o outro''\footnote{Esta frase foi dita por Miguel
  Reale Jr., um dos juristas signatários do pedido de impeachment de
  Dilma Roussef , quando realizava sua manifestação na Comissão Especial
  do Impeachment no Senado Federal no dia 28 de abril de 2016.
  Disponível em:
  http://veja.abril.com.br/brasil/reale-defende-impeachment-no-senado-crime-de-responsabilidade-sem-punicao-e-golpe/
  (Acesso em 27/10/17).}.

Para além do protagonismo parlamentar na deposição da Presidenta eleita,
o poder judiciário teve também participação crucial nesse processo. O
\versal{STF} se negou a exercer o seu papel de limitar os abusos do Parlamento ao
longo do processo fraudulento de impeachment, mesmo quando
provocado\footnote{Logo após consumado o impedimento da Presidenta
  Dilma, a sua defesa impetrou um Mandado de Segurança no \versal{STF} pedindo a
  anulação do impedimento. A liminar foi negada e a ação, até os dias
  nos quais se escreve este artigo, dormita nas gavetas da Corte.}, sob
o argumento de que se tratava de uma decisão ``política'' e de que não
deveria intervir, lavando as suas mãos.

Ademais, para que o processo de impeachment da Presidenta Dilma fosse
possivel, foi necessário um intenso processo de criminalização do seu
partido e do seu governo, proporcionado por intensa campanha midiática e
por ação seletiva e arbitrária do Judiciário federal, da Polícia Federal
e do Ministério Público Federal.

Ao longo do ano de 2016 o Jornal O Globo estampava sucessivas manchetes
e editoriais de apoio ao golpe parlamentar, assim como fizeram também
quase todos os jornais da grande mídia (e entre eles a Folha de São
Paulo, o Estadão e a Revista Veja) . A Rede Globo de Televisão teve
papel decisivo e protagonista por meio principalmente dos seus programas
de notícias e jornalismo. O Jornal Nacional dedicou edições inteiras
para noticiar e analisar vazamentos seletivos e escutas ilegais enviadas
diretamente pelo juiz Sergio Moro, responsável pela Operação
Lava"-Jato\footnote{A Operação Lava"-Jato é uma mega operação deflagrada
  pela Polícia Federal no ano eleitoral de 2014 e que no âmbito do
  judiciário federal vem sendo centralizada e coordenada pelo juiz
  federal Sergio Moro, com atuação do \versal{STF} nos casos de políticos com
  foro privilegiado. O foco central é a investigação sobre esquemas de
  propina praticados na Petrobrás envolvendo políticos e empreiteiras.
  Trata"-se de uma operação polêmica, ainda não concluída quando da
  redação deste artigo, que tem se tornado notória pelo uso explícito
  que faz de delações premiadas obtidas de suspeitos mantidos
  indefinidamente em prisão preventiva, de vazamentos ilegais à
  imprensa, de atuação em parceria entre promotoria e magistratura
  contra os réus, de cerceamento de direitos da defesa, de ações
  espetaculares cobertas pela mídia e de condenações severas que em
  muitos casos baseiam"-se tão somente em delações.}. Também deu destaque
para investigações ainda em andamento do Ministério Público Federal
voltadas contra o Ex"-Presidente Lula, seu partido e o governo da
Presidenta Dilma, ao mesmo passo em que dava pouco espaço e importância
às denúncias e delações envolvendo empresários que apoiavam a oposição e
políticos da oposição, entre eles o candidato do Partido da Social
Democracia Brasileira (\versal{PSDB}) derrotado em 2014. O auge do espetáculo
midiático ocorreu na noite de 16 de março quando Moro enviou grampos
ilegais de conversas entre a Presidenta Dilma e o Ex"-Presidente Lula,
feitos na própria Presidência da República, diretamente à Rede Globo de
Televisão, contendo conversas particulares e privadas que são
manipuladas e expostas à execração pública em pleno Jornal
Nacional\footnote{Na conversa ilegalmente gravada e divulgada a
  Presidenta Dilma avisava o Ex"-Presidente Lula de que um emissário
  levaria até ele o termo da sua posse como Ministro da Casa Civil, para
  que pudesse utilizá"-lo ``em caso de necessidade'' até a sua chegada a
  Brasília para efetivamente tomar posse no cargo. À época o
  Ex"-Presidente não era réu em qualquer processo, não havendo portanto
  qualquer impedimento para que Dilma o nomeasse Ministro, estando tal
  atitude dentro da sua legítima discricionariedade como governante
  eleita. Com a sua nomeação como Ministro, o Ex"-Presidente adquiriria
  foro privilegiado e teria eventuais denúncias apreciadas e formuladas
  a partir da Procuradoria Geral da República diante do Supremo Tribunal
  Federal. No \versal{STF}, o Ministro Gilmar Mendes em decisão monocrática e
  liminar simplesmente suspendeu a posse do Ex"-Presidente, situação que
  não mais foi revertida, mesmo com a ausência de qualquer razoabilidade
  ou fundamento legal para sustentá"-la.}. O crime praticado por Moro é
ignorado pelo Conselho Nacional de Justiça e pelo \versal{STF}, contentando"-se
este último com um simples pedido de desculpas.

Amplos setores da Polícia Federal, em trabalho conjunto com o Judiciário
e o Ministério Público Federal, no bojo da Operação Lava"-Jato, levaram
adiante Operações de investigação, conduções coercitivas\footnote{Um dos
  momentos mais tensos em toda a escalada judicial"-midiática que
  preparou o golpe foi a condução coercitiva do Ex"-Presidente Lula no
  dia 04 de março de 2016. Sem que fosse réu, sem que houvesse sido
  intimado ou se negado a prestar depoimento no âmbito das
  investigações, com todo o aparato repressivo e midiático, Lula foi
  levado coercitivamente do interior do seu apartamento para o aeroporto
  de Congonhas em São Paulo, onde por fim acabou por fazer o seu
  depoimento e ser liberado em seguida. Havia aparato já designado para
  que ele fosse levado à Curitiba, mas, ao que parece, a forte reação
  popular no próprio aeroporto ou algum outro motivo não esclarecido,
  impediu que assim ocorresse.}, prisões e de execução de mandados de
busca e apreensão que se voltaram prioritariamente contra o próprio
governo da Presidenta eleita e seu Partido, por mais frágeis e
inconsistentes que fossem as acusações, enquanto os documentos e
delações que envolveram políticos dos partidos favorecidos com a
deposição da Presidenta Dilma, em especial o \versal{PSDB} e o \versal{PMDB} (Partido do
Movimento Democrático Brasileiro), foram sistematicamente ignorados. A
operação assumiu explicitamente um viés seletivo\footnote{Importa
  esclarecer que não se pretende nesse artigo negar ou afirmar o
  envolvimento de políticos do Partido dos Trabalhadores, assim como de
  outras siglas, em práticas de corrupção. O ponto que aqui interessa é
  notar a instrumentalização do sistema de justiça em prol de objetivos
  políticos que contam com a necessidade de rupturas institucionais e de
  expedientes de exceção e antidemocráticos, adotando como bandeira
  assumida a perigosa fórmula de que os fins justificam os meios.}
apoiado basicamente em delações de corrupção obtidas a partir de prisões
provisórias sem prazo para acabarem, além de terem praticado inúmeras
ações ilegais e irregulares como vazamentos para a imprensa, prisões
baseadas em indícios frágeis e escutas ilegais, inclusive de escritórios
de advocacia que representavam os réus. O \versal{STF} convalidou todos os atos
da Operação Lava Jato.

Quando provocado o \versal{STF} e outras instâncias superiores convalidaram todas
as evidentes ilegalidades praticadas em especial pelo juiz Sergio Moro.
O episódio mais intenso neste sentido foi a decisão do Tribunal Regional
Federal da 4a Região em representação disciplinar feita por advogados
contra este juiz, tomada em setembro de 2016. Por 13 votos a 1, os
juízes deste Tribunal decidiram que a Operação Lava"-Jato está lidando
com situações excepcionais e que portanto exigem ``soluções
excepcionais'', e não podem ser tratadas pelo direito comum\footnote{Eis
  o trecho da decisão que autoriza a suspensão do direito (exceção) no
  caso da Lava"-Jato, sem que o relator pareça ter qualquer ideia mais
  apurada do que isto significa. É patente a confusão que ele faz entre
  casos especiais ou excepcionais e estado de exceção:

  De início, impõe"-se advertir que essas regras jurídicas só podem ser
  corretamente interpretadas à luz dos fatos a que se ligam e de todo
  modo verificado que incidiram dentro do âmbito de normalidade por elas
  abrangido. É que a norma jurídica incide no plano da normalidade, não
  se aplicando a situações excepcionais, como bem explica o jurista Eros
  Roberto Grau:

  \begin{quote}
  A 'exceção' é o caso que não cabe no âmbito da 'normalidade' abrangida
  pela norma geral. A norma geral deixaria de ser geral se a
  contemplasse. Da 'exceção' não se encontra alusão no discurso da ordem
  jurídica vigente. Define"-se como tal justamente por não ter sido
  descrita nos textos escritos que compõem essa ordem. É como se nesses
  textos de direito positivo não existissem palavras que tornassem
  viável sua descrição. Por isso dizemos que a 'exceção' está no
  direito, ainque que não se encontre nos textos normativos do direito
  positivo. Diante de situações como tais o juiz aplica a norma à
  exceção 'desaplicando"-a', isto é, retirando"-a da 'exceção {[}Agamben
  2002:25{]}. A 'exceção' é o fato que, em virtude de sua anormalidade,
  resulta não incidido por determinada norma. Norma que, em situação
  normal, o alcançaria (\versal{GRAU}, E. R. Por que tenho medo dos juízes (a
  interpretação/aplicação do direito e os princípios). 6ª ed. refundida
  do Ensaio e Discurso sobre a Interpretação/Aplicação do Direito. São
  Paulo: Malheiros, 2013. p. 124-25).
  \end{quote}

  Ora, é sabido que os processos e investigações criminais decorrentes
  da chamada ``Operação Lava"-Jato'', sob a direção do magistrado
  representado, constituem caso inédito (único, excepcional) no direito
  brasileiro. Em tais condições, neles haverá situações inéditas, que
  escaparão ao regramento genérico, destinado aos casos comuns. Assim,
  tendo o levantamento do sigilo das comunicações telefônicas de
  investigados na referida operação servido para preservá"-la das
  sucessivas e notórias tentativas de obstrução, por parte daqueles,
  garantindo"-se assim a futura aplicação da lei penal, é correto
  entender que o sigilo das comunicações telefônicas (Constituição, art.
  5º, \versal{XII}) pode, em casos excepcionais, ser suplantado pelo interesse
  geral na administração da justiça e na aplicação da lei penal. A
  ameaça permanente à continuidade das investigações da Operação
  Lava"-Jato, inclusive mediante sugestões de alterações na legislação,
  constitui, sem dúvida, uma situação inédita, a merecer um tratamento
  excepcional." (Brasil. Tribunal Regional Federal (4a Região). P.A.
  \versal{CORTE} \versal{ESPECIAL} \versal{N}º 0003021-32.2016.4.04.8000/\versal{RS}. Relator Des. Romulo
  Pizzolatti. Disponível em:
  http://www.conjur.com.br/2016-set-23/lava-jato-nao-seguir-regras-casos-comuns-trf
  (Acesso em 27/10/2017).}. O relator chega até mesmo a citar Giorgio
Agambem para definir o Estado de exceção, embora o faça indevidamente já
que interpreta ser a sua descrição do Estado de exceção uma hipótese
necessária em alguns casos e não uma denúncia do alastramento do seu
padrão pelo mundo.

A deposição da Presidenta Dilma Roussef, assim como todas as
consequencias que vieram depois para o país em termos de retrocessos e
fragilização democrática, necessitou de um ambiente institucional de
normalização do abuso de poder por parte do Judiciário, bem como da sua
convergência com abusos de poder praticados por outros agentes públicos,
entre os quais membros do Ministério Público, da Polícia Federal e
parlamentares.

\section{Considerações Finais}

No processo de justiça transicional brasileiro, ainda em curso, muitas
ações importantes foram realizadas, ainda que tardiamente, mas o
bloqueio da pauta da responsabilização, tanto administrativa quanto
penal, e a ausência de reformas públicas e legais mais efetivas no
repúdio à instrumentalização das instituições estatais e de setores
estratégicos como a mídia e o sistema de justiça, parecem ter
contribuído significativamente para a interrupção do processo
democrático iniciado em 1988.

É por demais simbólico que logo no segundo dia após consumado o processo
fraudulento de impeachment, mais precisamente no dia 02 de setembro de
2016, o então Ministro da Justiça Alexandre de Moraes operou um
desmantelamento da Comissão de Anistia, com a dispensa unilateral e não
justificada de seis dos seus membros mais antigos e a nomeação de vinte
novos membros, dos quais nenhum é reconhecido por atuar no campo dos
Direitos Humanos, o que fez sem qualquer consulta à sociedade civil
organizada, como movimentos de familiares de mortos e desaparecidos
políticos, organizações de direitos humanos, movimentos sociais, e sem a
anuência dos conselheiros dispensados. Em toda a sua existência a
mudança na composição do conselho sempre se deu a partir da espontânea
decisão dos membros mais antigos em saírem e a partir da consulta aos
movimentos e organizações mais envolvidos com a pauta.

Houve nas primeiras reuniões do novo grupo uma clara tentativa por parte
da atual equipe administrativa e de alguns dos novos membros em alterar
em desfavor dos anistiandos uma série de entendimentos já consolidados
na Comissão de Anistia. O ápice deste processo foi a declaração à
imprensa do então novo Conselheiro Alberto Goldman de que não deveria
haver a reparação pecuniária aos perseguidos, já que a reparação teria
sido a própria redemocratização do país\footnote{Ver aqui a reportagem:
  https://oglobo.globo.com/brasil/novo-membro-da-comissao-de-anistia-contra-pagamento-perseguidos-1-20744424
  (Acesso em 27/10/2017).}. Movimentos brasileiros por Verdade, Memória
e Justiça reagiram em notas\footnote{Ver:
  \emph{https://bit.ly/2A19Drv};
  \emph{https://glo.bo/2ys2gYc};
  \emph{https://bit.ly/2pM0uND};
  e também:
  \emph{https://bit.ly/2NwIUGL}.
  Nota Pública da Comissão Estadual da Memória e Verdade Dom Helder
  Câmara (\versal{CEMVDHC}): Disponível em: \emph{https://bit.ly/2pJrqgM}
  (Acesso em 27/10/2017).}, e devido à pressão o referido Conselheiro
acabou pedindo o desligamento da Comissão, mas o seu entendimento,
hostil ao programa de reparações brasileiro, é compartilhado por alguns
dos novos Conselheiros e pela própria equipe administrativa da Comissão,
empossada tão logo Alexandre de Moraes assumiu o Ministério da Justiça.

Desde que Michel Temer assumiu o poder, a Comissão tem estado
praticamente estagnada em todas as suas atividades. Foram pouquíssimas
as sessões ocorridas até o primeiro semestre de 2017. O Conselho,
incluindo"-se aí a Presidência da Comissão, perdeu completamente a
ingerência sobre as sessões, estando todo o andamento e todos os
projetos da Comissão nas mãos da equipe administrativa, constituída de
modo completamente independente em relação à própria Presidência do
Conselho, o que contrasta com o modo anterior de funcionamento da
Comissão desde as suas origens.

As Caravanas da Anistia foram interrompidas. O Edital Marcas da Memória
não foi renovado e não há qualquer perspectiva na sua continuidade. O
Projeto Clínicas do Testemunho não conta no horizonte com qualquer
indício de renovação pelo governo brasileiro. E a construção do Memorial
da Anistia resta interrompida e inconclusa\footnote{Por óbvio, tais
  fatos apresentam o estado da questão à época de finalização deste
  artigo, mais precisamente no mês de outubro de 2017.}.

O golpe de 2016 articula simbolicamente um esforço revisionista de
suavização do golpe civil"-militar de 1964. Para além dos efeitos óbvios
neste sentido que a paralisação de toda a pauta justransicional
acarreta, as manifestações civis que pediram a derrubada da Presidenta
eleita trouxeram consigo setores expressivos que pediam a volta da
ditadura militar, tida por eles como um período sem corrupção e duro
para com os comunistas ou membros da esquerda (agora identificados com o
Partido dos Trabalhadores). Toda a forte campanha de estigmatização do
\versal{PT} e das suas principais lideranças, conduzida ao longo desse processo,
teve o condão de suavizar as ilegalidades e abusos de poder praticados
contra eles, bem como justificar um juízo político sem amparo
constitucional, em claro paralelismo com a ruptura havida em 1964.

Nessa chave, ainda é preciso mencionar que a despeito da ampla base
parlamentar do governo instalado em 2016, e justamente por isto capaz de
operar retrocessos antipopulares e antissociais na legislação, a
instabilidade política e econômica apenas se agravou. Nesse cenário, um
general da ativa das Forças Armadas, Hamilton Mourão, depois apoiado
pelo então Comandante do Exército brasileiro, General Eduardo
Villas"-Boas, invocou em uma palestra tornada pública pelos meios
virtuais, a possibilidade de uma ``intervenção militar'' caso o judiciário
brasileiro não afaste os políticos envolvidos em maus feitos, e na mesma
ocasião fez uma defesa do papel das Forças Armadas durante a ditadura
vivida pelo país\footnote{https://www.sul21.com.br/jornal/em-palestra-general-mourao-fala-em-possibilidade-de-intervencao-militar-no-pais/
  e também:
  http://www1.folha.uol.com.br/poder/2017/09/1920079-comandante-do-exercito-descarta-punir-general-que-sugeriu-intervencao.shtml
  (Acesso em 27/10/2017).}. O fato teve repercussão nacional e encontrou
amplo apoio nas redes sociais, não raro com discursos revisionistas
sobre o significado da ditadura civil-militar\footnote{https://www.cartacapital.com.br/politica/para-historiadora-intervencao-militar-no-brasil-201cnao-pode-mais-ser-descartada201d
  (Acesso em 27/10/2017).}.

De todas as forças que continuam a operar pela normalização do golpe
parlamentar e das suas consequências, uma das maiores responsabilidades
cabe ao Poder Judiciário, na medida em que claramente abriu mão do seu
papel contramajoritário e de defensor dos direitos e garantias
constitucionais\footnote{Entre tantas decisões polêmicas de chancela de
  retrocessos de direitos e garantias, cabe mencionar o caso notório da
  autorização de prisão por condenação não transitada em julgado.
  Confirmando decisão em caso específico que havia sido tomada em 17 de
  fevereiro de 2016, o \versal{STF} decidiu permitir a execução provisória da
  pena de prisão após condenação em segundo grau mesmo com recurso às
  instâncias superiores ainda em trâmite. E o fez a despeito dos Art.5°,
  \versal{LVII} da Constituição Federal e do Art.283 do Código de Processo Penal
  que fixam a necessidade do trânsito em julgado para que alguém possa
  ser considerado culpado e cumprir a sua pena, permitindo"-se apenas as
  excepcionalidades da prisão preventiva e da provisória. Esta decisão
  foi reforçada no dia 05 de outubro de 2016 com efeito geral no
  julgamento da Ação Declaratória de Constitucionalidade 43. Na sua
  manifestação o Ministro Gilmar Mendes chegou a dizer que a resposta à
  preocupação dos advogados com a presunção da inocência seria a
  Lava"-Jato, visto que a decisão do \versal{STF} possui o efeito prático de
  antecipar a prisão de políticos presos pela Operação já referida.
  Note"-se que a cláusula constitucional relativizada e restringida é uma
  cláusula pétrea, assim declarada pela própria Constituição no Art.60,
  parágrafo 4°.}.

A par da ausência de depurações administrativas no corpo do Poder
Judiciário após o regime autoritário, seja em relação ao pessoal
integrante do aparato burocrático seja em relação aos próprios
magistrados, importa aqui destacar a manutenção no regime democrático de
uma expectativa moralizante a respeito da atuação jurisdicional
combinada com sua presença cada vez maior nas funções de mediação
institucional e social.

Para um diagnóstico coerente com os que poderiam ser apontados como os
cânones democráticos mais básicos, sejam eles relativos à soberania
popular ou ao espaço conferido à participação da sociedade civil
organizada e à permeabilidade às demandas populares, não basta partir"-se
apenas das definições conceituais reservadas ao papel do Poder
Judiciário em um Estado de Direito. É preciso problematizar suas
continuidades históricas, sua estrutura elitista, hierárquica e pouco
permeável ao exercício democrático, o que assume cores especiais no
contexto latino"-americano\footnote{\versal{ZAFFARONI}, Eugenio Raúl.
  \textbf{Poder Judiciário}: crise, acertos e desacertos. São Paulo:
  Revista dos Tribunais, 1995.}.

Como lembra Cittadino\footnote{\versal{CITTADINO}, Gisele. \textbf{Poder
  judiciário, ativismo judicial e democracia}. Alceu (\versal{PUCRJ}), v. 5, \versal{N}º
  9, p. 105-113, jul./dez. 2004.} em um país como o Brasil, dificilmente
se pode invocar a existência de uma comunidade de valores que possa ser
perscrutada pela inteligência e sensibilidade superiores de algum
magistrado, ou que esteja afinada a alguma tradição constitucional. A
história constitucional brasileira é permeada por rupturas e
continuidades que não autorizam a pressuposição quanto à existência de
algum tipo de tradição. Tampouco é factível supor"-se uma comunidade
ética de valores compartilhados no contexto de sociedades profundamente
marcadas pela desigualdade e pela assimetria nas relações de poder (se é
que seria possível fazê"-lo em relação a qualquer sociedade
contemporânea), sem falar no intenso pluralismo que as caracterizam.

Em contextos assim, o compromisso maior e necessário do Poder Judiciário
deve ser, de um lado, o de concretizar a Constituição a partir dos seus
próprios marcos republicanos, abrindo mão da busca de um denominador
moral objetivo que esteja para além ou para aquém da referência
constitucional, e controlando com especial atenção os seus próprios
arroubos ativistas, e de outro, a abertura e a permeabilidade aos grupos
sociais populares organizados voltados a pautas emancipatórias de
diminuição das desigualdades históricas e ao respeito e ampliação dos
direitos fundamentais. Como bem adverte Ingeborg Maus,

\begin{quote}
Quando a justiça ascende ela própria à condição de mais alta instância
moral da sociedade, passa a escapar de qualquer mecanismo de controle
social -- controle ao qual normalmente se deve subordinar toda
instituição do Estado em uma forma de organização política democrática.
No domínio de uma Justiça que contrapõe um direito ``superior'', dotado de
atributos morais, ao simples direito dos outros poderes do Estado e da
sociedade, é notória a regressão a valores pré"-democráticos de
parâmetros de integração social.\footnote{\versal{MAUS}, Ingeborg. Judiciário
  como superego da sociedade -- o papel da atividade jurisprudencial na
  ``sociedade orfã''. Tradução de Martonio Lima e Paulo Albuquerque.
  \textbf{Novos Estudos}, \versal{N}º 58, p.183-202. nov. 2000. p.187.}
\end{quote}

Não se trata de negar ao judiciário a necessidade de que exerça a
interpretação da lei ou de querer regressar a parâmetros positivistas ou
de literalidade, especialmente em um marco constitucional
principiológico, mas sim que se abstenha de ostentar as categorias
objetivas da moralidade social e da escuta do ``clamor popular'', acabando
por confundir interpretação com subjetivismo ou decisionismo. Autonomia
e independência judiciais não devem ser compreendidas como pretextos
para abusos de poder. Dado o seu histórico de complacência autoritária,
é imprescindível que se opere uma democratização na própria estrutura
administrativa do poder judicial\footnote{A esse respeito, ver o sucinto
  artigo de \versal{ESCRIVÃO FILHO} sobre o caráter centralizador, verticalizante
  e oligárquico da estrutura judicial no Brasil, citando"-se por exemplo:
  a inexistência do sufrágio direto exercido pelos magistrados e
  servidores para a escolha dos presidentes dos tribunais de justiça;
  órgãos de corregedoria controlados pelos tribunais e conformadores de
  uniformidades condizentes com o padrão político e ideológico adotado;
  presença diminuta de representantes das minorias sociais na composição
  dos quadros (mulheres, negros, indígenas, \versal{LGBT}'s, etc); controle
  externo limitado, tímido, tardio, incompleto e claudicante. Ainda
  relativamente ao processo de escolha dos juízes da instância judicial
  nacional mais elevada, o autor aponta para uma diferença nos critérios
  de composição dos juízes da Corte Suprema argentina e nos do Supremo
  Tribunal Federal no Brasil. Enquanto naqueles se prevê expressamente o
  compromisso do futuro magistrado do Tribunual com os Direitos Humanos
  (em acordo com o Decreto Presidencial \versal{N}º 222/2003), exemplo também
  presente na Constituição boliviana de 2009, nestes nada se menciona
  sobre a necessidade de tal compromisso. (\versal{ESCRIVÃO FILHO}, 2015,
  p.39-40).} e em relação à sua atividade, ampliando os controles
sociais e democráticos, buscando"-se criar verdadeiras pontes de diálogos
e construção entre a magistratura e os movimentos sociais, sem o que se
esvaem a legitimidade e a soberania popular.

Independente dos interesses e influências internacionais que influenciam
a ruptura institucional ocorrida em 2016 no Brasil, ao examinar"-se o
processo a partir das próprias contradições e dificuldades internas do
país, nota"-se, por tudo o que já se descreveu aqui, um claro destaque
para o papel concomitante do poder judiciário em omitir"-se no controle
dos atos parlamentares e em praticar atos de abuso de poder e de
violação dos marcos legais e constitucionais, aspecto este combinado à
realização de uma justiça de transição parcial e bloqueada nas pautas da
responsabilização e da reforma das instituições, que seguiu de
perto\footnote{Paulo Abrão e Marcelo Torelly apresentam em minúcias o
  argumento que o processo justransicional brasileiro foi conduzido pelo
  processo da anistia. Ver: \versal{ABRÃO}, Paulo; \versal{TORELLY}, Marcelo D. O programa
  de reparações como eixo estruturante da justiça de transição no
  Brasil. In: \versal{REÁTEGUI}, Félix (Org.). \textbf{Justiça de Transição} -
  manual para a América Latina. Brasília: Comissão de Anistia; New York:
  International Center for Transitional Justice, 2011. p.473-516.} a
própria ambiguidade do processo de anistia no Brasil.

\section{Referências Bibliográficas}

\versal{ABRÃO}, Paulo; \versal{TORELLY}, Marcelo D. O programa de reparações como eixo
estruturante da justiça de transição no Brasil. In: \versal{REÁTEGUI}, Félix
(Org.). \textbf{Justiça de Transição} -- manual para a América Latina.
Brasília: Comissão de Anistia; New York: International Center for
Transitional Justice, 2011. p.473-516.

\versal{ALVES}, Maria Helena Moreira. \textbf{Estado e oposição no Brasil
(1964-1984)}. 3.ed. Petrópolis: Vozes, 1984.

\versal{ASSUMPÇÃO}, Eliane Maria Salgado (org.). \textbf{O Direito na História:}
o caso das mãos amarradas. Porto Alegre: \versal{TRF} 4a. Região, 2008.

\versal{BARBÉ}, Carlos. Golpe de Estado. In: \versal{BOBBIO}, Norberto; \versal{MATTEUCCI}, Nicola;
\versal{PASQUINO}, Gianfranco (orgs.). \textbf{Dicionário de Política.} 5.ed.
Tradução de Carmen C. Varrialle\ldots{} {[}et al{]}. Brasília: Universidade
de Brasília, 1993.p.545-547.

\versal{BRASIL}. Secretaria Especial dos Direitos Humanos. Comissão Especial
sobre Mortos e Desaparecidos Políticos. \textbf{Direito à verdade e à
memória}. Brasília: Secretaria Especial dos Direitos Humanos, 2007.

\versal{CITTADINO}, Gisele. \textbf{Poder judiciário, ativismo judicial e
democracia}. Alceu (\versal{PUCRJ}), v. 5, \versal{N}º 9, p. 105-113, jul./dez. 2004.

\versal{CITTADINO}, Gisele; \versal{PRONER}, Carol; \versal{RAMOS} \versal{FILHO}, Wilson; \versal{TENEMBAUM}, Marcio
(Orgs.). \textbf{A resistência ao golpe de 2016}. Bauru: Canal 6, 2016.

\versal{COELHO}, João Gilberto Lucas. A Garantia das Instituições.
\textbf{Caderno \versal{CEAC}/UnB}, Ano 1, \versal{N}º1, 1987. p.37-45. {[}Constituinte:
temas em análise{]}.

\versal{COELHO}, Maria José H.; \versal{ROTTA}, Vera (orgs.). \textbf{Caravanas da
Anistia:} o Brasil pede perdão. Brasília: Ministério da Justiça;
Florianópolis: Comunicação, Estudos e Consultoria, 2012.

\versal{GRECO}, Heloísa Amélia. \textbf{Dimensões fundacionais da luta pela
Anistia}. 2009. 456f. {[}Tese de Doutorado{]} -- Curso de Pós"-Graduação
das Faculdades de Filosofia e Ciências Humanas da Universidade Federal
de Minas Gerais. Belo Horizonte. 2003.

\versal{MAUS}, Ingeborg. Judiciário como superego da sociedade -- o papel da
atividade jurisprudencial na ``sociedade orfã''. Tradução de Martonio Lima
e Paulo Albuquerque. \textbf{Novos Estudos}, \versal{N}º58, p.183-202. nov. 2000.

\versal{PEREIRA}, Anthony W. \textbf{Ditadura e repressão}: o autoritarismo e o
estado de direito no Brasil, no Chile e na Argentina. São Paulo: Paz e
Terra, 2010.

\versal{PÉREZ"-LÍÑÁN}, Anibal. \textbf{Juício político al presidente y nueva
inestabilidad política en América Latina}. Buenos Aires: Fondo de
Cultura Economica, 2009.

\versal{RODEGHERO}, Carla Simone; \versal{DIENSTMANN}, Gabriel; \versal{TRINDADE}, Tatiana.
\textbf{Anistia ampla, geral e irrestrita}: história de uma luta
inconclusa. Santa Cruz do Sul: \versal{EDUNISC}, 2011.

\versal{SANTOS}, Roberto Lima; \versal{BREGA} \versal{FILHO}, Vladimir. Os reflexos da
"judicialização" da repressão política no Brasil no seu engajamento com
os postulados da justiça de transição. In: \textbf{Revista Anistia
Política e Justiça de Transição}. Brasília, \versal{N}º1, p.152-177, jan./jun.
2009.

\versal{SCHINKE}, Vanessa Dorneles. \textbf{Judiciário e autoritarismo}: regime
autoritário (1964-1985), democracia e permanências. Rio de Janeiro:
Lumen Juris, 2016.

\versal{SPIELER}, Paula; \versal{QUEIROZ}, Rafael Mafei Rabelo (Coords.).
\textbf{Advocacia em tempos difíceis} -- ditadura militar 1964-1985.
Curitiba: edição do autor, 2013.

\versal{SERRANO}, Pedro Estevam Alves Pinto. \textbf{Autoritarismo e golpes na
América Latina} -- breve ensaio sobre jurisdição e exceção. São Paulo:
Alameda, 2016.

\versal{SILVA} \versal{FILHO}, José Carlos Moreira da. \textbf{Justiça de Transição} -- da
ditadura civil"-militar ao debate justransicional -- direito à memória e à
verdade e os caminhos da reparação e da anistia no Brasil. Porto Alegre:
Livraria do Advogado, 2015.

\versal{SILVA} \versal{FILHO}, José Carlos Moreira da. A Comissão de Anistia e a
concretização da justiça de transição no Brasil -- repercussão na mídia
impressa brasileira: jornal O Globo, 2001 a 2010.~\versal{SILVA} \versal{FILHO}, José
Carlos Moreira da; \versal{ABRÃO}, Paulo; \versal{TORELLY}, Marcelo D. (Orgs).
\textbf{Justiça de Transição nas Américas} -- olhares interdisciplinares,
fundamentos e padrões de efetivação. Belo Horizonte: Fórum, 2013.

\versal{SKIDMORE}, Thomas. \textbf{Brasil:} de Castelo a Tancredo. 8.ed. Rio de
Janeiro: Paz e Terra, 1988.

\versal{SOUSA} \versal{JUNIOR}, José Geraldo de. Soberania e Direitos: processos sociais
novos? \textbf{Caderno \versal{CEAC}/UnB}, Ano 1, \versal{N}º1, 1987. p.9-16.
{[}Constituinte: temas em análise{]}.

\versal{SOUSA} \versal{JUNIOR}, José Geraldo de. Triste do Poder que não pode.
\textbf{Caderno \versal{CEAC}/UnB}, Ano 1, \versal{N}º1, 1987. p.25-31. {[}Constituinte:
temas em análise{]}.

\versal{SOUSA} \versal{JUNIOR}, José Geraldo de; \versal{SILVA} \versal{FILHO}, José Carlos Moreira da;
\versal{PAIXÃO}, Cristiano; \versal{FONSECA}, Lívia Gimenes Dias da; \versal{RAMPIN}, Talita
Tatiana Dias (Orgs.). \textbf{O Direito Achado na Rua}: Introdução
Crítica à Justiça de Transição na América Latina. Brasília: UnB, 2015.

\versal{ZAFFARONI}, Eugenio Raúl. \textbf{Poder Judiciário}: crise, acertos e
desacertos. São Paulo: Revista dos Tribunais, 1995.
