\chapter*{Prefácio}
\addcontentsline{toc}{chapter}{Prefácio, \scriptsize{por Paulo Pimenta}}

\hedramarkboth{Prefácio}{}

\begin{flushright}
\emph{Paulo Pimenta}\footnote{Jornalista, técnico agrônomo, deputado
federal e líder do \versal{PT} na Câmara dos Deputados (2018-2019).}
\end{flushright}

Trinta anos após a Constituição mais avançada que o Brasil construiu --
e, ainda assim, bem aquém do que pretendia e merecia o povo brasileiro
no processo constituinte -- ao longo de quase dois séculos enquanto
nação independente, vivemos um cenário muito semelhante ao da Velha
República.

A consolidação da modernidade civilizatória esperada para o século \versal{XXI}
foi freada e revertida bruscamente por um novo pacto das elites,
fenômeno que já vivenciamos algumas vezes na nossa história. Voltamos a
ser governados, como na virada do século \versal{XIX} para o \versal{XX}, por prepostos
dos ``coronéis'' da Casa Grande, devidamente vigiados e protegidos por
jagunços, que hoje também atendem pela alcunha de milicianos.

Aos inimigos da ``ordem pública'', que reivindicam alternativas
políticas ao \emph{status quo}, a solução empregada é a polícia, tal
qual como na época dos oligarcas do café e do leite. Agentes públicos de
alguns setores do Judiciário, do Ministério Público e da Polícia Federal
-- especialmente magistrados, procuradores e policiais envolvidos na
Operação Lava Jato -- são apenas a versão contemporânea das forças de
repressão dos nossos primeiros tempos de República.

Está mais atual do que nunca a frase atribuída ao presidente Washington
Luís: ``A questão social é caso de polícia''. Se depender do bloco de
poder que governa o Brasil hoje, movimentos sociais -- cujas
contribuições para um mundo mais justo, democrático e solidário são
reconhecidas internacionalmente -- serão tratados como ``terroristas''.

A presente obra, com a qual honrosamente contribuo por meio deste
prefácio, é um documento histórico de extrema relevância e, mais do que
isso, trata"-se de libelo em defesa da democracia, do Estado de Direito e
das garantias e liberdades fundamentais, hoje ameaçadas pela associação
perversa entre o neofascismo e a agenda ultraneoliberal posta em curso
desde o afastamento ilegal da presidenta Dilma Rousseff.

Desde 2015, com a assunção de Eduardo Cunha à presidência da Câmara dos
Deputados, esta Casa tornou"-se a arena central da disputa política
brasileira. Na condição de parlamentar, inclusive de líder da Bancada do
\versal{PT} em 2018 e 2019, pude testemunhar de muito perto todas as manobras
feitas para derrocar o governo da presidenta Dilma Rousseff, para atacar
o significado político e histórico do Partido dos Trabalhadores e, pior
de tudo, para destruir todo o alicerce de direitos trabalhistas e
sociais conquistado pela sociedade brasileira em mais de um século de
lutas.

Os discursos que abrem esta coletânea, composta de textos de importantes
lutadores e protagonistas da história do Brasil, são leituras
obrigatórias para se compreender o caminho percorrido pelo país nas
últimas três décadas.

Em seu pronunciamento emblemático de uma era que tinha início a 5 de
outubro de 1988, Ulysses Guimarães ressaltava o valor da coragem,
afirmando que ela ``é a matéria"-prima da civilização''. Essa virtude foi
muito bem personificada por Dilma Rousseff ao enfrentar cara a cara, sem
subterfúgios, em sessão do Senado Federal, aqueles que não traíram
apenas o seu mandato conquistado nas urnas, mas, sobretudo, a expressão
legítima e soberana da vontade popular depositada nas urnas em outubro
de 2014.

``Se alguns rasgam o seu passado e negociam as benesses do presente, que
respondam perante a sua consciência e perante a história pelos atos que
praticam'', registrou a presidenta naquele 29 de agosto de 2016. A sua
consciência estava -- e segue até hoje -- tranquila diante da
conspiração armada por diversos atores políticos que não podiam mais
aceitar que seus interesses, subalternos e entreguistas em âmbito
internacional, fossem deixados de lado para que prevalecessem os anseios
do povo brasileiro, corporificados nos princípios e direitos sociais
inscritos na Carta Magna.

Este documento promulgado a 5 de outubro de 1988 teve a contribuição
decisiva e direta da Bancada do Partido dos Trabalhadores. Embora
contasse com apenas 16 parlamentares, liderados por Luiz Inácio Lula da
Silva, foi graças à intensa e qualificada atuação aos petistas, sempre
articulados e fortalecidos pela sociedade civil, que foram aprovados os
artigos que promovem direitos da cidadania frente aos objetivos
mercantis do que Ulysses Guimarães apontou como ``campanha mercenária''
daqueles que tinham ``suas burras abarrotadas com o ouro de seus
privilégios e especulações''.

O deputado Lula, eleito para a Assembleia Nacional Constituinte com a
maior votação do País, mostrou ao Brasil que era possível fazer política
tendo como bússola a superação dos problemas crônicos de uma sociedade
extremamente desigual e injusta.

Anos depois, e após três derrotas eleitorais, o presidente Lula
transformou em realidade a esperança de dezenas de milhões de
brasileiras e brasileiros que sonhavam por uma vida melhor. Renda mínima
para garantir o direito à alimentação e erradicar a fome; universidades
e escolas técnicas federais; energia elétrica em todos os rincões do
País; moradia digna subsidiada pelo Estado; crédito facilitado para a
agricultura familiar; infraestrutura para alavancar a economia e o
desenvolvimento regional; crescimento sólido do \versal{PIB} com acelerada
distribuição de renda e diminuição da extrema pobreza, entre tantas
outras ações, premiadas e replicadas no mundo inteiro, são marcas
indeléveis do legado de Luiz Inácio Lula da Silva como presidente da
República.

O maior líder popular da história política brasileira foi também o mais
importante chefe de Estado que o país já teve. Essa opinião é aceita até
mesmo por adversários políticos. Mais do que isso, tal tese é amplamente
reconhecida pela comunidade internacional e pela população que foi
diretamente beneficiada pelas políticas públicas dos governos Lula e
Dilma.

A mesma percepção se dá quanto às condenações sem provas -- bem como a
prisão ilegal e toda uma série de arbitrariedades -- impostas contra o
ex"-presidente, vítima do lawfare, estratégia política de perseguição
jurídica que vem sendo largamente utilizada na América Latina na última
década e tem como objetivos centrais a destruição da imagem pública e o
cerceamento de direitos políticos de lideranças populares.

A perseguição contra Lula é abordada de forma profunda e detalhada neste
livro, mas vale parafrasear Ulysses Guimarães, que citou ao final do seu
discurso icônico na promulgação da Constituição o deputado Rubens Paiva,
sequestrado pela ditadura militar inaugurada em 1964 e desaparecido
político: ``A sociedade é Lula, não os facínoras que o prenderam''.

Boa leitura, que será também uma cátedra de História, de Ciência
Política e de Direito, dada a qualidade dos autores dos textos contidos
nesta obra.
