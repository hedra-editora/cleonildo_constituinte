\textbf{DESDOBRAMENTOS HERMENÊUTICO-CONSTITUCIONAIS DO
\emph{IMPEACHMENT} DA EX-PRESIDENTE DILMA ROUSSEF NO CONTEXTO DA
``PÓS-DEMOCRACIA''}

\textbf{*BRUNO GALINDO}

\emph{Professor Associado da Faculdade de Direito do Recife/Universidade
Federal de Pernambuco (UFPE)}

\emph{Doutor em Direito pela UFPE/Universidade de Coimbra-Portugal
(PDEE)}

\emph{Conselheiro Estadual da OAB/PE}

\begin{enumerate}
\def\labelenumi{\arabic{enumi}.}
\item
  \textbf{As tentativas de um ``estado da arte'' no 30º aniversário da
  Assembleia Constituinte 1987-1988: do Estado democrático de direito à
  ``pós-democracia''}
\end{enumerate}

Em meio a todas as dificuldades que nosso país vem passando na política
e no direito, não poderia ser em melhor hora a publicação dessa obra
coordenada por Cleonildo Cruz e Liana Cirne Lins. Foi com imensa honra e
felicidade que recebi o convite para contribuir com tão importante e
oportuna obra.

Escrever análises em tempos como o nosso é sempre muito arriscado. Estar
no ``olho do furacão'' traz a dificuldade de não se ter um
distanciamento suficiente para analisar com precisão os fenômenos de seu
entorno. Entretanto, a academia não pode se furtar a fazê-lo, sob pena
de se tornar hermética e levada a devaneios metafísicos sem correlação
com a realidade.

É com esse espírito que pretendo traçar as linhas que se seguem.

Cabem precipuamente algumas indagações: temos realmente o que comemorar
nesse 30º aniversário da instalação da Assembleia Constituinte
1987-1988? O Estado democrático de direito preconizado pela Carta
promulgada em 5 de outubro de 1988 ainda resta incólume? Estaríamos
vivendo uma ``pós-democracia'' (com aproximações à perspectiva de Pierre
Dardot e Christian Laval -- cf. Dardot \& Laval: 2016, pp. 379ss.) ou
uma ``pausa democrática'' (nos dizeres do Ministro aposentado do STF
Carlos Ayres Britto\footnote{\url{http://politica.estadao.com.br/noticias/geral,brasil-vive-pausa-democratica-para-freio-de-arrumacao--diz-ayres-britto,10000027535},
  acesso: 28/08/2017.}) com a construção de uma espécie de
``hermenêutica de resultados'' em lugar dos postulados interpretativos
do Estado democrático de direito?

Tendo como pano de fundo o recente processo de \emph{impeachment} que
resultou no afastamento definitivo da ex-Presidente Dilma Roussef em
agosto de 2016, tentemos esboçar algumas reflexões a respeito das
indagações.

Desde o ano passado quando publiquei meu livro ``\emph{Impeachment} à
luz do constitucionalismo contemporâneo'', venho acompanhando e
refletindo sobre as consequências hermenêuticas do referido processo,
concluído em 31 de agosto de 2016.

Destaco que, dentre inúmeras situações inusitadas e controversas
ocorridas durante o processo, o seu desfecho produziu mais duas.

A primeira: na Sessão de Julgamento, foi acolhido requerimento para
votação em separado quanto à pena a ser aplicada, bifurcando a pena
prevista no art. 52, parágrafo único, da Constituição, para considerar a
perda do cargo em um escrutínio e a suspensão dos direitos políticos por
8 anos em outro. Considerada a divisão, os senadores decidiram pela
maioria constitucionalmente exigida a favor da aplicação da primeira
pena e da rejeição da segunda, fazendo uma espécie de dosimetria do
conteúdo do dispositivo, algo não expressamente previsto em seu texto.

A segunda: no dia 1 de setembro, tão somente um dia após a conclusão do
processo, o Presidente da Câmara dos Deputados Rodrigo Maia, no
exercício temporário da Presidência da República, sanciona a Lei
13332/2016, publicada no dia seguinte. Esta Lei amplia consideravelmente
as possibilidades do Presidente da República editar decretos de
suplementação orçamentária sem autorização prévia do Congresso Nacional.
Um dia antes, a Presidente Dilma Roussef foi destituída do cargo por ter
sido considerada culpada -- além das chamadas ``pedaladas fiscais'' no
caso do Plano Safra - pelo crime de responsabilidade de: editar 3
decretos de suplementação orçamentária sem autorização prévia do
Congresso Nacional.

Muitos afirmam que estaríamos diante de um verdadeiro golpe de Estado
disfarçado, percepção reforçada pela divulgação dos áudios telefônicos
dos diálogos entre o Senador Romero Jucá e Sérgio Machado, assim como a
entrevista do atual Presidente da República Michel Temer afirmando que o
então Presidente da Câmara dos Deputados Eduardo Cunha teria arquivado a
Denúncia se tivesse obtido os votos favoráveis do Partido dos
Trabalhadores na Comissão de Ética daquela Casa\footnote{\url{http://exame.abril.com.br/brasil/dilma-usara-entrevista-de-temer-como-prova-contra-impeachment/},
  acesso: 17/04/2017.}. Outros, que estamos em situação de plena
normalidade democrática e que, apesar da crise política, as instituições
estão funcionando e há melhorias institucionais, especialmente no
combate à corrupção e à impunidade de criminosos. As análises políticas
são eivadas de passionalismo e, verdade seja dita, não é simples
refletir sobre um fenômeno em meio a ele, quando os acontecimentos estão
em curso. As possibilidades de falhas aumentam exponencialmente.

O único diagnóstico possivelmente consensual é o de que estamos vivendo
um momento de profundas incertezas jurídicas e políticas. No campo
jurídico, aparentemente a gradativa edificação dogmática e hermenêutica
constitucional de mais de duas décadas de vigência da Constituição de
1988 está abalada por casuísmos interpretativos e abruptas rupturas
paradigmáticas. E o processo de \emph{impeachment} da Presidente Dilma é
provavelmente o epicentro dessas questões, embora elas resvalem em
outros campos da interpretação constitucional, a exemplo das mutações
constitucionais em torno da presunção de não culpabilidade e das
inversões processuais e procedimentais em prejuízo do réu dentro do
contexto da denominada ``Operação Lava Jato''.

Nessa perspectiva, é necessária uma leitura hermenêutica e teórica do
delineamento constitucional do \emph{impeachment} no Brasil e sua
(des)continuidade diante do Caso Dilma Roussef. Este ensaio tenta,
portanto, apontar algumas possibilidades de compreensão fenomênica sem
deixar de alertar para os graves problemas decorrentes do abandono de
importantes postulados hermenêuticos do Estado democrático de direito.

\begin{enumerate}
\def\labelenumi{\arabic{enumi}.}
\item
  \textbf{\emph{Impeachment}: sobre antigas e novas possibilidades
  hermenêuticas}
\end{enumerate}

O que comumente tem sido denominado \emph{impeachment} (impedimento, em
tradução literal do inglês) a partir da origem anglo-americana do
instituto, nada mais é do que uma forma constitucional de destituição de
um detentor de poder político através de um procedimento jurídico
específico com fundamentos formais e materiais na própria constituição.
A princípio, não se trata de golpe de Estado ou ruptura institucional,
apesar de, em termos de análise política, ser possível a utilização
deste e de outros mecanismos constitucionalmente previstos para
disfarçar rupturas materiais e substantivas com a constituição e com o
ordenamento jurídico subjacente (cf. Martinez: 2013; Pérez-Liñan: 2007,
pp. 49ss.; Holmes: 2016a).

Antes do processo de \emph{impeachment} da Presidente Dilma Roussef, a
tendência teórica predominante apontava a uma natureza mista
(político-administrativa e criminal) do impedimento constitucional.
Agora, a dúvida sobre sua natureza jurídica, bem como se seu caráter é o
de um instrumento predominante ou exclusivamente político ou judicial
penal, inevitavelmente aumenta, pois os dados da realidade
constitucional não podem ser ignorados em sua interpretação, como
adverte há certo tempo a metódica estruturante de Friedrich Müller
(Müller: 2000, pp. 100ss.; Galindo: 2003, pp. 152ss.). No Brasil, isso
se dá, dentre outras coisas, pela denominação e formas estipuladas para
tal processo, bastante associadas às formas judiciais e terminologias
correlatas (no Brasil, o termo ``crimes de responsabilidade'', usado
para infrações que, em princípio, seriam político-administrativas), e,
por outro lado, pelo fato de a competência jurisdicional nesse processo,
ser excepcionalmente conferida a colegiados políticos em sentido
estrito, já que esses julgamentos são feitos por casas parlamentares
investidas de funções jurisdicionais e não por tribunais e juízes
regulares.

Fazer uma leitura hermenêutica constitucional do \emph{impeachment}
exige uma reflexão sobre como os elementos clássicos e contemporâneos da
interpretação jurídica estão presentes nas possibilidades de sua
concretização constitucional. Para tal, precisamos de dogmática
hermenêutica e de sólida fundamentação em termos de teoria da
interpretação jurídica (Ferraz Jr.: 2003, pp. 255ss.). Do contrário,
conquistas civilizatórias importantes do constitucionalismo, como a
força normativa da constituição, a segurança jurídica, a máxima
efetividade dos direitos fundamentais e a supremacia de princípios como
o democrático e o republicano sucumbirão ao ``canto das sereias''
hermenêutico da volatilidade argumentativa da política ou da moral, ao
poder momentâneo, ao argumento de autoridade e, no limite, a novas e
veladas formas de autocracia. Se não se quer voltar à constituição
``folha de papel'' e a prevalência pura e simples dos ``fatores reais do
poder'', tal como já denunciara Ferdinand Lassalle no século XIX, ou o
retrocesso de uma transformação da Carta de 1988 em uma constituição
semântica no sentido loewensteiniano (Lassalle: 1998, p. 51;
Loewenstein: 1964, pp. 218-219; Galindo: 2015, pp. 98-99).

Nesse contexto, destaquei em meu estudo que o \emph{impeachment} se
afigura como instrumento essencialmente político-criminal no contexto
constitucional brasileiro, fazendo-se necessária a demonstração jurídica
do cometimento de crime comum ou de responsabilidade como justa causa
para o processo. Tal fundamento é imprescindível, pois o
\emph{impeachment} não serve para solução de desavenças políticas ou de
substituição da disputa eleitoral, bem como não é substitutivo de voto
de desconfiança ou de referendo revogatório/\emph{recall} (Bahia, Silva
\& Oliveira: 2016, p. 34). Do mesmo modo, procedimentalmente há o
exercício do contraditório e da ampla defesa e a realização de juízos de
admissibilidade, pronúncia e mérito, tal como em processos penais em
geral, seguindo o devido processo legal, não podendo ser esse processo
algo meramente ritualístico e amorfo. Em termos substantivos, para que
se configure a justa causa, é necessário observar a questão da
tipicidade, pois, ao menos em tese, o ato deve ser típico, antijurídico
e culpável, ainda que na modalidade de crime de responsabilidade.

Todavia, com o Caso da ex-Presidente Dilma Roussef, é possível que a
leitura hermenêutica feita precise sofrer uma revisão analítica.

\begin{enumerate}
\def\labelenumi{\arabic{enumi}.}
\item
  \textbf{O \emph{impeachment} da Presidente Dilma Roussef e o ``canto
  das sereias'' hermenêutico: mutação constitucional à paraguaia?}
\end{enumerate}

O recente processo de \emph{impeachment} se deu em um contexto de
abruptas e profundas alterações em entendimentos, compreensões e
conceituações acerca do fenômeno diante das referências clássicas e
contemporâneas do constitucionalismo e da dogmática jurídica em geral.
Isso traz grandes desafios adicionais a uma teoria constitucional do
\emph{impeachment.}

Em termos gerais, não se pode ignorar que o \emph{impeachment} quase
sempre surge a partir de componentes políticos de grande insatisfação
com o governo por parte de vários setores políticos, econômicos e
sociais. Não foi diferente nos casos brasileiros, tanto de Fernando
Collor como de Dilma Roussef, coincidindo com a ausência de êxitos
econômicos, gestões administrativas problemáticas e denúncias de
corrupção no governo. As profundas dificuldades de negociação política
com o Congresso Nacional também foram evidenciadas em ambos os casos, o
que ocasionou drástica diminuição no apoio do Parlamento aos Presidentes
em questão.

As semelhanças, contudo, não vão muito além dessas. Em termos
constitucionais, os dois processos foram bem diversos. Enquanto no Caso
Collor, as controvérsias foram mais voltadas a questões de fundo e não
dividiram significativamente a comunidade jurídica, o Caso Dilma foi
extremamente polêmico, causando divisões fractais a partir da própria
existência ou não de justa causa ao processo. Em termos: a própria
configuração dos atos da Presidente como crimes de responsabilidade
segue sendo controversa, dadas as frequentes variações hermenêuticas
realizadas pelos apoiadores do \emph{impeachment} para justificá-lo
juridicamente, e, de outro lado, a reação de boa parte da comunidade
jurídica, que se voltou contra tais justificativas.

Politicamente, desde o início do segundo mandato da Presidente Dilma
Roussef, a sua condenação em um processo de \emph{impeachment} foi
defendida por setores da oposição (cf. Martins: 2015, pp. 16-18). A
constante deterioração nas relações políticas com o Congresso Nacional,
as frequentes avarias à imagem do Governo nas investigações da Operação
Lava Jato, aliadas às grandes dificuldades no campo econômico,
mantiveram politicamente acesa a chama de um processo de impedimento. E,
em termos jurídicos, isso ganhou considerável força com o Parecer Prévio
do Tribunal de Contas da União que recomendou ao Congresso Nacional a
reprovação das contas do Governo referentes ao exercício de 2014 face às
chamadas ``pedaladas fiscais''. Tal opinativo, embora até o momento
(agosto de 2017) não tenha sido aprovado pelo Congresso, foi um dos
fundamentos da Denúncia protocolada pelos juristas Hélio Bicudo, Janaína
Paschoal e Miguel Reale Jr. (Bicudo; Reale Jr. \& Paschoal: 2015, p.
62). Contudo, o recebimento da Denúncia foi em termos mais restritos,
limitando-se às ``pedaladas'' (prática ilegal de desinformações
contábeis e fiscais) do ano de 2015 (Caso específico do Plano Safra) e
de 6 Decretos de suplementação orçamentária sem autorização prévia do
Congresso Nacional, em aparente violação à CF, art. 85, V a VII, e à Lei
1079/1950, arts. 4º, V e VI; 9º, 3 e 7; 10, 6 a 9; e 11, 3.

Uma das dificuldades desse processo sempre foi a de realizar a devida
filtragem hermenêutica dos dispositivos legais, interpretando-os à luz
da Constituição e do Estado democrático de direito. Analisar a
consistência dos principais argumentos da Denúncia recebida implica em
contextualizá-los adequadamente, inclusive quanto aos seus elementos
interpretativos, em especial os elementos histórico, genético,
teleológico e sistemático. Ademais, a se considerar a genealogia da Lei
1079/1950 (o seu ``DNA parlamentarista'', em particular -- cf. Queiroz:
2015), sua textura excessivamente aberta e os aspectos de direito
comparado da questão e concluindo pela natureza político-criminal do
processo de \emph{impeachment}, em termos constitucionais, impõe-se uma
interpretação restritiva da Lei dos crimes de responsabilidade (Galindo:
2016, pp. 81-82).

Contudo, os desdobramentos concretos do processo em relação à
ex-Presidente Dilma em quase nada seguiram essa leitura hermenêutica
constitucional.

As ``pedaladas'' de 2015 referentes ao chamado Plano Safra foram um dos
fundamentos jurídicos do \emph{impeachment.} É de se registrar que à
época da conclusão do processo, em agosto de 2016, sequer havia sobre as
ditas ``pedaladas'' Parecer Prévio opinativo da Corte de Contas, nem
análise do Congresso Nacional. Aliás, outro Parecer, o do Ministério
Público Federal, chega a expressamente afirmar a necessidade de, em
relação à Tomada de Contas de 2015, ``se aguardar o andamento dos
trabalhos do TCU antes de se emitir uma opinião sobre a configuração
penal dos fatos e delimitação de responsabilidades''.\footnote{\url{http://www.mpf.mp.br/df/sala-de-imprensa/docs/arquivamento-pedaladas-pic},
  p. 34. Acesso: 10/09/2016.}

O Parecer Ministerial indica ainda que no caso dos potenciais crimes
tipificados no art. 359-A do Código Penal, de modo bastante assemelhado
às regras previstas na Lei de Responsabilidade Fiscal e na Lei
1079/1950, não poderia haver ampliação do conceito de ``operação de
crédito'', dentre outras coisas, em face do princípio da legalidade
estrita e da proibição da analogia \emph{in malam parte}. Assevera ainda
a ausência de dolo e até mesmo a semelhança entre as ``pedaladas''
praticadas ao longo dos anos.\footnote{http://www.mpf.mp.br/df/sala-de-imprensa/docs/arquivamento-pedaladas-pic,
  pp. 29-31. Acesso: 10/09/2016.}

Em verdade, isso já fora dito por vários renomados especialistas do
cenário jurídico nacional, como Misabel Machado Derzi, Heleno Torres,
Carlos Valder Nascimento, Ricardo Lodi Ribeiro, Geraldo Prado, Juarez
Tavares e vários outros (Ribeiro: 2015a; Ribeiro: 2015b; Prado \&
Tavares: 2015). O próprio Min. Augusto Nardes, Relator do Parecer Prévio
que recomendou a reprovação das Contas de 2014, destacou em entrevista
dada ao Jornal ``O Estado de São Paulo'' que se trata de uma efetiva
mudança paradigmática da jurisprudência do TCU.\footnote{\url{http://economia.estadao.com.br/noticias/geral,sera-muito-dificil-justificar-distorcoes--diz-ministro-do-tcu--imp-,1718369},
  acesso: 21/12/2015.}

Em casos como esses, em que o crime de responsabilidade é, no mínimo,
algo duvidoso, a leitura hermenêutica constitucional exige do julgador a
utilização de técnicas jurisprudenciais como o \emph{prospective
overruling}, que implica, com a mudança no precedente, uma mudança na
solução para os casos futuros, como um alerta de que o TCU não mais
toleraria a partir dali as ditas ``pedaladas fiscais'', sem, no entanto,
retroagir com o novo entendimento, sendo tal técnica perfeitamente
aplicável em casos como o das decisões e pareceres da Corte de Contas
(Guedes \& Pádua: 2015; cf. tb. Souza: 2006, pp. 160ss.).

No caso da abertura de créditos suplementares sem autorização legal,
parece igualmente desacertado considerá-la como fundamento a um processo
de \emph{impeachment}, como é controversa até mesmo a própria ideia de
que a abertura no caso tenha sido de fato à revelia da Lei Orçamentária
Anual. Por mais que seja uma execução orçamentária problemática, as
revisões de metas ocorridas durante a execução do orçamento como fator
tacitamente autorizativo de abertura de créditos suplementares por
decreto, já que o próprio Congresso Nacional, responsável pela aprovação
da lei orçamentária, modifica a meta fiscal \emph{a posteriori}, tendo
sido, aliás, o que ocorreu no caso da ex-Presidente, acarretando a
convalidação legal dos referidos Decretos (cf. Ribeiro: 2015a).

Ainda mais grave se afigura considerá-los como crime de responsabilidade
se for considerado o fato de que 3 dos 6 Decretos inicialmente em
questão foram excluídos do objeto do processo após minuciosa perícia
técnica realizada pelo próprio Senado Federal. Foi verificado no laudo
pericial que o valor total das suplementações foi de apenas R\$ 2,3
bilhões. Assim, da denúncia original, cujos decretos somavam R\$ 96
bilhões, apenas R\$ 2,3 bilhões permaneceram sob suspeita de terem
ofendido a meta fiscal. Os 3 Decretos remanescentes, portanto,
representaram apenas 0,1\% da despesa total.

Fazer uma leitura hermenêutica constitucional do \emph{impeachment}
implica em considerar alguns imprescindíveis elementos
político-constitucionais e de teoria da constituição. A tripla opção
histórica do próprio povo pelo sistema presidencialista de governo
(plebiscitos em 1963 e 1993 e Assembleia Constituinte em 1987-1988)
exige ao intérprete considerar esse sistema de governo, em que há a
responsabilidade presidencial republicana, mas o chefe de Estado, com
legitimação popular direta, não pode ficar vulnerável a maiorias
parlamentares ocasionais. Igualmente, não há em nosso sistema
presidencialista, por opção do constituinte (originário e derivado),
mecanismos como o referendo revogatório de mandato ou a autorização
constitucional do \emph{impeachment} por ``mau desempenho'', como em
Constituições como a colombiana, a boliviana ou a paraguaia. Em tal
perspectiva, não se afigura razoável que a leitura hermenêutica da Carta
de 1988 possa admitir que todo e qualquer ato presidencial que, p. ex.,
atente contra o cumprimento da lei ou o exercício de um direito
individual seja automaticamente passível de \emph{impeachment}. Muitas
vezes, em torno de uma lei de constitucionalidade discutível o seu
descumprimento poderia ser até mesmo uma atitude governamental de
preservação da Constituição, o que pode vir a ser confirmado pela
Suprema Corte caso, em julgamento definitivo de mérito, reconheça a
inconstitucionalidade da lei descumprida.\footnote{Tal situação
  ironicamente ocorreu em relação ao primeiro caso de \emph{impeachment}
  que chegou a ser julgado nos EUA, em que o então Presidente Andrew
  Johnson escapou da condenação por um voto, em 1867. Posteriormente,
  lei de conteúdo bastante semelhante à que Johnson fora acusado de
  descumprir foi declarada inconstitucional pela Suprema Corte daquele
  país, em 1926, no Caso \emph{Myers x United States} (Galindo: 2016, p.
  30; Tribe: 2000, pp. 176-178; Ackerman: 2001, pp. 178ss.).}

Em adendo, afirme-se que atos ilegais ou inconstitucionais
cotidianamente são editados pelos poderes executivos nos 3 níveis da
Federação. Mas para eles, o sistema normativo possui meios regulares de
impugnação das ilegalidades e inconstitucionalidades, como o controle de
constitucionalidade, o controle judicial da legalidade e até mesmo a
sustação dos atos normativos do poder executivo que exorbitem do poder
regulamentar (CF, art. 49, V). Em tais situações, é claro que o
presidente da República agiu de modo ilegal ou inconstitucional, mas nem
por isso se cogita fundamentar um processo de impedimento contra o chefe
do executivo nacional (Santos: 2015; Araújo, Santos \& Teixeira: 2015;
Neves: 2015, p. 31).

Diante do exposto, vê-se que é extremamente controverso caracterizar as
``pedaladas fiscais'' do Plano Safra e os 3 Decretos de suplementação
orçamentária, ao final convalidados pela Lei 13199/2015, como crimes de
responsabilidade (Ribeiro: 2015b, pp. 25-26). A sua reiterada e tolerada
prática por governos anteriores e mesmo pelo da ex-Presidente Dilma no
primeiro mandato, ainda que com diferenças quanto ao seu aspecto
sistemático e quantitativo, não autoriza a sua inclusão no rol dos
crimes de responsabilidade por via interpretativa em uma espécie de
analogia e retroatividade \emph{in} \emph{malam partem}. Não condiz com
o Estado democrático de direito adotar posições de ampliação
hermenêutica de hipóteses criminalizantes.

\begin{quote}
Apesar disso, as ``pedaladas'' referentes ao Plano Safra e os 3 Decretos
de suplementação orçamentária legalizados \emph{a posteriori} foram
formalmente admitidos e aceitos pelo Senado Federal como crimes de
responsabilidade e fundamentaram juridicamente a condenação da
ex-Presidente. Sem aplicação de um \emph{prospective overruling}, sem
observar a legalidade e a tipicidade estritas e promovendo analogia
\emph{in malam parte}, além da aplicação da retroatividade em prejuízo
da ré, tudo muito distante do arquétipo hermenêutico do Estado
democrático de direito e da Carta cidadã de 1988.
\end{quote}

Por tais razões, a ex-Presidente ingressou com mandado de segurança no
Supremo Tribunal Federal contra a decisão do Senado em destitui-la do
cargo. Ao mesmo tempo, outros atores políticos, especialmente partidos
que antes estavam na oposição, ingressaram com outros mandados de
segurança contra a aplicação ``fatiada'' das penas do art. 52, parágrafo
único, da CF. Em ambos os casos, os Relatores, Mins. Teori Zavascki e
Rosa Weber, negaram as liminares pleiteadas, mantendo, na íntegra, a
decisão do Senado em ambos os aspectos.

Entretanto, a decisão interlocutória dada pelo então Relator, o falecido
Min. Teori Zavascki, gera certa perplexidade, pois não somente exprime
uma já esperada posição política de autocontenção do STF,\footnote{Sobre
  os conceitos de ativismo e autocontenção, cf. Lima: 2014.} negando-se
a interferir na decisão de competência constitucional de outro poder,
mas adentra o mérito da questão, estipulando pela via interpretativa,
verdadeira mutação constitucional (Cf. Ferraz: 1986, pp. 56-57;
Canotilho: 2002, pp. 1214-1216) da construção histórica política e
jurisprudencial em torno do impedimento no sistema presidencialista de
governo, pois no horizonte interpretativo da referida decisão vê-se
pouca proximidade a uma hermenêutica jurídica afeita ao Estado
democrático de direito preconizado pela Constituição de 1988.

Curiosamente, o Ministro inicia a partir de uma premissa correta: a de
que a configuração isolada de uma das condutas da Lei 1079/1950 não
seria suficiente à tipificação do crime de responsabilidade, apontando
para a gravidade que teria que resultar da conduta típica. Porém, logo a
seguir, afirma que a tipificação não deve se cingir aos tipos mais
estritos oriundos do fechamento normativo do direito penal. Ao
contrário, deve ter a possibilidade de uma imputação subjetiva alargada
o suficiente para alcançar condutas diversas do Presidente da República
que possam ferir a Constituição, sem se ater ao que denominou de
``transposição acrítica'' dos padrões jurídicos do direito penal. É de
se observar especialmente essa passagem:

\begin{quote}
Ocorre que a configuração, isoladamente, de uma das condutas previstas
entre os arts. 5º e 11 da Lei 1.079/50, tampouco haverá de ser
necessariamente suficiente para resultar na decretação do impedimento de
um Presidente da República. A tipificação de um crime de
responsabilidade deve capturar uma realidade que vai muito além da
microdelinquência, para ser capaz de indicar um descompromisso grave com
as responsabilidades inerentes ao cargo de Presidente da República,
refletindo uma aguda perturbação de bens jurídicos cardeais para o
funcionamento da República e da Federação. Justamente por isso, ela não
deve mimetizar à risca a racionalidade aplicada nos domínios do direito
penal, que exige um fechamento normativo mais estrito das condutas
hipotetizadas pelos ``tipos incriminadores''. O ``tipo de
responsabilidade'', diferentemente, deve ser capaz de clinicar uma
espécie de realidade aumentada, provendo elementos que permitam uma
imputação subjetiva com suficiente clareza da conduta, sem perder a
sensibilidade para as consequências que decorreram deste ato para
preceitos fundamentais da Constituição Federal, dentre os quais aqueles
sediados nos incisos do art. 85 da CF. São estes os bens jurídicos
imediatamente tutelados pelas normas que definem os crimes de
responsabilidade e o processo de \emph{impeachment}, o que torna
inadequada a transposição acrítica, para esses institutos, do
estreitamento dogmático que caracteriza os padrões jurídicos do direito
penal, voltados à proteção de direitos pessoais fundamentais,
notadamente os relacionados à liberdade de ir e vir (STF, MS 34371-MC,
Rel. Min. Teori Zavascki, DJe 12/09/2016).
\end{quote}

Ao privilegiar tais aspectos, parece que estiveram ausentes do horizonte
hermenêutico do saudoso Ministro o princípio democrático, a soberania
popular e a proteção constitucional ao exercício de determinados cargos
contra a ingerência política indevida e sem presença de uma justa causa,
incluídas aí garantias relevantes da sociedade em relação a esses
cargos, como a vitaliciedade da magistratura, a estabilidade do servidor
público efetivo e o mandato dos representantes e governantes eleitos
pelo povo.\footnote{``O processo de crime de responsabilidade não pode
  atender ao interesse privatizado de um/alguns, uma vez que aquele que
  é acusado também possui a legitimidade advinda do voto popular. De
  igual forma, há que se lembrar de um princípio próprio ao sistema
  presidencialista, o da ``impossibilidade de censura legislativa do
  Presidente da República'', segundo o qual, no regime presidencialista,
  o Presidente -- assim como o Governador e o Prefeito -- não são
  responsabilizáveis ante o Parlamento, mas tão somente perante o
  público de cidadãos, pelo não cumprimento de planos/projetos de
  campanha ou pelo insucesso de políticas/ações tomadas, se essas não
  configurarem crime -- comum ou de responsabilidade. Assim é que crises
  econômicas ou aumento do desemprego, por exemplo, não são causas
  constitucionais de retirada do Presidente de seu cargo. O
  \emph{Impeachment}, assim, não é meio de eternização da disputa
  eleitoral e nem deve ser usado como substituto da crítica parlamentar
  sobre os atos do Poder Executivo.'' (Bahia, Silva \& Oliveira: 2016,
  p. 27).} São bens jurídico-constitucionais tão relevantes quanto todos
aqueles elencados no art. 85 da CF e não podem ser solenemente ignorados
na interpretação de um mecanismo como o \emph{impeachment}.

Na esteira da lição de Ingo Sarlet, faz sentido admitir conceitos
constitucionais materialmente abertos quando tratamos de direitos
fundamentais, especialmente a abertura preconizada por um dispositivo
como o art. 5º, § 2º, da Carta de 1988, considerando direitos
fundamentais implícitos e decorrentes do regime e dos princípios
constitucionais, exatamente para incluir o que não foi expressamente
previsto nos textos constitucionais, mas que se compatibilizam com suas
diretrizes normativas (Sarlet: 2006, pp. 92ss.). Contudo, permitir essa
``abertura material'' no sentido inverso, parece demasiado perigoso para
bens jurídico-constitucionais tão relevantes como os referidos.

Sendo a Lei 1079/1950 tão lacônica, o intérprete que tenha no horizonte
hermenêutico o Estado democrático de direito preconizado pela Carta de
1988 precisaria adotar exatamente o oposto do que fez o Ministro: primar
pelo respeito às garantias, à presunção constitucional de inocência
(\emph{in dubio pro reo}) e à soberania popular expressa nas eleições
periódicas, pois pelo demonstrado, o \emph{impeachment} presidencialista
não se desenvolveu com a ideia de deixar o presidente vulnerável a
maiorias parlamentares de ocasião, mas de possibilitar sua
responsabilização em graves situações de convergência política e
jurídica pela sua destituição, não se admitindo que apenas um desses
dois aspectos esteja presente para que ocorra uma condenação.

Desse modo, a tendência demonstrada pela decisão interlocutória do Min.
Teori Zavascki é a de uma mutação constitucional à paraguaia inspirada
no precedente do ex-Presidente Fernando Lugo naquele país. Porém, com um
agravante: diferentemente da Constituição do Paraguai que prevê
expressamente o ``\emph{mal desempeño de sus funciones}'' como justa
causa ao seu \emph{juicio politico} (art. 225), não há essa mesma
previsão constitucional em nossa Carta, o que deixa ainda mais frágeis
os fundamentos especificamente jurídicos do \emph{impeachment} da
ex-Presidente Dilma Roussef (cf. Galindo: 2016, pp. 35ss.; Balbuena
Pérez: 2013, pp. 380ss.; Lezcane Claude: 2012).

Parece que resolvemos adotar uma espécie de ``hermenêutica de
resultados'' na qual os postulados da interpretação constitucional do
Estado democrático de direito são radicalmente relativizados, dando
lugar a excessos de voluntarismo judicial, desconsiderando a essencial
inegabilidade dos pontos de partida da dogmática jurídica (Ferraz Jr.:
2003, pp. 83ss.), que, no caso, é o arquétipo hermenêutico do Estado
democrático de direito delineado pelo trabalho dos constituintes de
1987-1988.

\begin{enumerate}
\def\labelenumi{\arabic{enumi}.}
\item
  \textbf{Indagação final: Hermes controlará Ulisses desamarrado?}
\end{enumerate}

Imaginando que o Deus grego Hermes chegasse do Olimpo em plena travessia
da nau grega pelo golfo das sereias descrita na Odisseia de Homero e,
confiante na força da virtude do herói Ulisses, aceitasse seu pedido e o
desamarrasse, permitindo-lhe plena liberdade ao ouvir o canto delas, o
que aconteceria? Como seria a substituição da inteligibilidade e
racionalidade da \emph{hermeneia} pela sedução avassaladora do canto
mágico letal das fadas marinhas?

Alegoricamente, é essa a encruzilhada em que se encontra o direito
constitucional brasileiro após o \emph{impeachment} e todo o
recrudescimento teórico a uma hermenêutica constitucional de resultados,
de cariz autoritário e baseada em argumentos de moral e de política
\emph{stricto sensu}, cada vez mais distante do modelo republicano,
democrático e humanista preconizado pelo \emph{rule of law} concebido
pelos constituintes de 1987-1988. A elasticização das possibilidades de
interpretação punitivista com base em uma confiança irrestrita nos
agentes ``da lei e da ordem'' tem solapado o pouco que consolidamos em
termos de hermenêutica constitucional do Estado democrático de direito e
conduzir os direitos fundamentais e a democracia, binômio civilizatório
constitucional essencial, a uma situação de significativa
constitucionalização simbólica ou de efeitos ``hipertroficamente
simbólicos'' das normas constitucionais em detrimento das ``expectativas
normativas congruentemente generalizadas'' (Neves: 2007, pp. 95ss.;
Galindo: 2006, pp. 144-145). Ou ainda, de uma guinada paradigmática no
caminho de uma semantização loewensteiniana da Constituição de 1988.

E o STF, tal como ocorre com os tribunais de democracias não
consolidadas e intérpretes de constituições nominalistas ou semânticas,
comportou-se na maior parte do tempo de modo omissivo ou mesmo
colaborativo com esse ``estado da arte'', com a exceção, talvez, de sua
decisão na Arguição de Descumprimento de Preceito Fundamental 378 sobre
parte dos procedimentos (Galindo: 2016, pp. 96ss.). Assim também ocorreu
em recentes crises políticas de idêntica controvérsia, como as de
Honduras e do Paraguai, nos episódios da deposição de Manuel Zelaya e do
\emph{juicio politico} de Fernando Lugo. Não à toa, também nesses países
se desenvolveu uma ``narrativa de golpe de Estado'', tal como vem
ocorrendo também no Brasil, com sérios questionamentos acerca da
legitimidade do processo (Proner: 2016, pp. 69-72).

Em minha modesta percepção, o STF permitiu, conscientemente ou não, a
abertura de uma ``Caixa de Pandora'' que pode consolidar essa mutação
constitucional à paraguaia, sob a qual a decisão jurídica torna-se mera
questão competencial e pode ser feita de acordo com argumentos de
política e de moral \emph{stricto sensu}, com cada vez menos espaço à
leitura hermenêutica constitucional do Estado democrático de direito,
tal como destacou Lenio Streck:

\begin{quote}
Assim eu pergunto: qual a diferençada postura daquele que defende ser o
ato de decisão do Congresso -- no caso de \emph{impeachment} --
puramente político-ideológico, de um jurista que admite raciocínios
puramente consequencialistas no ato de uma decisão judicial qualquer? Um
parlamentar pode decidir o futuro da nação ``conforme a sua consciência
individual'', ou melhor, ``mera conveniência político-eleitoral''? Ele
pode ignorar ``o jurídico''? O político se basta? O discurso moral
supera o direito?

A partir da tese (aceita por considerável parcela dos juristas) de que o
\emph{impeachment} é um instituto político (e não jurídico), não há
diferença alguma para o decisionismo historicamente praticado pelas
cortes brasileiras, e aquilo que é sustentado por muitos doutrinadores.
Tudo se transforma em raciocínios consequencialistas, do tipo ``decido e
depois busco o fundamento para justificar a escolha (arbitrária)
(Streck: 2016, p. 228).
\end{quote}

Embora tenhamos a esperança de ainda contar com Hermes para controlar
Ulisses desamarrado e inebriado com o sedutor e mortífero canto das
sereias de um constitucionalismo potencialmente semântico, os movimentos
político-jurídicos ocorridos até o momento tem sido desanimadores. A
``pós-democracia'' parece forte, encampando novas e sofisticadas formas
de autoritarismo, sem tanques nas ruas, mas com o esvaziamento concreto
de muitas conquistas civilizatórias do modelo constitucional de 1988.

No 30º aniversário da instalação da Assembleia Constituinte, faz-se
necessária a retomada do espírito democrático e humanista daqueles
homens e mulheres e tentar reaprender e ensinar os caminhos do Estado
democrático de direito contra pós-democracias e autoritarismos de todos
os gêneros.

\textbf{REFERÊNCIAS}

\textbf{Livros, artigos e textos:}

ACKERMAN, Bruce. We the people 2 -- transformations.
Cambridge/Massachusets: Belknap Press of Harvard University Press, 2001.

ANASTASIA, Antonio. Relatório Final no Processo de \emph{Impeachment} da
Presidente Dilma Roussef, 2016.

ARAÚJO, Marcelo Labanca; SANTOS, Gustavo Ferreira \& TEIXEIRA, João
Paulo Allain. Sobre o impeachment. In:
\url{http://novoconstitucionalismo.blogspot.com.br/2015/12/sobre-o-impeachment.html},
(acesso: 16/12/2015), 2015.

BAHIA, Alexandre Gustavo Melo Franco de Moraes; SILVA, Diogo Bacha e \&
OLIVEIRA, Marcelo Andrade Cattoni de: O \emph{impeachment} e o Supremo
Tribunal Federal: história e teoria constitucional brasileira.
Florianópolis: Empório do Direito, 2016.

BALBUENA PÉREZ, David-Eleuterio. El juicio político en la Constitución
paraguaya y la destitución del Presidente Fernando Lugo. In:
\emph{Revista de Derecho Político}. Madrid: UNED, nº 87, pp. 355-398,
2013.

BICUDO, Hélio Pereira; REALE JR., Miguel \& PASCHOAL, Janaína Conceição.
Denúncia contra a Presidente da República Dilma Vana Roussef, 2015.

BROSSARD, Paulo. O impeachment. 3ª ed. São Paulo: Saraiva, 1992.

CANOTILHO, José Joaquim Gomes. Direito Constitucional e Teoria da
Constituição. 6ª ed. Coimbra: Almedina, 2002.

DARDOT, Pierre \& LAVAL, Christian. A nova razão do mundo -- ensaio
sobre a sociedade neoliberal (trad. Mariana Echalar). São Paulo:
Boitempo, 2016.

FERRAZ, Anna Candida da Cunha. Processos informais de mudança da
Constituição. São Paulo: Max Limonad, 1986.

FERRAZ JR., Tércio Sampaio. Introdução ao Estudo do Direito -- Técnica,
Decisão, Dominação. 4ª ed. São Paulo: Atlas, 2003.

GALINDO, Bruno. \emph{Impeachment} à luz do constitucionalismo
contemporâneo. Curitiba: Juruá, 2016.

GALINDO, Bruno. Constitucionalismo e justiça de transição: em busca de
uma metodologia de análise a partir dos conceitos de autoritarismo e
democracia. In: Revista da Faculdade de Direito da Universidade Federal
de Minas Gerais\emph{.} Belo Horizonte: UFMG, nº 67, pp. 75-104, 2015.

GALINDO, Bruno. Direitos fundamentais (análise de sua concretização
constitucional). Curitiba: Juruá, 2003.

GUEDES, Jefferson Gadús \& PÁDUA, Thiago Aguiar de. Pedaladas
jurisprudenciais do TCU ou prospective overruling? In:
\url{http://www.conjur.com.br/2015-ago-16/pedaladas-jurisprudenciais-tcu-ou-prospective-overruling},
(acesso: 11/12/2015), 2015.

HOLMES, Pablo. Por que foi um golpe. In:
\url{http://www.criticaconstitucional.com.br/por-que-foi-um-golpe/},
(acesso: 04/09/2016), 2016a.

HOMERO. Odisséia (trad. Manuel Odorico Mendes). Digitalização da 3ª
edição: eBooksBrasil, 2009.

LASSALLE, Ferdinand. A essência da constituição (trad. Walter Stönner).
4ª ed. Rio de Janeiro: Lumen Juris, 1998.

LEZCANO CLAUDE, Luis. Sobre el ``juicio político'' al Pdte. Fernando
Lugo Méndez. In:
\url{https://luislezcanoclaude.wordpress.com/2012/06/27/sobre-el-jui-2/},
(acesso: 04/09/2016), 2012.

LIMA, Flávia Santiago. Jurisdição constitucional e política (ativismo e
autocontenção no STF). Curitiba: Juruá, 2014.

LOEWENSTEIN, Karl. Teoría de la Constitución (trad. Alfredo Gallego
Anabitarte). Barcelona: Ariel, 1964.

\section{\texorpdfstring{MARTINEZ, Rafael. El juicio político en América
Latina: un golpe de estado encubierto. In:
\url{http://www.condistintosacentos.com/el-juicio-politico-en-america-latina-un-golpe-de-estado-encubierto/},
(acesso: 22/12/2015),
2013.}{MARTINEZ, Rafael. El juicio político en América Latina: un golpe de estado encubierto. In: http://www.condistintosacentos.com/el-juicio-politico-en-america-latina-un-golpe-de-estado-encubierto/, (acesso: 22/12/2015), 2013.}}\label{martinez-rafael.-el-juicio-poluxedtico-en-amuxe9rica-latina-un-golpe-de-estado-encubierto.-in-httpwww.condistintosacentos.comel-juicio-politico-en-america-latina-un-golpe-de-estado-encubierto-acesso-22122015-2013.}

MARTINS, Ives Gandra da Silva. Responsabilidade dos agentes públicos por
atos de lesão à sociedade -- inteligência dos §§ 5º e 6º do artigo 37 da
CF -- improbidade administrativa por culpa ou dolo -- disciplina
jurídica do `impeachment' presidencial (artigo 85 inciso V da CF) --
Parecer, 2015.

MÜLLER, Friedrich. Métodos de trabalho do direito constitucional (trad.
Peter Naumann). 2ª ed. São Paulo: Max Limonad, 2000.

NEVES, Marcelo. Parecer sobre \emph{impeachment}, 2015.

NEVES, Marcelo. A constitucionalização simbólica. 2ª ed. São Paulo: WMF
Martins Fontes, 2007.

PÉREZ-LIÑAN, Aníbal. Presidential impeachment and the new political
instability in Latin America. Cambridge: University Press, 2007.

PRADO, Geraldo \& TAVARES, Juarez. Parecer sobre \emph{impeachment},
2015.

PRONER, Carol. Golpe branco no Brasil: Dilma alerta na ONU. In: A
resistência ao golpe de 2016 (orgs.: PRONER, Carol; CITTADINO, Gisele;
TENENBAUM, Márcio \& RAMOS FILHO, Wilson). Bauru: Canal 6, pp. 69-73,
2016.

QUEIROZ, Rafael Mafei Rabelo. Impeachment e Lei de Crimes de
Responsabilidade: o cavalo de Troia parlamentarista. In:
\url{http://brasil.estadao.com.br/blogs/direito-e-sociedade/impeachment-e-lei-de-crimes-de-responsabilidade-o-cavalo-de-troia-parlamentarista/},
(acesso: 17/12/2015), 2015.

RIBEIRO, Ricardo Lodi. Pedaladas hermenêuticas no pedido de impeachment
de Dilma Roussef. In:
\url{http://www.conjur.com.br/2015-dez-04/ricardo-lodi-pedaladas-hermeneuticas-pedido-impeachment},
(acesso: 11/12/2015), 2015a.

RIBEIRO, Ricardo Lodi. Parecer sobre o Pedido de \emph{Impeachment} da
Presidente Dilma Roussef, 2015b.

SANTOS, Gustavo Ferreira. Processo de impeachment é político? In:
\url{http://www.diariodepernambuco.com.br/app/noticia/vida-urbana/2015/12/19/interna_vidaurbana,617370/processo-de-impeachment-e-politico.shtml},
(acesso: 20/12/2015), 2015.

SARLET, Ingo Wolfgang. A eficácia dos direitos fundamentais. 6ª ed.
Porto Alegre: Livraria do Advogado, 2006.

SOUZA, Marcelo Alves Dias de. Do precedente judicial à súmula
vinculante. Curitiba: Juruá, 2006.

STRECK, Lenio Luiz. A questão de teorias jurídicas meramente descritivas
ou de como o positivismo jurídico influencia na crise política
brasileira. In: A resistência ao Golpe de 2016 (orgs.: PRONER, Carol;
CITTADINO, Gisele; TENENBAUM, Márcio \& RAMOS FILHO, Wilson). Bauru:
Canal 6, pp. 221-228, 2016.

TRIBE, Laurence. American constitutional law. 3ª ed. New York: New York
Foundation Press, vol. I, 2000.

\textbf{Sites consultados:}

\url{http://economia.estadao.com.br/noticias/geral,sera-muito-dificil-justificar-distorcoes--diz-ministro-do-tcu--imp-,1718369},
acesso: 21/12/2015.

\url{http://exame.abril.com.br/brasil/dilma-usara-entrevista-de-temer-como-prova-contra-impeachment/},
acesso: 17/04/2017

\url{http://www.mpf.mp.br/df/sala-de-imprensa/docs/arquivamento-pedaladas-pic},
acesso: 10/09/2016.

\url{http://politica.estadao.com.br/noticias/geral,brasil-vive-pausa-democratica-para-freio-de-arrumacao--diz-ayres-britto,10000027535},
acesso: 28/08/2017.
