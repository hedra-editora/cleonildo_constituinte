\chapter*{A fotografia constitucional de 1988}

\addcontentsline{toc}{chapter}{A fotografia constitucional de 1988,\\
\scriptsize{por José Eduardo Cardozo}}

\emph{José Eduardo Martins Cardozo}\footnote{Advogado e
Professor de Direito. Ex"-Ministro da Justiça e Ex"-Advogado
Geral da União. Advogado no processo de \emph{impeachment} da Presidenta
Dilma Rousseff}

\section{Observações preliminares}

Ainda hoje me recordo do forte impacto que tive quando, iniciando meu
curso de direito, li as reflexões de Ferdinand Lassalle sobre as
Constituições\footnote{\emph{O que é uma Constituição?} Tradução de
  Ricardo Rodrigues Gama. Campinas/\versal{SP}: Russell Editores, 1a. ed. eBook,
  2013.}. Para o ilustre pensador alemão do século \versal{XIX}, na sua essência,
uma Constituição será sempre ``\emph{a soma dos fatores reais do poder
que regem um país}''\footnote{\emph{Op. cit.,} posição 355
  (\emph{eBook}). Para Lassalle, ``\emph{os fatores reais de poder que
  regulam no seio de cada sociedade são essa força ativa e eficaz que
  informa tods as leis e instituições jurídicas da sociedade em apreço,
  determinando que não possam ser, em substância, a não ser tal como
  elas são}'' (posição 250).}. Todos os países possuiriam sempre uma
Constituição real e verdadeira, integralizada pelos ``\emph{fatores
reais e efetivos que regem a sociedade}'' \footnote{\emph{Op. cit.,}
  posição 451 (\emph{eBook})} e que não são necessariamente as
``\emph{constituições escritas nas folhas de papel}''\footnote{\emph{Op.
  cit.,} posição 583 (\emph{eBook})}.

As relações entre estas ``\emph{duas Constituições}''\footnote{\emph{Op.
  cit.,} posição 446 (\emph{eBook})} (a que ``\emph{representa a soma
dos fatores reais de poder}'' e as escritas nas ``\emph{folhas de
papel}''), a seu ver, não se apresentariam sempre do mesmo modo. Quando
a Constituição escrita ``\emph{responde aos fatores reais do poder}'',
ela seria ``\emph{invulnerável}''. Afinal, proclama, ``\emph{com uma
Constituição dessas ninguém brinca se não quer passar mal}''\footnote{\emph{Op.
  cit.,} posição 777 (\emph{eBook})}. Mas ``\emph{onde a Constituição
escrita não corresponder à real, irrompe inevitavelmente um conflito que
é impossivel evitar e no qual, mais dia ou menos dia, a Constituição
escrita, a folha de papel, sucumbirá necessariamente perante a
Constituição real, a das verdadeiras forças vitais do país}''\footnote{\emph{Op.
  cit.,} posição 693 \emph{(eBook)}}\footnote{\emph{Op. cit.,} posição
  369 (\emph{eBook})}. E, dentro dessa percepção, afirma o autor que:
\emph{``se isso acontecer, se esse divórcio existir, a Constituição
escrita está liquidada; não existe Deus, nem força capaz de
salvá"-la''}\footnote{\emph{Op. Cit.,} posição 788 \emph{(eBook).} É
  expressivo o texto do autor quando afirma: ``\emph{podem os meus
  ouvintes plantar no seu quintal uma macieira e segurar no seu tronco
  um papel que diga: `Esta árvore é uma figueira. Bastará esse papel
  para transformar em figueira o que é macieira? Não, naturalmente. E
  embora conseguissem que seus criados, vizinhos e conhecidos, por uma
  razão de solidariedade, confirmassem a inscrição existente na árvore
  de que o pé plantado era uma figueira, a planta confirmassem a
  inscrição existente na árvore de que o pé plantado era uma figueira, a
  planta continuaria sendo o que realmente era e, quando desse frutos,
  destruiriam estes a fábula produzindo maçãs e não figos'}. E conclui:
  \emph{``igual acontece com as constituições. De nada servirá o que se
  escrever numa folha de papel, se não se justifica pelos fatos reais e
  efetivos de poder''} (\emph{Op. cit.}, posição 753 \emph{no eBook})}\emph{.}

A partir destas considerações Lassalle chega a uma importante conclusão.
Para ele, ``\emph{os problemas constitucionais não são problemas de
direito, mas do poder; a verdadeira Constituição de um país somente tem
por base os fatores reais e efetivos do poder que naquele país regem, e
as constituições escritas não têm valor nem são duráveis a não ser que
exprimam fielmente os fatores do poder que imperam na realidade social:
eis aí os critérios fundamentais que devemos sempre lembrar''}\footnote{\emph{Op.
  Cit.,} posição 809 (\emph{eBook}).}\emph{.}

Com Lassalle comecei a ver e a experienciar ao longo da minha vida que o
\emph{direito} e o \emph{poder} são realidades indissociáveis, por mais
que o \emph{modus} com que a ciência jurídica é ensinada nos nossos
bancos universitários busque nos adestrar para que operemos as normas
jurídicas como realidades dogmáticas, impenetráveis nos seus fundamentos
e na ideologia que constitui a sua própria dimensão axiológica.
Habitualmente aprendemos a interpretar as normas jurídicas, sem
refletirmos sobre a dimensão maior da sua própria historicidade.
Aprendemos o que é o direito e a sermos seus operadores, mas quase nunca
refletimos, com profundidade, sobre o que ele é ou o porque ele existe.

Com o passar dos anos e de muitas outras leituras, passei a ver as
constituições escritas, em alguma medida, como verdadeiras
``\emph{fotografias}'' em que a câmera do legislador constituinte
registra em ``\emph{folhas de papel}'' os ``f\emph{atores reais do
poder}''. A imagem normativa registrada expressa o estado evolutivo da
essência das relações de poder existentes na sociedade fotografada,
inclusive das concepcões dominantes e dos valores por ela acolhidos
naquele específico momento histórico.

Todavia, as constituições escritas são fotografias tiradas por câmeras
que possuem uma dimensão ainda hoje inimaginável, mesmo com os recursos
da moderna tecnologia. Constituindo um conjunto de normas jurídicas, os
textos constitucionais, além de registrar as imagens dos ``\emph{fatores
do poder}'' existentes, guardam sempre com a realidade uma interação
profundamente dialética. Ao contrário das fotografias normais que apenas
se limitam a registrar ``\emph{o que é}'' a realidade fotografada, ou
seja, o ``\emph{plano do ser}'' da imagem captada, as regras
constitucionais, por serem enunciados jurídicos afirmados no plano do
``\emph{dever ser}'', incidem dialeticamente sobre a realidade no
período ``pós"-fotografado'', condicionando"-o em diferentes perspectivas.
De um lado, atuam no mundo dos fatos para \textbf{\emph{conservar}} os
``\emph{fatores do poder}'' cuja imagem normativa foi registrada,
negando a possibilidade da sua destruição. E de outro, atuam para
\textbf{\emph{orientar}} a produção dos fatos da vida para que sejam
aquilo que ``\emph{deve ser}'', mesmo que ainda ``\emph{nada seja}'' ou
``\emph{exista}'' no momento em que a imagem fotografada é captada.

Tomemos um exemplo para melhor explicar o que acabou de ser dito.

Os pais de uma jovem de 16 anos, no dia em que se realizará um evento
comemorativo, tiram uma foto dela vestida para uma apresentação de
\emph{ballet}. Aquela foto registrará uma imagem da moça, com o porte
físico que possuia naquele momento da sua juventude, com as feições do
seu rosto e as emoções que deixava transparecer. Sempre que for vista
pelos que presenciaram o evento, a fotografia servirá para ativar a
memória acerca daquele momento especial da vida da jovem. Já para os que
não estiveram presentes, servirá para recordar ou para conhecer o corpo,
as feições e as emoções daquela mulher hoje idosa que um dia foi jovem.

Por óbvio, aquela imagem não tem o ``\emph{poder}'' de conservar a moça
fisicamente como ela era. Nem de fazer dela uma bailarina para todo o
sempre, ou de fazê"-la permanentemente alegre ou insegura como estava
naquele dia. Seguramente, seus traços e seu corpo serão alterados pelo
decurso do tempo. Poderá ela vir a ser uma bailarina ou jamais vir a
dançar novamente. Poderá vir a ser uma pessoa feliz ou infeliz,
sorridente ou sisuda, segura ou insegura, em situações da vida que em
nada se vincularão ao momento registrado naquela foto. As fotografias
não tem o poder de aprisionar dentro de si mesmas a realidade
fotografada, nem comportam uma edição para fazer do futuro o que não
existe no presente.

As câmeras normativas que registram as fotografias constitucionais
operam em outra dimensão. Fotografam os \emph{fatores reais do poder,}
não com o objetivo de serem vistos no futuro, como um alimentador visual
da memória histórica de um tempo que já se foi. As fotografias
normativas, por natureza, sempre se destinam a ``\emph{aprisionar}'' as
relações de poder que registram, para mantê"-las e conservá"-las no tempo.
Simultaneamente, de forma paradoxal, incluem nesse registro realidades
que ainda não estão dadas e que estão no plano de um mero
``\emph{vir"-a"-ser}'', mas que as ``\emph{relações de poder}
\emph{existentes}'', no estado em que foram fotografadas, impõem,
desejam ou consentem que devam se verificar no futuro.

Para nos referirmos ao \emph{poder de aprisionamento da realidade em
curso} pelas normas constitucionais, por simplificação expositiva,
empregaremos a denominação de ``\emph{efeito mantenedor}'' dessas
normas. Já para rotularmos o poder que possuem estas mesmas normas de
buscar tornar realidade o que ainda não é, doravante, utilizaremos a
denominação ``\emph{efeito transformador}''\footnote{O próprio Lassalle,
  ao que tudo indica, não ignorava esse efeito ``transformador'' das
  normas constitucionais. Com efeito, ele chega, \emph{en passant,} a
  dizer que ``\emph{não desconheceis também o processo que se segue para
  transformar esses escritos em fatores reais do poder, transformando"-os
  dessa maneira em fatores jurídicos}''(\emph{op. cit.,} posição 367 do
  \emph{eBook})}.

\section{A Constituição de 1988 e o seu efeito
``mantenedor}''

\subsection{Os Antecedentes históricos da Constituição de 1988}

Com o golpe militar de 1º de abril de 1964\footnote{Por razões óbvias de
  comunicação social, o golpe militar se proclamou como
  ``\emph{Revolução Brasileira de 31 de março de 1964}''.}, a nova
configuração dos ``\emph{fatores reais do poder}'' da sociedade
brasileira, imposta pela força inexpugnável dos canhões e das baionetas,
exigiram um redesenho normativo que substituisse a democrática
Constituição Brasileira de 1946. Para dar ``legalidade'' e
``legitimidade'' às ações consolidadas desse Golpe de Estado, foram
editados os denominados ``\emph{Atos Institucionais}'' que assumiam a
condição de serem as normas supremas do país, prevalecendo inclusive
sobre o disposto na Carta Constitucional até então vigente\footnote{O
  Ato Institucional nº 1 foi promulgado em 9 de abril de 1964, pelo com
  o objetivo de ``declarar mantida a Constituição de 1946 e as
  Constituições estaduais e respectivas Emendas, com as modificações
  constantes deste Ato''. Assinaram este ato o Gen. Ex Arthur da Costa e
  Silva, o Ten. Brig. Francisco de Assis Correia de Mello e o Vice"-Alm.
  Augusto Hamann Rademaker Grunewald.}.

Passados dois anos do golpe, por meio do Ato institucional nº 4, de 7 de
dezembro de 1966, o Presidente da República Humberto Castello Branco
outorgou ao Congresso Nacional poderes constituintes ilimitados e
soberanos para elaborar uma nova Constituição que retratasse o novo
quadro de poder instalado no país. Com os membros da oposição já
afastados das bancadas parlamentares, convocou então o Poder Legislativo
para se reunir extraordinariamente, de 12 de dezembro daquele mesmo ano
a 24 de janeiro do ano subsequente com o objetivo de que houvesse a
\emph{``discussão, votação e promulgação do projeto de Constituição por
ele apresentado}''.

No próprio dia 24 de janeiro de 1967 foi aprovado, sem alterações
relevantes, o texto encaminhado pelo Presidente da República. Nasceu
então, naquele ano, a Constituição Federal que pretendeu dar feição
normativa institucional ao poder ditatorial, centralizando o poder e
afirmando uma real supremacia do Poder Executivo sobre os demais Poderes
do Estado. Foi a primeira fotografia constitucional do golpe.

A abusividade do poder ditatorial, porém, não se continha dentro dos
limites da sua própria constituição. Os ``\emph{fatores reais do
poder}'', em face da conjuntura social e política, exigiam uma maior
legitimação do arbítrio, com sacrifício de direitos que ainda pudessem
ser invocados contra ele.

Tomando como pretexto um duro discurso feito na Câmara dos Deputados
pelo então Deputado Moreira Alves contra a ditadura militar, foi editado
o ``\emph{Ato Institucional nº 5}'', indubitavelmente, a maior expressão
normativa do arbítrio ditatorial daquele período.

Assinado pelo Presidente da República, General Arthur Costa e Silva,
publicado em 13 de dezembro de 1968, ele veio a ``\emph{manter a
Constituição de 24 de janeiro de 1967 e as Constituições
estaduais''}\footnote{Art. 1º.}, com as modificações dele constantes. E
dentre estas modificações estabeleceu:

\begin{enumerate}
\item{}a possibilidade do Presidente da República decretar o recesso do
Congresso Nacional, das Assembléias Legislativas e das Câmaras de
Vereadores, por ato complementar, em estado de sítio ou fora dele,
``\emph{só voltando os mesmos a funcionar quando convocados pelo
Presidente da República}''\footnote{Art. 2º, \emph{caput.}};

\item{}a possibilidade do Poder Executivo correspondente, durante a
decretação do recesso parlamentar, ``\emph{legislar em todas as
matérias}''\footnote{Art. 2º, §1º.};

\item{}a possibilidade do Presidente da República decretar a
``\emph{intervenção}'' em Estados e Municípios, nomeando
``\emph{interventores}'' para exercerem as atribuições que coubessem,
respectivamente, a Governadores e a Prefeitos\footnote{Art. 3º};

\item{}a possibilidade do Presidente da República, ``\emph{ouvido
previamente o Conselho de Segurança Nacional, e sem as limitações
previstas na Constituição}'', suspender ``\emph{os direitos politicos de
quaisquer cidadãos pelo prazo de 10 anos e cassar mandatos eletivos
federais, estaduais e municipais}''\footnote{Art. 4º.} (suspendendo,
inclusive, com essa medida, o privilégio de foro ou função, o direito de
votar e de ser votado em eleições sindicais), bem como de proibir
``\emph{atividades ou manifestação sobre assunto de natureza política}''
e de aplicar medidas de segurança de ``\emph{liberdade vigiada}'', de
``\emph{proibição de frequentar determinados lugares}'', de
``\emph{domicílio determinado}'' e ``\emph{outras restrições ou
proibições relativamente ao exercício de quaisquer outros direitos
públicos ou privados}''\footnote{Art. 5º. As medidas de liberdade
  vigiada, proibição de frequenter determinados lugares e domicílio
  determinado poderia ser aplicadas diretamente pelo Ministro de Estado
  da Justiça, sendo vedada a apreciação da sua decisão pelo Poder
  Judiciário (art. 5º, §2º).} ;

\item{}a suspensão das garantias constitucionais de vitaliciedade,
inamovibilidade e estabilidade, podendo o Presidente da República
demitir, remover, aposentar ou por em disponibilidade ``\emph{quaisquer
titulares das garantias referidas nesse artigo}''\footnote{Art. 6º. As
  prerrogativas de demição, remoção, aposentadoria e colocação em
  disponibilidade também eram aplicáveis a todos os entes da Federação e
  extensíveis também aos ocupantes de cargos ou empregos na
  Administração Indireta, e aos militares e policiais militares que
  também poderiam ser transferidos para a reserva (art. 6º, §§ 1º e 2º)};

\item{}a possibilidade do Presidente da República, após investigação,
decretar ``\emph{o estado de sítio e prorrogá"-lo}\footnote{Art. 7º}'' e
\emph{``decretar o confisco de bens de todos quantos tenham enriquecido
ilicitamente, no exercício de cargo ou função pública''}\footnote{Art.
  8º};

\item{}a suspensão da garantia de \emph{habeas corpus,} ``\emph{nos casos de
crimes politicos, contra a segurança nacional, a ordem econômica e
social e a economia popular}'';\footnote{Art. 10.}
\end{enumerate}

Note"-se que todas estas situações, por determinação expressa do art. 11
do \emph{Ato Institucional} nº 5, poderiam vir a ser decididas, com
exclusão ``\emph{de qualquer apreciação judicial}'', inclusive no que
diz respeito a seus ``\emph{respectivos efeitos}''.

Diante desta nova realidade, não se fez esperar um novo ``redesenho'' da
ordem jurídico"-constitucional, a partir do ``endurecimento'' do regime
militar. Após o afastamento da Presidência da República do General
Arthur da Costa e Silva em razão de ``\emph{lamentável e grave
enfermidade}'' e a declaração da vacância do seu cargo\footnote{A
  expressão ``\emph{lamentável e grave enfermidade}'' foi utilizada no
  primeiro ``considerando'' justificador da Edição do Ato Institucional
  nº 16, de 14 de outubro de 1969. Este mesmo ato institucional declarou
  a vacância do cargo de Presidente da República (art. 1º). Nele se
  determinou ainda que enquanto não fosse realizada a eleição e posse do
  novo Presidente da República, a Chefia do Poder Executivo seria
  exercida ``\emph{pelos Ministros Militares}'' (art. 3º). Ainda de
  acordo com este ato, a ``\emph{eleição do Presidente e Vice"-Presidente
  da República}'' seria realizada no dia 25 do corrente mês de outubro,
  pelos Membros do Congresso Nacional (art. 4º).

  Note"-se que também aqui os atos praticados com base neste Ato
  Institucional ficavam ``\emph{excluídos de apreciação judicial}''(art.
  8º).

  Em 25 de outubro de 1969, foi eleito Presidente da República o General
  Emílio Garrastazu Médici, vindo a tomar posse no dia seguinte. O
  General Arthur da Costa e Silva faleceu em 17 de dezembro de 1969.}, a
Junta de Governo Provisório que assumiu o exercício do governo\footnote{Formada
  pelos Ministros da Marinha de Guerra, do Exército e da Aeronáutica
  (art. 3º, do Ato Institucional nº 16).}, em 17 de outubro de 1969,
procedeu à edição de uma Emenda Constitucional que alterou profundamente
o texto da Constituição de 1967\footnote{Como o Congresso nacional havia
  sido colocado em recesso pelo Ato Complementar nº 38, de 13 de
  dezembro de 1968, o Poder Executivo estava autorizado a legislar sobre
  todas as matérias (art. 2º, §1º, do Ato Institucional nº 5), com base
  nos poderes outorgados pelo Ato Institucional nº 16 (art. 3º). Em
  decorrência disso, a Emenda Constitucional nº 1/1969 foi promulgada
  pelos Ministros Militares.}.

Com efeito, a Emenda Constitucional nº1 procedeu, em larga medida, à
incorporação no texto constitucional de medidas tomadas pelos Atos
Institucionais. Por isso, de acordo com a opinião de muitos juristas,
passou a ser qualificada como uma ``\emph{nova Constituição}'' (`` a
Constituição de 1969''), e não apenas como uma simples ``\emph{emenda
constitucional}''.

Sob a sua vigência, após a posse do novo Presidente da República, o
General Emílio Garastazu Médici, o país passou pelo período mais
arbitrário do regime militar, sendo por isso referido, frequentemente,
como ``\emph{os anos de chumbo}''. Repressão ao movimento social e
estudantil, o desaparecimento de presos politicos e as torturas
tornaram"-se prática comum nos órgãos repressivos do Estado, sempre
acompanhadas do cerceamento absoluto à liberdade de expressão e de
imprensa.

É importante observar que somente muitos anos depois, durante o período
de ``abertura lenta e gradual'' do regime militar, é que ``\emph{os
fatores reais do poder}'' passaram a indicar a necessidade de ser
gradativamente extirpados os ``entulhos normativos'' impostos pela fase
mais brutal da ditadura. Em 13 de outubro de 1978, no governo de Ernesto
Geisel,\footnote{Ernesto Beckmann Geisel assumiu a Presidência da
  República em 15 de março de 1974, tendo o seu governo sido
  caracterizado pela amenização gradativa do rigor do regime militar.
  Foi sucedido pelo General João Batista de Figueiredo, eleito em 15 de
  outubro de 1978, que deu continuidade ao período de ``abertura
  política'' gradual.} foi promulgada a Emenda Constitucional nº 11 à
Constituição de 1967 (ou de 1969, como visto, ao ver de muitos juristas)
que, ao lado de outras medidas, determinou ficarem ``\emph{revogados os
Atos institucionais e complementares, no que contrriarem a Constituição
Federal, ressalvados os efeitos dos atos praticados com base neles, os
quais estão excluídos de apreciação judicial}''\footnote{Art. 3º.}.

\subsection{Constituição de 1988: a fotografia de um Estado
Democrático de Direito}

Após um período de luta popular intensa pela redemocratização do país, a
mudança dos ``\emph{fatores reais do poder}'' indicavam uma nova
realidade institucional e jurídica a ser fotografada. O processo de
negação do Estado autoritário e ditatorial nascido do golpe de Estado de
1º de abril de 1964, teve o seu momento final de consolidação com a
entrada em vigor da nossa vigente Constituição Federal, em 5 de outubro
de 1988.

A transição lenta e gradual realizada ao longo do período final da
ditadura militar ensejou uma interessante dinâmica no processo
constituinte responsável pelo nascimento da Carta Constitucional mais
democrática da nossa história. Embora tendo como pecado original o fato
de não ter sido elaborada e aprovada por uma Assembléia Nacional
Constituinte, uma vez que foi o próprio Congresso Nacional que foi
investido dos poderes de elaborá"-la\footnote{Um dos graves problemas
  decorrentes desse ``pecado original'' foi a manutenção, pela
  Constituição de 1988, das vigas mestras do sistema politico até então
  vigente, apesar do seu anacronismo e das suas distorções
  representativas. Dificilmente as instituições humanas, por si, alteram
  as regras principais que estabelecem as suas relações de poder, a
  partir de iniciativa dos seus próprios beneficiários. Foi o que
  ocorreu com o Congresso Constituinte, na elaboração da Carta
  Constitucional de 1988.}, o seu resultado foi a consolidação
institucional indiscutível de um verdadeiro ``\emph{Estado Democrático
de Direito}''.

Superando o período ditatorial, a sociedade brasileira queria impor
limites ao exercício do poder estatal e estabelecer a democracia.
Ansiava por direitos que fossem respeitados, por um Legislativo forte e
por um Poder Judiciário autônomo.

Tudo isso foi afirmado, com fortes tintas, no texto da nova Carta
Constitucional.

Já na sua abertura a Constituição de 1988 proclamou que ``\emph{a
República Federativa do Brasil, formada pela união indissolúvel dos
Estados e Municípios e do Distrito Federal, constitui"-se em
\textbf{Estado Democrático de Direito}}''\footnote{Art. 1º,
  \emph{caput.} O destaque, em negrito, é nosso.}, tendo por fundamentos
além da sua natural e desejada soberania e do seu comprometimento com os
valores sociais do trabalho e da livre iniciativa, ``a
\emph{cidadania}'', a ``\emph{dignidade da pessoa humana}'' e o
``\emph{pluralismo politico}'' \footnote{Art. 127 e segs.}. Estabeleceu
um significativo elenco de direitos e garantias individuais e
coletivos\footnote{Estão concentrados no art. 5º da C.F., mas podem ser
  encontrados em vários outros dispositivos do texto constitucional}, em
larga medida, voltados ao estabelecimento de limites claros às condutas
do próprio Estado. Não permitiu que nenhuma ``\emph{lesão}'' ou
``\emph{ameaça}'' a direito pudesse vir a ser excluída da
``\emph{apreciação do Poder Judiciário}''\footnote{art. 5º, \versal{XXXV}.},
dando a esse Poder garantias, prerrogativas e uma sólida estrutura
predefinida no próprio texto constitucional\footnote{Art. 92 e segs.}. O
Poder Legislativo ganhou uma autonomia e uma função de controle em
relação ao Executivo, com tal magnitude, que muitos chegaram a sugerir
que a nossa Carta teria ``flertado'' com o parlamentarismo\footnote{Art.
  44 e segs.}. O Ministério Público ganhou significativa entronização
constitucional, afirmando"-se institucionalmente como dotado de autonomia
funcional, de garantias, de competências e de prerrogativas
absolutamente inovadoras na história do Estado Brasileiro. E a
Administração Pública, Direta e Indireta, ganhou limites objetivos de
tal ordem que muitos chegaram a afirmar que o princípio da
``\emph{indisponibilidade pública}'' propiciador das ``\emph{contenções
jurídicas}'' estatais em situações que contrastam com a liberdade
privada, parecia ter ganho tintas valorativas mais fortes do que as
atribuídas ao princípio da ``\emph{supremacia do interesse público}'',
vetor que orienta a possibilidade do administrador público ter
\emph{prerrogativas juridicas} que o mundo privado não
disfruta\footnote{Seguimos aqui os passos de Celso Antônio Bandeira de
  Mello quando o ilustre administrativista pátrio afirma que na base do
  regime jurídico"-administrativo devem ser encontrados dois princípios
  basilares: o princípio da \emph{supremacia do interesse público} e o
  da \emph{indisponibilidade do interesse público} (\emph{Curso de
  Direito Administrativo,} 33a. ed, p.29 e segs.(São Paulo: Malheiros,
  2016). O primeiro gera, em defesa do primado dos interesses públicos,
  um conjunto de prerrogativas jurídicas que deve ser deferido em favor
  daqueles que exercem a função administrativa, em situação
  absolutamente distinta dos que perseguem apenas a satisfação de
  interesses privados. O segundo, propicia limitações objetivas de
  conduta para o administrador público, justamente pelo fato dele ser um
  mero preposto da coletividade, apenas autorizado a agir dentro dos
  estreitos limintes que a lei lhe assinala.

  É fato que hoje existe uma acalorada disputa sobre o significado real
  destes princípios no direito adminsitrativo moderno. Todavia, temos
  por convicção que o Direito Administrativo, nos moldes em que se
  estruturou a partir do nascimento do modelo que julgamos oportuno
  denominar de ``\emph{Estado de Direito}'', tem na base da definição do
  seu regime jurídico estes dois princípios, em um âmbito dinâmico de
  contraposição recíproca. Cada Constituição, naturalmente, dará a eles
  o peso que julga conveniente e oportuno, de acordo com a realidade
  histórica e social que agasalha. Mas eles continuam a existir, a nosso
  ver, como pontos centrais de definição de todas as regras e institutos
  próprios deste particular campo do direito.}.

Ponto que merece destaque reside ainda no fato de que a nossa
Constituição estabeleceu um nível de ``petrificação'' inédita na nossa
história, buscando evitar que as conquistas democráticas do \emph{Estado
Democrático de Direito} que prefigurou não pudessem ser eliminadas no
futuro. Com efeito, o seu art. 60, §4º, estabeleceu que o texto da nossa
lei maior não poderá ser alterado por meio de ``\emph{emendas
constitucionais}'' em relação ``\emph{a forma federativa do Estado}'',
``\emph{o voto direto, secreto, universal e periódico}'', ``\emph{a
separação dos Poderes}'' e no que concerne ainda a todos ``\emph{os
direitos e garantias individuais}'' nela estabelecidos. Nesses casos,
apenas uma nova Constituição que rompa com a ordem jurídica estabelecida
na Carta de 1988, ou seja, que tenha sua formação advinda da
manifestação de um novo \emph{poder constituinte originário} poderá
tocar nesses pontos por ela qualificados como jurídicamente
``\emph{intangíveis}''.

De muito não se precisa para se avaliar que a nossa vigente Constituição
Federal resolveu assegurar no âmbito das possibilidades determinadas
pelos \emph{fatores reais do poder} que se configuraram após o fim do
regime militar, a Democracia e o Estado de Direito em sólida dimensão
institucional e jurídica. Reconhecendo as dificuldades que a nossa
história documenta para que uma tal realidade pudesse ser conquistada e
registrada, a fotografia constitucional então realizada pelo processo
constituinte utilizou tintas fortes para que esta realidade pudesse ser
conservada e não alterada em dias futuros. Registrou um sólido e
institucionalizado \emph{Estado Democrático de Direito,} firmando
amarras jurídicas para que em dias futuros não viessemos a deixar
escapar pelo vão dos dedos essa conquista histórica.

Deveras, a Carta Constitucional de 1988 foi uma conquista inédita no
nosso album de fotografias constitucionais, ao menos na dimensão da
perspectiva e do forte colorido da imagem fotografada. Uma conquista que
se desejava, e se amoldava inteiramente à realidade social, econômica e
política vivenciada em 5 de outubro de 1988, e aos desejos de liberdade
de uma sociedade que amargou dolorosos ``\emph{anos de chumbo}'' sob as
botas autoritárias de um regime militar ilegítimo.

Nisto reside a principal dimensão ``\emph{mantenedora}'' da realidade
ditada pelos ``\emph{fatores reais do poder}'' que deram nascimento à
fotografia ``\emph{Constituição"-cidadã}''.

\section{A Constituição de 1988 e o seu efeito
``transformador''}

Embora seja uma questão polêmica no âmbito da ciência jurídica e da
própria ciência política, temos como correta a afirmação de que o
denominado \emph{Estado de Direito,} desde o momento das primeiras
experiências normativas que o caracterizaram como um modelo diferenciado
de exercício do poder politico ao longo do século \versal{XVIII}\footnote{Devem
  ser aqui destacadas a constituição estadonidense e as constituições
  francesas que se seguiram à revolução burguesa de 1789. Deixamos aqui
  de fazer referência à \emph{rule of law} inglesa, em decorrência da
  aguda polêmica que existe sobre se este modelo se enquadraria ou não
  dentro do modelo de \emph{Estado de Direito}. Reseervamos a abordagem
  desta polêmica para estudo próprio, de maior folego, onde todas as
  intrincadas nuances da matéria possam ser adequadamente discutidas.},
e da sua própria teorização pela doutrina alemã a partir do início do
século \versal{XIX}, assumiu diferentes configurações, em especial dentro do que
se convencionou denominar de \emph{Estado Liberal} e de \emph{Estado
Social.} Temos, pois, como aceitável a denominação \emph{Estado de
Direito Liberal} e \emph{Estado de Direito Social}, na medida em que são
conceitualmente identificadas como espécies do gênero \emph{Estado de
Direito.}

Nascido tambéem ao final do século \versal{XVIII}, em momento histórico
coincidente com o nascimento do Estado de Direito, o \emph{Estado de
Direito Liberal} tem como característica central a sua não intervenção
na economia. Por essa premissa, a nova classe dominante da sociedade
capitalista, a burguesia, garantia a não atuação estatal nas estruturas
próprias do modo de produção que caracterizava este modelo de sociedade.
A \emph{livre iniciativa} e a \emph{livre concorrência} deveriam ser as
molas propulsoras da vida social, de modo a que a \emph{mão invisível}
do mercado a tudo harmonizaria. A regra era o ``\emph{lassez"-faire,
lassez"-passer; le monde va de lui même''} \footnote{``\emph{Deixar
  passar, deixar fazer, o mundo caminha por si só}''.}\emph{.}

Naturalmente, este ``\emph{Estado de Direito}'' de matiz liberal, tinha
uma feição \emph{negativa,} ou seja de \emph{inação contida} de
exercício do poder frente ao cidadão. Ele deveria assegurar direitos
fundamentais para o desenvolvimento da sociedade capitalista a todos os
indivíduos, tais como a vida, a liberdade, a propriedade, e a igualdade
perante a lei (isonomia formal), em princípio, por uma geral obrigação
estatal de \emph{non facere.}

O desenvolvimento acelerado da sociedade capitalista, todavia, agudizou
as contradições próprias inerentes a esse modelo de sociedade. Condições
degradantes e desumanas que atingiam a classe trabalhadora, implicando
em tensões, disputas e conflitos que ensejaram uma mudança nos
\emph{fatores reais do poder}. A ``\emph{mão invisível}'' do mercado
demonstrava não harmonizar interesses antagônicos e conflituosos,
gerando utopias e ações de contestação aberta ao modo de produção
capitalista, à propriedade privada e às estruturas jurídicas e sociais
que sobre ele foram e continuavam a ser edificadas. Impunha"-se que o
\emph{Estado de Direito} deixasse de ser apenas ``\emph{negativo}'' para
passar a ser \emph{positivamente} atuante, intervencionista, na busca da
redução das desigualdades sociais.

Nasce assim, o \emph{Estado de Direito Social}, para muitos uma forma
mais avançada de se conquistar a ``\emph{justiça social}'' nas
sociedades capitalistas. Para outros, mais a esquerda, uma forma do
capital ceder os anéis para não perder os dedos, em face ação dos
movimentos revolucionários marxistas que ganharam corpo logo ao início
do século \versal{XX}\footnote{Costuma"-se afirmar que a primeira Constituição a
  assegurar os denominados direitos sociais foi a Constituição Mexicana
  de 1917, embora seja sempre lembrada e referida a Constituição de
  Weimar (alemã) de 1919.}.

Podemos, assim, afirmar que a principal diferença entre as duas espécies
de modelos de \emph{Estado de Direito} (\emph{Estado de Direito Liberal}
e \emph{Estado de Direito Social}) reside no fato de que em uma a
atuação do aparelho estatal frente a sociedade deve ser fundamentalmente
\emph{negativa,} limitadora da sua atuação (\emph{Estado Liberal}),
enquanto que na outra (\emph{Estado Social}), além da ação negativa
respeitadora de limites ao exercício do poder, deve existir uma atuação
\emph{positiva}, intervencionista, na busca de se garantir um universo
maior de direitos sociais (por ex: direito ao trabalho, a saúde, a
educação, ao lazer, a cultura, etc.), eliminar desigualdades e
desenvolver políticas de enfrentamento da exclusão social.

Muito não se precisa argumentar para que se demonstre que o modelo de
\emph{Estado Democrático de Direito} afirmado na nossa Constituição de
1988, realizou uma clara opção pelo \emph{Estado Social}, abandonando
com nitidez a feição \emph{restritiva} e \emph{negativa} do Estado
Liberal.

Já no seu artigo 3º a nossa vigente lei maior aponta que a erradicação
da pobreza e da marginalização, a redução das ``\emph{desigualdade
sociais e regionais}'' e a promoção do ``\emph{bem de todos, sem
preconceitos}'' é um dos objetivos maiores da Repúbica Federativa do
Brasil. Em outros dispositivos consagra como direitos sociais ``\emph{a
educação, a saúde, a alimentação, o trabalho, a moradia, o lazer, a
segurança social, a proteção à maternidade e à infância, a assistência
aos desamparados, na forma desta Constituição}''\footnote{Art. 6º.}, e
um extenso rol de concretos direitos trabalhistas\footnote{Art. 7º}. Ao
tratar da ordem econômica e social, assegura como um dos seus
fundamentais princípios, ao lado da ``\emph{livre iniciativa}'', da
``\emph{propriedade privada}'' e da ``\emph{livre concorrência}'', a
``\emph{função social da propriedade}'' e a ``\emph{defesa do
consumidor}''\footnote{Art. 170}, além de admitir que certas atividades
econômicas devam estar submetidas a autorização e a regulação
estatal\footnote{Art. 170, parágrafo único.} e que o Estado, por meio de
empresas estatais, em certos casos, possa realizá"-las sob regime do
direito privado\footnote{Art. 173.}. Assegura ainda que ``\emph{a lei
reprimirá o abuso do poder econômico que vise a dominação dos mercados,
à eliminação da concorrência e ao aumento arbitrário dos
lucros''}\footnote{Art. 173, §4º.}\emph{,} admitindo inclusive que para
fins urbanísticos possa haver a desapropriação de imóveis urbanos que
desatendam a sua função social\footnote{Art. 182.}, ou em casos
semelhantes no âmbito rural, a utilização dessa mesma forma de aquisição
compulsória da propriedade para a realização de programas de reforma
agrária pela União\footnote{Art. 184 e segs.}.

Nesse aspecto, é importante ressaltar que a nossa lei maior dedica todo
um Título ao tratamento da ``\emph{Ordem Social}''\footnote{Título \versal{VIII}},
onde destina diferentes capítulos para a fixação de regras
constitucionais sobre a \emph{seguridade social}, a \emph{saúde}, a
\emph{previdência social}, a \emph{assistência social}, a
\emph{educação}, a \emph{cultura,} ao \emph{desporto}, a \emph{ciência e
teconologia}, a \emph{comunicação social}, ao \emph{meio ambiente}, a
\emph{familia, criança, adolescente, jovem e idoso}, e ainda aos
\emph{índios}.

Assumiu, portanto, o legislador constitucional de 1988, a tarefa de
qualificar o Brasil como um \emph{Estado de Direito Social}. E é nessa
perspectiva que se afirma a dimensão \emph{transformadora} da Carta
Constituição Federal de 1988. Pretende fazer de um país marcado
historicamente pela pobreza, pelo analfabetismo, pelo preconceito, pela
desigualdade econômica profunda e pela exclusão social, um país
igualitário e justo, respeitando os parâmetros básicos do modo de
produção capitalista.

\section{A crise da fotografia constitucional de 1988}

Diante de todo o exposto, fica evidente a perspectiva juridico"-política
da fotografia registrada pela Constituição de 1988, a partir da
redefinição dos \emph{fatores reais do poder} que ocorreu no período que
se segue ao final da ditadura militar. Um \emph{Estado Democrático de
Direito} a ser mantido e um \emph{Estado social} estimulador de
profundas transformações nos marcos de uma sociedade capitalista, foi a
paisagem normativa fotografada pelos congressistas constituintes que
discutiram e aprovaram a nossa vigente Carta Constitucional.

Resta saber agora se essa fotografia, após quase 30 anos do momento em
que foi tirada pelos parlamentares constituintes, ainda guarda adequação
com a realidade vivenciada em nosso país nos primeiros anos do século
\versal{XXI}.

A ninguém é dado desconhecer que vivemos hoje no Brasil um momento de
aguda crise. Não é uma crise econômica ou política similar àquelas que
tivemos ao longo da nossa história mais recente. É, segundo penso, uma
crise complexa, incomum, aguda, que poderá vir atingir a essência da
nossa institucionalidade, se não for bem compreendida e solucionada de
forma pacífica.

Seria equivocado imaginar"-se que as sementes desta crise apenas
germinaram, até o momento, no solo brasileiro. Inegavelmente, ao menos
sob dois aspectos, ela remonta à totalidade do globo terrestre.

Não podemos deixar de reconhecer, contudo, o tempero apimentado que foi
adicionado a esta crise globalizada pelos atuais viventes da antiga Ilha
de Vera Cruz, com as suas idiossincrasias e a sua cultura ainda pouco
apegada a valores democráticos, humanos e republicanos mais profundos.

No que diz respeito aos mencionados fatores externos, acreditamos que o
modelo de \emph{Estado Democrático de Direito} se encontra hoje em crise
em todo planeta. E o Brasil, não sendo uma ilha isolada do mundo, não
poderia jamais por ela passar incólume. Do mesmo modo, o modelo de
\emph{Estado Social}, no atual período histórico, também está submetido,
frequentemente, aos choques ditados pelas ondas neoliberais que seguem o
ritmo das marés de um mundo complexo e contraditório que a cada dia se
transforma por um processo de globalização frenética.

Meditemos incialmente sobre a crise do modelo de \emph{Estado
Democrático de Direito}. E a seguir sobre a crise do \emph{Estado
Social.}

\subsection{A crise do Estado Democrático de Direito}

Ouso afirmar que todos os pilares que fundamentaram e ainda fundamentam
a existência dos denominados \emph{Estados Democráticos de Direit}o
estão hoje em crise. Não apenas no Brasil, mas em todos os países que
adotam esse modelo de organização do exercício do poder político.

De início, podemos afirmar que o \emph{princípio da legalidade} está em
cheque. Para alguns, muito próximo, talvez, de um verdadeiro
``\emph{cheque"-mate}''.

O Parlamento, independentemente da qualidade das suas legislaturas, por
definição, sempre será um órgão plural e lento, onde nem sempre a
aprovação de decisões por uma maioria é rápida. Em um mundo governado
pela globalização, marcado pela rapida interação determinada pelos
modernos processos tecnológicos, a falta de lei para fundamentar
decisões ágeis é quase sempre uma realidade. Se por um lado
\emph{ninguém pode fazer ou deixar de fazer alguma coisa a não ser em
virtude lei}, de outro, as vezes, o Estado necessita agir de imediato
para dar uma resposta aos fatos da vida ou aos reclamos da sociedade.
\emph{E a lei, muitas vezes \ldots{}. não existe}.

Jamais haverá a possibilidade de ``\emph{vácuo}'' no exercício do poder
politico. Se quem o tem não o exerce no momento em que necessita ser
exercido, diante das necessidades impostas \emph{a partir dos fatores
reais do poder,} alguém o exercerá, fazendo as suas vezes. Tem sido
assim na história do Estado de Direito.

E a lei? \emph{A lei\ldots{} ora, a lei}!

Algumas vezes, porém, digamos a bem da verdade, não é a inércia do
Legislativo que determina que outros Poderes do Estado
``\emph{invadam}'' indevidamente a esfera legislativa. Outras
circunstâncias, amparadas \emph{nos fatores reais do poder} o
determinam, ``\emph{desconstitucionalizando}'' de fato o que está
``\emph{constitucionalizado}'' de direito.

Tem sido frequente que o Executivo ou entes personalizados que a ele
estão vinculados, em alguns momentos específicos da vida política e
institucional de muitos países, ``\emph{invadam}'' com seus atos e
regulamentos, \emph{espaços legislativos próprios}, mesmo não existindo
nenhum amparo constitucional para isso. A retórica jurídica, como
sempre, nesses casos, se ocupa de tentar justificar os arbítrios, sob a
ótica da \emph{modernidade} exigida pelos tempos.

Quando isso ocorre, naturalmente, só o Poder Judiciário pode impor o
limite institucional devido ao \emph{abuso de poder} perpetrado pelo
Poder Executivo, dando cor e vida ao sistema de \emph{checks and
balances} inerente à um modelo Estado onde ninguém deve deter o poder
absoluto. Se ele não age, por razões conjuntuais ou ditadas pelos
\emph{fatores reais de poder} momentâneos, o autoritarismo ganha corpo e
o que era democrático se transforma em ditatorial. E, no caso, pouco
importarão os dizeres que porventura se encontrem elegantemente escritos
na ``\emph{folha de papel}''. Onde se lê que deveria existir um
\emph{Estado Democrático de Direito,} na realidade, o que era ``\emph{de
direito}'' se transmuda em ``\emph{de retórica}'', e o que era
\emph{democrático} se transfigura em \emph{autoritário}.

Sem temer desafiar os que se julgam mais sábios e à frente do seu
próprio tempo, temos que reconhecer, porém, que nos atuais trajes
formulados por estilistas da moda juridica, é o Judiciário que,
frequentemente, a pretexto de dar aplicação a vagos princípios
constitucionais, assume uma tarefa \emph{ativista} de
``\emph{legislar}'', substituindo o Legislativo. Amparados em sólidas,
eruditas e ousadas teorias, os magistrados não se intimidam diante da
própria institucionalidade constitucionalmente consagrada em seus países
e passam a produzir normas jurídicas traçando o que deve ser seguido
como ``\emph{lei}'' quando, propriamente, o decidido, nunca foi sequer
cogitado pelos representantes democraticamete eleitos pelo povo, e nem
se insere, para os que querem ter na \emph{common law} o parâmetro
imperial da modernidade, no âmbito de precedentes judiciais anteriores
ou nos costumes do seu povo. Partem, consciente ou inconscientemente, da
patológica idéia de que o sistema jurídico tem vida própria, como se
fosse um ``\emph{ente}'' que nascido do ventre do seu criador original,
o povo, dele se desliga para determinar, \emph{per se,} de forma
onisciente, o que deve, pode, ou não deve este mesmo povo fazer.

Só que o ordenamento jurídico, a não ser por meio de ficções
ilusionistas a que se apegam alguns juristas para justificar o que
muitas vezes não pode ser justificado, não é pessoa dotada de
``\emph{vida}'' ou de ``\emph{querer}'' próprio. Ele não é humano, mas
um produto da cultura humana. Sua vida não independe das mentes que o
produziram e que o interpretam. Não é um ``\emph{ser em si e que existe
para si}'', como se fosse uma divindade que paira acima da
temporalidade, intocável pelas paixões humanas. Não pode ser visto como
um oráculo cujos dizeres serão apenas decifráveis a partir da
``\emph{pedra de roseta''} descoberta pelos hermeneutas, os únicos
detentores dos sacros segredos guardados à distância do conhecimento de
todos, à semelhança do que ocorria na medieval biblioteca retratada por
Umberto Eco no seu célebre ``\emph{O Nome da Rosa}''.

Os ordenamentos jurídicos dos Estados Democráticos de Direito, quer
queiram ou não alguns dos ativistas modernos, não passam de ser um
conjunto de regras idealizadas e aprovadas, a partir de decisões que
apenas deveriam ser tomadas pelos representantes eleitos pelo povo, com
o objetivo de orientar a vida humana e social. Quem lhes empresta vida e
vontade próprias, todavia, ao interpretarem o que dizem estas regras,
serão sempre seres mortais e falíveis, ou seja, os juizes. Como
quaisquer seres humanos eles jamais conserguirão ser ``\emph{neutros}'',
embora no exercício das suas funções estejam proibidos de ser
``\emph{parciais}'', isto é, de agirem como ``\emph{parte}''. São eles,
de fato, que decidem, em última instância, o que ``dizem'' os princípios
constitucionais. E o fazem sempre de acordo com a sua visão de mundo,
com a sua ideologia, e com o seu querer pessoal.

Por isso, a pretexto de ``\emph{dizerem}'' o direito, jamais poderiam os
magistrados, invocando construções retóricas engenhosas ou mirabolantes,
\emph{``criar''} o direito \emph{com caráter inovador primário}, como se
fossem legisladores. Não receberam do povo a missão de decidir em seu
nome o que seus representantes nunca decidiram. Em especial quando criam
o direito ``inovador'' decidindo \emph{contra legem} ou mesmo contra o
sentido óbvio das próprias normas constitucionais, valendo"-se de uma
aparente ponderação de princípios para fazer do \emph{direito posto} um
\emph{direito por eles imposto}.

É claro que as fronteiras delimitadoras do ativismo e da submissão à lei
não são fáceis de serem estabelecidas. Mas são as aberrações, abudantes
em todo o mundo, que tem feito muitos levantarem as bandeiras da
democracia bradando que hoje estamos entrando num período onde
predominaria \emph{a ``ditadura dos juizes''}. Se é verdade que todo o
homem que tem o poder tende a dele abusar, como dizia Montesquieu, e que
o poder deveria limitar o poder, a questão da ausência de limites
externos ao agir do Poder Judiciário estaria demolindo o clássico
\emph{princípio da separação de Poderes,} outro dos fundamentos
basilares do \emph{Estado Democrático de Direito}.

A bem da verdade, reconheçamos, há ainda aqui os ``progressistas'' que
invocando a mesma democracia, aplaudem esse ativismo normativo produtor
de inovações primárias. Dizem que por ele o ``\emph{Deus"-ordenamento}''
ganha vida própria e agilidade para fazer as transformações necessárias
em favor dos excluidos e dos oprimidos. A contenção aos limites da lei
deveria ceder espaço a um ``\emph{consequencialismo}'' judicial que
atenderia ora aos ``\emph{reclamos da sociedade}'', ora aos
\emph{``valores humanistas''.}

Não se pode deixar de reconhecer, por vezes, que esse ``\emph{ativismo
normativo primário}'', afirmado a partir de um espaço jurídico vazio ou
de uma destemida ação \emph{contra legem,} tem produzido importantes
transformações, especialmente no plano de valores humanistas. Esse
pragmatismo, porém, não pode justificar aquilo que na prática significa
a exclusão da palavra ``\emph{Democrático}'' da expressão \emph{Estado
Democrático de Direito}. Magistrados não podem falar pelo povo (como se
fossem o \emph{Führer} dos novos tempos), na medida em que não foram
eleitos ou escolhidos para fazerem opções políticas em seu nome. Foram
investidos em cargos públicos para aplicarem o que já foi constituído
como ``norma'' por aqueles que possuem legitimidade democrática de
fazê"-lo, nos termos da institucionalidade vigente.

Aliás, mesmo os pragmáticos ``democratas'' que apaludem o \emph{ativismo
progressista,} pressupondo as vezes de forma inconsciente que os fins
justificam os meios, deveriam se dar conta de algo que recentemente um
professor acusado de dogmático pelos que assistiam a uma sua exposição
acabou lembrando aos ``bem"-intencionados humanistas'' que o ouviam.
Lembrou, em um singelo exemplo, que quando se abre a porta de uma
gaiola, torna"-se impossível escolher os pássaros que dela fugirão.
Aberta a gaiola da legalidade democrática, os pássaros que fugirão, em
certos momentos, poderão ser os ``\emph{humanistas}'' e os
``\emph{progressistas}''. Mas pela mesma porta aberta fugirão,
certamente, os pássaros \emph{repressores}, os \emph{violadores de
direitos humanos}, os \emph{predadores preconceituosos das vidas e das
liberdades}, ou ainda os \emph{aniquiladores autoritários das garantias
consagradas nos textos constitucionais}. Afinal, pássaros conservadores
e fascistas também voam.

Por isso, o respeito à legalidade democrática pode ser mais lento para a
conquista das transformações sociais e humanas, porque exige uma luta
mais intensa, de convencimento, de desmascaremento ideológico e de
derrota de preconceitos enraizados. Mas, indiscutivelmente, é mais
segura, menos elitista e mais respeitadora do sagrado direito da maioria
de decidir como quer viver, mesmo que pareça injusta e indevida a alguns
a decisão. Para o bem ou pelo mal, para o justo ou para o injusto, é
melhor viver na democracia do que sob o império autoritário de alguns
iluminados. O mundo não será mais feliz quando os filósofos ou os juizes
forem reis, ou quando os reis forem filósofos ou juízes, ao contrário do
que parecem pensar alguns. Ele será sempre melhor quando não houver reis
de qualquer natureza.

Além de tudo o que foi dito, devemos lembrar que \emph{a democracia
representativa}, razão última de ser do próprio princípio da legalidade,
está em crise em decorrência de uma outra questão. No mundo que é
informado e interage pela \emph{web,} a cada dia que passa, eleger
pessoas e pagar seus salarios para que falem em nome do povo, parece
algo esquizofrênico e distante do cotidiano. Se negócios bilionários
podem ser decididos pelos própios titulares das obrigações que assumem
com um simples apertar de botões, se pessoas podem em tempo real receber
todo o universo de informações que chega aos parlamentares ou
governantes no momento em que vão, em nome do povo, decidir se aprovam
ou não uma dada decisão, por que este mesmo povo não poderia chamar para
si, diretamente, a responsabilidade de governar ou de legislar
diretamente sobre suas próprias vidas?

É fato que talvez questões dessa natureza não estejam sendo
racionalmente pensadas por todos os que reclamam dos representantes que
elegem para governar ou para legislar em todo o mundo. Mas é evidente
que seu descontentamento sobre o que se decide em seu nome é uma
realidade crescente. O desejo de protagonismo absoluto no exercício do
poder politico parece uma realidade que cresce dia a dia, muitas vezes
ganhando as ruas em manifestações sem líderes, sem pauta, onde o fio
condutor único, muitas vezes inconsciente, parece ser a idéia de que
``\emph{aqueles que elegi não me representam quando decidem sobre a
minha vida}''.

Sem dúvida, por força de tudo o que foi exposto, pode"-se afirmar que o
modelo do \emph{Estado Democrático de Direito}, em todo o mundo, está em
crise, sendo impossível, no momento, dizer se e quando ela será
superada.

\subsection{A crise do Estado Social}

Parece inegável que nas últimas décadas do século \versal{XX}, e durante os anos
que deram início ao século \versal{XXI}, o \emph{Estado Social} tem sofrido
ataques frontais dos que desejam o retorno da \emph{mão invisível do
mercado} para tudo regular.

Há quem diga que o fim da \versal{URSS}, da bipolaridade mundial, e da própria
redução dos riscos impostos pelo ``marxismo revolucionário'' teria
liberado os velhos fantasmas liberais para sairem dos túmulos em que se
encontravam seupultados. Teriam então voltado a assombrar o mundo,
pretendendo aniquilar, para o pavor e o desespero de muitos, os direitos
conquistados pela classe trabalhadora e pela própria sociedade, em
diferentes dimensões. Se no passado cederam"-se os anéis para não se
perder os dedos, afirma"-se que agora os fantasmas carcomidos teriam
renascido e lutam, com vigor, para recuperar os velhos anéis cedidos.

Independentemente do simplismo ou não desta explicação, a verdade é que
a defesa de um ``Estado mínimo'', da redução de direitos, da
desregulamentação e do não intervencionismo na economia, passou a ser
uma realidade permanente, ora se impondo com força, ora reduzindo de
intensidade em todo o mundo. Se por um lado as crises da economia,
postas agora em âmbito global, e os agudos conflitos sociais que tomaram
as ruas em muitos países, implicaram no arrefecimento das lutas travadas
pelos segmentos \emph{neoliberais}, por outro, a intolerância, o
preconceito contra os excluídos, a xenofobia, o inchaço e a ineficiência
da máquina estatal, o mito ideológico da insuperável eficiência do mundo
privado, acompanhados das próprias dificuldades econômicas enfrentadas
na manutenção de programas e serviços sociais, na geração de empregos em
uma sociedade cujo desenvolvimento tecnologico avança a cada dia, e de
pagamento de aposentadorias para uma população que teima em viver mais,
tem fragilizado, em muitos momentos, a estabilidade dos \emph{Estados
Democráticos de Direito} estruturados sob a feição de \emph{Estados
Sociais}.

A disputa, a respeito, permanece em aberta em todo o mundo, configurando
uma própria crise entre as espécies (\emph{Estado Liberal} e
\emph{Estado Social}), dentro da própria crise do gênero (\emph{Estado
Democrático de Direito}). Uma situação de difícil diagnóstico e de quase
impossível previsão acerca do que ocorrerá no futuro, já que os
\emph{fatores reais do poder} oscilam de um lado para o outro\emph{,} em
cada país e no mundo, sem que se possa dizer que algum dos lutadores
tenha ido ou irá a nocaute.

\subsection{A crise brasileira}

Seria impossível que a crise do \emph{Estado Democrático de Direito} e a
crise do \emph{Estado Social} não atingissem de frente a imagem
fotografada pela Constituição de 1988, produzindo incompatibilidades
entre esta e as realidades que normativamente pretende manter e
transformar. Todavia, o distanciamento entre o ``\emph{ser}'' dinâmico
da vida política e institucional brasileira e o ``\emph{dever ser}''
constitucional alargou"-se imensamente por razões peculiares e históricas
determinadas por mudanças acentuadas nos nossos \emph{fatores reais do
poder} ao longo dos últimos anos.

Na primeira década do século \versal{XXI} vivemos um período de grande
transformação social, decorrente da boa utilização governamental do
modelo de \emph{Estado Social.} Foram desenvolvidos programas que
tiraram milhões de brasileiros da linha da miséria, assegurou"-se a
estabilidade econômica, atingiu"-se o pleno emprego, afirmou"-se no plano
internacional a soberania e a importância do Brasil como um país
emergente. Os êxitos e a prosperidade elevaram a autoestima dos
brasileiros a patamares poucas vezes sentidos em outros momentos da
nossa história. Brasileiros antes excluídos completamente do mercado de
consumo, começaram a consumir, ingressando, sem aviso prévio, em numa
nova classe media e fazendo girar a economia.

O quadro social e econômico alterava"-se tão velozmente que a própria mão
de obra especializada não atendia as necessidades do mercado interno,
havendo quem defendesse, sob a resistência das entidades corporativas, o
ingresso de profissionais estrangeiros no país, com a finalidade de que
se pudesse dar atendimento ao processo gerador de demandas econômicas
vivenciadas no período. O Brasil passava a ser um destino preferencial
dos que buscavam refúgio, dos desempregados, ou daqueles que buscavam
uma qualidade de vida melhor para si e para suas famílias, em um mundo
turbulento, marcado por agudas crises econômicas, sociais e políticas.

Uma nova realidade, até então desconhecida, percorria os ares e os
subterrâneos da nossa vida social, inflando corações e mentes, elevando
a autoestima de um povo que historicamente sempre viveu sob o domínio de
um autêntico ``\emph{complexo de vira"-lata}''\footnote{Expressão criada
  pelo escritor Nelson Rodrigues, para se referir à inferioridade em que
  o brasileiro se coloca, voluntariamente, em frente ao resto do mundo.}.

É também inegável, contudo, que ao longo da segunda década do século
\versal{XXI}, gradativamente, veio a se instalar, em segmentos significativos da
nossa população, um sentimento de profunda insatisfação. As intensas
mudanças sociais não se fizeram acompanhar, com a mesma velocidade, de
uma melhoria imediata no plano da infraestrutura e da oferta dos
serviços prestados ao conjunto da sociedade, tanto na sua dimensão
quantitativa como qualitativa. Aeroportos abarrotados de pessoas que
antes viajavam de ônibus ou não viajavam, automóveis congestionando e
paralisando o trânsito das grandes cidades, celulares que não
funcionavam em decorrência da sobrecarga das redes de comunicação, ao
lado de tantas outras situações, geravam uma nova e desconhecida forma
de mal estar social.

Segundo alguns estudos demonstraram, esta veloz transformação social,
reflexamente, propiciava, em larga medida, um crescente mau humor, ou
seja, um indesejável e inesperado efeito colateral decorrente do
desenvolvimento e do sucesso das políticas inclusivas vivenciadas pelo
conjunto da sociedade. Os que não ascenderam ou não prosperaram tanto na
escala social, viam a ascensão dos que consideravam socialmente
``inferiores'' a patamares equivalentes ou mais próximos aos seus,
pleiteando serviços, oportunidades e espaço nas ruas que antes não lhes
pertenciam, em moldes nunca antes pretendidos e disputados naquela
intensidade. E se sentiam incomodados por isso, vivenciando, a cada novo
dia, a sensação amarga e frustrante de estarem sendo nivelados àqueles
que tinham aprendido a considerar, desde que nasceram, como socialmente
``inferiores''. Já os que ascenderam a uma nova classe média,
frequentemente, se sentiam tolhidos e frustrados por não poderem
desfrutar, na plenitude das suas possibilidades, do novo patamar de
consumo, da acessibilidade ao conjunto de serviços e de uma nova e
desejada qualidade de vida. Como as ruas das grandes cidades e das
estradas de rodagem, a sociedade, em alguns dos seus segmentos, estava
congestionada agora pelos novos e velhos atores, gerando reações
passionais de frustração, nem sempre fáceis de serem compreendidas pela
razão.

De outro lado, ainda, esse sentimento de mau humor passou a ser
alimentado em dose vertiginosa pela aparente ``descoberta coletiva'' da
existência da corrupção em nosso país. É fato que qualquer pessoa que
tenha minimamente se ocupado em conhecer a história brasileira, ou não
tenha vivido dentro de um quarto fechado e escuro ao longo de toda a sua
existência, jamais poderá dizer que desconhecia que a corrupção sempre
correu livre, leve e solta, na nossa vida política e empresarial.
Somente um pateta ingênuo, ou um hipócrita farsante, poderá afirmar, em
sã consciência, que a existência de um sistema politico corroído nas
suas entranhas pela corrupção, no Brasil, teria tido seu nascimento
apenas a partir da eleição de um governo, ou da chegada ao poder de um
partido, nos primeiros anos do século \versal{XXI}.

Todavia, patetas e hipócritas sempre existiram em todos os países do
mundo, e em todos os momentos da história da humanidade. No Brasil,
também não foi diferente. Especialmente quando o que corria nos esgotos
da política brasileira foi posto a céu aberto, podendo ser observado a
olho nú e combatido de frente. Órgãos da grande mídia, a soldo ou por
compromissos politicos"-ideológicos, se encarregavam de confundir a
``descoberta'' de algo preexistente, com o ``nascimento'' de uma
realidade ilícita. Disseminavam a intolerância, o ódio generalizado
contra ``os criadores da corrupção'', vazando seletivamente
investigações, jogando tintas em fatos ainda não investigados que
envolviam o seus alvos preferenciais, ou mesmo criando inverdades,
propulsionados pelos seus interesses pouco nobres. Incentivavam a
execração pública daqueles que não estavam sob a sua proteção, e
colocavam sob as suas asas aqueles que lhes interessava, no mundo
politico ou empresarial. Agindo como discípulos de Goebbels, criaram
herois e vilões, anjos e demônios, incentivando, nos rincões mais
soturnos da alma humana, o maniqueísmo passional, generalizador e
violento. Semearam, em terras de tolerância fertil, as sementes do
fascismo que hoje germina e floresce a olhos vistos, assombrando os que
não esperavam reviver, ao menos nessa fase evolutiva da nossa história,
novamente, esta amarga e insólita experiência.

Se olharmos racionalmente e de modo objetivo os fatos do período,
encontraremos, uma explicação evidente para o desnudar dos rios fétidos
que corriam pelos subterraneos da vida política nacional, naquele
particular momento da nossa história. Novos mecanismos legislativos e
administrativos de transparência da atuação estatal e de garantia da
autonomia institucional de órgãos de controle e de investigação de
práticas delituosas, foram sendo criados, curiosamente, pelos próprios
governos que seriam acusados, mais tarde, por escandalizadas vestais
hipocritamente virgens, de terem ``\emph{criado a corrupção no país}''.

Foram estes novos instrumentos institucionais que geraram uma exposição
e um enfrentramento da improbidade e do desapego à \emph{res publica}
por governantes, parlamentares, administradores e líderes da iniciativa
privada, em dimensão talvez nunca antes vivenciada de forma permanente e
continua pelo nosso povo.

A corrupção que estruturalmente sempre caracterizou o nosso sistema
político, na sua feição ativa e passiva, e os crimes de colarinho
branco, passaram a ser colocados sob a luz do sol, sendo descobertos e
investigados por um Estado brasileiro que abandonou a rota dos
``engavetadores da República'' e passou a seguir um novo rumo. Ricos e
detentores do poder político começaram a ser réus de processos
criminais, em situação anterior apenas reservada aos pobres ou
excluídos.

Essa nova realidade, infelizmente, deu ensejo a posturas messiânicas e
prepotentes de alguns agentes públicos que se consideravam
``iluminados'', e a encenação perversa de verdadeiras farsas burlescas
por parte de outros. O mau humor e a indignação ganhou definitivamente
as ruas e propulsionou protestos que envolveram multidões. Bradava"-se
pela necessidade de se ter no país uma cópia tupiniquim da operação
``\emph{Mãos limpas''} realizada da Itália, reproduzindo"-se os seus
métodos e as suas ações, como se fossem um bom receituário. Esquecia"-se,
porém, de se fazer uma análise mais detida sobre o que representou para
aquele país europeu essa operação, tanto quanto à sua vida política
democrática, como no que concerne ao próprio combate à corrupção em si
mesmo considerado.

Naquele mesmo período, uma crise econômica decorrente de fatores
internacionais, de equívocos governamentais, ou de ambos, também teve
início. Foi o suficiente para que os \emph{fatores reais do poder}
entrassem em choque aberto, muitas vezes desconhecendo o caminho do
precipício em que se aventuravam, tanto no plano econômico, como no
institucional.

A \emph{marcha da insensatez} sempre foi uma característica da vida
política humana, como ensinou a célebre historiadora autodidata
estadonidense Barbara Tutchmnan, no seu afamado livro que leva esse
nome\footnote{\emph{A Marcha da Insensatez -- de Tróia ao Vietnã.} São
  Paulo: José Olympio Editora, 4a. ed, 1985.}. Não seria diferente no
Brasil do século \versal{XXI}, especialmente quando se conhece o pragmatismo
oportunista de setores das nossas elites e o real descompromisso da
nossa cultura politica com a democracia e o Estado de Direito.

Nesse cenário de mau humor, de crise econômica e de investigação da
corrupção explorado políticamente por setores da grande mídia, se deu a
eleição de 2014. Um duro embate acabou ocorrendo entre os que defendiam
a continuidade do projeto governamental iniciado pelo governo do
Presidente Luis Inácio Lula da Silva e os setores neoliberais, na
disputa de segundo turno eleitoral travado entre a então Ministra Dilma
Rousseff (\versal{PT}) e o senador Aécio Neves (\versal{PSDB}). Venceu a continuidade do
projeto por uma pequena margem de votos.

Os setores neoliberais e conservadores sabiam que tinham perdido uma
oportunidade histórica para retomar o controle do poder político do
país. Uma situação propícia como aquela talvez não voltasse a se repetir
tão cedo. Por isso, logo que foi anunciada a vitória de Dilma Rousseff,
tudo fizeram para deslegitimar o resultado das urnas. Acusações de
fraude eleitoral e outras tentativas, foram utilizadas para evitar a
posse da primeira mulher eleita Presidente da República. Não lograram
êxito.

Logo após a posse, em 1º de janeiro de 2015, os setores derrotados nas
urnas iniciaram uma frenética construção de um \emph{impeachment}
presidencial, mesmo não sabendo qual fato poderiam invocar para tanto.

Só que a versão brasileira da operação \emph{Mãos Limpas,} a brasileira
``\emph{Lava"-Jato},'' colocava em situação desconfortável, a cada dia
mais, a classe política brasileira. A situação se agravava a todo
instante, pela revelação de novos fatos que indicavam que setores
expressivos da vida política e empresarial nacional, de todos os matizes
ideológicos, poderiam vir a ser atingidos pelas investigações em curto
espaço de tempo.

O governo de Dilma Rousseff, porém, era ``excesivamente republicano'',
ao ver de alguns importantes líderes politicos e empresariais, para
impedir a continuidade das investigações. Impunha"-se, assim, para alguns
temerosos do seu próprio futuro, que ``\emph{um novo governo}'' o
fizesse, estancando\emph{``a sangria da classe política brasileira}''.

Sob o comando explícito de Eduardo Cunha, e ``semioculto'' de Michel
Temer (o Vice"-Presidente eleito na chapa com Dilma Rousseff para a
execução do programa de continuidade ao governo do Presidente Lula), e a
atuação orquestrada da sua \emph{entourage} comum, uniram"-se
neoliberais, conservadores e os sequiosos de evitar a continuidade da
``sangria'', sob a benção de setores expressivos da grande mídia.

Um golpe de Estado passou a ser premeditado e articulado, primeiro sob
as sombras do poder, e mais tarde, sob a luz do sol. Um golpe executado
sem armas e sem o uso das forças militares como em 1964, mas que seria
consumado por meio da formação de uma substantiva maioria parlamentar,
de ações da grande mídia e por uma retórica debilitada que buscava
justificar, perante os olhos estupefatos do mundo, o abuso e a afronta
democrática que se cometida.

O objetivo do golpe não era uma mudança do regime democrático, mas
apenas o resgate de uma vitória que não havia sido conseguida por meio
das urnas e a paralisação de uma investigação incômoda para a classe
política brasileira.

Desse modo, os \emph{fatores reais de poder} que passaram a ser
dominantes nessa nova conjutura política e social, não tiveram nenhum
pudor em atropelar a \emph{folha de papel} escrita, como norma
constitucional, em 1988. Um \emph{impeachment} presidencial sem crime de
responsabilidade, e motivado por pretextos retóricamente alinhavados,
foi consumado.

Golpes de Estado sempre deixaram e sempre deixarão sequelas. Mesmo,
diga"-se, se tiverem natureza parlamentar e forem executados por meio de
um processo de \emph{impeachment} presidencial. Como alguém já disse, a
cassação de um mandato presidencial outorgado pelas urnas sempre
equivalerá a um terremoto politico. Contudo, um \emph{impeachment}
presidencial ralizado sem quaisquer fundamentos jurídicos que o
justifique, ao arrepio da Constituição, passando"-se a titularidade do
governo para quem não tinha legitimidade democrática, assume, em um
Estado Democrático de Direito, a dimensão de um terremoto politico
permanente, com abalos sísmicos ininterruptos. Dificilmente as
instituições sobreviverão a uma realidade como essa sem fissuras
profundas, ou sem a possibilidade de desmoronamento das suas vigas
mestras.

Nos \emph{Estados Democráticos de Direito}, sem legitimidade das urnas,
não há governo que consiga satisfatoriamente exercer sua função. E em
uma tal ambiência desastrada, não há poder institucional que se mantenha
intacto e dentro do exercício contido e delimitado das suas próprias
funções. Os agentes politicos, em tais condições de vento e temperatura,
sempre perderão os limites que as normas lhes fixavam. Afinal, no reino
do ``\emph{vale tudo}'' derivado de um \emph{coup d'État}, as regras do
jogo democrático não estão mais pressupostas ou são reconhecidas pelos
operadores de quaisquer dos Poderes do Estado.

Por isso, dado o \emph{impeachment} de Dilma Rousseff, o que era
claramente previsível, infelizmente, ocorreu. As instituições do
\emph{Estado de Direito}, cuja dimensão \emph{democrátic}a foi eclipsada
ou subtraida do meio da expressão grafada na \emph{folha de papel} de
1988, parecem desconhecer o lugar que deveriam ocupar. Magistrados
passaram a se comportar como parlamentares, depois que parlamentares não
souberam se comportar como magistrados ao julgar uma Presidenta da
República. Conflitos abertos com o empunhar de bambus e flexas, passaram
a existir entre o Ministério Público, o Poder Legislativo e o Poder
Executivo, ganhando rotina cotidiana e as principais páginas dos
jornais. Um governo acuado, composto por homens brancos conservadores,
atingido por uma impupularidade histórica e aparentemente irreversível,
naufraga sob graves denúncias de corrupção. Seu único plano de governo é
manter"-se no poder a qualquer preço. Um Presidente da República
denunciado criminalmente por fatos gravíssimos, e um Congresso que se
recusa a autorizar que ele seja investigado, geram perplexidade em todos
os rincões civilizados do planeta. Especialmente quando se atenta para o
fato de que esse é exatamente o mesmo Congresso que cassou o mandato de
Dilma Rousseff por questões técnicas de gestão orçamentária, sem a prova
mínima da prática de qualquer ato ilícito doloso que tenha praticado, e
por atos que haviam sido também realizados pelos governos que a
antecederam.

Em situação antagônica ao determinado pelas urnas, o governo Temer
passou a seguir rigorosamente o programa do candidato derrotado Aécio
Neves, hoje também destroçado na sua imagem política e pública por
fortíssimas denúncias de corrupção. Direitos trabalhistas e sociais
passaram a ser atacados pelo governo e pela maioria parlamentar que o
apoia. Mudanças na Constituição que atentam contra o Estado Social foram
promovidas e aprovadas. Direitos adquiridos são descartados e programas
de combate à exclusão social são ostensivamente desativados, sem
qualquer pudor democrático ou constitucional.

O desapego à Constituição de 1988, ao \emph{Estado Democrático de
Direito} e ao \emph{Estado Social} nela consagrados, tornou"-se uma
realidade própria dos nossos dias. Até o momento, a bem da verdade, só
não se conseguiu evitar a ``s\emph{angria da classe política
brasileira}''. O resto do programa neoliberal do golpe parlamentar de
2016 parece seguir impávido e destemido, sob a regência de um governo
que teme perder o seu batimento cardiaco se não vier a realizá"-lo.

Esse profundo desapego constitucional e institucional também se faz
presente nos processos judiciais. Condena"-se por convicções, não se
exigindo provas. Presume"-se a culpa quando o princípio constitucional é
o da presunção da inocência. Prende"-se, a torto e a direito, sem
condenação transitada julgado. Prende"-se provisoriamente, sem a
configuração dos pressupostos legais, para atender aos anesios do senso
comum, ou para que investigados delatem. Divulgam"-se os resultados das
delações, atingindo"-se a imagem dos denunciados, antes mesmo de qualquer
investigação, sem nenhum amparo legal.

A adoção das bem intencionadas premissas do ativismo judicial passou a
aniquilar garantias constitucionais a pretexto do combate à corrupção,
como se essa luta virutosa, para ser bem feita, exigisse a eliminação
dos igualmente virtuosos valores decorrentes da noção de \emph{Estado
Democrático de Direito}. Os aplausos punitivos passionais parecem valer
mais do que a realidade probatória e o que está escrito na \emph{folha
de papel} de 1988\emph{.} Já dizia Montesquieu que até a virtude exige
limites para que não se abuse do poder. No Brasil, onde juízes que
jamais seriam neutros tornaram"-se parciais, isso parece ter sido
esquecido no fundo de um album de fotografias que se encontra guardado
em algum baú, escondido a sete chaves, do influenciável senso comum.

Mas a \emph{folha de papel} ainda está escrita e segue vendida nas
livrarias aos renitentes que ainda querem relembrar, fazer valer ou
conhecer suas esquecidas palavras, apesar dos \emph{fatores reais do
poder} hoje dominantes não recomendarem a ninguém a sua sediciosa
leitura.

\section{Uma nova fotografia constitucional?}

Em 5 de outubro de 1988, eu acreditava que a nova Constituição tinha
criado um indestrutível \emph{Estado Democrático de Direito} e que
lançava as bases efetivas para um conjunto de transformações importantes
decorrentes do modelo de \emph{Estado Social} que estabeleceu para o
país. Imaginava que ao longo da minha existência dificilmente os
\emph{fatores reais do poder} mudariam a ponto de ser necessária a
instituição de uma nova Carta Constitucional. Pensava ainda que caso um
poder constituinte originário viesse, no futuro, a se manifestar, seria
para registrar ainda maiores avanços na fotografia de uma democracia
mais plural e mais radicalizada, e na conquista de avanços substantivos
no campo de uma igualdade real que viesse a ocupar o espaço normativo
hoje estabelecido em favor da igualdade meramente formal. Não
acreditava, por fé e por convicção, na possibilidade de retrocessos que
pudessem atingir as vigas mestras da Constituição"-cidadã.

Há quase trinta anos atrás, reconheço agora, eu era um otimista ingênuo.
A vida, para minha tristeza, se mostrou perversa com as minhas crenças
ou convicções. O modelo de Estado desenhado na nossa lei maior hoje se
encontra em crise no mundo e reclama atenção especial daqueles que
defendem a continuidade, o aprofundamento da democracia e a defesa de
valores humanistas. O golpe de Estado parlamentar que violentou a nossa
Constituição e atingiu as nossas instituições, assegurou a marcha da
retomada neoliberal e autoritária no país. As garantias e os direitos
que nos bancos universitários passei a conhecer e a defender, no mesmo
momento da vida em que lia Lassalle, passaram nos dias atuais a ser
ignorados ou pisoteados, retórica ou pragmaticamente, inclusive por
membros do próprio Poder que foi concebido, na origem de tudo, para
fazê"-los respeitar.

Hoje também tenho visto, à direita e à esquerda, vozes se erguendo na
defesa da instauração de um novo processo constituinte para a criação de
uma nova Carta Constitucional que substitua a \emph{folha de papel}
datada de 5 de outubro de 1988. Motivados por avaliações diversas, e por
perspectivas politico"-ideológicas distintas, afirmam os defensores desta
tese que os \emph{fatores reais do poder} foram substantivamente
alterados e que, por isso, uma nova fotografia constitucional deve ser
registrada.

Temo imaginar que essa tese possa vingar, no próximo período, diante de
uma possível continuidade do esfacelamento institucional que vivemos no
\emph{pós"-impeachment} de Dilma Rousseff\emph{.} O atual estado de
coisas, ao que me parece, não indica, até o momento, a consolidação de
uma nova configuração dos \emph{fatores reais do poder} hoje
predominantes na sociedade brasileira. Vivemos, nos dias que se seguem,
uma realidade política em disputa. Uma forte e dolorida disputa,
permeada por desatinos e intolerâncias. Uma disputa sobre a manutenção
do \emph{Estado Democratico de Direito} e dos processos próprios de
transforção assegurados pelo \emph{Estado Social}, definidos a partir da
substância axiologica e de conteúdo consignadas na \emph{folha de papel}
aprovada pelo processo constituinte de 1988.

A convocação de um processo constituinte, agora, talvez aumente confusão
política em que vivemos, amplie de modo irremediável as fissuras
institucionais já existentes e a intranquilidade que nos paralisa. Uma
nova fotografia constitucional exige uma consolidação de cenários que
ainda hoje não podem ser captadas na sua inteireza e profundidade pelas
lentes normativas que a nossa sociedade poderia dispor para a captura de
uma nova imagem constitucional. Ainda não está dado quem ganhou e quem
perdeu, no jogo que está sendo jogado. Se hoje parece que o
autoritarismo, o conservadorismo e a intolerância avançam, a verdade é
que as forças políticas que os antagonisam não estão destruídas, não
debandaram e possuem um forte enraizamento social.

Por isso, acredito que podemos e devemos lutar, com vigor, pela
Constituição de 1988, fazendo com que as suas regras que definem o nosso
Estado como Democrático de Direito e Social, voltem a ser seguidas e
aplicadas. Que se conserve o que ela determina deva ser conservado e que
se transforme o que ela determina deva ser transformado. Ainda há muito
a ser consquistado em avanços democráticos e sociais sob o seu império.
Combatamos, de frente e com vigor, os retrocessos golpistas e o
autoritarismo, pouco importando de onde promanem.

A luta contra a barbarie, pela dignidade humana, contra o autoritarismo
e pela democracia sempre valem pena. E continuará a valer, se a alma
continuar a não ser pequena.
