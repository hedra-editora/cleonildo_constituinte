\part{Discursos}

\chapter*{Discurso do deputado Ulysses Guimarães, presidente da Assembleia
Nacional Constituinte, em 5 de outubro de 1988, por ocasião da
promulgação da Constituição Federal\footnote{Anais da Assembleia Nacional Constituinte – \versal{CEDI} (Centro de Documentação e Informação da Câmara Federal).}}

\addcontentsline{toc}{chapter}{Ulysses Guimarães, 5 de outubro de 1988}
\hedramarkboth{Discurso -- Ulysses Guimarães}{}

Exmo. Sr. Presidente da República, José Sarney; Exmo. Sr. Presidente
do Senado Federal, Humberto Lucena; Exmo. Sr. Presidente do Supremo
Tribunal Federal, ministro Rafael Mayer; Srs. Membros da Mesa da
Assembleia Nacional Constituinte; eminente Relator Bernardo Cabral;
preclaros Chefes do Poder Legislativo de nações amigas; insignes
Embaixadores, saudados no decano D. Carlo Furno; Exmos. Srs. Ministros
de Estado; Exmos. Srs. Governadores de Estado; Exmos. Srs. Presidentes
de Assembleias Legislativas; dignos Líderes partidários; autoridades
civis, militares e religiosas, registrando o comparecimento do Cardeal
D. José Freire Falcão, Arcebispo de Brasília, e de D. Luciano Mendes de
Almeida, Presidente da \versal{CNBB}; prestigiosos Srs. Presidentes de
confederações, Sras. e Srs. Constituintes; minhas senhoras e meus
senhores:

Estatuto do Homem, da Liberdade, da Democracia. Dois de fevereiro de
1987: ``Ecoam nesta sala as reivindicações das ruas. A Nação quer mudar,
a Nação deve mudar, a Nação vai mudar.'' São palavras constantes do
discurso de posse como Presidente da Assembleia Nacional Constituinte.
Hoje, 5 de outubro de 1988, no que tange à Constituição, a Nação mudou.

A Constituição mudou na sua elaboração, mudou na definição dos poderes,
mudou restaurando a Federação, mudou quando quer mudar o homem em
cidadão, e só é cidadão quem ganha justo e suficiente salário, lê e
escreve, mora, tem hospital e remédio, lazer quando descansa. Num país
de 30.401.000 analfabetos, afrontosos 25\% da população, cabe advertir:
a cidadania começa com o alfabeto.

Chegamos! Esperamos a Constituição como o vigia espera a aurora.
Bem"-aventurados os que chegam. Não nos desencaminhamos na longa marcha,
não nos desmoralizamos capitulando ante pressões aliciadoras e
comprometedoras, não desertamos, não caímos no caminho. Alguns a
fatalidade derrubou: Virgílio Távora, Alair Ferreira, Fábio
Lucena, Antonio Farias e Norberto Schwantes.

Pronunciamos seus nomes queridos com saudade e orgulho: cumpriram com o
seu dever. A Nação nos mandou executar um serviço. Nós o fizemos com
amor, aplicação e sem medo. A Constituição certamente não é perfeita.
Ela própria o confessa, ao admitir a reforma. Quanto a ela, discordar,
sim.

Divergir, sim. Descumprir, jamais. Afrontá"-la, nunca. Traidor da
Constituição é traidor da Pátria. Conhecemos o caminho maldito: rasgar a
Constituição, trancar as portas do Parlamento, garrotear a liberdade,
mandar os patriotas para a cadeia, o exílio, o cemitério.

A persistência da Constituição é a sobrevivência da democracia. Quando,
após tantos anos de lutas e sacrifícios, promulgamos o estatuto do
homem, da liberdade e da democracia, bradamos por imposição de sua
honra: temos ódio à ditadura. Ódio e nojo.

Amaldiçoamos a tirania onde quer que ela desgrace homens e nações,
principalmente na América Latina. Assinalarei algumas marcas da
Constituição que passará a comandar esta grande Nação.

A primeira é a coragem. A coragem é a matéria"-prima da civilização. Sem
ela, o dever e as instituições perecem. Sem a coragem, as demais
virtudes sucumbem na hora do perigo. Sem ela, não haveria a cruz, nem os
evangelhos. A Assembleia Nacional Constituinte rompeu contra o
establishment, investiu contra a inércia, desafiou tabus. Não ouviu o
refrão saudosista do velho do Restelo, no genial canto de Camões.

Suportou a ira e perigosa campanha mercenária dos que se atreveram na
tentativa de aviltar legisladores em guardas de suas burras abarrotadas
com o ouro de seus privilégios e especulações.

Foi de audácia inovadora a arquitetura da Constituinte, recusando
anteprojeto forâneo ou de elaboração interna. O enorme esforço é
dimensionado pelas 61.020 emendas, além de 122 emendas populares,
algumas com mais de 1 milhão de assinaturas, que foram apresentadas,
publicadas, distribuídas, relatadas e votadas, no longo trajeto das
subcomissões à redação final.

A participação foi também pela presença, pois diariamente cerca de 10
mil postulantes franquearam, livremente, as 11 entradas do enorme
complexo arquitetônico do Parlamento, na procura dos gabinetes,
comissões, galeria e salões.

Há, portanto, representativo e oxigenado sopro de gente, de rua, de
praça, de favela, de fábrica, de trabalhadores, de cozinheiros, de
menores carentes, de índios, de posseiros, de empresários, de
estudantes, de aposentados, de servidores civis e militares, atestando a
contemporaneidade e autenticidade social do texto que ora passa a
vigorar. Como o caramujo, guardará para sempre o bramido das ondas de
sofrimento, esperança e reivindicações de onde proveio.

A Constituição é caracteristicamente o estatuto do homem. É sua marca de
fábrica. O inimigo mortal do homem é a miséria. O Estado de Direito,
consectário da igualdade, não pode conviver com estado de miséria. Mais
miserável do que os miseráveis é a sociedade que não acaba com a
miséria.

Tipograficamente é hierarquizada a precedência e a preeminência do
homem, colocando"-o no umbral da Constituição e catalogando"-lhe o número
não superado, só no art. 5º., de 77 incisos e 104 dispositivos.

Não lhe bastou, porém, defendê"-lo contra os abusos originários do Estado
e de outras procedências. Introduziu o homem no Estado, fazendo"-o credor
de direitos e serviços, cobráveis inclusive com o mandado de injunção.
Tem substância popular e cristã o título que a consagra: ``a
Constituição Cidadã''.

Vivenciados e originários dos Estados e Municípios, os Constituintes
haveriam de ser fiéis à Federação. Exemplarmente o foram.

No Brasil, desde o Império, o Estado ultraja a geografia. Espantoso
despautério: o Estado contra o País, quando o País é a geografia, a base
física da Nação, portanto, do Estado. É elementar: não existe Estado sem
país, nem país sem geografia. Esta antinomia é fator de nosso atraso e
de muitos de nossos problemas, pois somos um arquipélago social,
econômico, ambiental e de costumes, não uma ilha. A civilização e a
grandeza do Brasil percorreram rotas centrífugas e não centrípetas. Os
bandeirantes não ficaram arranhando o litoral como caranguejos, na
imagem pitoresca mas exata de Frei Vicente do Salvador. Cavalgaram os
rios e marcharam para o oeste e para a História, na conquista de um
continente.

Foi também indômita vocação federativa que inspirou o gênio do
Presidente Juscelino Kubitschek, que plantou Brasília longe do mar, no
coração do sertão, como a capital da interiorização e da integração.

A Federação é a unidade na desigualdade, é a coesão pela autonomia das
províncias. Comprimidas pelo centralismo, há o perigo de serem
empurradas para a secessão. É a irmandade entre as regiões. Para que não
se rompa o elo, as mais prósperas devem colaborar com as menos
desenvolvidas. Enquanto houver Norte e Nordeste fracos, não haverá na
União Estado forte, pois fraco é o Brasil. As necessidades básicas do
homem estão nos Estados e nos Municípios. Neles deve estar o dinheiro
para atendê"-las.

A Federação é a governabilidade. A governabilidade da Nação passa pela
governabilidade dos estados e dos municípios. O desgoverno, filho da
penúria de recursos, acende a ira popular, que invade primeiro os paços
municipais, arranca as grades dos palácios e acabará chegando à rampa do
Palácio do Planalto.

A Constituição reabilitou a Federação ao alocar recursos ponderáveis às
unidades regionais e locais, bem como ao arbitrar competência tributária
para lastrear"-lhes a independência financeira. Democracia é a vontade da
lei, que é plural e igual para todos, não a do príncipe, que é
unipessoal e desigual para os favorecimentos e os privilégios.

Se a democracia é o governo da lei, não só ao elaborá"-la, mas também
para cumpri"-la, são governo o Executivo e o Legislativo.

O Legislativo brasileiro investiu"-se das competências dos Parlamentos
contemporâneos. É axiomático que muitos têm maior probabilidade de
acertar do que um só. O governo associativo e gregário é mais apto do
que o solitário.

Eis outro imperativo de governabilidade: a coparticipação e a
corresponsabilidade. Cabe a indagação: instituiu"-se no Brasil o
tricameralismo ou fortaleceu"-se o unicameralismo, com as numerosas e
fundamentais atribuições cometidas ao Congresso Nacional? A resposta
virá pela boca do tempo. Faço votos para que essa regência trina prove
bem.

Nós, os legisladores, ampliamos nossos deveres. Teremos de honrá"-los. A
Nação repudia a preguiça, a negligência, a inépcia. Soma"-se à nossa
atividade ordinária, astante dilatada, a edição de 56 leis
complementares e 314 ordinárias. Não esqueçamos que, na ausência de lei
complementar, os cidadãos poderão ter o provimento suplementar pelo
mandado de injunção.

A confiabilidade do Congresso Nacional permite que repita, pois tem
pertinência, o slogan: ``Vamos votar, vamos votar'', que integra o
folclore de nossa prática constituinte, reproduzido até em horas de
diversão e em programas humorísticos.

Tem significado de diagnóstico a Constituição ter alargado o exercício
da democracia, em participativa além de representativa. É o clarim da
soberania popular e direta, tocando no umbral da Constituição, para
ordenar o avanço no campo das necessidades sociais. O povo passou a ter
a iniciativa de leis. Mais do que isso, o povo é o superlegislador,
habilitado a rejeitar, pelo referendo, projetos aprovados pelo
Parlamento. A vida pública brasileira será também fiscalizada pelos
cidadãos. Do Presidente da República ao Prefeito, do senador ao
Vereador.

A moral é o cerne da Pátria. A corrupção é o cupim da República.
República suja pela corrupção impune tomba nas mãos de demagogos, que, a
pretexto de salvá"-la, a tiranizam. Não roubar, não deixar roubar, pôr na
cadeia quem roube, eis o primeiro mandamento da moral pública. Pela
Constituição, os cidadãos são poderosos e vigilantes agentes da
fiscalização, através do mandado de segurança coletivo; do direito de
receber informações dos órgãos públicos, da prerrogativa de petição aos
poderes públicos, em defesa de direitos contra ilegalidade ou abuso de
poder; da obtenção de certidões para defesa de direitos; da ação popular, que pode ser
proposta por qualquer cidadão, para anular ato lesivo ao patrimônio
público, ao meio ambiente e ao patrimônio histórico, isento de custas
judiciais; da fiscalização das contas dos Municípios por parte do
contribuinte; podem peticionar, reclamar, representar ou apresentar
queixas junto às comissões das Casas do Congresso Nacional; qualquer
cidadão, partido político, associação ou sindicato são partes legítimas
e poderão denunciar irregularidades ou ilegalidades perante o Tribunal
de Contas da União, do Estado ou do Município. A gratuidade facilita a
efetividade dessa fiscalização.

A exposição panorâmica da lei fundamental que hoje passa a reger a Nação
permite conceituá"-la, sinoticamente, como a Constituição coragem, a
Constituição cidadã, a Constituição federativa, a Constituição
representativa e participativa, a Constituição do Governo síntese
Executivo"-Legislativo, a Constituição fiscalizadora. Não é a
Constituição perfeita. Se fosse perfeita, seria irreformável. Ela
própria, com humildade e realismo, admite ser emendada, até por maioria
mais acessível, dentro de 5 anos.

Não é a Constituição perfeita, mas será útil, pioneira, desbravadora.
Será luz, ainda que de lamparina, na noite dos desgraçados. É caminhando
que se abrem os caminhos. Ela vai caminhar e abri"-los.

Será redentor o caminho que penetrar nos bolsões sujos, escuros e
ignorados da miséria. Recorde"-se, alvissareiramente, que o Brasil é o
quinto país a implantar o instituto moderno da seguridade, com a
integração de ações relativas à saúde, à previdência e à assistência
social, assim como a universalidade dos benefícios para os que
contribuam ou não, além de beneficiar 11 milhões de aposentados,
espoliados em seus proventos.

É consagrador o testemunho da \versal{ONU} de que nenhuma outra Carta no mundo
tenha dedicado mais espaço ao meio ambiente do que a que vamos
promulgar. Sr. Presidente José Sarney: V.Exa. cumpriu exemplarmente o
compromisso do saudoso, do grande Tancredo Neves, de V.Exa. e da Aliança
Democrática ao convocar a Assembleia Nacional Constituinte.

A Emenda Constitucional nº26 teve origem em mensagem do Governo, de
V.Exa., vinculando V.Exa. à efemeridade que hoje a Nação celebra.

Nossa homenagem ao Presidente do Senado, Humberto Lucena, atuante na
Constituinte pelo seu trabalho, seu talento e pela colaboração fraterna
da Casa que representa. Sr. Ministro Rafael Mayer, Presidente do Supremo
Tribunal Federal, saúdo o Poder Judiciário na pessoa austera e modelar
de V.Exa. O imperativo de ``Muda Brasil'', desafio de nossa geração, não
se processará sem o conseqüente ``Muda Justiça'', que se
instrumentalizou na Carta Magna com a valiosa contribuição do poder
chefiado por V.Exa. Cumprimento o eminente Ministro do Supremo Tribunal
Federal, Moreira Alves, que, em histórica sessão, instalou em 1º de
fevereiro de 1987 a Assembleia Nacional Constituinte.

Registro a homogeneidade e o desempenho admirável e solidário de seus
altos deveres, por parte dos dignos membros da Mesa Diretora, condôminos
imprescindíveis de minha Presidência.

O Relator Bernardo Cabral foi capaz, flexível para o entendimento, mas
irremovível nas posições de defesa dos interesses do País. O louvor da
Nação aplaudirá sua vida pública.

Os Relatores Adjuntos, José Fogaça, Konder Reis e Adolfo Oliveira,
prestaram colaboração unanimemente enaltecida.

Nossa palavra de sincero e profundo louvor ao mestre da língua
portuguesa Prof. Celso Cunha, por sua colaboração para a escorreita
redação do texto.

O Brasil agradece pela minha voz a honrosa presença dos prestigiosos
dignitários do Poder Legislativo do continente americano, de Portugal,
da Espanha, de Angola, Moçambique, Guiné Bissau, Príncipe e Cabo Verde.
As nossas saudações.

Os Srs. Governadores de Estado e Presidentes das Assembleias
Legislativas dão realce singular a esta solenidade histórica. Os Líderes
foram o vestibular da Constituinte. Suas reuniões pela manhã e pela
madrugada, com autores de emendas e interessados, disciplinaram,
agilizaram e qualificaram as decisões do Plenário. Os Anais guardarão
seus nomes e sua benemérita faina.

Cumprimento as autoridades civis, eclesiásticas e militares, integrados
estes com seus chefes, na missão, que cumprem com decisão, de prestigiar
a estabilidade democrática.

Nossas congratulações à imprensa, ao rádio e à televisão. Viram tudo,
ouviram o que quiseram, tiveram acesso desimpedido às dependências e
documentos da Constituinte. Nosso reconhecimento, tanto pela divulgação
como pelas críticas, que documentam a absoluta liberdade de imprensa
neste País. Testemunho a coadjuvação diuturna e esclarecida dos
funcionários e assessores, abraçando"-os nas pessoas de seus excepcionais
chefes, Paulo Affonso Martins de Oliveira e Adelmar Sabino. Agora
conversemos pela última vez, companheiras e companheiros constituintes.

A atuação das mulheres nesta Casa foi de tal teor, que, pela edificante
força do exemplo, aumentará a representação feminina nas futuras
eleições.

Agradeço a colaboração dos funcionários do Senado --- da Gráfica e do
Prodasen.

Agradeço aos Constituintes a eleição como seu Presidente e agradeço o
convívio alegre, civilizado e motivador.

Quanto a mim, cumpriu"-se o magistério do filósofo: o segredo da
felicidade é fazer do seu dever o seu prazer. Todos os dias, meus amigos
constituintes, quando divisava, na chegada ao Congresso, a concha
côncava da Câmara rogando as bênçãos do céu, e a convexa do Senado
ouvindo as súplicas da terra, a alegria inundava meu coração.

Ver o Congresso era como ver a aurora, o mar, o canto do rio, ouvir os
passarinhos. Sentei"-me ininterruptamente 9 mil horas nesta cadeira, em
320 sessões, gerando até interpretações divertidas pela não"-saída para
lugares biologicamente exigíveis. Somadas as das sessões, foram 17 horas
diárias de labor, também no gabinete e na residência, incluídos sábados,
domingos e feriados.

Político, sou caçador de nuvens. Já fui caçado por tempestades. Uma
delas, benfazeja, me colocou no topo desta montanha de sonho e de
glória. Tive mais do que pedi, cheguei mais longe do que mereço. Que o
bem que os Constituintes me fizeram frutifique em paz, êxito e alegria
para cada um deles.

Adeus, meus irmãos. É despedida definitiva, sem o desejo de retorno.

Nosso desejo é o da Nação: que este Plenário não abrigue outra
Assembleia Nacional Constituinte. Porque, antes da Constituinte, a
ditadura já teria trancado as portas desta Casa.

Autoridades, Constituintes, senhoras e senhores, A sociedade sempre
acaba vencendo, mesmo ante a inércia ou antagonismo do Estado.

O Estado era Tordesilhas. Rebelada, a sociedade empurrou as fronteiras
do Brasil, criando uma das maiores geografias do Universo.

O Estado, encarnado na metrópole, resignara"-se ante a invasão holandesa
no Nordeste. A sociedade restaurou nossa integridade territorial com a
insurreição nativa de Tabocas e Guararapes, sob a liderança de André
Vidal de Negreiros, Felipe Camarão e João Fernandes Vieira, que cunhou a
frase da preeminência da sociedade sobre o Estado: ``Desobedecer a
El"-Rei, para servir a El"-Rei''.

O Estado capitulou na entrega do Acre, a sociedade retomou"-o com as
foices, os achados e os punhos de Plácido de Castro e dos seus
seringueiros.

O Estado autoritário prendeu e exilou. A sociedade, com Teotônio Vilela,
pela anistia, libertou e repatriou.

A sociedade foi Rubens Paiva, não os facínoras que o mataram.

Foi a sociedade, mobilizada nos colossais comícios das Diretas"-já, que,
pela transição e pela mudança, derrotou o Estado usurpador.

Termino com as palavras com que comecei esta fala: a Nação quer mudar.

A Nação deve mudar. A Nação vai mudar.

A Constituição pretende ser a voz, a letra, a vontade política da
sociedade rumo à mudança.

Que a promulgação seja nosso grito:

--- Mudar para vencer! Muda, Brasil!
