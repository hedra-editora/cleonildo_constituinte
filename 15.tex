\chapter*{Os 30 anos da Constituição Brasileira\\
\emph{A resistência popular vai parir a reconstrução democrática nas
ruas e nas urnas}}

\addcontentsline{toc}{chapter}{Os 30 anos da Constituição Brasileira,
\scriptsize{por Carmen Foro}}
\hedramarkboth{Os 30 anos da constituição brasileira}{}

\begin{flushright}
\emph{Carmen Foro}\footnote{Vice"-Presidenta da Central Única dos Trabalhadores.}
\end{flushright}

\section{A sede social por democracia}

A Constituição Federal de 1988 trouxe avanços extremamente
significativos, ainda que ela ainda estivesse aquém dos anseios das
classes populares, especialmente do movimento sindical e social.

A participação, fruto da efervescência social da época, deu o tom da
Constituinte de 1987. Movimentos organizados do campo, da cidade e da
floresta elaboraram propostas, articularam parlamentares e defenderam
direitos constituintes da Carta Magna.

A base da \versal{CUT} e da Contag teve participação ativa e decisiva nos os
rumos que tomou a Constituição de 1988 no que se refere aos direitos
trabalhistas e previdenciários e aos avanços do campo. Assim como um sem
número de organizações urbanas e rurais e partidos no campo da esquerda
que se fizeram ouvir pelos parlamentares constituintes.

Aqui quero destacar a importância dos movimentos feministas e de
mulheres na luta para a efetivação da igualdade de direitos e
oportunidades. Em 1988, a luta das mulheres concentrou"-se no
envio da Carta das Mulheres Brasileiras aos
Constituintes. As mulheres estavam na liderança das reivindicações e das
manifestações e asseguraram no texto um dos princípios fundamentais:
``promover o bem estar de todos sem preconceito de origem, sexo, raça,
cor, idade e quaisquer outras formas de discriminação'' é uma diretriz
que tem muito da participação comprometida delas.

Vale resgatar que vivíamos no país as consequências de um momento
histórico de mudanças estruturais na dinâmica do capitalismo mundial.
Altas taxas de juros, os Tigres Asiáticos, a reestruturação produtiva
iniciada no final dos anos 70 e inicio dos anos 80 no Brasil e com ela,
o pacote de desregulamentação do trabalho, precarização, rotatividade,
achatamento de salários e, claro, demissões em massa.

\section{O Novo Sindicalismo}

Como resposta, surge no Brasil o Novo Sindicalismo. Vigoroso, autônomo,
combativo e classista, veio na missão de enfrentar e reverter o
ataque e a repressão sofrida pelos sindicatos e movimentos organizados
desencadeados pela ditadura, além de enfrentar o arrocho salarial
estabelecido nos anos de chumbo e conquistar avanços e direitos
trabalhistas.

Foi um sindicalismo que chegou para ficar, fortaleceu e intensificou a
abertura política e os rumos da transição de uma ditadura civil"-militar
para uma democracia com a inclusão dos trabalhadores. E que, além disso, revelou
inúmeras lideranças que ajudaram a formar uma esquerda aguerrida no
país, dentre elas Luís Inácio Lula da Silva, que, anos depois de operário
e sindicalista, torna"-se Presidente da República, numa trajetória que
nos orgulha imensamente.

A Internet mal havia chegado ao Brasil -- nós nem conhecíamos, era
restrita a algumas universidades --, mas nós acompanhávamos o que acontecia
mundialmente à classe trabalhadora por intermédio das inestimáveis
parcerias e solidariedade do sindicalismo internacional, que sempre nos
colocou a par das ameaças e avanços da exploração perpetrada na
``gestão'' internacional do trabalho.

Naquele momento, estávamos construindo a redemocratização, após 21 anos
de chumbo, de ditatura civil"-militar com assassinatos, torturas,
desagregações familiares, pessoas desaparecidas (até hoje) e outras
tantas traumatizadas. Portanto, havia em nós uma sede gigante
por avanços, por democracia. A sociedade ainda se reorganizava nos
espaços de luta por direitos. Mas não nos contentaríamos com pouco.

\section{1980: Uma década histórica}

Construir a nova Constituição era, então, o nosso terceiro grande aporte
de esperança na transformação da sociedade daquela importante década
para o povo brasileiro. O primeiro foi a Fundação do Partido dos
Trabalhadores em 1980, o segundo a Fundação da Central Única dos
Trabalhadores, em 1983, e o terceiro, a possibilidade concreta de que
nossos direitos estivessem reconhecidos na Constituição da República
Federativa do Brasil, em 1988.

A abertura democrática ratificada pela Constituição Federal era um valor
sobre o qual sabíamos que tínhamos que nos debruçar, propor, resistir e
aprovar. Cabia a nós, a sociedade civil organizada, este enfrentamento. Foi
absolutamente necessária a radicalização na defesa de propostas que
trouxessem concretamente a inclusão dessas maiorias, que, na prática,
eram invisibilizadas enquanto sujeitos sociais.

Fez a diferença a participação popular massiva de nós mulheres, e de nossas
lutas por igualdade de direitos. De nós, negras e negros
historicamente escravizados, que geramos com nossa força de trabalho a
riqueza deste país, mas que não éramos reconhecidos enquanto sujeitos
detentores de direitos.

Foi essencial dar visibilidade aos direitos das crianças e adolescentes,
à juventude, assim como ao reconhecimento e às demarcações das terras
indígenas, quilombolas e de populações tradicionais, compreendendo sua
importância histórica para o nosso país, sua cultura e identidade.

Foi, portanto, um momento em que a sociedade também se enxergou de
maneira mais coletiva: mais do que apenas se enxergar,
foi o momento da sociedade se perceber, se tatear, construir junto o que
de fato nós, povo brasileiro, queríamos ser enquanto nação. Com
instituições fortes, com democracia, com direitos, nesse sentido também o
movimento sindical teve papel fundamental na conquista desses avanços.

Destaco aqui parte da resolução do 2º Congresso Nacional da \versal{CUT},
realizado de 31 de julho a 3 de agosto de 1986:

\begin{quote}
O 2º Congresso da \versal{CUT} conclama os trabalhadores da cidade e do
campo a uma ampla mobilização unitária, uma campanha nacional de lutas,
com as seguintes bandeiras:\\
-- Terra, salário, emprego e liberdade\\
-- Direito irrestrito de greve, liberdade e autonomia sindical (Convenção
87 da \versal{OIT})\\
-- Não pagamento da dívida externa\\
-- Participação popular na Constituinte.
\end{quote}

\section{A conquista de avanços históricos, apesar da correlação de
forças}

Dois anos antes, portanto, a \versal{CUT} prepara suas bases para essa
importante missão. Na sociedade havia também essa expectativa de avanços
na legislação no que tangia aos direitos sociais, dos quais estávamos
abortados e que perpassavam questões centrais para a classe trabalhadora
nos aspectos dos direitos humanos e da justiça social, trabalhista,
ambiental, dentre outras. E, com essa responsabilidade latente de
transformação histórica, passamos a constituir, do nosso jeito, a nossa
expressão na nova Constituição.

A correlação de forças era extremamente desfavorável em um Congresso
Nacional com forte presença conservadora, como a \versal{UDR}, a bancada
religiosa fundamentalista, os representantes das empresas e dos bancos.
Eles eram, como hoje, maioria naquele Congresso.

Com todos os limites que podem ser apontados, apesar de não ser a
Constituinte livre, democrática, soberana e exclusiva que defendíamos,
ela trouxe inúmeros avanços, e conseguimos assegurar caminhos
progressistas. Um exemplo no campo dos direitos foi o
estudo, pesquisa e a pressão popular aliado ao trabalho intenso e
dedicado dos parlamentares comprometidos com a classe trabalhadora que
resultou na Seguridade Social na Constituição Federal.

A Seguridade Social é o sistema tríade que envolve a Assistência, a
Saúde e a Previdência Social.

A Assistência Social depôs o que historicamente se convencionou ser
``comiseração'', ``clientelismo'' e outras desassistências que davam
ares de caridade e generosidade ao que a Constituição de 1988 regula,
passando a tratar enquanto direito social.

Por meio de um aparato normativo com uma Política Nacional de
Assistência Social (\versal{PNAS}) e um Sistema Único de Assistência Social
(\versal{SUAS}) foram definidas as diretrizes da política à partir do direito
cidadão, ressignificando, assim, a relação com a assistência social
brasileira.

Capítulo à parte na Constituinte foi a militância dos movimentos
populares de saúde, em que participavam profissionais de saúde,
estudantes, membros da comunidade, e religiosos com visões progressistas.
Se debatia sobre saúde, mas também sobre direitos e, de lá, surgiram
lideranças que defendiam outra visão de saúde pública e de sociedade.

Os movimentos populares de saúde havia muitos anos já defendiam a
bandeira da luta antimanicomial, a universalização do atendimento
público e o controle social, ou seja, a saúde enquanto direito e de
forma descentralizada. Surge o Sistema Único de Saúde, o \versal{SUS}, um
modelo exitoso e admirado internacionalmente, que, ainda que com muitas
limitações em sua implementação, precisa ser defendido
incondicionalmente, pois representa o acesso universal e público à
saúde, com qualidade.

A exitosa bandeira da Seguridade Social fecha o ciclo com a Previdência.
Dentre outros avanços vou me ater aos da Previdência Rural à
Constituição de 1988: aposentadoria aos 55 anos para mulher e 60 para os
homens, e o salário maternidade para as mulheres rurais.

Cito a experiência e a construção histórica da Seguridade Social como
exemplo, dada sua grandeza e abrangência do conceito de garantia do
Estado à proteção social como direito universal, e em uma prática de
melhoria de vida da população brasileira. Orgulha que construção se
dê a partir da histórica luta popular até a conquista de espaço no
texto da Constituição de 1988.

Da mesma forma, isso ocorre com outras categorias, sobre outras questões. A
universalização da educação pública, o direito de greve, a retomada da
representação sindical, os avanços no campo, todos frutos de lutas populares,
da força e do poder transformador que emana dos movimentos sociais.

Quando conquistamos espaços na Constituinte, esses movimentos estavam
prontos para apresentar propostas sólidas e viáveis. Estavam,
principalmente, prontos enquanto ferramenta de pressão popular, para
influenciar decisivamente na elaboração da Constituição.

O mesmo posso dizer sobre a questão da preservação ambiental e sobre o
direito dos povos da floresta. Naquele momento tínhamos já um forte
movimento sindical e social, em torno dessas bandeiras e como membro da
Direção Nacional da \versal{CUT}, símbolo e liderança, o querido e saudoso
sindicalista e ambientalista Chico Mendes. Conhecido e premiado no mundo
inteiro, ele representou as demandas sociais e ambientais dando
visibilidade aos povos e movimentos da floresta, o que em muito ajudou a
colocar o debate ambiental, sindical e social num novo patamar,
internacional e de significativo respeito à floresta, a quem vive nela e
a tudo o que dela se recebe.

À época da Constituinte, em 1987, Chico veio à São Paulo, e com
companheiros seringueiros do Acre lançou a estratégia da ``Aliança dos
Povos da Floresta''. Quanto à Constituição, Chico viu sua promulgação, o
nascimento da Carta Magna brasileira. Pouco mais de dois meses depois,
em 22 de dezembro de 1988, foi assassinado, e a Aliança dos Povos da
Floresta só viria a se concretizar 2 anos depois.

Óbvio que as mudanças na sociedade brasileira não se deram pela simples
promulgação da Carta Magna. Ao longo desses 30 anos são inúmeras
legislações intraconstituição que exigiram enorme fôlego e disposição
da sociedade para o enfrentamento de sua aprovação. A negociação
coletiva para o setor público é um dos muitos exemplos de lutas que não
foram regulamentadas com a Constituição, até porque a correlação de
forças dentro do Congresso Nacional é a mesma que antes impedia tais
avanços.

\section{Sociedade civil forte e resistente}

Até hoje, muitas das conquistas lá registradas não se tornaram
realidade, sem falar que vivemos um momento da nossa história em que
querem revogá"-la aos trancos e barrancos. O que quero chamar
atenção aqui é para o fato de que os movimentos sociais, os movimentos
feministas, de mulheres e o sindical, de tão bem
articulados e preparados, inclusive do ponto de vista teórico e
intelectual, conseguiram, em grande medida, dobrar a resistência daquele
Congresso, majoritariamente conservador.

Foram abaixo"-assinados, emendas, greves, passeatas, protestos, caravanas
à Brasília em direção ao Congresso Constituinte, cartazes de denúncia dos traidores
da Constituição, corpo a corpo com os parlamentares, uma luta hercúlea.

Insisto nesse resgate, para que tomemos como exemplo nossas iniciativas
e nossa própria luta à época como alavanca para mais um longo período de
resistências que se farão necessárias à partir dos novos ataques à
Constituição Federal, aos direitos trabalhistas e previdenciários, às
suadas conquistas do povo brasileiro.

Cabe lembrar, entretanto, que há uma diferença substantiva entre aquele
momento e este que vivemos. Lá, saíamos de um cenário de obscuridade e
estávamos unidos pela esperança. Foram necessários outros 20 anos para
que chegássemos a um período que foi uma espécie de catarse, a eleição
do primeiro governo democrático e popular.

Hoje, vivemos uma situação inversa. Saímos de um período de grandes
conquistas -- ainda que com as limitações e erros do exercício de
governar -- para adentrarmos num amargo retrocesso, violentados por um
processo criminoso de ataque à democracia e pelo desrespeito ao voto de
mais de 54 milhões de brasileiros.

Das mãos de um Congresso Nacional de maioria vendida e corrupta, surge
um impeachment comprado, um presidente golpista, um judiciário cúmplice,
uma imprensa manipuladora e perversa, empresários que financiaram eles
próprios o golpe para beneficiarem"-se da fragilidade política a que
se submete o país e fazerem valer seus interesses, derrubando conquistas
históricas. E tudo isso somado a uma operação policial parcial, que
quebrou na base as estruturas de desenvolvimento do país, comprometendo,
inclusive, sua soberania.

Diante disso, é preciso que saibamos que o que une as duas épocas é que
temos o acúmulo das lutas e que vivemos, novamente, um período de
necessidade de reconstrução.

\section{Do Golpe às Reformas: o desmonte dos direitos constitucionais}

No apagar das luzes da democracia causada pelo golpe de 2016, já em 15
de dezembro, uma primeira medida dá o tom dos retrocessos: A promulgação
da Emenda Constitucional 95 conhecida como \versal{PEC} do Teto. Trata"-se de uma
emenda que, dentre outras questões e impactos, congela por 20 anos os
investimentos públicos. Sob a desculpa de ``limitar gastos'' e trazer
``equilíbrio às contas'', será aplicada nos orçamentos fiscais e da
seguridade social, o que traz imensos retrocessos ao povo mais pobre e
dependente das políticas públicas no país.

É preciso que reflitamos com maior profundidade sobre o que de fato representa
esse ataque aos direitos. Estamos sofrendo intensas investidas do
capital internacional por todos os lados. São escutas telefônicas,
compras de terras por grupos estrangeiros, ataques à nossa soberania,
ameaças às nossas riquezas e bens naturais, pressões no parlamento, um
sem fim de sinais que nos indicam mais uma face da disputa entre o
capital e o trabalho em que o capital internacional e nacional buscam
baratear a mão de obra, destruir a organização sindical e,
principalmente, ressignificar o papel do Estado, reduzindo direitos
constitucionais por nós conquistados.

\section{O pacote de desmontes: A regulamentação predatória da
terceirização}

A prova final de que o objetivo do conluio era o desmonte de políticas
públicas, direitos trabalhistas, sociais e civis conquistados no Brasil
no mínimo nos últimos 100 anos -- a partir da primeira greve geral em
1917 até a Constituição Federal de 1988 -- foi apresentada.

Medidas como a Emenda Constitucional 95 fixando teto e, na prática,
congelando os investimentos públicos, ancoraram"-se na alegação da
urgência por programas de ajuste fiscal, ``necessários para a geração de
emprego e desenvolvimento econômico'' e, portanto, na pseudomodernização
das relações de trabalho.

A partir de então foi desferido o segundo ataque pós"-golpe de 2016 aos
direitos da classe trabalhadora: a aprovação da regulamentação
predatória e legitimadora da terceirização desenfreada, inclusive na
atividade fim.

\section{Uma história classista e coletiva}

Os impactos da precarização advinda da terceirização já vinham sendo
sentidos no cotidiano laboral e enfrentados pelas entidades sindicais
havia um bom tempo -- em especial desde os anos 80 e 90 -- quando sob o
argumento da necessidade de especialização, qualificação, focalização,
modernização, inovações tecnológicas e competitividade, muitas empresas
passaram a se valer do artifício da terceirização somada ao trabalho
temporário para o desenvolvimento de atividades que em tese seriam
sazonais.

As denúncias decorrentes da terceirização são inúmeras e vastas, sendo
que vários e especializados estudos se dedicaram a estudá"-las: vão desde
assédio moral, extrema vulnerabilidade laboral, adoecimento, mortes,
péssimas e inferiores condições de trabalho e salários, rotatividade,
``falência'' repentina de empresas intermediárias de mão de obra para
deixar de pagar direitos, trabalhadoras e trabalhadores submetidos a
jornadas extenuantes e a anos seguidos sem férias, multiplicidade de
exercício de funções diferentes daquelas para as quais houve a
contratação e muitas outras atrocidades. Enfim, as exceções para as
quais a terceirização poderia ser aplicada viram regra geral para
situações em que essa forma de contratação é a maquiagem da quebra de
direitos e da exploração ilegal do trabalho.

O fato é que a terceirização avança fortemente nas relações de trabalho,
com o objetivo (nem sempre dito) de cortar os custos do trabalho e
ampliar a margem de lucro do capital. A terceirização inclusive adentra
no terreno da atividade"-fim, porém, até então, sem conseguir a desejada
validação jurídica.

O foco empresarial é a derrubada da Súmula 331 do \versal{TST} -- Tribunal
Superior do Trabalho, que proíbe a terceirização na atividade"-fim, assim
como o desejo da eliminação de reclamações trabalhistas sobre a
usurpação de direitos nos processos de terceirização.

Na Central Única dos Trabalhadores, instalamos em 2004 o ``\versal{GT}
sobre Terceirização'', envolvendo vários ramos de atividade com
trabalhadores do campo e da cidade, por meio de suas federações e
confederações, com o objetivo de pensar uma estratégia mais bem
estruturada, para além das denúncias e medidas judiciais, e com base em
três frentes de ação: \emph{organização sindical, negociação coletiva e
legislação}.

Vários projetos de lei com interesses patronais subjacentes tramitavam no
Congresso Nacional. Decidimos, portanto, fazer um com a visão e os
interesses classistas. Em 2007, após a conclusão dos debates do ``\versal{GT}
Sobre Terceirização'', o deputado federal Vicente Paulo da Silva, o
Vicentinho, em diálogo com sindicalistas da Central, apresentou uma
propositura na Câmara dos Deputados: o projeto de lei 1621 de 2007, com
base na proposta elaborada em grupo.

O \versal{PL} 1621 enfrenta explicitamente os nós críticos e deflagradores dos
processos de precarização das relações de trabalho embutidos na
terceirização. Obviamente, é atacado pelo parlamento composto em sua
maioria por representantes patronais e no meio sindical não alcança
apoio integral, pois parte do sindicalismo defendia uma legislação mais
flexível e permissiva para a terceirização, talvez levados pelo engodo
de que haveria perda de postos de trabalho caso uma boa e forte
legislação enfrentasse e coibisse efetivamente este processo.

\section{A maioria golpista do parlamento ataca novamente}

Nos últimos anos, o \versal{PL} 4330/2004, do deputado Sandro Mabel, tramita e
avança ameaçadoramente no Congresso Nacional. O movimento sindical, a
Anamatra (Associação dos Magistrados do Trabalho) e o Ministério Público
do Trabalho se mobilizam pela sua não aprovação e tentam construir
alterações e emendas para enfrentar seu conteúdo que objetiva eliminar
as travas que garantem os direitos trabalhistas. Ele é aprovado na
Câmara e segue para o Senado como \versal{PLC}30/2015, cujo relator é o senador
Paulo Paim. O senador, em diálogo com diversos segmentos, realiza
audiências públicas objetivando sua melhoria e alteração.

Porém, em 23 de março de 2017, é aprovado no Congresso Nacional algo
inusitado: é desengavetado o \versal{PL} 4302/98 do Executivo, enviado por
Fernando Henrique Cardoso quando presidente da República, e que vai
flexibilizar e regulamentar dispositivos da lei 6019/1974 e regulamentar
o trabalho temporário, a intermediação de mão de obra por prestadoras de
serviços nos processos de terceirização. Este projeto havia tramitado em
regime de urgência sendo que a mensagem 389/2003 do Presidente Luís
Inácio Lula da Silva havia pedido o seu arquivamento, o que nunca foi
apreciado, mas retirou"-se o regime de urgência. Porém, com o golpe, os
setores conservadores ganham força, ele é retomado e aprovado. Sua
redação final fica bastante prejudicial à defesa dos direitos
trabalhistas e vem chancelar amplamente a terceirização, no campo e na
cidade, no setor privado e público.

E então nos deparamos com a terceirização, antes abordada com base em
entendimento jurisprudencial (Súmula 331 do \versal{TST}) -- segundo o qual apenas
era lícita quando abrangendo atividade"-meio --, passará a estar regulada
por lei, com possibilidade de terceirização de qualquer atividade.

Estabelece critérios formais para a validade dos contratos e para o
funcionamento das empresas de prestação de serviços terceirizados ou
temporários, além da possibilidade de contratação de trabalho temporário por
empresas rurais, ampliando o prazo legal para contratação de um mesmo
trabalhador temporário -- que passa de 90 dias (corridos) prorrogáveis
por mais 90, para 180 dias (consecutivos ou não) prorrogáveis por mais
90.

\section{O golpe no golpe: A Reforma Trabalhista e Sindical}

Mal começamos a administrar os dilemas causados pela terceirização,
a precarização nas relações de trabalho atinge seu grau máximo. Recebemos
a fatura do ``pato amarelo'', e não saiu nada barata.

A Reforma Trabalhista \emph{e Sindical} aprovada pelo Congresso que
deveria defender os interesses do povo brasileiro, sancionada
pelo presidente ilegítimo Michel Temer, elimina direitos históricos,
além dos conquistados pelas trabalhadoras e trabalhadores na
Constituição de 1988, retrocedendo, muitos deles, para antes da grande
Greve de 1917.

São mudanças profundas na legislação trabalhista que, alterada em mais
de 100 pontos, ficou destroçada, pondo abaixo anos de lutas e conquistas
históricas dos trabalhadores ante os conflitos inerentes às relações
entre o capital e o trabalho, passando a atender sobremaneira aos
interesses dos patrões. Estas mudanças impactam profundamente os
direitos e a vida da trabalhadora e do trabalhador brasileiros.

Este golpe enfiado goela abaixo dos trabalhadores provoca uma completa
desproteção dos seus direitos conquistados, enquanto promove um profundo
desmonte do papel do Estado do ponto de vista social, eliminando sua
proteção e interferindo, inclusive, na organização sindical.

\section{Uma síntese das mais importantes alterações na legislação}

Altera dois princípios elementares do Direito do Trabalho, que deveria
assegurar proteção à parte mais frágil da relação capital e trabalho:
\emph{princípio da hipossuficiência} bem como o \emph{princípio da
prevalência da norma mais favorável,} ao trabalhador, enquanto institui
a \emph{``intervenção mínima na autonomia da vontade coletiva''} que
parte da falsa ideia de que as partes (empresa e indivíduo) são
equivalentes em poder e meios para realizar acordos, o que já seria uma
completa inverdade no pleno emprego, imagine em uma realidade de mais de
14 milhões de desempregados no país. A mudança prevê, portanto, uma
chancela que passará a tornar juridicamente legal tudo o que vai contra
os interesses dos trabalhadores.

A Reforma Trabalhista afeta as condições de trabalho na sua essência:

Na \emph{contratação} com o Regime de Tempo Parcial, trabalho
temporário, intermitente, estabelecendo convenientemente para o patrão o
teletrabalho, dentre outros.

Nas \emph{jornadas de trabalho}, alterando regras em prejuízo dos
trabalhadores como na hora extra, jornada 12x36, intervalos
intrajornada, nos bancos de horas, desconsiderando a jornada \emph{in
itinere} (do itinerário do trabalhador de casa para o trabalho e
vice"-versa).

Alterando as \emph{condições de trabalho e salário}: reduzindo o
conceito de salário, permitindo o parcelamento de férias em 3 vezes, a
redução da incorporação das gratificações, a presença de lactantes em
locais insalubres, e transferindo ao Estado o que seria responsabilidade
da empresa, como as expensas de afastamento da trabalhadora, colocando em
negociação individual entre a trabalhadora e o patrão as pausas para a
amamentação.

E nos casos de \emph{demissão}, revoga direitos à assistência nos casos
de rescisão, desobriga a presença do sindicato, inclusive em casos de
demissão coletiva, permite a quitação de débitos de demissão voluntária
e legitima o que se convenciona chamar de ``comum acordo''. Essas são
algumas das alterações que vão todas, sem exceção, contra os direitos da
classe trabalhadora brasileira.

A Reforma prevê, ainda, alterações profundas nas negociações coletivas,
flexibilizando direitos (prevalência do negociado sobre o legislado),
promovendo a negociação individual com patrões em uma relação que não
estabelece condição de igualdade, alterando profundamente as normas de
regulação do trabalho e inclusive prevalecendo o acordo coletivo sobre a
convenção coletiva, além de proibir a ultratividade.

Na \emph{organização sindical} elimina a contribuição sindical sem
estabelecer uma taxa negocial a ser defendida pelos trabalhadores em
assembleia, como historicamente é bandeira da \versal{CUT}, e permite a
representação no local de trabalho, porém proibindo o vínculo com o
sindicato, cerceando assim a organização por local de trabalho.

Além disso, reduz o papel e o acesso da \emph{Justiça do Trabalho} nas
negociações coletivas, reduz o acesso gratuito à Justiça do Trabalho,
estabelece multas e custos judiciais aos trabalhadores reclamantes. Como
se pode perceber nesse breve resumo, a Reforma Trabalhista é devastadora
quanto aos direitos e sentencia a necessidade urgente da luta de
classes, que se faz necessária para o enfrentamento a esse retrocesso.

\section{E agora, o que vem por aí?}

O cenário é de um governo ilegítimo que aprova questões estruturantes
para a sociedade e os rumos do país, decidindo legislações que irão
modificar por 20 anos a realidade do povo brasileiro a exemplo da \versal{EC} 95
que com um teto limitador impedirá a melhoria de salários, e
potencializará a queda da qualidade dos serviços públicos prestados,
restringindo direitos, impedindo novas oportunidades e acessos a quem
mais necessita, especialmente à saúde e educação públicas.

Aprova a péssima regulamentação irrestrita e ampla da terceirização, nos
crava uma Reforma Trabalhista que é mais que um pesadelo. E como temos
reagido a tudo isso?

A \versal{CUT}, em março de 2015, foi a primeira entidade do campo sindical e
popular a ir para as ruas, a despeito de ter críticas e divergências com
o Governo, denunciar que a intenção de derrubar o Governo da Presidenta
Dilma Rousseff, sob alegação de ``pedaladas fiscais'', era pautada no
objetivo -- obviamente não explicitado -- do desmonte de políticas
sociais de inclusão e equidade e na destruição de direitos trabalhistas.

Está tendo luta, resistência, e ainda mais haverá. As pesquisas de
intenção de voto, bem como as pesquisas de índices de popularidade do
governo ilegítimo, mostram que está ficando cada vez mais nítido para o
povo brasileiro que a atuação golpista de tirar e impedir uma presidenta
eleita de exercer seu legítimo mandato tinha na verdade a intenção de
ressuscitar o projeto neoliberal, derrotado por 12 anos seguidos nas
urnas.

A rejeição ao golpista Michel Temer é a maior da história republicana. E
os setores progressistas acumulam forças em defesa da democracia e do
respeito à Constituição. O fato é que para eles, ainda falta desferir o
golpe fatal do desmonte das Políticas Públicas para o povo brasileiro: A
Reforma da Previdência.

Tratada falsamente como vilã para os governos antidemocráticos, a
Previdência Pública é falsamente tratada como deficitária. No entanto,
especialistas comprovam por meio de uma Comissão Parlamentar de
Inquérito que a previdência é superavitária e que seu déficit ocorre
devido a desonerações às empresas, manipulações contábeis e incentivos
fiscais.

O governo ilegítimo se contradiz ao alegar rombo nos caixas da
Previdência Pública enquanto desonera setores de seus interesses, reduz
a dívida previdenciária de ruralistas e estimula setores ligados à
Previdência Privada.

As perdas com o pacote de generosidade do governo ilegítimo para
determinados setores chega a 10 bilhões de reais, segundo dados da \versal{CPI}.
A Reforma da Previdência é mais um golpe que se aproxima. O povo
brasileiro, já está escaldado pelas reformas anteriores precisa impedir
com firmeza e determinação.

Maior vitória para o campo na Constituição de 1988, a Previdência Rural
e Especial reduziu as históricas desigualdades no campo, incluindo
cidadãos ao acesso a este direito. A luta por sua manutenção terá que
partir de todas as partes do país, organizações, segmentos e movimentos
da sociedade civil organizada.

O povo brasileiro deve dar uma resposta à altura do descaso, desrespeito
e desvalorização dos parlamentares que votaram contra os direitos da
classe trabalhadora e da população em geral. Dentre elas, a privatização
das estatais, sendo este mais um ataque à soberania brasileira e mais
uma manifestação do entreguismo deste governo.

A \versal{CUT} que desde o começo posicionou"-se incondicionalmente na defesa da classe trabalhadora, recusando"-se a negociar com esse governo, dentre tantas outras iniciativas contra todos esses ataques lançou sua \emph{Campanha Nacional pela Anulação da Reforma Trabalhista}, que coletou assinaturas para um Projeto de Iniciativa Popular, e entregou na Câmara Federal no dia 11 novembro de 2017, dia que entrou em vigor da Lei 13.467 de 13 de julho de 2017. A Reforma Trabalhista tem gerado desemprego e a miséria aos que estão na base da pirâmide social, a classe trabalhadora que sustenta com sua força de trabalho essa nação. O povo brasileiro deve reagir e a resposta mais fácil de ser compreendida mais uma vez virá das ruas e das urnas. Quem viver verá. 

%SUBSTITUIDO POR CLEONILDO
%A \versal{CUT} que desde o começo posicionou"-se incondicionalmente na defesa da
%classe trabalhadora recusando"-se a negociar com esse governo, dentre
%tantas outras iniciativas contra todos esses ataques lança sua
%\emph{Campanha Nacional pela Anulação da Reforma Trabalhista} com uma
%coleta de assinaturas para um Projeto de Iniciativa Popular e pretende
%entregá"-lo em novembro de 2017, quando entra em vigor da Lei 13.467 de
%13 de julho de 2017, que levará ao desemprego e a miséria os que estão
%na base da pirâmide social, a classe trabalhadora que sustenta com sua
%força de trabalho essa nação. O povo brasileiro deve reagir e a resposta
%mais fácil de ser compreendida mais uma vez virá das ruas e das urnas.
%Quem viver verá.

\section{Referências Bibliográficas}

\begin{Parskip}
\versal{MOTTA DAU}, Denise; \versal{RODRIGUEZ}, Iram Jácome; \versal{CONCEIÇÃO},
Jefferson José da (Orgs.). \emph{Terceirização no Brasil: do discurso da
inovação à precarização do trabalho (atualização do debate e perspectivas)}.
São Paulo: Annablume; \versal{CUT}, 2009.

\versal{TEIXEIRA}, Marilane Oliveira; \versal{RODRIGUES}, Hélio;
\versal{COELHO}, Elaine D'Ávila (Orgs.).
\emph{Precarização e Terceirização: faces da mesma realidade?} São
Paulo: Sindicato dos Químicos, 2016.

\emph{Reforma trabalhista: desafios para o sistema brasileiro de relações de
trabalho}. Subsídio Curso Dieese sobre Reforma trabalhista. Curso
realizado em 27 de setembro de 2017.

\versal{CUT}. \emph{Resoluções do 2º Congresso Nacional da \versal{CUT}}.
Disponível em \textless{}\emph{https://bit.ly/2N6ZO00}\textgreater{}
(Acesso em 2/10/2017).

\versal{DIEESE}. \emph{Privatização do setor de saneamento no Brasil}. Nota Técnica:
Nº 183 -- junho de 2017. Disponível em 
\textless{}\emph{https://bit.ly/2N7nLEA}\textgreater{}
(Acesso em 5/10/2017).

 \versal{STAL}; \versal{ÁGUA} \versal{DE} \versal{TODOS}; \versal{PSIRU};
 \versal{MULTINACIONAIS} \versal{OBSERVATORY}. \emph{Veio para ficar: A
  remunicipalização da água como uma tendência mundial}, 2014. Disponível em
\textless{}\emph{https://bit.ly/2kTPGfq}\textgreater{}
(Acesso em 05/10/2017).

 \versal{CUT}. \emph{Campanha Nacional pela Reforma Trabalhista}. Disponível em
\textless{}\emph{https://bit.ly/2f8VXmR}\textgreater{}
(Acesso em 4/10/2017).
\end{Parskip}