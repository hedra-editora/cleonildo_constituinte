\chapter*{A atuação política do juiz:\\
\emph{Uma análise à luz da Operação Lava"-Jato}}

\addcontentsline{toc}{chapter}{A atuação política do juiz,
\scriptsize{por Mariana de Carvalho Milet}}
\hedramarkboth{A atuação política do juiz}{}

\begin{flushright}
\emph{Mariana de Carvalho Milet}\footnote{Juíza do Trabalho do Regional do Trabalho da 6ª Região (\versal{TRT"-PE}).}
\end{flushright}


%\section{Introdução}


A Constituição Federal de 1988 consagrou, no art.2º, a separação dos
poderes do Estado, os quais devem atuar independente e harmonicamente
entre si. Ao longo do texto constitucional criam"-se mecanismos de
controle recíproco, sempre como garantia de perpetuidade do Estado
Democrático de Direito.

O poder é soberano, entretanto, na esteira dos ensinamentos de
Aristóteles na obra \emph{A Política}, para o Estado bem exercer a sua
soberania deve delegar suas funções. No Brasil, as funções do Estado são
distribuídas entre os poderes Legislativo, Executivo e Judiciário.

O protagonismo do Judiciário tem aumentado ultimamente na esfera
criminal. A Operação Lavajato\footnote{Ao longo do texto utilizaremos a
  expressão ``Lavajato'', assim gravada por ser o modo que o juiz Sérgio
  Moro a escreve em seus textos.} que tem desvendado e trazido à tona um
enorme esquema de corrupção entre os políticos que exercem ou exerceram
suas funções no Legislativo e no Executivo e os principais dirigentes de
grandes empreiteiras que atuam no país, serve de mola propulsora para
que os holofotes se voltem para o Judiciário.

Concomitantemente, observa"-se que as leis não têm se atualizado na mesma
velocidade em que surgem as demandas sociais. Cabe ao Judiciário, no
exercício da Jurisdição transformadora, desempenhar um papel
reformador,\textbf{~}quando as estruturas sociais concretas já não
atenderem razoavelmente aos princípios constitucionais maiores que
cristalizam os grandes valores da vida social organizada.

Na antiguidade, os juízes governavam as cidades, confundindo"-se a figura
do político com a do juiz. Entretanto, desde o século \versal{XVI}, com o
aperfeiçoamento pela sociedade das ideias de Montesquieu acerca da
tripartição dos poderes, o papel do juiz vem sendo modificado.

Cabe ao juiz, enquanto representante do Estado no que concerne à função
julgadora, garantir a normatividade e a efetividade da regra jurídica. A
atuação do juiz deve tornar o Direito tão democrático que será capaz de
alcançar a realidade social.~ Dessa maneira, o Estado, na pessoa do
juiz, deverá garantir a promoção da justiça, porquanto interpretará o direito considerando a pluralidade da
sociedade contemporânea, na solução dos conflitos sociais.

Nem tudo o que é posto para ser decidido pelos juízes, seja na rotina da
magistratura e especificamente em relação àqueles que atuam na Operação
Lavajato, está literalmente colocado na lei. Estes juízes têm sido cada
dia mais provocados a fazer interpretações legais que legitimem os seus
atos de modo a buscarem a efetividade necessária para o combate à
corrupção.

O presente trabalho se presta a apreciar como tem sido a ação do
Judiciário ao longo da operação Lavajato, bem como se os magistrados que
conduzem os processos estão se valendo do modelo ativista para decidir,
analisando"-se os reflexos sociais, políticos e econômicos da atuação.

Nesse diapasão, o objetivo precípuo desse texto é demonstrar até que
ponto a atuação ativista de um juiz pode impactar em toda a estabilidade
social, econômica e política de um país. Não temos qualquer pretensão de
fazer uma abordagem ufanista, como se a Operação Lavajato viesse
solucionar todos os problemas morais do país, tampouco pessimista, de
modo que essa investigação não produza qualquer efeito no que concerne à
garantia da apregoada moralidade perseguida pelo Estado Democrático de
Direito. Tenta"-se fazer uma explanação equilibrada e crítica do papel da
operação Lavajato e de como a atuação do Judiciário pode impactar no
contexto da sociedade contemporânea brasileira.

\section{Ativismo judicial}

\subsection{Soberania Popular e Constituição}

Em um Estado Democrático é necessário assegurar ao povo a coexistência
da soberania popular com as facções políticas inerentes ao exercício
legítimo da Democracia. Entretanto, os precursores da ideia, Alexander
Hamilton, James Madison e John Jay, nos artigos organizados no livro \emph{O
Federalista}, entendiam que para garantir a soberania popular era
preciso estabelecer uma nova ordem jurídica.

Constatou"-se que é inerente ao exercício da soberania a existência de
grupos políticos que podem agir de encontro ao interesse coletivo. Era
preciso controlar a ação desses grupos, sem que eles deixassem de
existir, pois também deveria ser garantida a liberdade de expressão e
associação. Para isso a vontade popular estaria posta e assegurada na
Constituição.

Claramente influenciado pelas ideias do contrato social de Jonh Locke,
onde as relações entre os indivíduos e o governo seriam asseguradas
pelo exercício do poder através de um soberano representante dos ideais
coletivos, o constitucionalismo norte"-americano tinha por escopo
assegurar o livre exercício dos interesses da maioria política,
existente no Legislativo, e as minorias, os proprietários e donos do
dinheiro.

A crítica que se faz a esse modelo representativo é que seria elitista e
conservador, pois afasta os cidadãos comuns das decisões coletivas. O
Estado correria o risco de abrir espaço para que os parlamentares
pudessem agir de modo a se tornarem ditadores, através da utilização de
leis para a imposição de interesses adversos aos da população. Esse seria
o principal ônus do constitucionalismo representativo, sobre o qual a
sociedade deveria estar sempre atenta.

Nesse contexto, o Judiciário surgiria como órgão composto por indivíduos
que teriam as virtudes necessárias para garantir que as decisões em nome
da coletividade prevaleçam. No modelo político constitucional norte"-americano, o sistema de freios e contrapesos previa que o Judiciário tinha a função principal de limitar os abusos dos legisladores e, quando
necessário, a estrutura prevista na Constituição.

Veja"-se que a função intervencionista do Judiciário surge de uma relação
de confiança entre sociedade e seus membros, posto que os juízes
ascendem à sua função com base no conhecimento jurídico que possuem,
demonstrado através da forma de ingresso no Poder e, ao longo da
carreira, pelas garantias de atuação que lhes são resguardadas.

No Brasil, desde a Constituição de 1981, verifica"-se a influência,
guardadas as devidas proporções, do constitucionalismo estadunidense,
posto que a democracia brasileira está construída sob os mesmos pilares
da norte"-americana, quais sejam, separação dos poderes, federalismo e
controle de constitucionalidade. Sempre no sentido de equilibrar os
ideais de exercício da soberania popular e normatividade, ao Judiciário
é conferido o poder de anular os atos dos demais poderes.

A atuação dos juízes, no arranjo institucional brasileiro, como mediador
de conflitos políticos, sempre foi reconhecida, posto que, tal como no
modelo estadunidense, há uma tendência à instrumentalização dos
conflitos sociais e políticos.

A Constituição Brasileira de 1988 é a que melhor retrata o papel do
poder Judiciário na democracia brasileira, dado o seu viés
redemocratizante. Segundo Flavia Danielle Santiago Lima:

\begin{quote}
{[}\ldots{}{]} neste modelo, ao assegurar autonomia ao Judiciário e expandir
suas competências -- exemplificada na adoção de um complexo sistema de
controle de constitucionalidade --, fortalecer outras instituições do
Direito (Ministério Público, Defensoria Pública, Advocacia Pública) e
canalizar o acesso das demandas políticas através destes meios, tem"-se
as condições para a expansão deste poder\footnote{\versal{LIMA}, F.
\emph{Ativismo e autocontenção no Supremo Tribunal Federal: uma
  proposta de delimitação do debate}. Tese de Doutorado em
  Direito. Universidade Federal de Pernambuco, 2013, p. 190. Disponível
  em: \textless{}\emph{https://bit.ly/2DNgtmb}\textgreater{}. Acesso em outubro de 2016.}.
\end{quote}

As várias Emendas Constitucionais aumentaram os poderes do órgão de
cúpula do Judiciário brasileiro, o Supremo Tribunal Federal (\versal{STF}).
Atualmente, diariamente nos deparamos com decisões prolatadas pelos
Ministros, as quais, via de regra, demonstram a interferência direta do
Judiciário nos demais poderes. O \versal{STF}, ao se posicionar juridicamente na
interpretação das Ações Diretas de Inconstitucionalidade e nas ações
propostas por partidos políticos, tem tido a função de estabelecer as
``regras do jogo'', ao se posicionar favoravelmente ou não a determinado
ato.

Podemos exemplificar a atuação do Ministro Luiz Roberto Barroso ao
indeferir liminar em ação proposta pelo \versal{PT} e pelo \versal{PC}do\versal{B} para declarar
a inconstitucionalidade da proposta de Emenda Complementar 241 (atual
\versal{PEC} 55). Partidos oponentes ao que está no Poder Executivo provocam o
Judiciário a se pronunciar sobre a criticável proposta de modificação
restritiva dos direitos individuais e coletivos e cabe ao Judiciário,
através do \versal{STF}, dizer se pode haver ou não a alteração constitucional,
influenciando tanto na ação do Executivo e Legislativo, quanto na vida
da população.

Percebe"-se, pois, que o fundamento de validade da Constituição
Brasileira é assegurar o pleno exercício da soberania popular,
promovendo o equilíbrio entre os interesses do povo de uma maneira geral
e a atuação política de uma minoria, detentora do capital produtivo. E,
sobre essa premissa surge a noção de ativismo judicial no contexto
jurídico político.

\subsection{Judicialização da política}

Ao se pensar em Direito e Política se tem logo a ideia de que os mesmos
não se misturam, havendo um muro separando os dois conceitos.
Entretanto, essa não é a realidade que vem sendo experimentada pela
população brasileira, mormente com a Constituição Federal de 1988. Ao
aplicar as normas jurídicas, direito e política influenciam um ao outro.
O mundo da subjetividade e discricionariedade (política) interfere no
plano da razão e da objetividade (direito) em diferentes nuances.

Considerando essa ideia, a expressão ``judicialização da política''
significa que o Poder Judiciário está atuando em questões relevantes do
ponto de vista social, moral ou político, ao decidir sobre elas em
caráter final.

Esse fenômeno é mundial e tem atingido todos os países, independente da
forma de governo adotada ou sistema jurídico. No Brasil, o recente
impeachment da presidente Dilma, em decorrência do desmonte da corrupção
promovido pela Operação Lavajato. Na Colômbia, também pode se citar a
atuação jurisdicional no combate à corrupção e modificação de práticas
políticas, bem como a proteção a minorias. Na França, foi anulado o
imposto do carbono, que incidiria sobre o consumo e a emissão de gases
poluentes, com forte reação do governo.

Barroso\footnote{\versal{BARROSO}, Luís Roberto. Constituição,
  democracia e supremacia judicial: Direito e Política no Brasil
  Contemporâneo. In: \emph{Revista da Faculdade de Direito da \versal{UERJ}}, nº 21, Rio de
  Janeiro, 2012, p. 7. Disponível em: \textless{}\emph{https://bit.ly/2BfwL5A}\textgreater{}. 
Acesso em setembro de 2016.} enumera como causas da plasticidade
dos conceitos de política e direito, e surgimento da judicialização da
política, a crise de representatividade e da funcionalidade do
parlamento, a ascensão institucional dos juízes e tribunais, bem como o
fato da preferência dos atores políticos, para que o Judiciário seja a
instância decisória de certas questões polêmicas, em relação às quais
exista desacordo moral razoável na sociedade, como por exemplo, a
demarcação das terras indígenas.

No Brasil, o sistema de controle de constitucionalidade e o modelo de
constituição analítico são os principais fatores que contribuem para o
fenômeno em abrangência. Outrossim, uma vez provocado, o Judiciário não
pode se furtar de promover a prestação jurisdicional e dar uma resposta
ao postulante. Então, a judicialização surge, nesse contexto, como uma
imposição aos juízes e tribunais, não sendo uma opção.

\subsection{Ativismo judicial propriamente dito}

Qualquer definição de ativismo judicial pode ser considerada excludente,
restritiva ou ampla, a depender do viés adotado, pois o conceito está
pautado nos ideais progressistas ou conservadores de quem o analisa.
Entretanto, é necessário expor qual a linha que iremos adotar para a
busca dos aspectos empíricos que se pretendem analisar neste trabalho.

A primeira vez em que se tem notícia da utilização da expressão ativismo
judicial foi no artigo jornalístico intitulado ``\emph{The Supreme
Court: 1947}'', publicado pelo historiador democrata Arthur Schlesinger
Jr. na revista \emph{Fortune Magazine.}

No artigo, o jornalista relata a divisão dos juízes da Suprema Corte dos
Estados Unidos em dois blocos: aqueles que defendiam a interpretação
restritiva dos casos postos em discussão, não cabendo ao Judiciário se
imiscuir em questões políticas, e aqueles favoráveis a decisões que
deveriam trazer em seu bojo preocupação com os efeitos políticos e
sociais das mesmas.

O estudo constata que os magistrados, a partir de sua formação
acadêmica e de visões particulares do direito, podem promover uma
revisão judicial de caráter expansivo.

Entretanto, o artigo também é o marco de uma discussão que persiste até
os dias atuais. Há aqueles partidários de que a atuação do magistrado
pautada não apenas na lei, mas em aspectos ideológicos, seria um prejuízo
para a democracia. De outra ponta, os simpatizantes da prática jurídica
afirmam que tal prática assegura o exercício das liberdades civis. Em
linhas gerais, o ativismo judicial representa a noção de uma
participação mais ampla e intensa do Judiciário na concretização dos
valores e princípios definidos na Constituição, com maior interferência
no espaço de atuação dos outros dois Poderes. Seria uma intervenção para
ocupar espaços vazios e nem sempre substituí"-los ou exercer a atividade
própria de outro poder.

No Brasil, o ativismo é recente e a doutrina tem tentado
delimitar o espaço de atuação do magistrado para adotar tal postura. O
respeito à Constituição de 1988 e ao arranjo democrático são princípios
basilares e limitativos do ativismo judicial. Segundo Gustavo Ferreira
Santos\footnote{\versal{SANTOS}, Gustavo Ferreira. \emph{Neoconstitucionalismo,
  poder judiciário e direitos fundamentais}. Curitiba: Juruá,
  2001, p. 92.}, o juiz, ao exercer o controle de inconstitucionalidade de
uma norma, deve ter consciência do aspecto político da decisão, sem que
crie o hábito de adentrar no espaço criativo que é próprio do
Legislador.

Também merece comentários a atividade oposta ao ativismo, qual seja, a
autocontenção judicial. Ao agir desta maneira, o Poder Judiciário tenta
interferir o mínimo possível na ação dos outros poderes.

Enquanto no ativismo judicial procura"-se utilizar uma interpretação
ampla da Constituição, buscando"-se todas as suas potencialidades,
inclusive criando regras a partir de enunciados jurídicos vagos. Na
autocontenção busca"-se conferir mais ação aos poderes Executivo e
Legislativo para que atuem dentro de suas competências.

Um aspecto interessante do ativismo judicial é o fato de a mesma corte
ou juiz poder prolatar decisões que respaldem o conceito e, em outras,
partir para o exercício da autocontenção, evitando se imiscuir na
atuação de outros poderes e restringindo suas decisões aos aspectos
jurídicos.

Nesse sentido elucida Barroso:

\begin{quote}
A principal diferença metodológica entre as duas posições está em que,
em princípio, o ativismo judicial legitimamente exercido procura extrair
o máximo das potencialidades do texto constitucional, inclusive e
especialmente construindo regras específicas de conduta a partir de
enunciados vagos (princípios, conceitos jurídicos indeterminados). Por
sua vez, a autocontenção se caracteriza justamente por abrir mais espaço
à atuação dos Poderes políticos, tendo por nota fundamental a forte
deferência em relação às ações e omissões desses últimos\footnote{Idem, p. 11.}.
\end{quote}

Merece destaque a crítica feita à adoção do ativismo judicial no sentido
de que surgiria a possibilidade de quebra da harmonia constitucional no
que diz respeito à separação entre os poderes, bem como a abertura para
o magistrado adotar convicções pessoais e interesses escusos para
direcionar as decisões proferidas, fazendo prevalecer convicções
políticas pessoais em detrimento da imparcialidade.

Assim, como toda a atividade jurídica, caso esse respaldo de atuação
voltado ao ativismo judicial conferido pelo \versal{STF} seja bem utilizado
pelos juízes, a sociedade poderá sim ver seus ideais democráticos e
soberanos realizados. Entretanto, não se pode olvidar dos riscos da
adoção do conceito, posto que os magistrados são seres humanos dotados
de convicções políticas e sociais, podendo valer"-se da técnica para
expressá"-las, sem que fique expressamente caracterizada a quebra da
imparcialidade.

\section{Operação Lava"-Jato}

\subsection{Gênese}

A Polícia Federal de Londrina, no estado do Paraná, recebeu uma denúncia
de Hermes Freitas Magnus, sócio da empresa Dunel Indústria e Comércio
Ltda. Segundo Magnus, em 2008, a empresa começou a crescer rapidamente
através do aumento de serviço. Era preciso expandir o negócio para poder
atender a demanda e, para tal, a empresa precisava de sócios com
condições de investir capital.

Foi então que o denunciante conheceu José Janene, deputado federal do
\versal{PP}, que aceitou investir um milhão de reais na empresa. Para tanto,
foi feita uma transferência de recursos da empresa \versal{CSA} Project Finance,
a qual passou a deter 50\% do capital da Dunel e pertencia a José Janene
e ao doleiro Alberto Youssef. Em alguns meses Magnus percebeu que seu
escritório estava sendo usado para entrega de dinheiro em espécie a
políticos e a nova empresa associada participava de contratos
superfaturados de obras públicas.

A Dunel estava envolvida em um esquema de lavagem de dinheiro e o sócio
fundador resolveu procurar a Polícia Federal e formular a denúncia.

Ao dar início às investigações, a Polícia Federal e o Ministério Público
Federal acharam estar diante de um dos tantos esquemas de lavagem de
dinheiro já investigados. Entretanto, à medida que expandiam as
investigações para atingir todas as empresas com participação de Alberto
Youssef, foram percebendo que estavam em um dos maiores esquemas de
corrupção já descobertos no Brasil.

É comum no mundo as operações policiais receberem nomes fictícios com o
intuito de resguardar os fatos e o sigilo da operação. Inicialmente, a
delegada Erika Mialik Marena denominou Lavajato a operação que
investigava os crimes praticados pela quadrilha de Charter, mas, posteriormente, o
nome passou a compor todas as investigações relacionadas.

Explica a delegada que propôs tal nome porque a quadrilha investigada
utilizava a rede de postos de gasolina de Youssef e oferecia todo tipo
de serviço de lavagem, de roupa a carros. Tudo que permitisse aumentar a
oferta de serviços para atrair demanda e assim criar condições de
realização do principal tipo de atividade, qual seja, lavagem de
dinheiro.

\subsection{O esquema}

A Petrobrás é uma empresa pública e como tal, para a contratação de
obras, submete"-se às regras previstas na lei 8.666 de 1983, ou seja,
apenas através de procedimento licitatório poderá contratar obras e
serviços.

Os diretores das maiores e melhores empreiteiras atuantes no Brasil,
entre elas Odebrecht, Engevix, \versal{OAS}, Mendes Junior, Queiroz Galvão, \versal{UTC},
Engesa, Iesa e Camargo Corrêa, reuniam"-se para apresentarem valores
superfaturados das obras, os quais teriam que serem aceitos pela
Petrobrás. Isso porque, embora não fossem justos, eram os preços que
tinham aparecido entre os licitantes e havia necessidade da contratação
da obra.

Para o esquema receber a aparência de lícito e não despertar o interesse
da sociedade, as empresas se alternavam e faziam um rodízio de modo que
todas ganhassem alguma licitação. Em consequência assumiriam a
responsabilidade pela execução de determinada obra.

Uma das obras mais superfaturadas foi a Refinaria Abreu e Lima em
Pernambuco, a qual contava com a participação de todas essas empresas,
que firmavam consórcio com a Estatal. O esquema era tão bem pensado que
na maioria das obras quase todas as empreiteiras participavam, de modo
que todas tivessem benefício.

Além da obra acima citada, o cartel também atuou na construção da
Refinaria Presidente Getúlio Vargas, no Paraná e no Complexo
Petroquímico do Rio de Janeiro.

E os agentes públicos que atuavam na Petrobrás e participavam do
esquema eram remunerados com 1\% a 5\% do valor dos contratos e dos
aditivos para que a prática do cartel permanecesse escondida.

Só esse esquema até o ponto acima descrito seria suficiente para deixar
a sociedade perplexa, mas o pior ainda estava por vir.

A nomeação dos diretores da Petrobrás ocorre por indicação do governo.
E, nesse ponto, entra a participação dos políticos. Eram nomeadas para
os cargos de direção da estatal pessoas coniventes com a participação
no esquema de corrupção e repasse de dinheiro obtido ilicitamente
(propina) para os partidos políticos.

Desse fato decorre a participação de Nestor Cerveró, indicado pelo \versal{PMDB}
para ocupar a diretoria internacional entre 2003 e 2008. Assim como a
de Renato Duque, ocupante da diretoria de serviços e indicado pelo \versal{PT},
no período de 2003 a 2012, e de Paulo Roberto Costa, diretor de
abastecimento entre 2004 e 2012, com interferência do \versal{PP}.

\subsection{As características da Lava"-Jato}

O juiz Sérgio Moro é o titular da 13ª Vara Criminal Federal de Curitiba
e é a autoridade que, juntamente com a Polícia Federal e a equipe de
procuradores da República, tem conduzido os principais processos da
operação. Sérgio Moro já tinha atuado nas investigações envolvendo
organizações criminosas como as do Banestado e do Mensalão, por exemplo.

O \emph{modus operandi} da atuação dos Procuradores da República,
juntamente com a Polícia Federal e a ação do juiz Sérgio Moro, tem tido
destaque principalmente pela rapidez em que os fatos vão acontecendo no
sentido de desmontar a corrupção e condenar os acusados.

O fio condutor da Lavajato tem sido as delações premiadas. O criminoso,
em troca de ver sua pena diminuída, revela traços do esquema de
corrupção denunciando detalhes dos fatos criminosos e, via de regra,
envolvendo outras pessoas.

Os advogados de defesa dos réus nos diversos processos oriundos da
Lavajato argumentam que nas delações os réus têm apenas mencionado os
fatos que lhes convêm. Entretanto, o Ministério Público Federal tem
ofertado e o juiz Sérgio Moro acolhido os depoimentos decorrentes de
colaboração premiada, tendo se observado, inclusive, as comutações de
diversas penas daqueles que aceitam fazer uso do instituto.

Interessante destacar a opinião de Sérgio Moro acerca da delação
premiada:

\begin{quote}
Sobre a delação premiada não se está traindo a pátria ou alguma espécie
de ``resistência francesa''. Um criminoso que confessa um crime e revela a
participação de outros, embora movido por interesses próprios, colabora
com a Justiça e com a aplicação de leis de um país. Se as leis forem
justas e democráticas, não há como condenar moralmente a delação; é
condenável, nesse caso, o silêncio\footnote{\versal{MORO}, Sérgio. Considerações
  sobre a Operação Mani pulite. \emph{Revista \versal{CEJ} (Brasília)}, v. 26,
  p. 56-62, 2004, P. 60.}.
\end{quote}

Também Celso de Mello, ministro do \versal{STF}, já se pronunciou sobre o
instituto e sua importância para a Lavajato, ao declarar a validade da
delação de Youssef:

\begin{quote}
A delação possibilitou penetrar nesse grupo que se apoderou do aparelho
do Estado, promovendo um assalto imoral, criminoso ao Erário e desviando
criminosamente recursos que tinham outra destinação, a destinação
socialmente necessária e aceitável. Os depoimentos desse agente como
meio de obtenção de provas revelaram"-se eficazes no afastamento desse
véu que encobria esse conluio de delinquentes, que estão agora sofrendo a
ação persecutória do Ministério Público\footnote{\versal{SCARPINO}, Luiz.
  \emph{Sérgio Moro: o homem , o juiz e o Brasil}. Ribeirão Preto, \versal{SP}:
  Novo Conceito, 2016, p. 76. A decisão está em sigilo no sítio
  do \versal{STF}. Logo, não foi possível consultá"-la em sua integralidade.}.
\end{quote}

Acerca da delação premiada, os benefícios do instituto para o andamento
da investigação em apreço têm sido válidos o suficiente para permitir
que a sua utilização suplante e desconsidere os argumentos levantados
pela defesa dos investigados, no sentido de que poderia ser conduzida
pelo colaborador da maneira que melhor o aproveitasse.

\subsection{A evolução da Lava"-Jato}

Atualmente a operação está na 42ª fase e também chegou em um momento
crucial, posto que começou a atingir políticos de renome no Brasil, tais
como o ex"-presidente da Câmara dos Deputados, Eduardo Cunha, o qual após
ter seu mandato cassado, foi preso por ordem do juiz Sérgio Moro.
Outrossim, o presidente em exercício Michel Temer figura como
investigado após a homologação da delação premiada de Joesley Batista.

Não se sabe se agora, quando os ``grandes nomes'' de políticos
brasileiros começarem a aparecer de modo a comprovar atitudes criminosas
dos mesmos, a operação irá sucumbir ou se a corrupção será realmente
detida. Mas, mesmo que encerrasse hoje, a Lavajato já tem o título de
maior operação brasileira realizada para o combate à corrupção.

Tudo começou quando, com o auxílio de grampos telefônicos, tornaram"-se
públicas quatro organizações criminosas lideradas por Carlos Habib
Chater, Nelma Kodama, Raul Srour e Alberto Youssef, em decorrência da
denúncia de Hermes Magnus.

Após Sérgio Moro ter autorizado algumas interceptações telefônicas,
descobriu"-se que Youssef tinha doado um veículo Land Rover Evoque para
Paulo Roberto Costa, então diretor de abastecimento da Petrobrás.

As primeiras medidas ostensivas da Lavajato, realizadas em março de
2014, destinaram"-se a cumprir 81 mandados de busca e apreensão, 18
mandados de prisão preventiva, 10 mandados de prisão temporária e 19
mandados de condução coercitiva, em 17 cidades de 6 estados e no
Distrito Federal.

Um desses mandados era destinado à empresa Costa Global, pertencente a
Paulo Roberto Costa. Para não arrombar as portas da empresa, a Polícia
Federal foi ao apartamento do proprietário pegar as chaves, enquanto
câmeras do local flagraram a filha e o cunhado de Costa entrando na
empresa e saindo com sacos de provas. Tal fato foi considerado
suficiente pelos Procuradores da República para denunciar Paulo Roberto
Costa e seus familiares por crime de obstrução à investigação de
organização criminosa.

Nesta primeira fase foram apreendidos 80 mil documentos, computadores e
celulares, os quais, aliados às conversas decorrentes de interceptações
telefônicas, culminaram em 12 denúncias, envolvendo 55 acusados pela
prática de crimes contra o Sistema Financeiro nacional, organização
criminosa, corrupção e peculato.

Concomitantemente foram propostas 15 medidas cautelares que culminaram
no deferimento do pedido de bloqueio de todos os bens dos acusados.

Todo esse material foi analisado pela equipe de Procuradores da
República e deu"-se início às ações penais, com o deferimento dos pedidos
de bloqueio, em apenas 1 mês.

Já em abril de 2014 foi noticiado outro momento importante da Lavajato,
para aprofundar as investigações sobre os doleiros. No dia 11 foram
cumpridos 23 mandados de busca e apreensão, 2 de prisão temporária, 6 de
condução coercitiva e 15 de busca e apreensão, em cinco cidades.

Em maio de 2014, a equipe da defesa dos investigados apresentou
reclamação perante o \versal{STF}. Na reclamação houve a acusação pelos advogados
de que o juízo da 13ª Vara Criminal de Curitiba estaria usurpando a
competência do \versal{STF} ao investigar políticos que detinham foro
privilegiado. O Ministro Teori Zavascki negou a existência de
inobservância do foro privilegiado, mas houve a paralisação das
investigações por um período.

Nesta época foram descobertos 23 milhões de dólares depositados em
bancos suíços em nome de Paulo Roberto Costa.

Já preso preventivamente, Paulo Roberto Costa decide, em agosto de 2014,
oferecer a primeira delação premiada da Lavajato. Costa procurou o \versal{MPF}
e ofereceu"-se a devolver toda a propina recebida, bem como informar
fatos novos em troca de benefícios. Como a delação envolvia políticos em
exercício e com foro privilegiado, o procurador geral da Republica
Rodrigo Janot autorizou a delação, que foi homologada pelo \versal{STF}.

Em seguida, Alberto Youssef também se ofereceu para falar. Nessa
esteira, outros acordos também foram colhidos e homologados pelo juízo
da 13ª Vara Criminal de Curitiba.

Em novembro de 2014 mais um avanço importante. Foram executados, pela
Polícia Federal em conjunto com a Receita Federal, 85 mandados, sendo 4
de prisão preventiva, 13 de prisão temporária, 49 de busca e apreensão e
9 de condução coercitiva, em diversas cidades do país, especialmente em
grandes e renomadas empresas de construção como Engevix, Mendes Júnior
Trading Engenharia, Grupo \versal{OAS}, Camargo Corrêa, Galvão Engenharia, \versal{UTC}
Engenharia, \versal{IESA} Engenharia, Construtora Queiroz Galvão e Odebrecht
Plantas Industriais e Participações.

Em dezembro do mesmo ano foram oferecidas mais 6 denúncias, incluindo o
ex"-diretor internacional da Petrobrás, Nestor Cerveró. Em janeiro de
2015, ao retornar de Londres com a família, Cerveró foi preso
preventivamente.

De acordo com o juiz Sérgio Moro, a prisão foi decretada para assegurar
a aplicação da lei penal, dada a possibilidade de Cerveró dissipar seu
patrimônio, dificultando futura punição, bem como possuir cidadania
espanhola, a qual teria sido omitida pelo denunciado. Também serviu de
fundamento para a prisão a existência de evidências de que a empresa
Jolmey, proprietária de imóvel em que o investigado residiu por vários
anos, pertence de fato ao ex"-diretor. Esta empresa também foi usada para
que Cerveró usufruísse no Brasil a propina recebida, dando"-lhe aparência
de dinheiro legítimo. Ademais, foi constatada a existência de operações
imobiliárias subvaloradas pela empresa.

Com o objetivo de produzir provas sobre pagamentos de propinas para
agentes públicos relacionados à diretoria de serviço da Petrobras e à \versal{BR}
Distribuidora, subsidiária da empresa, em fevereiro de 2015 foi
deflagrada nova fase da Lavajato, que culminou na denúncia de 47
parlamentares, a qual foi acolhida pelo ministro do \versal{STF} Teori Zavascki
em março de 2015. Deu"-se início a uma linha paralela de investigação de
agentes políticos com foro privilegiado.

Em outubro de 2015, Teori Zavascki determinou o desmembramento de parte
dos processos da Lavajato que apuravam irregularidades em contratos para
a construção da usina nuclear Angra 3. O que não foi, no entanto, uma
decisão festejada pelo \versal{MPF}, que entendia que o desmembramento de
processos para além do Paraná e Brasília poderia comprometer o andamento da
operação.

Paralelamente ao curso das ações existentes em Brasília e em Curitiba, e
da ação penal instaurada na 7ª Vara Federal do Rio de Janeiro, o
Ministério Público Federal no Rio de Janeiro deu início ao
aprofundamento das investigações, pois se constatou que o esquema era
mais amplo que o núcleo que foi objeto da denúncia inicial. Diante da
complexidade das investigações, em junho de 2016 foi criada uma
força"-tarefa para investigar supostos crimes de corrupção, desvio de
verbas e fraudes em licitações e contratos na Eletronuclear, subsidiária
da Eletrobrás.

Dentre os acusados na operação estavam grandes nomes da política
brasileira tais como Renan Calheiros, Eduardo Cunha, Delcídio Amaral e o
ex"-presidente Lula. Segundo o \versal{MPF}, estes nomes estão diretamente
envolvidos nos esquemas políticos ligados ao Partido Progressista, ao
Partido dos Trabalhadores e ao Partido do Movimento Democrático
Brasileiro.

\subsection{A Lava"-Jato em números}

Oficialmente, a Lavajato teve início em 17/3/2014 e até o dia 1/8/2017
já foram instaurados 1.765 procedimentos. Foram realizadas 844 buscas e
apreensões, 210 conduções coercitivas, 97 prisões preventivas, 104
prisões temporárias e 6 prisões em flagrante\footnote{\versal{BRASIL}. Ministério Publico
  Federal. Disponível em: \textless{}\emph{https://bit.ly/2qa4nvC}\textgreater{}. Acesso
  em 27/7/2017.}.

O \versal{MPF} formulou 279 pedidos de cooperação internacional, firmou 158
colaborações premiadas com pessoas físicas, 10 acordos de leniência e 1
termo de ajustamento de conduta.

Como resultado das investigações foram feitas 65 acusações criminais
contra 277 pessoas, sendo que já foram prolatadas sentenças
condenatórias pela prática de crimes contra o Sistema Financeiro
Internacional, corrupção, lavagem de dinheiro, tráfico de drogas, entre
outros. Foram apresentadas 8 acusações de improbidade administrativa
contra 50 pessoas físicas, 16 empresas e um partido político.

Até o momento contabilizam"-se 157 condenações, as quais correspondem a
1.563 anos, 7 meses e 5 dias.

Financeiramente, a operação envolve a investigação de R\$6,4 bilhões
decorrentes de pagamento de propina, sendo que 10,3 bilhões são alvo por
acordo de colaboração, R\$756,9 milhões por repatriação e 3,2
bilhões decorrentes da apreensão de bens e valores.

\subsection{A publicidade da Lava"-Jato}

Dada a grandiosidade da operação, a Lavajato não é julgada por apenas um
magistrado. Entretanto, o principal rosto condutor das decisões
judiciais tem sido Sérgio Moro. Até agora vê"-se uma tímida atuação do
\versal{STF} em relação aos políticos envolvidos, detentores de foro
privilegiado, expedindo ordem restritiva de liberdade apenas para
Delcídio do Amaral. Todos os outros mandados de prisão têm sido
expedidos pelo juiz Sérgio Moro.

Então, não se pode olvidar que o magistrado é o representante da
operação tanto na mídia brasileira quanto na internacional. E, por assim
ser, em 2015 recebeu diversos prêmios, merecendo o destaque obtido pelo
jornal \emph{New York Times} como uma das dez pessoas mais influentes do mundo.

É fato, outrossim, que a mídia tem se colocado em favor da atuação do
magistrado no combate à corrupção. Tal situação tem sido crucial para o
sucesso da operação.

Embora seja discreto e não fale diretamente para os meios de
comunicação, apenas ministrando palestras em eventos acadêmicos e aulas
como professor concursado na Universidade Federal do Paraná, Sérgio Moro
menciona ser favorável a atuação da mídia no sentido de divulgar o
andamento das investigações e cumprimento de ordens expedidas pela
Justiça.

Entretanto, é preciso destacar que o papel dos meios de comunicação na
sociedade se resume a informar ao público sobre o que acontece no mundo.
O fato deve ser descrito e apurado da maneira mais verídica possível e
isenta de opinião do profissional. Nesse sentido, dispõe o art. 4º do
Código de Ética do Jornalismo.

Não tem sido essa a postura dos jornalistas no que tange à Lavajato.
Embora os fatos descobertos sejam estarrecedores, o jornalista deve
evitar propagar suas opiniões de modo a influenciar a opinião pública.

Notório é o interesse social no que se refere ao desvio
de verbas públicas. E, utilizando esse entendimento, é latente a
articulação da mídia com a Lavajato.

Pessoas públicas e políticos não investigados têm vindo a público ganhar
a simpatia popular para que haja uma comoção no sentido de continuação
das investigações. Mormente agora que a operação avança para atingir os
grandes políticos brasileiros, ter o povo a favor da investigação pode
assumir até o cunho eleitoreiro desejado pelos não investigados.

Por outro lado, merece destaque que também existe o grupo que critica e
até ofende os atos praticados pela Lavajato.

Opiniões divergentes fazem parte do Estado Democrático de Direito.
Entretanto, mesmo não se pronunciando diretamente na mídia, Moro tem
dado respostas indiretas às críticas em suas decisões e cada vez mais
alarmando a sociedade com a necessidade de tornar os atos da Lavajato
públicos, a exemplo da decisão que retirou o sigilo das conversas
telefônicas entre Lula e Dilma.

Sobre as investigações, o Ministério Público Federal tem página de
internet atualizada em seu portal e o procurador Deltan Dallagnol usa o
twitter para divulgar atos de combate à corrupção bem como a campanha do
\versal{MPF} de ``10 medidas contra a corrupção''.

Todas as operações deflagradas pela Lavajato, em questão de segundos, já
são noticiadas e seguidas pelos jornalistas, e algumas são alvo de
grande comoção, como ocorreu com a condução coercitiva do presidente
Lula.

O juiz é movido por princípios, assim como a Ciência do Direito.
Entretanto, toda ação judicial deve respeitar os limites éticos e
morais. Não é porque o caso investigado seja totalmente imoral e mereça
todo o combate que esteja ao alcance da humanidade que hão de serem
desrespeitados os limites da atuação ética.

Nesse sentido, entendemos que devem os meios de comunicação receberem os
freios necessários daqueles que atuam no caso, sob pena de estarmos
diante de uma situação totalmente conduzida pelo clamor social ao
arrepio dos limites jurídicos.

\subsection{Os efeitos da Lava"-Jato}

\subsubsection{Efeitos econômicos e sociais}

A operação Lavajato atualmente é fato corriqueiro no cotidiano do
brasileiro, seja pela desestabilidade que tem causado no contexto
político brasileiro, seja pelas dificuldades que tem acrescentado no dia
a dia material de todos.

Em razão de todo o esquema de corrupção ter se formado em torno da
Petrobrás, a Lavajato acelerou o processo de desvalorização econômica da
empresa. Em meados de 2015, a Petrobrás teve suas ações de mercado
reduzidas a menos de metade do valor contábil da empresa.

Não se pode, porém, atribuir todo o desgaste da Petrobrás à Lavajato. A
causa maior e direta da situação financeira da estatal decorre do abuso
de poder pelo Governo Federal enquanto acionista controlador da
companhia. Cabe ao governo, de acordo com o art. 238 da lei nº 6404 de
1976, estabelecer a política econômica e financeira das estatais.

Entre 2011 e 2014, a intervenção do governo na política de preços da
companhia foi desastrosa. A empresa não tinha como sustentar por tanto
tempo a baixa no preço de mercado dos combustíveis e paridade dos mesmos
com o mercado externo e promover investimentos bilionários, com recursos
próprios, no Programa de Aceleração do Crescimento (\versal{PAC}).

Entretanto, a Lavajato, ao atingir diretamente a estrutura de comando da
Petrobrás, serviu como catalisador de uma instabilidade que já vinha se
desenvolvendo.

Outrossim, o impacto da paralisação nas obras e serviços que vinham
sendo realizadas aumentou o desemprego de uma maneira gritante.

Apenas a título exemplificativo, em Pernambuco, os efeitos decorrentes
do encerramento apressado das obras da Refinaria Abreu e Lima provocou a
dispensa imotivada e sem custeio de verbas decorrentes do encerramento
do pacto empregatício de aproximadamente 30.000 trabalhadores. A queda
de 1\% do \versal{PIB} brasileiro é atribuída pelos membros do Poder Executivo à
operação Lavajato.

Quanto aos efeitos sociais, como na maioria das crises históricas, quem
sofre mais é o lado hipossuficiente da relação jurídica estatal, a
sociedade.

Para controlar a crise econômica desencadeada no país, repita"-se, da qual
a Lavajato tem sido a mola propulsora, embora o pano de fundo tenha sido
a política de coalizão iniciada na primeira gestão da presidente Dilma,
cabe ao governo adotar medidas regressivas que impactam diretamente no
poder econômico da população.

Os índices de desemprego têm crescido. Sem emprego a população em
desespero tende a delinquir, o que impacta diretamente no crescimento da
violência.

Sob os auspícios de baluarte da moralidade, a Lavajato tem causado um
desequilíbrio financeiro e social sem precedentes na história do país.
Ademais, a população tem experimentado momentos de retrocesso que há
doze anos, tempo em que o governo social esteve no poder, não se
imaginava.

Tudo isso nos faz questionar até que ponto uma investigação contra a
corrupção deve ser feita de maneira tão veloz e açodada, sem que se leve
em conta os impactos causados na população já tão desgastada com a
corrupção. Vê"-se, pois, que estamos duplamente punidos, seja pelos
desvios de dinheiro público, seja pelas medidas adotadas pelo governo
para controlar a situação sócioeconômica do país, os quais impactam
diretamente na população.

\subsubsection{Efeitos políticos}

Ao destruir os esquemas corruptivos e começar a investigar um número
indeterminado, ainda, de políticos, a Lavajato atingiu de tal forma o cenário
da política brasileira que já é latente o fato de que se tornou uma
investigação muito mais política que jurídica.

Desde a primeira eleição de Lula à Presidência, o país já estava
dividido politicamente entre esquerdistas e direitistas. As tensões
decorrentes da bipolarização aumentavam a cada eleição e, no primeiro ano
da Lavajato, o candidato da direita, Aécio Neves, perdeu por menos de
1\% de diferença de votos para a candidata da esquerda, Dilma.

Os resultados das eleições geraram um desespero na direita, no sentido
de que eleitoralmente não seria possível o retorno ao poder. Pediu"-se
recontagem de votos, tentou"-se anular a eleição via \versal{TSE}. Até que a mídia
traz a figura de Moro como o representante maior da direita e permite
que seja traçado um poderoso discurso de combate à corrupção.

Com a operação Lavajato o discurso abstrato das jornadas de junho de
2013 concretizou"-se. Agora, os direitistas conseguiram o apoio popular
que buscavam há 16 anos. As pessoas iam para as ruas protestar pelo fim
da corrupção, não apenas o povo, mas principalmente a elite que sempre
vinha votando nos candidatos de direita.

Quanto à parcela carente da população que aderiu ao movimento pró"-Moro,
há de se dizer que o país já estava em crise, o governo tinha adotado
uma agenda regressiva para contenção de gastos e já não era possível
controlar a inflação e o surgimento de postos de trabalho tão bem quanto antes. Tudo isso enfraquecia a crença no desgastado governo esquerdista.

Nesse contexto, não restou outra alternativa à presidenta eleita que não
aderir ao propalado discurso de que a moralidade venceria qualquer tipo
de corrupção. Foi dado apoio irrestrito pelo governo às operações, até
porque esse era o principal discurso de Dilma, o combate à corrupção,
fazendo parte da agenda de governo.

Toda a situação política para o retorno da direita ao poder estava
armada. Dilma foi acusada de praticar pedaladas fiscais e editar
decretos sem numeração em afronta à lei de Responsabilidade Fiscal.
Eduardo Cunha capitaneou a galope o processo de impeachment na Câmara
dos Deputados. Dias após, teve seu próprio mandato cassado pelos seus
pares e atualmente está preso por ordem do juiz Sérgio Moro, em razão da
prática de diversas condutas ilícitas, entre elas recebimento de
propina. No Senado Federal, o impeachment foi confirmado, tendo como
líder Renan Calheiros, denunciado em 10 processos da Lavajato.

Entretanto, até hoje não foi efetivamente demonstrada a prática das
referidas pedaladas fiscais.

Michel Temer, também investigado e já denunciado, assumiu o cargo de
Presidente da República. Ressalte"-se que a presidente Dilma não é
investigada na operação, mas mesmo assim foi afastada pela ação dos
inúmeros parlamentares investigados e denunciados, sem nenhuma
credibilidade pública para a condução do processo de afastamento.

Percebe"-se, pois, que o principal efeito político da Lavajato foi causar
instabilidade também no campo político, a tal ponto de culminar no
afastamento de uma presidenta legitimamente eleita. Situação esta que
fere o Estado Democrático de Direito apregoado pela Constituição Federal
de 1988

\section{Análise de algumas decisões judiciais~proferidas
no âmbito da~Operação Lava"-Jato}

Nessa sessão pretendemos estudar algumas das decisões de maior
repercussão proferidas nos processos que compõem a operação Lavajato e
se as mesmas expressam as características do ativismo judicial aqui já
analisadas.

Os critérios de escolha das decisões foram a repercussão da mídia ao
serem prolatadas, bem como os efeitos sociais aliados aos impactos no
desmonte dos arranjos políticos voltados à prática da corrupção.
Outrossim, a análise será restrita a constatação ou não de ativismo
judicial no julgado em apreço, sem que seja feita qualquer valoração
jurídica do teor da decisão.

Importante destacar que a análise da decisão condenatória do ex"-presidente Lula, não obstante tenha sido uma das mais esperadas da operação, não será aqui analisada, seja pela estrutura da decisão, a qual foi proferida em 288 páginas, seja pela grandiosidade da mesma, a qual merece estudo apartado.

\subsection{Decisões do juiz Sérgio Moro}

\subsubsection{Condução coercitiva do ex"-presidente Lula}

Em março de 2016 a população brasileira acordou com a notícia de que o
ex"-presidente da República, Luís Inácio Lula da Silva, estaria sendo
conduzido coercitivamente por agentes da Polícia Federal. Estes estariam
cumprindo a ordem expedida pelo juiz Sérgio Moro, para prestar
depoimento no inquérito policial referente à operação Lavajato.

No despacho, o referido magistrado apresentou como razão para o
deferimento do pedido a manutenção da ordem pública. Motiva a
necessidade da medida da seguinte forma:

\begin{quote}
{[}\ldots{}{]} Embora o ex"-presidente mereça todo o respeito, em virtude da
dignidade do cargo que ocupou (sem prejuízo do respeito devido a
qualquer pessoa), isso não significa que está imune à investigação, já
que presentes justificativas para tanto, conforme exposto pelo \versal{MPF} e
conforme longamente fundamentado na decisão de 24/02/2016 (evento 4)~no
processo 5006617"-29.2016.4.04.7000. Por outro lado, nesse caso, apontado
motivo circunstancial relevante para justificar a diligência, qual seja
evitar possíveis tumultos como o havido recentemente perante o Fórum
Criminal de Barra Funda, em São Paulo, quando houve confronto entre
manifestantes políticos favoráveis e desfavoráreis ao ex"-Presidente e
que reclamou a intervenção da Polícia Militar [\ldots{}]\footnote{\versal{BRASIL}.
  7ª Vara Criminal Federal de Curitiba. Processo
  nº 5007401-06.2016.4.04.7000/\versal{PR}. Requerente: Ministério Público Federal.
  Requeridos Luís Inácio Lula da Silva e Marisa Letícia Lula da Silva.
  Juiz Sérgio Fernando Moro. Curitiba, 29 de fevereiro de 2016.
  Disponível em: \textless{}\emph{http://www.jfpr.jus.br}\textgreater{}.
  Acesso em 16/10/2016.}.
\end{quote}

A condução coercitiva está prevista no art. 201, §1º do Código de
Processo Penal, e não há menção do mesmo na Constituição Federal de 1988,
daí já começa a polêmica acerca da recepção ou não do instituto pelo
ordenamento jurídico. Entretanto, iremos considerar que o referido
artigo do \versal{CPP} está em plena vigência e pode ser aplicado caso
preenchidos os seguintes requisitos legais.

O que chama atenção na decisão é que o juiz de primeiro grau cita
e transcreve excertos do julgamento proferido pelo \versal{STF} no qual consta
expressa menção à possibilidade de atuação do Judiciário como intérprete
na aplicação da Constituição a situações não descritas expressamente em
seu texto e sem que tenham sido contempladas pelo legislador ordinário.
Assim decidiu o Ministro Ricardo Lewandowiski no teor do julgado
constante na decisão do juiz Sérgio Moro:

\begin{quote}
\versal{IV} -- Desnecessidade de invocação da chamada teoria ou doutrina
dos poderes implícitos, construída pela Suprema Corte norte"-americana e
incorporada ao nosso ordenamento jurídico, uma vez que há previsão
expressa, na Constituição e no Código de Processo Penal, que dá poderes
à polícia civil para investigar a prática de eventuais infrações penais,
bem como para exercer as funções de polícia judiciária
[\ldots{}]\footnote{Idem.}.
\end{quote}

Ao transcrever o julgado do \versal{STF}, o magistrado demonstra que, no que
for preciso, utiliza a doutrina do ativismo judicial para interpretar a
norma constitucional no que for conveniente para o andamento da
investigação.

Sobre a decisão o Ministro Marco Aurélio Mello fez pronunciamento
na mídia, esclarecendo que a medida da condução coercitiva é instituto
que só deve ser utilizado em última instância, quando haja no processo
demonstração expressa de que o intimado tenha se recusado a
comparecer\footnote{\versal{GLOBO}. Disponível em: \textless{}\emph{https://glo.bo/2t3HpIa}\textgreater{}.
Acesso em 16/10/2016.}.

Não há nos autos comprovação de que Lula tenha se recusado a
comparecer para depor. Da análise da decisão vê"-se, pois, que o juiz
extrapolou os limites jurídicos para atingir o seu objetivo de oitiva,
via inquérito policial, do ex"-presidente da República. Juridicamente, a
decisão é devidamente fundamentada e respaldada em julgado do \versal{STF}, logo,
carente de qualquer repreensão jurídica. Entretanto, para que
conseguisse levar Lula para ser interrogado, o juiz preferiu fazer uma
interpretação jurídica claramente ativista, posto que pautada em
fundamento extralegal, qual seja a exclusiva alegação de manutenção de
ordem pública. Ao invés de simplesmente determinar o comparecimento do
político para prestar depoimento.

Finalmente, merece destaque, mais uma vez, o fato de o julgado
transcrito pelo magistrado convalidar o entendimento já exposto no texto
da prática ativista pelo \versal{STF} de maneira expressa.

\subsubsection{Liberação dos áudios da conversa do ex"-presidente Lula
com Dilma Rousseff, presidente à época}

Inicialmente, cabe destacar o contexto em que a decisão a ser analisada
foi proferida. Após a condução coercitiva de Lula, os políticos ligados
ao \versal{PT} temeram a prisão imediata do ex"-presidente Lula, visto que um dos
argumentos de defesa seria a condução parcial do processo pelo
magistrado.

Para evitar tal decisão, a então Presidente da República, Dilma Rousseff,
declarou que iria nomear Lula como Ministro da Casa Civil, a fim de que
Lula passasse a gozar de foro privilegiado e qualquer decisão judicial
em relação a ele fosse proferida por algum ministro do \versal{STF}. O poder
decisório seria retirado, assim, da tutela de Sérgio Moro.

O juiz, para evitar que tal situação se consolidasse, retirou o sigilo
das interceptações telefônicas do número vinculado, teoricamente, a um
assessor do ex"-presidente Lula. Não obstante já existir requerimento
antigo nos autos, formulado pelo Ministério Público Federal, o
magistrado proferiu a decisão no dia em que teve notícia da aceitação do
cargo de Ministro da Casa Civil pelo investigado.

Merece destaque no despacho o seguinte trecho:

\begin{quote}
Como tenho decidido em todos os casos semelhantes da assim
denominada Operação Lavajato, tratando o processo de apuração de
possíveis crimes contra a Administração Pública, o interesse público e a
previsão constitucional de publicidade dos processos (art. 5º, \versal{LX}, e
art. 93, \versal{IX}, da Constituição Federal) impedem a imposição da
continuidade de sigilo sobre autos. O levantamento propiciará assim não
só o exercício da ampla defesa pelos investigados, mas também o saudável
escrutínio público sobre a atuação da Administração Pública e da própria
Justiça criminal. A democracia em uma sociedade livre exige que os
governados saibam o que fazem os governantes, mesmo quando estes buscam
agir protegidos pelas sombras. Isso é ainda mais relevante em um
cenário de aparentes tentativas de obstrução à justiça, como reconhecido
pelo Egrégio Supremo Tribunal Federal, ao decretar a prisão cautelar do
Senador da República Delcídio do Amaral Gomez, do Partido dos
Trabalhadores, e líder do Governo no Senado, quando buscava impedir que
o ex"-diretor da Petrobrás Nestor Cuñat Cerveró, preso e condenado por
este Juízo, colaborasse com a Justiça, especificamente com o Procurador
Geral de Justiça e com o próprio Supremo Tribunal Federal\footnote{\versal{BRASIL.
  PEDIDO DE QUEBRA DE SIGILO DE DADOS E/OU TELEFÔNICO
  Nº 5006205-98.2016.4.04.7000/PR}. Requerente: Ministério Público
  Federal. Acusados: \versal{L.I.L.S. PALESTRAS, EVENTOS E PUBLICACOES LTDA.},
  Instituto Luiz Inácio Lula da Silva, Elcio Pereira Vieira, Clara Levin
  Ant, Paulo Tarciso Okamoto e Luiz Inácio Lula da Silva. Juiz Sérgio
  Fernando Moro. Curitiba, 16 de março de 2016. Disponível em:
  \textless{}\emph{http://www.jfpr.jus.br}\textgreater{}. Acesso em 1/11/2016.}.
\end{quote}

Mais uma vez o magistrado, ao decidir, faz menção a ato praticado pelo \versal{STF}
e fundamenta sua decisão em artigos analíticos, de conceitos
amplíssimos, da Constituição, bem como em princípios fundamentais, como,
por exemplo, a publicidade e o interesse público, que têm suas aplicações
reguladas por normas constitucionais.

Ressalte"-se que o art. 8º da lei n° 9.296 de 24 de julho de 1996, ao
tratar das interceptações telefônicas, prevê expressamente que deverão
ser respeitados os sigilos das gravações, transcrições e interceptações
telefônicas.

Lógico que a divulgação dos áudios, no sentido de tornar públicos e
lançar para divulgação da mídia a conversa de uma presidente, cujo
processo de impeachment estava em curso, e de um ex"-presidente, ligado ao
mesmo partido da presidenta, tem um viés muito mais político do que
jurídico. Ressalte"-se que o fundamento de decisão exclusivamente em
texto constitucional abre espaço para interpretações não pautadas exclusivamente na
racionalidade jurídica, dado que a Constituição, por
estar no topo do ordenamento jurídico e ser a norma fundamental, da qual
todas as outras se originam, tem alguns conceitos amplos e vagos, tais
como o da publicidade, tão evidenciado pelo magistrado.

Cabe esclarecer que tal interpretação da Constituição Federal de 1988,
criando hipótese de retirada de sigilo de grampos telefônicos
exclusivamente para garantir a publicidade de situação envolvendo a
chefe do Poder Executivo, gerou um pedido de esclarecimentos do Ministro
do \versal{STF} Teori Zavascki para o magistrado de 1º grau.

Na resposta, Moro pede desculpas e esclarece que a decisão não teve o
condão de causar constrangimento para nenhuma autoridade detentora de
foro privilegiado. Sustenta que ainda que tenha feito uma interpretação
jurídica equivocada, em nenhum momento ultrapassou os limites da
investigação jurídico"-criminal, a qual tinha como figura exclusiva a
pessoa de Lula\footnote{\versal{ESTADÃO}. Disponível em:
  \textless{}\emph{https://bit.ly/2TuxOG8}\textgreater{}.
  Acesso em 19/10/2016.}. Embora tenha havido o pedido de desculpas, o
juiz conseguiu manter o processo sob sua tutela, dado que Lula não foi
empossado Ministro da Casa Civil.

Mais uma vez, da análise do conteúdo jurídico da decisão e do contexto
histórico, constatamos a existência de uma interpretação da Constituição
além do que está legislado, em prol da alegada satisfação do interesse
público. Tal atitude, portanto, enquadra"-se em hipótese de ativismo
judicial.

\subsection{Decisões do \versal{STF}}

Em razão da investigação acerca do esquema de corrupção da Petrobrás ter
atingido políticos no exercício do múnus público, entre eles deputados
federais e senadores, existem ações da Lavajato em trâmite no \versal{STF}.

Ao analisarmos algumas das decisões proferidas pela máxima instância do
Judiciário constatamos que, apesar do \versal{STF} ser um órgão cujas decisões
são voltadas em sua maioria para o ativismo judicial, no que concerne à
Lavajato essa não tem sido a postura adotada, via de regra.

Verifica"-se que as decisões proferidas pelo \versal{STF} em relação às
investigações, busca e apreensões, são facilmente prolatadas pelos
Ministros. Entretanto, as medidas restritivas de liberdade dos políticos
não têm merecido tanta atenção, principalmente, após a Lavajato estar
atingindo políticos de renome na esfera pública. Em razão do Ministro
Teori Zavascki ter falecido em acidente aéreo em dezembro de 2016, o
Ministro Edson Fachin assume a relatoria dos casos que envolvem a
Lavajato. O referido ministro tem proferido decisões mais arrojadas e
ativistas. Entretanto, observa"-se uma tendência à divisão dos processos
da operação em debate entre os Ministros da casa. Tal atitude faz com
que prevaleçam as decisões com aplicação restrita da lei, sem o viés do
ativismo judicial.

Via de regra, o \versal{STF} mantém as prisões decretadas por Moro e também
prolata decisões em outros processos que respaldam a atuação de Moro.
Nesse aspecto, merece destaque a decisão proferida nos autos do \versal{HC}
126.292 que relativizou o princípio norteador da presunção de inocência.
Segundo o \versal{STF}, o réu condenado em segunda instância pode começar a
cumprir a pena, sem que seja necessário o trânsito em julgado da decisão
penal condenatória. O relator, Ministro Teori Zavascki, justificou a
modificação da jurisprudência consolidada no \versal{STF} no sentido de que os
recursos de natureza extraordinária não comprometem o núcleo da não
culpabilidade do réu, o qual já é devidamente exaurido até a decisão de
segunda instância.

Até agora, após mais de três anos de Lavajato, só foi deferido um pedido
de prisão preventiva de político no exercício da função, o então senador
à época Delcídio do Amaral. O pedido de prisão de Eduardo Cunha só foi
apreciado por Moro após a perda do mandato pelo político e consequente
perda do foro privilegiado. Foi deferido pelo juiz de primeira instância
pelos fundamentos já existentes há seis meses e expostos na petição
apresentada ao \versal{STF}.

Em sequência, apreciaremos as decisões que determinaram a prisão de
Delcídio do Amaral, a que negou o pedido de restrição de liberdade dos
senadores Renan Calheiros e Romero Jucá e do ex"-presidente José Sarney e
a que cassou os direitos políticos de Aécio Neves.

\subsubsection{Ordem de prisão de Delcídio do Amaral}

Na decisão proferida na ação cautelar nº 4039, o Ministro Teori Zavascki
determina a prisão do senador Delcídio do Amaral. Fundamenta suas razões
no conjunto probatório apresentado pelo Ministério Público Federal.

Segundo o Ministro, o Senador estaria em conluio com o advogado Edson
Ribeiro para que Nestor Cerveró não firmasse acordo de colaboração
premiada. Ofereceram pagamento de R\$50.000,00 por mês à família de
Cerveró, bem como traçaram um plano de fuga do preso, através do
Paraguai, para que o mesmo chegasse à Espanha, país que Nestor Cerveró
também tem nacionalidade. Para o Ministro houve a clara prática de
obstrução à instrução criminal, dado que os atos praticados por Delcídio
e Edson Ribeiro configuravam uma conduta mafiosa e visavam obstruir o andamento
das investigações da Lavajato. Nestor Cerveró foi nada menos que o
diretor internacional da Petrobrás.

Todos esses fatos estão demonstrados na gravação de conversa mantida, em
um quarto de hotel em Brasília, entre Delcídio do Amaral, Edson Ribeiro
e Bernardo Cerveró, filho de Nestor. A gravação foi feita por Bernardo
sem o conhecimento dos outros dois participantes.

É importante ressaltar que a Constituição Federal de 1988, assim como tem
regras protetivas ao magistrado no exercício da função, estabelece
condições para a prisão do parlamentar em exercício. Segundo o art. 53,
§2º da Carta de 1988, apenas em caso de flagrante delito de crime
inafiançável é possível a prisão de congressista. Numa interpretação
literal do dispositivo em consonância com o art. 5º, \versal{LVII} da \versal{CF} de 1988,
como a nova redação do Código de Processo Penal considera inafiançáveis
apenas os crimes hediondos e equiparados, seria impossível a prisão de
parlamentar antes do trânsito em julgado de decisão penal condenatória.

Entretanto, o Ministro Teori Zavascki, mesmo assim, decretou a prisão do
senador Delcídio do Amaral. Desde 1985, foi a primeira prisão de membro
do Congresso Nacional legitimamente eleito.

Na decisão, de 39 laudas, o Ministro considera que a regra
constitucional deve ser relativizada, dado que foi elaborada no período
de início da democracia, quando ainda havia medo do retorno da ditadura.
Argumenta que neste momento do Brasil a democracia já estaria
consolidada, não podendo o parlamentar ser investigado criminalmente e
não poder estar sujeito à prisão cautelar. Não obstante o entendimento
do Ministro, o texto do art. 53, §2º permanece escrito da mesma maneira
que o constituinte originário previu.

Ademais, como fundamento da decretação da prisão, a decisão também
considera a validade de uma gravação feita por Bernardo Cerveró. O
Ministro também relativiza o disposto no art. 5º, \versal{LVI} para considerar
que não é prova ilícita a obtida sem conhecimento dos participantes,
posto que ausente cláusula legal de sigilo ou reserva de conversação.

Nessa decisão, a qual consideramos a mais ``agressiva'' do \versal{STF} no
contexto da Lavajato, é possível vermos tanto no contexto da análise da
aceitação da gravação ambiental como prova lícita, quanto nos
fundamentos propriamente da prisão, uma postura ativista do \versal{STF} no que
tange a aplicação da Constituição a situações não descritas
expressamente em seu texto e sem que tenham sido contempladas pelo
legislador ordinário.

\subsubsection{Indeferimento de pedido de prisão de Renan Calheiros,
José Sarney e Romero Jucá}

Em maio de 2016 o Ministério Público Federal ajuizou ação cautelar
postulando o deferimento de ordem restritiva de liberdade em relação aos
senadores Renan Calheiros, Romero Jucá e ao ex"-presidente José Sarney.
Argumenta que os três políticos estariam praticando manobras para
impedir o andamento da Lavajato. A conduta estaria tipificada no art.
2º, § 1º, da lei nº 12.850/2013. Tais fatos foram colhidos de
depoimentos decorrentes de colaboração premiada de José Sérgio de
Oliveira Machado e filhos.

Na decisão, o Ministro Teori Zavascki entendeu que não havia
materialidade da conduta. Sustenta que nos depoimentos colhidos a
conduta dos políticos teria sido apenas indicar alguém para conversar
com os Ministros envolvidos em processos da operação Lavajato no
sentido de obter decisões favoráveis.

As provas colhidas pelo \versal{MPF} neste requerimento decorrem de colaboração
premiada, homologada pelo \versal{STF}. Veja"-se que na decisão anteriormente
analisada, o juízo de convicção do Ministro foi baseado em gravação
ambiental e depoimento prestado pelo filho de Nestor Cerveró, o qual
poderia ter interesse na obtenção de algum benefício para o pai, que já
estava preso.

Valorando o conteúdo das duas provas produzidas, sem adentrar no mérito
da tipicidade das condutas ou da gravidade dos fatos, convém demonstrar
como um mesmo tribunal pode adotar uma postura ativista ou conservadora,
a depender da postura que o magistrado pretende adotar.

Renan Calheiros responde a 10 processos no \versal{STF} por condutas
potencialmente criminosas decorrentes da operação Lavajato. Mesmo
assim, o Ministro entendeu que não havia prova contundente da
materialidade da prática de ato ilícito. Já Delcídio do Amaral, à época
não tinha nenhum processo e ainda nem tinha havido a colaboração
premiada de Cerveró.

Sabe"-se que a decretação da prisão em análise poderia causar um
desequilíbrio político sem precedentes no que concerne à harmonia da
atuação entre os três poderes. Entretanto, seria coerente com o discurso
e atuação de Moro no que concerne ao combate à corrupção.

\subsubsection{Decisão que limita o exercício dos direitos políticos~por~Aécio Neves}

Em 19/5/2017, o ministro Edson Fachin determinou a prisão cautelar de
Andrea Neves da Cunha, Frederico Pacheco de Medeiros e Mendherson Souza
Lima, aplicando medidas diversas da prisão ao Senador da República Aécio
Neves da Cunha.

O senador é alvo de sete investigações no Supremo. Cinco delas já faziam
parte da lista de Fachin\footnote{\textless{}\emph{https://bit.ly/2ofg8Ol}\textgreater{}.}, abertas a partir de delações da empreiteira Odebrecht. Em uma delas, será investigado por ter pedido vantagens indevidas para a
campanha dele à Presidência em 2014. Outras duas foram abertas a partir
de delação do senador cassado Delcídio do Amaral, sobre Furnas e o
Mensalão.

A decisão proferida pautou"-se na delação premiada de Joesley Batista,
que apresentou gravação do senador pedindo 2 milhões de reais para
financiar sua defesa na Lavajato. O senador ficou impedido de praticar
atos parlamentares, teve o passaporte apreendido e ficou proibido de ter
contatos com outros parlamentares.

Até então nenhuma decisão do \versal{STF} teve tal viés. O Ministro, com base nas
disposições da lei penal e com uma interpretação dos fatos, não decretou
a prisão do senador, mas restringiu o exercício dos direitos políticos e
impôs medidas restritivas de liberdade diversas da prisão.

Entretanto, os autos da ação cautelar 4327 foram redistribuídos, ficando
o Ministro Fachin com a relatoria apenas do Inquérito Policial que
investiga o presidente Michel Temer, neste caso específico.

Com a redistribuição, o Ministro Marco Aurélio Mello, em 30/6/2017, sem
que tivesse sido juntada qualquer prova diversa aos autos, reconsiderou
a decisão do Ministro Fachin, restabelecendo todos os direitos políticos
a Aécio Neves, bem como afastando as demais restrições de liberdade
impostas.

Na decisão, fundamenta o Ministro Marco Aurélio:

\begin{quote}
É mais que hora de a Suprema Corte restabelecer o respeito à
Constituição, preservando as garantias do mandato parlamentar. Sejam
quais forem as denúncias contra o senador mineiro, não cabe ao \versal{STF}, por
seu plenário e, muito menos, por ordem monocrática, afastar um
parlamentar do exercício do mandato. Trata"-se de perigosíssima criação
jurisprudencial, que afeta de forma significativa o equilíbrio e a
independência dos Três Poderes. Mandato parlamentar é coisa séria e não
se mexe, impunemente, em suas prerrogativas [\ldots{}].
\end{quote}

O Ministro acrescenta que é incabível o afastamento do exercício do
mandato, em liminar, sem a existência~de processo"-crime contra o
parlamentar, porquanto no momento da decisão do ministro Edson Fachin
ainda não havia denúncia contra o senador~Aécio referente ao caso em
questão.

Observe"-se que, diante dos mesmos fatos, os dois ministros tiveram
posturas e interpretação legal diversa. Para o Ministro Fachin
era mais importante a aplicação da lei penal em detrimento da garantia
constitucional do exercício do mandato parlamentar e, tal interpretação,
sendo legal -- posto que pautada nos princípios da moralidade,
publicidade e aplicação da lei penal, todos previstos
constitucionalmente --, foi reconsiderada também com base na Constituição.

Estamos diante de duas decisões contrárias em um mesmo caso. Uma que
representa um avanço na interpretação constitucional, a qual foi feita
para garantir a moralidade, diante das provas coligidas e do clamor
social de ver a atuação do juiz no sentido de garantir a ordem pública e
demonstrar que o Judiciário estaria agindo politicamente na tentativa de
justificar a vontade popular de combate a corrupção.

Já a decisão do Ministro Marco Aurélio afigura"-se extremamente jurídica,
respaldada apenas nos ditames constitucionais de cumprimento do mandato
parlamentar. Ressalte"-se que, nesse caso, se o povo, quem escolhe os
seus representantes, soubesse de tais fatos, provavelmente não elegeria
o senador Aécio Neves.

\section{Conclusão}

A Operação Lavajato colocou o país em um momento político diferenciado,
porquanto movimenta todas as funções do Poder do Estado. As estruturas
políticas do Brasil estão todas desequilibradas. Não existe mais
estabilidade no Legislativo, cujos membros investigados atacam o
Judiciário; enquanto o Executivo, que ascendeu ao poder por manobra do
Legislativo, ao arrepio da escolha popular, tem suas ações limitadas por
tal Poder.

Tudo isso reflete na sociedade, que se submete às ações dos três
Poderes, que argumentam que estão assim agindo para equilibrar o Estado,
sem de fato revelar o interesse real que os move.

Outro fato que merece destaque é que nos países desenvolvidos, os juízes
têm uma conduta discreta na atuação e na vida social, até por uma
exigência da função que exercem. A imparcialidade e a discrição social
andam de mãos dadas e não tem como ser dissociadas da vida privada. Mas,
na operação Lavajato, a imprensa atua como se fizesse parte do processo.
Os juízes que atuam de forma colegiada, até mesmo antes de se
pronunciarem nos autos, já concedem entrevistas, seja para mencionar
como irão agir no processo, seja para criticar as decisões dos colegas.

Segundo Jessé de Souza\footnote{\versal{SOUZA}, Jessé. \emph{A radiografia do
  golpe.} Rio de Janeiro, \versal{RJ}: LeYa, 2016, p 118}, a Lavajato fez eclodir
no país não apenas a judicialização da política, mas a politização da
Justiça. Sempre que houver predominância da política sobre o direito
este perde sua autonomia.

De fato, o desmonte da teia de corrupção existente no Brasil era
necessário. Diante de tantas distorções e por estarmos acostumados ao
``jeitinho brasileiro'' ou à ideia de que no Brasil ``tudo acaba em
pizza'', é difícil crer que a Lavajato cumprirá o papel de acabar com a
corrupção, como é a pretensão de Sérgio Moro.

A Operação Lavajato impôs ao país uma nova regra social, trocamos um
governo corrupto por outro tão corrupto quanto, haja vista que também
está sendo investigado. Mas também trocamos um governo social, que
permitia que o pobre sonhasse como uma melhoria de vida, por outro que
prega a estagnação das classes sem qualquer perspectiva de melhora. As
medidas até então declaradas são retrógradas, conservadoras e resvalam
diretamente no povo.

E se estivéssemos vivendo toda essa situação para que acabasse a
corrupção poderíamos dizer que teria valido a pena vivenciá"-la.
Mas não acabará, porque o problema da corrupção não é partidário,
mas é social, endêmico e sistêmico.

Das lições do historiador Leandro Karnal\footnote{\versal{BRASIL}. Disponível em:
  \emph{\textless{}https://bit.ly/2TErTOR\textgreater{}}.
  Acesso em 1 de novembro de 2016.} percebemos que não basta um juiz,
com uma atuação ativista, tentar banir do poder os agentes corruptos. É
preciso que haja uma reforma social, para que o povo seja educado a não
corromper: a criança não ache normal colar na prova; o pai não pense que
está fazendo um bem para o filho assinando um atestado médico que
justifique a ausência na avaliação; ou que o adolescente não ache normal
comprar a entrada do cinema primeiro que os colegas porque furou a fila.

Não é porque a corrupção está enraizada no Brasil desde a colonização,
visto que os portugueses não vieram para cá visando o nosso bem, mas
para enriquecer com nossos bens, como disse o Padre Antônio Vieira no
sermão do Bom Ladrão, que não cremos que uma nova ordem social possa ser
imposta.

A operação Lavajato não representará o termo da corrupção, pois esta só
findará quando a sociedade estiver sendo educada para arcar com a
responsabilidade de assumir seus erros e o compromisso com o social e
não apenas com o individual. O modo como isso será feito é uma
incógnita.

Entretanto, merece destaque o fato de que a Lavajato trouxe tal discurso
à tona e plantou na sociedade o debate acerca da ética na atuação
profissional, consubstanciada nas práticas anticorruptivas. Para que a
semente germine é preciso que existam resultados efetivos na Lavajato,
que seja validada a ideia de que vilipendiar a sociedade é uma atitude
que nos atinge mediatamente.

Ante o exposto, percebemos que a atuação política dos juízes, embora
consolidada mundialmente, deve existir de maneira a assegurar a
existência da Democracia e a paz social, nunca a instabilidade. A
Lavajato, embora tenha suas falhas e polêmicas, principalmente em
decorrência da atuação ativista dos juízes ou de autocontenção do \versal{STF},
despertou no povo a ideia de combate à corrupção. A população cada vez
mais clama pela reconstrução de novas bases de Poder. Desejamos que o
tempo seja aliado do povo e traga não apenas a esperança, mas meios para
a reconstrução de um país voltado para a prática do social, com uma
melhor distribuição de riquezas e educação.

\section{Referências Bibliográficas}

1. \versal{BARROSO}, Luís Roberto. \emph{Constituição, democracia e
supremacia judicial: Direito e Política no Brasil Contemporâneo}.
Revista da Faculdade de Direito da \versal{UERJ}, nº 21, Rio de Janeiro, 2012.
Disponível em:
\emph{\textless{}https://bit.ly/2Sjch6F}\textgreater{}.
Acesso em setembro de 2016.

2. \versal{CALAMANDREI}, Piero. \emph{Eles, os juízes, vistos por um advogado}.
São Paulo, \versal{SP}: Martins Fontes, 1995.

3. \versal{GHIRARDI}, André Garcez. \emph{Petrobras: as causas da crise, além da
Lava Jato}. Disponível em:
\emph{\textless{}https://bit.ly/2y 9y4lj\textgreater{}}.
Acesso em outubro de 2016.

4. \versal{JORG}, Janes e \versal{CARVALHO}, Marcelo S. de. \emph{Os trabalhadores diante da
operação Lava Jato}. Disponível em:
\emph{\textless{}https://bit.ly/2TAl7K0\textgreater{}}.
Acesso em 11 de março de 2015.

5. \versal{LIMA}, Flávia Danielle Santiago. \emph{Ativismo e autocontenção no
Supremo Tribunal Federal: Uma proposta de delimitação do debate}.
Repositório Institucional da \versal{UFPE}, fevereiro de 2013. Disponível em:
\emph{\textless{}https://bit.ly/2RPqJ16\textgreater{}}.
Acesso em setembro de 2016.

6. \versal{NETTO}, Vladimir. \emph{Lava Jato: O juiz Sérgio Moro e os
bastidores da operação que abalou o Brasil}. Rio de Janeiro:
Primeira Pessoa, 2016.

7. \versal{SCARPINO}, Luiz. \emph{Sérgio Moro: O homem, o juiz e o
Brasil}. Ribeirão Preto: Novo Conceito, 2016.

8. \versal{SOUZA}, Jessé. \emph{A radiografia do golpe}. Rio de Janeiro:
LeYa, 2016
