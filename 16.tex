\chapter*{O fenômeno do \emph{Lawfare} sobre as óticas do direito e da política:\\
\emph{Uma análise do caso Lula}}
\addcontentsline{toc}{chapter}{O fenômeno do \emph{Lawfare} sobre as óticas
do direito e da política,\\
\scriptsize{por Victor Fialho}}
\hedramarkboth{O fenômeno do \emph{Lawfare} sobre as óticas\ldots{}}{}

\begin{flushright}
Victor Fialho\footnote{Jurista e chefe de gabinete da deputada federal Marília Arraes, é vice"-líder da bancada do \versal{PT} na Câmara Federal.}
\end{flushright}

O termo \emph{lawfare} foi cunhado, em 2001, em artigo escrito pelo
general Charles Dunlap, membro da \emph{Duke Law School} e da Força
Aérea norte"-americana, quando buscou legitimar as ações e estratégias
militares através do uso da lei, com finalidade totalmente pragmática,
que possibilitou que o país pudesse atingir seus objetivos de guerra
adaptados ao século \versal{XXI}. Tais objetivos fizeram com que setores da
inteligência e da espionagem de Washington pudessem intensificar a
guerra contra o terror, após o atentado de 11 de setembro de 2001.

A partir do anúncio de guerra contra o terrorismo, os Estados Unidos
iniciaram um processo de perseguição contra a rede al"-Qaeda e o seu
líder, Osama Bin Laden -- antigo aliado nas disputas geopolíticas locais.
Desta forma, os Estados Unidos expandiram seu campo de batalha no mundo
oriental, a partir de uma estratégia de defesa e de segurança nacional,
atendendo principalmente aos objetivos comerciais e políticos,
pressionando governos e induzindo mudanças de regime.

A definição de \emph{lawfare} foi muito além da incorporação de um
hiper"-legalismo nas operações militares, pois, o próprio autor do termo
o definiu como ``o uso da lei como uma arma de guerra'' e ``um método de
guerra em que a lei é usada como meio de realizar um objetivo militar''\footnote{\versal{DUNLAP}, 2008.}. No entanto, como transformar a adesão à lei como uma
estratégia que atenda aos propósitos do combatente? A resposta pode ser
encontrada no trabalho de Carl von Clausewitz em sua análise tripartite:
o povo, o governo e as forças armadas. Na sua visão, não bastava ter o
controle das forças armadas ou do governo, teria de ter adesão popular
para obter o sucesso da estratégia de guerra\footnote{\versal{WERNER}, 2010.}.

Em seu livro \emph{Guerra do tesouro}, Juan Zarate relata como o poder estatal
dos Estados Unidos se utilizou de uma variedade de legislações para
atacar financeiramente os adversários do século \versal{XXI}, desencadeando uma
nova estratégia para consolidar uma guerra financeira e fiscal com o
mesmo objetivo que todas as outras guerras em contextos semelhantes:
aniquilar o inimigo\footnote{\versal{WATERS}, 2008.}. A partir do lançamento do
\emph{Lawfare Institute,} em Londes, há agora uma concentração de
estudos desse fenômeno de aniquilação do inimigo com fins políticos.

Na América Latina, o processo de judicialização da política emana do
consenso sobre a corrupção como um problema endêmico do aparelho estatal\footnote{\versal{DOS SANTOS}, 2017.}. Para obter sucesso, requer"-se uma articulação com a
mídia que opera para fabricar o consenso contra ou a favor de certas
personalidades, grupos ou setores políticos. A aceitação ou
desmoralização do adversário político são especialmente verdadeiras ao
nível da opinião pública\footnote{\versal{BERGER}, 2018.}.

No Brasil, o estudo do fenômeno \emph{lawfare,} aplicado ao caso do
ex"-presidente Luiz Inácio Lula da Silva, faz parte de um processo de
Supremacia do Poder Judiciário que ocorre desde a instalação da Operação
Lava Jato no Brasil, cujo objetivo tinha o combate ao crime
institucionalizado na Petrobras, envolvendo políticos, diretores e
funcionários de carreira. A ação moralizante anticorrupção ocorreu com a
unidade entre juízes, promotores, a Polícia Federal e a grande mídia,
que foi grande responsável por aumentar o apoio popular da operação. O
Poder Judiciário tornou"-se, nos últimos anos, um poderoso instrumento
para implantar, quase sem limitações, estratégias de desestabilização e
perseguição política; este é o único dos demais poderes que não deriva
da vontade popular.

A operação Lava Jato foi gestada a partir de um programa de
aconselhamento oferecido pelo governo dos Estados Unidos para membros do
Poder Judiciário do Brasil, trazendo métodos de obtenção de evidências e
do uso da figura incomum de denúncia em larga escala. Além do destaque
do juiz Sérgio Moro como um dos alunos desses programas de treinamento\footnote{\versal{LUBAN}, 2017.}. Vale ressaltar que é impossível tratar sobre o
\emph{lawfare} no Brasil sem citar a importância que o juiz Sérgio Moro
adquiriu durante todo o processo de investigações.

Moro passou a atuar como parte interessada no processo, sem isenção,
exercendo o papel de um juiz acusador, provocando apelo popular para que
a população saísse às ruas em defesa da Operação Lava Jato, nos diversos
atos convocados pelo Movimento Brasil Livre -- \versal{MBL}, por todo o Brasil.
Ao pedir demissão do seu cargo, abandonou a Magistratura para ocupar o
cargo de Ministro da Justiça e Segurança Pública do governo do atual
presidente do Brasil Jair Bolsonaro, que reposiciona o país no contexto
global de guinada à direita em forte aliança com os Estados Unidos da
América e Israel.

Concomitantemente a todo o processo de perseguição política e judicial
contra o ex"-presidente Lula, é cabível também chamar a atenção para o
que vem ocorrendo com o Brasil desde o impeachment, em 2016, da
sucessora de Lula, a ex"-presidente Dilma Rousseff, quando o seu vice
Michel Temer assumiu a presidência e tomou medidas focadas na redução
substancial dos gastos sociais em longo prazo e na eliminação dos
direitos trabalhistas.

O ex"-presidente Michel Temer retomou a discussão sobre antigas
privatizações em diversos setores, além de ter aliado a lei de partilha
do petróleo após a crise com a Petrobras, o que fez com que
multinacionais estrangeiras avançassem na divisão do poder do pré"-sal.
No atual governo, do presidente Jair Bolsonaro, além da pauta econômica
muito identificada com o presidente anterior, há uma pauta mais
conservadora perante os costumes e de bastante confronto com os direitos
humanos e da preservação do meio ambiente.

Rubens Casara, juiz do Tribunal de Justiça do Estado do Rio de Janeiro,
sustenta, em seu livro \emph{Estado pós"-democrático: neo"-obscurantismo e
gestão dos indesejáveis}, que o atual momento é de um contexto de
pós"-democracia, que pode ser compreendido como uma consequência da
desconsideração dos valores democráticos observada no contexto atual de
diversos países, desaparecendo limites éticos e jurídicos para uma
elevação da concepção neoliberal. Os interesses dos que detêm o poder
econômico passaram a ser incompatíveis com a manutenção da concepção
democrática que surgiu após a Segunda Guerra e, por consequência,
direitos e garantias constitucionais tornaram"-se desinteressantes ao
desenvolvimento do mercado e à consolidação do projeto capitalista de
Estado, passando a extinguir direitos e assumindo uma feição
pré"-moderna, visto que o poder econômico volta a se identificar com o
poder político.

Dessa forma, o discurso de combate à corrupção e toda a abordagem
midiática sobre as investigações durante a operação Lava Jato serviram
para a não valorização da democracia no Brasil\footnote{\versal{DE SOUZA RODRIGUES},
2017.}, quando nos momentos de denúncia das perseguições políticas
cometidas contra o ex"-presidente Lula. E o objetivo? O impedimento do
ex"-presidente Lula em disputar as eleições presidenciais de 2018 e a
derrota de seu candidato.

\pagebreak

\section{Referências Bibliográficas}

\begin{Parskip}
\versal{DUNLAP JR}, Charles J. Lawfare today: A perspective. \emph{Yale J.
Int´l Aff.}, v.3, p. 146, 2008.

\versal{WERNER}, Wouter G. The curious career of lawfare. \emph{Case W. Res. J.
Int´l L}., v. 43, p. 61, 2010.

\versal{WATERS}, Christopher. Beyond Lawfare: Juridical Oversight of Western
Militaries. \emph{Alta. L. Rev}., v. 46, p. 885, 2008.

\versal{DOS SANTOS}, Adelino Pereira. O ofidiário e as jararacas: acontecimento
midiático, discurso político e deslizamento de sentido. \emph{A cor
das Letras}, v. 16, n. 1, p. 103-111. 2017.

\versal{BERGER}, Christa. O golpe da mídia: a crítica ao jornalismo no discurso
de intelectuais. \emph{Revista Observatório}, v. 4, n.1, p. 307-326,
2018.

\versal{LUBAN}, David. Lawfare and Legal Ethics in Guatanamo. \emph{Stan. L.
Rev}., v. 60, p. 1981, 2007.

\versal{CASARA}, Rubens. \emph{Estado Pós"-Democrático -- Neo"-Obscurantismo e
Gestão dos Indesejáveis}. 2017.
\end{Parskip}