\chapter*{A Constituição Federal de 1988 e a efetividade dos direitos sociais}

\addcontentsline{toc}{chapter}{A Constituição Federal de 1988 e a efetividade
dos direitos sociais,\\ \scriptsize{por Wilson Ramos Filho e Nasser Ahmad Allan}}
\hedramarkboth{A C.F. de 1988 e a efetividade dos direitos sociais}{}

\begin{flushright}
\emph{Wilson Ramos Filho}\footnote{Doutor pela Universidade Federal do
  Paraná (\versal{UFPR}) e Pós"-doutor pela Escola de Altos Estudos em Ciências
  Sociais de Paris. Professor de Direito do Trabalho da \versal{UFPR} e de
  Direitos Sociais da Universidad Pablo de Olavide, em Sevilha.
  Presidente do Instituto de Defesa da Classe Trabalhadora (Declatra).}
e \emph{Nasser Ahmad Allan}\footnote{Pós"-Doutorando no Programa de Pós"-Graduação
  em Direito da Universidade Federal do Rio de Janeiro -- \versal{UFRJ}, vinculado
  ao grupo \emph{Configurações Institucionais e Relações de Trabalho} --
  \versal{CIRT}. Doutor e Mestre em Direitos Humanos e Democracia pela
  Universidade Federal do Paraná. Advogado trabalhista em Curitiba.
  Diretor institucional do Instituto de Defesa da Classe Trabalhadora
  (Declatra).}
\end{flushright}

No ano de 2018, a Constituição Federal brasileira completou trinta anos
de existência, sem que os setores mais progressistas da sociedade
brasileira possam celebrar. Com o decorrer do tempo, pôde"-se constatar,
com consternação, pesar e, por que não dizer, revolta, que as garantias
de direitos fundamentais, como os sociais, previstas no texto
constitucional, foram negligenciadas da agenda política dos
representantes da população nos Poderes Executivo e Legislativo, assim
como relativizadas no Poder Judiciário.

Nesse cenário, não parece exagerado afirmar que pouco se concretizou do
Estado de Bem"-Estar Social projetado quando da Assembleia Nacional
Constituinte.

Compreender o direito \emph{nunca} como ``resultado neutro de uma
decisão arbitrária do poder'', mas sim como fruto de ``um processo
dinâmico de conflito de interesses que, desde diferentes posições de
poder, lutam por conduzir seus anseios e valores, ou seja, seu
entendimento das relações sociais, a lei''\footnote{\versal{HERRERA}, J.
  \emph{La reinvención de los Derechos Humanos}. Atrapasueños:
  Sevilla"-Espanha, s/d. p. 101.}, isto é, como resultante da correlação
de forças na luta de classes, permite uma melhor aproximação às
ambiguidades do texto constitucional, apreendidas entre os primados de
um Estado de Bem"-Estar Social e o ideário neoliberal.

Durante a Assembleia Nacional Constituinte consta\-tou"-se existirem
parlamentares mais identificados com os anseios da sociedade por uma
carta constitucional que se inclinasse a estruturar um Estado
Democrático de Direito, rompendo com o ranço autoritário da ditadura
civil"-militar que recém se superava, estabelecendo assim um rol amplo de
direitos civis e políticos, além de direitos que resultassem em
prestações materiais positivas por parte do Estado à população, os
direitos sociais.

Em contraposição a aqueles, em flagrante antagonismo de classes, havia
um grupo representativo dos interesses do grande capital que resistia à
inclusão de qualquer dispositivo de avanço social, e que pretendia
incorporar no texto constitucional um modelo de Estado Neoliberal, onde
prevalecesse a racionalidade da eficiência, instigada pela competição, e
pela privatização das atividades estatais essenciais, sendo assumidas,
evidentemente, pelos setores mais fortes da iniciativa privada.

Resultou dessa correlação de forças uma Constituição híbrida em vários
pontos. Para exemplificar, ao mesmo tempo em que garante a proteção
contra a dispensa arbitrária na relação de emprego, relegou"-se à lei
infraconstitucional a possibilidade de lhe atribuir indenização
compensatória (artigo 7º, \versal{I}, da \versal{CF}), o que conduziu ao esvaziamento do
conteúdo desse direito, pela posição jurisprudencial dominante da
Justiça do Trabalho. Igualmente, podem ser mencionados os dispositivos
que privilegiaram a negociação coletiva de trabalho como instrumentos
para ampliação de direitos, porém, o legislador constituinte originário
encarregou"-se de admitir a possibilidade também por negociação coletiva
de flexibilizar direitos em temas centrais ao capital: salário (artigo
7º, \versal{VI}, da \versal{CF}) e jornada de trabalho (artigo 7º, \versal{XIII} e \versal{XIV}, da
\versal{CF})\footnote{\versal{RAMOS} \versal{FILHO}, Wilson. \emph{Direito Capitalista do
  Trabalho: histórias, mitos e perspectivas no Brasil}. São Paulo: \versal{LT}r,
  2012. p. 339.}.

Talvez, a síntese do hibridismo constitucional possa ser vislumbrada no
inciso \versal{III} do artigo 1º da Constituição Federal, no qual são colocados, lado
a lado, como fundamentos da República Federativa do Brasil, os valores
sociais do trabalho e da livre iniciativa.

A despeito disso, se comparada às anteriores, pode"-se asseverar que a
Carta Magna de 1988 implantou, no plano do direito, grande avanço
social, ao considerar os direitos sociais como fundamentais,
tratando"-os, portanto, como direitos humanos.

Na definição do artigo 6º, entre os direitos sociais, estariam abrangidos
``a educação, a saúde, a alimentação, o trabalho, a moradia, o
transporte, o lazer, a segurança, a previdência social, a proteção à
maternidade e à infância, a assistência aos desamparados''\footnote{\versal{BRASIL}.
  Constituição da República Federativa do Brasil de 1988. Artigo 6º.
  Disponível em: \textless{}\emph{https://bit.ly/1dFiRrW}\textgreater{}.
  Acesso em 24. Set. 2017.}. Cabem ainda mencionar como direitos sociais
os previstos em outras regras constitucionais, como ao meio"-ambiente
sustentável (artigo 225) e à cultura (artigos 5º, \versal{IX}, e 215 a 217).

Com a abertura democrática, a primeira eleição direta para a Presidência
da República conduziu ao poder Fernando Collor de Mello, cujo mandato
foi interrompido por um processo de impedimento que o levou à renúncia.
No entanto, durante os dois anos em que efetivamente governou o país,
constatou"-se a tomada de medidas legislativas e administrativas de cunho
neoliberal, o que, praticamente, impediu a concretização dos direitos
sociais previstos na Constituição.

O mesmo pode"-se afirmar em relação ao período em que Fernando Henrique
Cardoso esteve à frente do país (1995--2002), quando se verificou a
intensificação de políticas consentâneas à racionalidade
neoliberal\footnote{O neoliberalismo pode ser compreendido como a
  ``razão do capitalismo contemporâneo'' e definido como um ``conjunto
  dos discursos, das práticas, dos dispositivos que determinam um novo
  modelo de governo dos homens segundo o princípio universal da
  competição''. In: \versal{LAVAL}, Christian; \versal{DARDOT}, Pierre. \emph{La nueva
  razón del mundo}: ensayo sobre a sociedad neoliberal. Tradução Alfonso
  Diez. Barcelona: Gedisa, p. 15.}, envolvendo, entre outras, a
privatização de parte da esfera pública e de atividades antes assumidas
pelo Estado, assim como de medidas legislativas que buscaram
flexibilizar direitos sociais da classe trabalhadora, em especial, no
tocante a formas de contratação e à jornada de trabalho\footnote{Para
  exemplificar podem ser citados o contrato de trabalho provisório e o
  banco de horas, ambos regulados pela Lei 9.601, de 1998.}.

A efetividade dos direitos sociais constitucionais permaneceu distante
da agenda política destes Governos, sendo, simplesmente, negligenciada
sob pretexto de que o país não detinha condições financeiras de suportar
as prestações materiais exigidas para o cumprimento das garantias
constitucionais. Em contrapartida, seguiu"-se com os pagamentos dos
exorbitantes juros da dívida interna e externa do país.

No Congresso Nacional, para atender os interesses do capital financeiro
e internacional, inúmeros projetos de lei e propostas de emenda
constitucional foram apresentados, com clara finalidade de restringir a
eficácia dos direitos sociais, sob as falsas justificativas de
``modernizar o país'' e de ``impedir a quebra das contas públicas'' e
outras falácias similares, tal qual a de que os direitos sociais
inibiriam o crescimento econômico\footnote{\versal{BELLO}, Enzo. Cidadania e
  direitos sociais no Brasil: um enfoque político e social. In:
  \emph{Espaço Jurídico}, Joaçaba, v. 8, n. 2, p. 133-154, jul./dez.
  2007.}. Para ficar só em um exemplo, e nesse horizonte
político"-ideológico que se insere a reforma da previdência social,
implantada com a Emenda Constitucional n. 20, de 1998, que restringiu,
sobremaneira, o direito à aposentadoria de brasileiros e brasileiras.

Mesmo durante os mandatos de Luís Inácio Lula da Silva (2003--2010) e
Dilma Rousseff (2011--abril/2016), por mais que se tenha estancado em
certa medida o avanço da racionalidade neoliberal, especialmente, em
decorrência de programas públicos de inclusão social, ainda assim, o
país manteve"-se muito distante das bases de existência de um Estado de
Bem"-Estar Social.

Na verdade, em maior ou menor medida, a depender de quem esteve à frente
do Poder Executivo Federal, foram os interesses do mercado, dos grandes
grupos econômicos nacionais e internacionais, que dominaram (e que,
atualmente, dominam ainda mais) a cena política no Brasil.

A inexistência de políticas públicas que permitissem a efetivação dos
direitos sociais constitucionais, como educação, saúde, cultura,
moradia, aposentadoria, entre outros, induziu uma parte um pouco
mais abastada da sociedade a buscar a prestação desses serviços na
iniciativa privada. Ainda, durante a gestão de Fernando Henrique Cardoso,
foi criada a figura jurídica das organizações sociais, entidades
privadas, que assumiram muitas das tarefas que deveriam ser executadas
pelo Estado, em atividades essenciais, tais como saúde, educação e
assistência social. Com isso, a cidadania social assumiu a feição de
mera relação de consumo.

Em outra perspectiva, como a Constituição Federal de 1988 admite a
eficácia jurídica imediata das normas que se referem a direitos
fundamentais (artigo 5º, § 1º), o que inclui os direitos sociais, a não
execução de prestações materiais positivas exigíveis do Estado foi
levada ao Judiciário, tornando"-se objeto de disputas judiciais.

Todos os anos milhares de ações judiciais são ajuizadas em face do Poder
Executivo, em seus três níveis, para que o Judiciário lhe determine a
concretização das garantias sociais previstas no texto constitucional. A
despeito de firmar"-se entendimento de que as normas que versam sobre
direitos sociais são autoaplicáveis e, portanto, exigíveis imediatamente
perante o Estado, o Judiciário vem mitigando sua eficácia com a adoção
dos princípios do mínimo existencial e da reserva do possível na solução
judicial dos casos.

O primeiro deles aponta o entendimento de que para não desrespeitar o
princípio da separação entre os poderes e a democracia, a interferência
do Poder Judiciário na esfera do Executivo, com a concessão judicial de
prestações materiais positivas em benefício da parte demandante, somente
seria plausível quando estivesse sob ameaça a garantia do mínimo
existencial desta, podendo ser concebida como a liberdade ou como a
dignidade da pessoa humana.

O princípio da reserva do possível atrela"-se à ideia de existir uma
constante tensão entre os direitos sociais, assegurados pela maior parte
das Constituições contemporâneas, e os orçamentos públicos destinados
pelos Estados para efetivação deles. Assim, para deferir qualquer
pretensão de satisfação de algum dos direitos sociais, os Juízes
deveriam considerar o impacto de suas decisões nas contas públicas e o
nível de interferência gerado na autonomia do Poder Executivo\footnote{Ibidem.},
o que leva não raramente ao insucesso de ações civis coletivas ou ações
civis públicas, ante o potencial econômico dessas demandas.

Pode"-se constatar também que mesmo as decisões judiciais reconhecendo e
conferindo direitos aos cidadãos em face do poder público não se
mostraram suficientes, como não são, a arrefecer os índices de
desigualdade econômica e social, exatamente em razão da eficácia
jurídica dessas decisões, por não produzirem efeitos para além das
partes litigantes. Privilegiou"-se, com isso, quem detém recursos
econômicos para acessar ao Judiciário, negligenciando a concessão de
cidadania a quem não pode arcar com os custos de um processo judicial.

Parece claro, portanto, que a efetividade dos direitos sociais
assegurados na Constituição Federal de 1988 está longe de ser garantida
pelo Poder Judiciário. As respostas obtidas pela via judicial mostram"-se
insuficientes, seja porque nem sempre reconhece o direito dos cidadãos e
cidadãs às prestações materiais requeridas, seja porque quando o faz
acaba por naturalizar e acentuar ainda mais as desigualdades econômicas
e sociais existentes no país, contribuindo para a segregação de duas
espécies de cidadania social, a de quem detém recursos econômicos, sendo
mais ampla, abrangente e efetiva, e a de quem não pode acessar ao
Judiciário, tratando"-se de uma forma de subcidadania.

Retomando as ideias lançadas no início deste artigo, se o direito
resulta da correlação de forças nas relações sociais de produção; é
resultado da luta de classes, o que dizer então da transformação de leis
em ações estatais? Mais do que ao Judiciário, a sociedade civil
organizada, em sindicatos, partidos políticos, associações, organizações
não governamentais ou movimentos sociais, deve mobilizar"-se para exigir
do Poder Executivo, em seus diferentes níveis, políticas públicas para
efetivação dos direitos sociais garantidos na Constituição Federal.

Tais direitos devem ser compreendidos como resultado das lutas sociais
do povo brasileiro por uma vida digna, sendo conquistados, portanto,
nas ruas. É nelas onde ele deverá exigir a efetividade de suas
conquistas!
