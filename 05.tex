\chapter{Morreu na Contramão atrapalhando o Sábado -- A Constituição, o
Golpe e as Reformas.}

\begin{flushright}
\emph{Maria Goretti Nagime Barros Costa}\footnote{Maria Goretti
Nagime Barros Costa é advogada, mestranda em
Sociologia Política na Universidade Estadual do Norte Fluminense Darcy
Ribeiro (UENF) e pós"-graduanda em Direitos Humanos e Estudos Críticos de
Direito no Conselho Latinoamericano de Ciências Sociais (CLACSO).}
\end{flushright}

A Constituição Federal de 1988 representa verdadeiro marco de um patamar
civilizatório.

Ela não é chamada de ``Constituição Democrática'' por acaso. Sua alma é
a democracia, e não só por estabelecer que ``todo poder emana do povo,
que o exerce por meio de representantes eleitos ou
diretamente''.\footnote{Constituição Federal de 1988, Artigo Primeiro,
  Parágrafo Único.}

Sua promulgação significou a superação de tempos sombrios - a ditadura -
em que o Brasil refletia a onda fascista do mundo. Sua filosofia
humanística e libertária, não só estabelecendo a democracia como seu
ponto central, mas prestigiando de forma geral os direitos sociais,
tornou"-a a mais importante Carta de Direitos da história do país.

Significou democracia também para o Direito do Trabalho, quando criou
condições para uma maior participação dos grupos sociais na produção de
normas jurídicas, valorizando convenções e acordos coletivos, a
negociação coletiva e a atuação sindical, inclusive protegendo da
despedida arbitrária dirigentes sindicais e membros de CIPAs (Comissões
internas de Prevenção de Acidentes) e protegendo as organizações
sindicais da intervenção do Estado.

Houve naquele momento algo muito maior do que a extensão de direitos aos
trabalhadores rurais, domésticos, avulsos, gestantes, etc -- o que, é
claro, não foi pouco. Foi inserido ali o olhar democrático conquistado a
duras penas naquele momento histórico"-político.

A Constituição, portanto, foi provocada e constituiu"-se como um legítimo
fruto dos movimentos sociais, políticos e correntes filosóficas daquela
época.

Nela foi reconhecido pela primeira vez que no contrato empregatício as
partes têm forças desiguais. Isso representou um marco na evolução do
pensamento social ligado à relação empregatícia. De fato, a própria
criação de leis trabalhistas objetiva a necessária proteção do
trabalhador justamente pela desigualdade de forças das partes.

\begin{quote}
``Toda estrutura normativa do Direito Individual do Trabalho constrói"-se
a partir da constatação fática da diferenciação social, econômica e
política básica entre os sujeitos da relação jurídica central desse ramo
jurídico específico. Em tal relação, o empregador age naturalmente como
\emph{ser coletivo}, isto é, um agente socioeconômico e político cujas
ações -- ainda que intraempresariais -- tem a natural aptidão de
produzir impacto na comunidade mais ampla.
\end{quote}

Em contrapartida, no outro polo da relação, inscreve"-se um ser
individual, consubstanciado no trabalhador que, como sujeito desse
vínculo sociojurídico, não é capaz, isoladamente, de produzir, como
regra, ações de impacto comunitário. Essa disparidade de posições na
realidade concreta fez emergir um Direito Individual do Trabalho
largamente protetivo, caracterizado por métodos, princípios e regras que
buscam reequilibrar, juridicamente, a relação desigual vivenciada na
prática cotidiana da relação de emprego.''\footnote{DELGADO, M. Godinho,
  2009, p. 181.}

Por isso, a partir da Constituição de 88, passou a vigorar no Direito do
Trabalho a ótica da noção do ser coletivo, em contraposição ao Direito
Civil, em que predomina a ótica de ser individual.

\begin{quote}
Conquistado este patamar civilizatório, seria -- e é -- constrangedor
falar"-se em retrocesso de garantias à parte mais vulnerável do contrato
empregatício, ou, mais objetivamente, retrocesso do pensamento social.
Sabemos que, historicamente, depois de conquistado direito do
trabalhador ou interpretação de proteção, daquele ponto não há que se
retroagir.
\end{quote}

No sistema capitalista é chamado de ``um bom empresário'' aquele que
gasta o mínimo e recebe de volta o máximo possível. Reconhecer o
trabalho daquele que realmente produz, o trabalhador, significa um
entrave, um obstáculo ao objetivo final do empresário: o maior lucro
possível. O salário e os direitos concedidos ao trabalhador são
comumente vistos não como investimentos, mas exatamente como ``gastos''.

A responsável pela a mágica de investir"-se pouco e lucrar"-se muito é a
exploração. De forma que, culturalmente, pagar ao trabalhador cada vez
menos e ter cada vez mais lucro não é visto como um sinal de ingratidão
ou falta de consciência de produção, mas tão somente uma decorrência
objetiva do sistema econômico vigente. Por isso as regras do Direito do
Trabalho foram criadas e são necessárias: para garantir que se respeite
minimamente a saúde física e mental do trabalhador.

O capitalismo ``substituiu o escravo antigo por `homens reduzidos ao
estado de produtos'''\footnote{CASARA, Rubens. Apresentação. In:
  TAVARES, Juarez; PRADO, Geraldo. \textbf{O Direito Penal e o Processo
  Penal no Estado de Direito: análise de casos}. Florianópolis: Empório
  do Direito, 2016.}\footnote{LACAN, Jacques. \textbf{O avesso da
  psicanálise.} Rio de Janeiro: Jorge Zahar, 1991, p. 35.}. Os direitos
dos trabalhadores, assim como todos os direitos fundamentais, ``antes
entendidos como trunfos civilizatórios contra maiorias de ocasião e
limites intransponíveis às perversões inquisitoriais, passaram a ser
percebidos pela população em geral {[}...{]} como obstáculos ao
desenvolvimento do mercado {[}...{]}\footnote{TAVARES, Juarez; PRADO,
  Geraldo. \textbf{O Direito Penal e o Processo Penal no Estado de
  Direito: análise de casos}. Florianópolis: Empório do Direito, 2016, p
  7.}

A Constituição de 88, portanto, em sua intenção nitidamente ligada ás
liberdades democráticas, inegavelmente constituiu em marco histórico do
pensamento na luta pelos direitos sociais, de forma a representar a
superação do terror da então recente ditadura militar e -- mais
especificamente em sua parte de normatização do Direito do Trabalho -- a
superação do forte passado escravocrata brasileiro.

No entanto, em 2016, um governo popular legitimamente eleito sofreu um
Golpe de Estado. Tomou o poder justamente o grupo que havia perdido por
sucessivas vezes as eleições diretas, símbolo da democracia estabelecida
através desta mesma Constituição.

As medidas implementadas a partir de então foram as propostas por aquele
grupo durante as eleições - exatamente as propostas refutadas pela
população através do voto.

Portanto, através de governo ilegítimo, formado por ministros e
propostas que perderam a eleição popular, as medidas e expressões das
covardias do período histórico anterior, a ditadura, vieram à tona.

Foram violados não pontualmente um ou outro ponto da Constituição, mas
exatamente a alma da Carta Magna -- A Democracia e os Direitos Sociais.

Iniciou"-se o desmonte das conquistas históricas: as reformas da
previdência, trabalhista, a lei da terceirização, entrega do petróleo e
gás do Pré-Sal, a venda de terras ilimitadas a estrangeiros, previsão de
redução drástica de investimentos em saúde e educação, etc.

O presidente da Câmara dos Deputados chegou a declarar publicamente que
"a justiça do trabalho não deveria nem existir".

O golpe e a decorrência do golpe -- as reformas -- não por acaso
significam igualmente a negação da Constituição de 88. A tomada da
Presidência da República por um governo ilegítimo e o retrocesso da
evolução na escala dos direitos do trabalhador significam a reação aos
direitos sociais conquistados.

O grupo político que perdeu a eleição não só tomou o poder como também
tenta tirar o candidato popular da disputa através de Lawfare. Tenta até
mesmo impedir novas eleições diretas. Não se pode ignorar a informação
que o candidato popular em questão, Luis Inácio (Lula), foi por duas
vezes presidente, e avaliado por todos os institutos de pesquisa do país
como o melhor presidente da história.

``Ficaram mais nítidas suas intenções pelos escandalosos cortes do
governo golpista nas pastas sociais. E Lula é justamente símbolo de um
projeto de inclusão social. Citemos como exemplo de sua gestão a
eliminação do trabalho escravo infantil na região do Nordeste, o que
provocou a ira da classe que era beneficiada com a
escravidão.''\footnote{NAGIME, Maria Goretti. Quem está acima da lei?
  In: \textbf{Comentários a uma sentença anunciada:} o processo Lula.
  Bauru: Projeto Editorial Praxis, 2017, p. 366.}

Sobre a Reforma Trabalhista, inegável a incidência em gravíssimo
retrocesso. Retoma ao tipo de poder individual do empregador próprio do
Código Civil de 1916, que já havia sido superado há 15 anos com a
promulgação de um Novo Código Civil, e que considerava a relação de
emprego como locação de serviços, dando poderes irrefutáveis ao
empregador. Nega, repita"-se, as conquistas sociais alcançadas previstas
na Constituição Democrática.

A aprovação da Reforma Trabalhista ocorreu um dia após já ter sido
tentada pela primeira vez, quando então perdeu a votação, o que, por si
só, já demonstra a torpeza e cumplicidade dos agentes envolvidos em sua
aprovação. Datou de um mês após ter sido aprovada a lei de Terceirização
Plena, também um grande retrocesso.

``Um exemplo da condição de ``igualdade'' em que se encontram
trabalhadores e empresários pode ser ilustrada pelo o próprio PL. O
relator da matéria, Rogério Marinho, acatando a ``pedidos'' do
agronegócio, incluiu no seu parecer o fim das~\emph{horas in itinere},
que são horas contadas no contrato quando o trabalho é de difícil acesso
ou sem transporte regular. Ou seja, enquanto o trabalhador não tem nem
transporte adequado para chegar ao trabalho, o patrão tem condições de
pagar o parlamentar para alterar uma lei em seu favor.''\footnote{CALLEGARI,
  Isabela Prado. Um golpe por dia, 2017. Disponível
  em:\textless{}https://www.cartacapital.com.br/politica/um-golpe-por-dia\textgreater{}.
  Acesso em: 21 de abr. 2017.}

A reforma prevê a retirada de vários direitos do trabalhador, mas
nenhuma previsão choca tanto quanto a de que ``o negociado deve
prevalecer sobre o legislado''. Retira"-se, assim, a força coercitiva das
normas trabalhistas protetivas. Ora, esta previsão elimina o próprio
Direito do Trabalho, ignora os motivos de sua criação e existência, além
de partir de um pressuposto falso e já superado: o de que as partes na
relação empregatícia negociam em pé de igualdade.

Esta previsão simplesmente torna as leis trabalhistas facultativas, o
que, na lógica capitalista de priorização do lucro, significa a total
negação de qualquer proteção ao trabalhador. Colocaria em cheque até
mesmo normas de proteção mínimas, como as ligadas à saúde e segurança.

Não haveria problema, por exemplo, em contratar alguém para trabalhar no
sol o dia inteiro em troca de um prato de comida. Bastaria combinar. Não
haveria nada a proteger o trabalhador assombrado pelo medo de morrer de
fome.

Nenhuma lei trabalhista historicamente conquistada teria validade,
afinal, assim foi negociado, e após a Reforma Trabalhista ``o negociado
deve prevalecer sobre o legislado''.

A Reforma Trabalhista serve aos que lucram com a miséria humana.

Ao aprovar a Reforma Trabalhista, os Deputados não tiveram o
constrangimento de representar não a população trabalhadora que os
elegeu, mas os barões da indústria e do agronegócio que patrocinaram --
e provavelmente continuarão patrocinando - suas campanhas.

Há uma correlação lógica inegável entre os três acontecimentos: (1) O
golpe de Estado que retira da presidência um governo popular eleito
através de voto, (2) um líder historicamente reconhecido por projetos
implementados de inclusão social e combate à fome ser ceifado das
eleições presidenciais, e (3) a aprovação de uma Reforma Trabalhista que
retira direitos dos trabalhadores em contramão a toda evolução histórica
do pensamento social. Os três acontecimentos são nítida demonstração da
fragilidade do Estado Democrático de Direito.

Foi enviada para votação, inclusive, projeto de lei que prevê a
legitimidade de se pagar o trabalhador rural com casa e comida. Um
triste espectro da escravidão. Um retrato do Golpe e das decorrências do
Golpe.