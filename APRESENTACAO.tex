\chapter{Apresentação}% DO LIVRO CONSTITUINTE 1987-1988: DA ABERTURA DEMOCRÁTICA AO GOLPE E PRISÃO DE LULA}

\emph{Constituinte 1987"-1988: da abertura democrática ao golpe e prisão de Lula} é um livro
que surgiu da necessidade de analisar a ruptura institucional que o país
atravessa, consolidada no impeachment de Dilma Rousseff, em 31 de agosto de
2016, e aprofundada na aprovação da desreforma
trabalhista e do congelamento dos gastos sociais por 20
anos, e em mais um golpe dentro do golpe: a sentença do parcial juiz Sérgio
Moro, confirmada pelos três desembargadores da 8ª Turma do Tribunal Regional Federal da
4ª Região (\versal{TRF}"-4), que manteve a condenação e ampliou
a pena de prisão do ex"-presidente Luiz Inácio Lula da Silva por corrupção passiva e lavagem
de dinheiro no caso do triplex no Guarujá.

Um julgamento comprovadamente sem crime, sem provas e recheado de nulidades jurídicas, em que
o Supremo Tribunal Federal, ao rejeitar o pedido de habeas corpus do ex"-presidente Luiz Inácio
Lula da Silva, negou"-lhe o direito da presunção de inocência que é cláusula
pétrea da Constituição Federal de 1988, rasgando"-a: lê"-se, no Artigo 5º, inciso
\versal{LVII}, que ``Ninguém será considerado culpado até o trânsito em julgado de sentença
penal condenatória''.

Por algum tempo, nutrimos a esperança de que a Constituição de 1988 tivesse encerrado
o ciclo de instabilidade política no país; de que a democracia
estava consolidada e caminhávamos para um aprimoramento
permanente das nossas instituições. Ledo engano na interpretação do
nosso processo histórico.

No ano que relembramos os 30 anos da promulgação da Constituição Brasileira,
é importante rememorar o discurso histórico de Ulysses Guimarães em 5 de
outubro de 1988, marca na história e na memória do povo brasileiro. Posso
até imaginar que Ulysses antevia que não podemos descumprir a Constituição:
``A nação nos mandou executar um serviço. Nós o fizemos com amor, aplicação e sem medo. A
Constituição não é perfeita. Ela própria o confessa, ao admitir a reforma. Quanto a ela,
discordar sim. Divergir, sim. Descumprir, jamais. Afrontá"-la, nunca. Traidor da Constituição
é traidor da pátria. Conhecemos o caminho maldito: rasgar a Constituição, trancar as portas
do parlamento, garrotear a liberdade, mandar os patriotas para a cadeia, o exílio, o cemitério.
A persistência da Constituição é a sobrevivência da democracia. Quando, após tantos anos de
lutas e sacrifícios, promulgamos o estatuto do homem, da liberdade e da democracia, bradamos
por imposição de sua honra: temos ódio à ditadura. Ódio e nojo.''

Não aceitamos a perseguição e manipulação da justiça para tirar o ex"-presidente Lula
da disputa eleitoral de 2018. Todo o processo movido contra ele é uma
farsa partidária de setores do sistema judicial, orquestrado pela Rede Globo, com
o objetivo de tirá"-lo do processo eleitoral.

Um golpe político, midiático e judicial do capital financeiro nacional e internacional, que
se esforça para ostentar uma aparência de observância das mesmas regras jurídicas que
viola, deixando clara a fragilidade da nossa democracia.

\smallskip
\asterisc
\smallskip

O presente livro traz um conjunto de análises no campo político,
jurídico, econômico e sindical das principais vozes que têm combatido o bom combate da defesa
da democracia, do ex"-presidente Luiz Inácio Lula da Silva e
contra o estado de exceção que estamos vivendo.

Contribuem com seguintes textos os intelectuais da luta democrática, cada um na sua trincheira
de luta pela restauração do Estado Democrático de Direito:

\begin{itemize}
\item ``Educação da Constituição Cidadã ao Golpe de 2016''. De
Heleno Araújo, presidente da \versal{CNTE} (Confederação Nacional dos Trabalhadores em Educação);
\item ``A Constituinte e o capitalismo brasileiro''. De Luiz Gonzaga Belluzzo, economista e
professor do Instituto de Economia da Universidade Estadual de Campinas;
\item ``A fotografia constitucional de 1988''. De José
Eduardo Cardozo, advogado e professor de Direito. Foi Ministro da Justiça e Advogado"-Geral da
União, além de ter defendido a presidenta Dilma Rousseff no processo de impeachment;
\item ``Morreu na contramão atrapalhando o sábado -- A Constituição, o golpe e as
reformas''. De Maria Goretti Nagime, jurista;
\item ``A Constituinte, as mulheres e o Golpe''. De Vivian Farias, Dirigente Nacional
do Partido dos Trabalhadores e da Fundação Perseu Abramo;
\item ``A Constituição Federal de 1988 e a efetividade dos direitos sociais''. Dos
juristas Wilson Ramos Filho e Nasser Ahmad Allan;
\item ``Desdobramentos hermenêutico"-constitucionais do impeachment da ex"-presidente Dilma
Rousseff no contexto da pós"-democracia''. De Bruno Galindo, jurista e professor de Direito da
\versal{UFPE};
\item ``A Atuação política do Juiz: Uma análise à luz da Operação Lava Jato''. De Mariana
de Carvalho Milet, Juíza do Trabalho;
\item ``A defesa desvirtuada -- 30 anos da Constituição Brasileira''. De Ademar Rigueira
Neto, advogado e jurista;
\item ``A Constituição Federal das garantias dos direitos sociais e o golpe político
desconstitutivo do trabalho no Brasil''. De Marcio Pochmann, economista e presidente
da Fundação Perseu Abramo;
\item ``Justiça de transição e usos políticos do Poder Judiciário
no Brasil em 2016: um golpe de Estado institucional?'' De José Carlos Moreira Filho, jurista e
professor da \versal{PUC}/\versal{RS};
\item ``Réquiem para a
Constituição de 1988''. De Rafael Valim, jurista e professor de direito da \versal{PUC}/\versal{SP};
\item ``A classe trabalhadora e a luta em defesa
intransigente da Constituição Brasileira''. De Carlos Veras, presidente da \versal{CUT}/Pernambuco;
\item ``Os 30 anos da Constituição Brasileira''. De
Carmen Foro, vice"-presidenta nacional da \versal{CUT}.
\end{itemize}

Por fim, o livro \emph{Constituinte 1987"-1988: da abertura democrática
ao golpe e prisão de Lula} inclui na íntegra as entrevistas
que compõem o documentário “Constituinte: 1987-1988”, lançado em 2012 e
publicado em livro em 2016 pelo \versal{CFOAB} (Conselho Federal da Ordem dos Advogados do
Brasil). Foram entrevistados: Fernando Henrique Cardoso, Luiz Inácio Lula da Silva, Marco
Maciel, Mauro Benevides, Nelson Jobim, Fernando Lyra, Paulo Paim, Vicentinho, Cristovam Buarque,
Roberto Freire, José Genoíno, Egídio Ferreira Lima, Maurílio Ferreira Lima, Marcos Terena, Jair
Meneguelli e o cientista político David Fleischer, de forma a desvendar os bastidores do
processo que elaborou a Constituição Brasileira de 1988.

É possível percorrer nas entrevistas relatos inéditos de como tudo ocorreu. Entre as várias
etapas, expõe episódios do governo Sarney e de sua relação com o congresso, da elaboração do
regimento interno, a composição dos partidos políticos, a participação popular, as emendas
populares, da virada regimental com a criação do “Centrão”, reforma agrária, movimento sindical,
entre outros.

O momento atual é propício para o resgate histórico da Assembleia Nacional Constituinte, que
foi um dos eventos mais importantes da história política do país. Nossa Constituição encerrou
um ciclo de instabilidade política no Brasil e a democracia parecia consolidada, até o inicio da
ruptura institucional iniciada no impeachment sem crime de Dilma Rousseff, em 31 de agosto de
2016, e o aniquilamento da Constituição Brasileira, feita pela condenação e prisão sem provas
do ex"-presidente Luiz Inácio Lula da Silva, tornando"-lhe preso político.

\section{Siglas e abreviações}

\begin{Parskip}
\versal{ANC} -- Assembleia Nacional Constituinte

\versal{BASA} -- Banco da Amazônia

\versal{BNB} -- Banco do Nordeste do Brasil

\versal{CD} -- Câmara dos Deputados

\versal{CEBs} -- Comunidades Eclesiais do Brasil

\versal{CGT} -- Confederação Geral dos Trabalhadores

\versal{CLT --} Consolidação das Leis do Trabalho

\versal{CNBB} -- Conferência Naciona dos Bispos do Brasil

\versal{CONTAG} -- Confederação Nacional dos Trabalhadores na
Agricultura

\versal{CUT} -- Confederação Geral dos Trabalhadores

\versal{CN} -- Congresso Nacional

\versal{DANC} -- Diário da Assembleia Nacional Constituinte

\versal{DIAP --} Departamento Intersindical de Assessoria Parlamentar

\versal{DVS --} Destaque para Votar em Separado

\versal{FCO --} Fundo Constitucional de Financiamento do Centro"-Oeste

\versal{FNE --} Fundo Constitucional de Financiamento do Nordeste

\versal{FNO --} Fundo Constitucional de Financiamento do Norte

\versal{FUNDEB --} Fundo de manutenção e Desenvolvimento da Educação
Básica e de valorização dos profissionais da educação

\versal{INSS --} Instituto Nacional de Seguridade Social

\versal{IPI --} Imposto sobre Produto Industrializado

\versal{IR --} Imposto de Renda

\versal{PROCON --} Programa de Orientação e Proteção ao Consumidor

\versal{SF} -- Senado Federal

\versal{STF --} Supremo Tribunal Federal

\versal{SUS --} Sistema Único de Saúde

\versal{TSE --} Tribunal Superior Eleitoral

\versal{UBES} -- União Brasileira dos Estudantes Secundaristas

\versal{UDR} -- União Democrática Ruralista

\versal{UNE} -- União Nacional dos Estudantes
\end{Parskip}

\section{Partidos políticos}

\begin{Parskip}
\versal{PCB} -- Partido Comunista Brasileiro

\versal{PC}do\versal{B} -- Partido Comunista do Brasil

\versal{PFL} -- Partido da Frente Liberal

\versal{PDC} -- Partido Democrático Cristão

\versal{PDS} -- Partido Democrático Social

\versal{PDT} -- Partido Democrático Trabalhista

\versal{PMDB} -- Partido do Movimento Democrático Brasileiro

\versal{PT} -- Partido dos Trabalhadores

\versal{PL} -- Partido Liberal

\versal{PMB} -- Partido Municipalista Brasileiro

\versal{PSB} -- Partido Socialista Brasileiro

\versal{PSC} -- Partido Social Cristão

\versal{PTB} -- Partido Trabalhista Brasileiro

\versal{PSDB} --Partido da Social Democracia Brasileira
\end{Parskip}