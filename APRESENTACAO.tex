\chapter{Apresentação}% DO LIVRO CONSTITUINTE 1987-1988: DA ABERTURA DEMOCRÁTICA AO GOLPE E PRISÃO DE LULA}

\emph{Constituinte 1987"-1988: da abertura democrática ao golpe e prisão de Lula} é um livro
que surgiu da necessidade de analisar a ruptura institucional que o país
atravessa, consolidada no impeachment de Dilma Rousseff, em 31 de agosto de
2016, e aprofundada na aprovação da desreforma
trabalhista e do congelamento dos gastos sociais por 20
anos, e em mais um golpe dentro do golpe: a sentença do parcial juiz Sérgio
Moro, confirmada pelos três desembargadores da 8ª Turma do Tribunal Regional Federal da
4ª Região (\versal{TRF}"-4), que manteve a condenação e ampliou
a pena de prisão do ex"-presidente Luiz Inácio Lula da Silva por corrupção passiva e lavagem
de dinheiro no caso do triplex no Guarujá.

Um julgamento comprovadamente sem crime, sem provas e recheado de nulidades jurídicas, em que
o Supremo Tribunal Federal, ao rejeitar o pedido de habeas corpus do ex"-presidente Luiz Inácio
Lula da Silva, negou"-lhe o direito da presunção de inocência que é cláusula
pétrea da Constituição Federal de 1988, rasgando"-a: lê"-se, no Artigo 5º, inciso
\versal{LVII}, que ``Ninguém será considerado culpado até o trânsito em julgado de sentença
penal condenatória''.

Por algum tempo, nutrimos a esperança de que a Constituição de 1988 tivesse encerrado
o ciclo de instabilidade política no país; de que a democracia
estava consolidada e caminhávamos para um aprimoramento
permanente das nossas instituições. Ledo engano da interpretação do
nosso processo histórico.
No ano que relembramos os 30 anos da promulgação da Constituição Brasileira,
é importante rememorar o discurso histórico de Ulysses Guimarães em 5 de
outubro de 1988, marca na história e na memória do povo brasileiro. Posso
até imaginar que Guimarães antevia que não podemos descumprir a Constituição:
A nação nos mandou executar um serviço. Nós o fizemos com amor, aplicação e sem medo. A
Constituição não é perfeita. Ela própria o confessa, ao admitir a reforma. Quanto a ela,
discordar sim. Divergir, sim. Descumprir, jamais. Afrontá"-la, nunca. Traidor da Constituição
é traidor da Pátria. Conhecemos o caminho maldito: rasgar a Constituição, trancar as portas
do parlamento, garrotear a liberdade, mandar os patriotas para a cadeia, o exílio, o cemitério.
A persistência da Constituição é a sobrevivência da Democracia. Quando, após tantos anos de
lutas e sacrifícios, promulgamos o estatuto do homem, da liberdade e da democracia, brandamos
por imposição de sua honra: temos ódio à ditadura. Ódio e nojo.

Não aceitamos a perseguição e manipulação da justiça para tirar o ex"-presidente Lula
da disputa eleitoral de 2018. Todo o processo movido contra ele é uma
farsa partidária de setores do sistema judicial, orquestrado pela Rede Globo, com
o objetivo de tirá"-lo do processo eleitoral.

Um golpe político, midiático e judicial do capital financeiro nacional e internacional, que
se esforça para ostentar uma aparência de observância das mesmas regras jurídicas que
viola, deixando clara a fragilidade da nossa democracia.

\asterisc

O presente livro traz um conjunto de análises no campo político,
jurídico, econômico e sindical das principais vozes que têm combatido o bom combate da defesa
da democracia, da defesa intransigente do ex"-presidente Luiz Inácio Lula da Silva e
contra o estado de exceção que estamos vivendo.

Contribuem com seguintes textos os intelectuais da luta democrática, cada um na suas trincheiras
de luta pela restauração do Estado Democrático de Direito: Os 30 anos da Constituição Brasileira,
Carmen Foro, vice"-presidenta nacional da \versal{CUT}; A classe trabalhadora e a luta em defesa
intransigente da Constituição Brasileira, Carlos Veras, presidente da \versal{CUT}/Pernambuco;
A Constituinte, as Mulheres e o Golpe, Vivian Farias, Dirigente Nacional do Partido dos
Trabalhadores e da Fundação Perseu Abramo; Educação da Constituição cidadã e golpe de 2016,
Heleno Araújo, presidente da \versal{CNTE} (Confederação Nacional dos Trabalhadores em Educação);
A defesa desvirtuada – 30 anos da Constituição Brasileira, Ademar Rigueira Neto, advogado e
jurista; Desdobramento hermenêutico"-constitucionais do impeachment da ex"-presidente Dilma
Rousseff no contexto da pós"-democracia, Bruno Galindo, jurista e professor de Direito da
\versal{UFPE}; A Atuação política do Juiz: Uma análise à luz da Operação Lava Jato, Mariana
de Carvalho Milet, Juíza do Trabalho; Ativismo judicial, anistia e a ruptura democrática no
Brasil – algumas pistas sobre o golpe de 2016, José Carlos Moreira Filho, jurista e professor da
\versal{PUC}/\versal{RS};  A constituição  Federal de 1988 e a efetividade dos direitos sociais,
juristas Wilson Ramos Filho e Nasser Ahmad Allan;  A fotografia constitucional de 1988, José
Eduardo Cardoso, Advogado e Professor de Direito. Ex"-ministro da Justiça e ex"-advogado"-Geral da
União. Advogado no processo de impeachment da presidenta Dilma Rousseff; Réquiem para a
Constituição de 1988, Rafael Valim, jurista e professor de direito da \versal{PUC}/\versal{SP};
Morreu na contramão atrapalhando o Sábado - A Constituição, o golpe e as Reformas, Maria Goretti
Nagime, jurista; A constituinte e o capitalismo brasileiro, Luiz Gonzaga Belluzzo, economista,
professor do Instituto de Economia da Universidade de Estadual Campinas; A Constituição Federal
das garantias dos direitos sociais e o golpe político desconstitutivo do trabalho no Brasil,
Marcio Pochmann, economista e presidente da Fundação Perseu Abramo.

Por fim, o livro Constituinte 1987-1988: Da abertura democrática ao golpe, inclui as entrevistas
na íntegra que compõem o documentário “Constituinte: 1987-1988”, lançado em 2012, tendo sido
publicado em livro em 2016, pelo \versal{CFOAB} (Conselho Federal da Ordem dos Advogados do
Brasil). Na íntegra as entrevistas: Fernando Henrique Cardoso, Luiz Inácio Lula da Silva, Marco
Maciel, Mauro Benevides, Nelson Jobim, Fernando Lyra, Paulo Paim, Vicentinho, Cristovam Buarque,
Roberto Freire, José Genoíno, Egídio Ferreira Lima, Maurílio Ferreira Lima, Marcos Terena, Jair
Meneguelli e o cientista político David Fleischer, entrevistas que desvendam os bastidores do
processo da Assembleia Nacional Constituinte, que elaborou a Constituição Brasileira de 1988.

É possível discorrer nas entrevistas os relatos inéditos de como tudo ocorreu. Entre as várias
etapas, expõe episódios do governo Sarney e sua relação com o congresso, a elaboração do
regimento interno, a composição dos partidos políticos, a participação popular, as emendas
populares, a virada regimental com a criação do “Centrão”, reforma agrária, movimento sindical,
entre outros.

O momento atual é propicio para o resgate histórico da Assembleia Nacional Constituinte, que
foi um dos eventos mais importantes da história política do País. Nossa Constituição encerrou
um ciclo de instabilidade política no Brasil e a democracia estava consolidada, até o inicio da
ruptura institucional iniciada no impeachment sem crime de Dilma Rousseff, em 31 de agosto de
2016, e o aniquilamento da Constituição Brasileira, feita pela condenação e prisão sem provas
do ex"-presidente Luis Inácio Lula da Silva, tornando"-lhe preso político.