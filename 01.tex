\textbf{INTRODUÇÃO}

\textbf{Os bastidores da Carta Magna}

A~Constituinte~de 1998 redigiu e aprovou a Constituição considerada a
mais liberal e democrática que o Pais já teve, merecendo o nome de
``Constituição Cidadã''. O presente livro, baseado no documentário de
mesmo nome, contém a íntegra de todas as entrevistas e desvenda os
bastidores do processo de elaboração da Constituição Brasileira.

``Constituinte~1987-1988'' revela relatos inéditos de como tudo ocorreu.
Entre as várias etapas, o livro expõe episódios do governo Sarney e sua
relação com o Congresso, a elaboração do regimento interno, a composição
dos partidos políticos, a participação popular, as Emendas Populares, a
virada regimental com a criação do ``Centrão'', reforma agrária,
movimento sindical, entre outros.

Há também o registro de quando o PT (Partidos dos Trabalhadores) votou
contra a redação final da Carta e a assinatura do documento, além de
votações e uma análise atual dos entrevistados sobre a Constituição.

A Assembleia Nacional Constituinte foi um dos eventos mais
importantes~da história política do País. Nossa Constituição encerrou um
ciclo de instabilidade política no Brasil e a democracia se consolidou
em toda a Nação. A íntegra dos depoimentos de quem fez e vivenciou todo
o processo da elaboração da nossa Constituição é de enorme relevância
histórica.

Destaca-se a riqueza de todos os relatos, entre os quais estão os do
ex-presidentes da República Fernando Henrique Cardoso (na época,
senador~constituinte) e Luiz Inácio Lula da Silva (na condição de
deputado~constituinte); de Nelson Jobim, que atuou como ministro nos
mandatos de FHC, Lula e da presidenta Dilma Roussef; do deputado
constituinte Fernando Lyra, ministro da Justiça no governo Sarney; e do
então senador~constituinte Marco Maciel, posteriormente vice-presidente
no governo FHC.~

O~dia 5 de outubro de 1988, data da promulgação da Constituição
Brasileira, ficou marcado na história e na memória do povo brasileiro,
oportunidade na qual o presidente da Assembleia Nacional Constituinte, o
deputado Ulysses Guimarães, bradou ``mudar para vencer. Muda Brasil''.

O governo Sarney e a instalação do processo~Constituinte~é abordado em
todas as entrevistas, com relatos contundentes do ex-presidente Fernando
Henrique Cardoso. O então senador lembra o clima de forte espírito
libertário que havia entre os parlamentares ante a
``Constituinte~Soberana''. Lembra também que pairava um sentimento de
desconfiança e medo de alguns setores econômicos da sociedade
brasileira.

Nelson Jobim, à época deputado, destaca que o modelo aprovado foi
``simbiótico''. ``Era um modelo tipicamente brasileiro. No Brasil não
encontramos rupturas. É um país de transição. Quando um regime se
supera, de dentro do próprio regime surgem as fórmulas de superação
desse regime. É o que aconteceu'', relata Jobim.

No rol de personagens que testemunham aquele momento histórico, há a
visão crítica do então parlamentar~constituinte~Egídio Ferreira Lima,
seguido de Fernando Lyra, que cita o episódio em que Ulysses Guimarães
teria ``violado'' a Constituinte~ao permaner com sua candidatura à
reeleição para a Presidência do Congresso Nacional.

Jobim relata~que houve o ``desenterro'' de um anteprojeto, usado por
Fernando Lyra na disputa com Ulysses,~demarcando a figura dos
``constituintes de 1ª e 2ª categoria''. Lyra perdeu a eleição, mas tal
iniciativa, na visão de Jobim, ``queimava a possibilidade de fazer um
regimento interno baseado no modelo da~Constituinte~de 1946''.

O livro apresenta o processo de elaboração do regimento interno. FHC
relata a formação das várias comissões que contavam com a participação
de praticamente todos os parlamentares Constituintes. Ele dá ênfase à
existência de uma comissão mais cobiçada: a Comissão de Sistematização.
O ex-presidente FHC lembra também que a~Constituinte~brasileira se
baseou na Constituição de Portugal.

José Genoíno, por sua vez, revela a existência de uma tática que visava
fortalecer as subcomissões temáticas a partir de pressões para garantir
a soberania da Constituição.

Os partidos políticos também são analisados pelos entrevistados.~O
grande partido da época, o PMDB, ``era o majoritário, mas não possuía
coesão'', como lembra FHC. O Mário Covas foi uma das figuras mais
importantes da Constituinte. Para Egídio Ferreira Lima, durante o
período da~Constituinte os partidos políticos ``eram meras legendas''.
E~menciona a divisão muito clara dos políticos que integravam as
chamados setores das ``esquerdas'', ``centroesquerdas'' e ``direitas''.~

Fernando Lyra é contundente quando afirma que os partidos não
funcionavam. Para justificar sua posição, lembra que vários liberais
votavam em teses defendidas pelos esquerdistas, enquanto conservadores
votavam com o ``Centrão''. Genoíno conta que o bloco de esquerda, mesmo
sendo minoria, fez uma articulação eficaz, ofensiva, de aliança com o
bloco do PMDB que contava com FHC, Nelson Jobim e Pimenta da Veiga.

O deputado constituinte Paulo Paim (PT) destaca a parceria feita pelo PT
com o PMDB e cita Mário Covas e Ulysses Guimarães como importantes
aliados.

Entre os diversos depoimentos há relatos do cientista político David
Fleischer. Através de suas pesquisas, revela que o PMDB não era o maior
partido da~Constituinte, mas sim a Arena.~Afirma que muitos políticos
passaram, na sua origem, pela Arena, PDS e, depois, pelo PMDB por pura
conveniência para fins eleitorais. Fleischer também cita o surgimento do
MUP -- Movimento de Unidade Progressista, que teria reunido cerca de 100
deputados.

A participação popular é comentada por todos os entrevistados,~da
presença da mulher na política à luta indígena, passando por questões
como a defesa das crianças e adolescentes, da cultura, do meio-ambiente,
da educação pública de qualidade e da reforma agrária.~O líder indígena
Marcos Terena resgata o episódio da entrada de índios no gabinete de
Ulysses Guimarães, quando um chefe coloca um cocar na cabeça do
parlamentar.~Diz que o que está escrito na Constituição Brasileira, até
hoje, é um documento sagrado.

As emendas populares foram o elo de legitimação entre o Congresso e o
País na visão de todos os entrevistados, que nunca haviam visto o
Parlamento Nacional tomado pelo povo a reivindicar seus direitos.

Mauro Benevides, à época deputado~constituinte~pelo PMDB, relata as
várias emendas com milhares de assinaturas vindas do povo. O
ex-presidente Lula lembra a cobrança e participação da sociedade
brasileira: ``foi um coisa maluca a pressão da sociedade''.

O então reitor da Universidade de Brasilia -- UNB, Dr. Cristovam
Buarque, lança duras críticas à Constituição de 1988, que, segundo ele,
ficou muito a favor dos grupos corporativos organizados. ``O povo
descalço, os excluídos, eles não participavam das audiências. Alguns os
representavam, mas muito poucos. Os Constituintes ouviram a opinião
pública, não o povo'', provoca.

As ideias de Buarque são contestadas por Paulo Paim, que lembra que o
movimento social cumpria um papel fundamental na Assembleia
Nacional~Constituinte. De acordo com Paim, foi a maior mobilização vista
por ele em mais de 20 anos de carreira política. ``A pressão popular foi
muito grande e isso se refletia, por exemplo, nas Comissões''.

Nelson Jobim apimenta a discussão ao lembrar que, ao contrário do que se
dizia sobre haver uma ``sociedade civil organizada'', o que existia eram
``grupos de interesses organizados'', que queriam ``pedaços'' do Estado
para si (nichos do Estado). Ele é enfático quando fala sobre a
inexistência de representação popular. Jobim também critica as Emendas
Populares, que, no seu entendimento, não se tratavam de um manifesto
popular amplamente debatido. ``Era um grupo de interesse organizado que
fazia uma coleta de assinaturas na praia de Copacabana, tentando
legitimar como sendo vontade do povo''. Para ele, tudo não passava de um
jogo político.

O então presidente da Central Única dos Trabalhadores (CUT), Jair
Meneghelli, também avalia como de pouca repercussão a mobilização
popular. ``Conseguimos os abaixo-assinados, mas sem muito efeito'',
lamenta, justificando a representação minoritária dentro da Assembleia
Nacional~Constituinte e o fato de as emendas não serem automáticas -- ou
seja, obter um milhão de assinaturas não significava inserção na Carta
Magna da Nação. Para ele, a sensação é de que os movimentos sociais
foram enganados.

O livro analisa, através de entrevistas, o trabalho de todas as
comissões, subcomissões e da Comissão de Sistematização~revelando as
disputas, tomadas de decisões pelos constituintes, partidos políticos,
frentes e grupos partidários, através das diversas opiniões dos
parlamentares~ a respeito de temas como, por exemplo, o sistema político
a ser adotado: parlamentarismo ou presidencialismo? Há declarações de
FHC, Lula, Roberto Freire.

Freire foi ardoroso defensor do parlamentarismo. ``Acredito ser o regime
que, mesmo ante de cenários de instabilidades governamentais, não se
correria o risco de sofrer restrições nas liberdades democráticas do
povo''. O deputado José Genoíno lembra que o PT se pronunciou favorável
ao presidencialismo, visto que estava bem madura a ideia de promover a
primeira campanha de Lula à Presidência da República.

Freire revela o arrependimento por seu então partido PCB haver
``embarcado'' com o PT e o PDT na disputa pelo mandato presidencial de
quatro anos para Sarney. ``A ideia fundamental, e me arrependo junto com
Mário Covas, era termos aceito junto com Ulysses Guimarães -- e Ulysses
admitiu discutir com José Sarney -- os cinco anos para Sarney em troca
do parlamentarismo''. Freire lembra ainda que tanto o PT como o PDT
queriam tirar dois anos de Sarney e revela a incapacidade de dialogar
para chegar ao que para ele seria ``a grande negociação'': o Brasil
passar a vigorar sob o parlamentarismo e mandato de cinco anos.

David Fleischer destaca ``o poder de fogo'' do governo Sarney para
sustentar os cinco anos de mandato. As cartas nas mangas se concentravam
nas emendas parlamentares para nomear cargos e, especialmente, a
liberação de concessões de canais de rádio e TV.~``Quase todos os
constituintes do Centrão foram beneficiados com estes canais'', afirma o
cientista político.

A questão do mandato de Sarney~é descortinada por todos,~contando os
detalhes da formatação do ``Centrão'', ou ``Centro Democrático'', grupo
majoritário na~Constituinte,~formado por uma parcela dos parlamentares
do PMDB, do PFL, do PDS e do PTB, além de partidos menores.

O Centrão, apoiado pelo Poder Executivo e representantes das tendências
mais conservadoras da sociedade, conseguiu influenciar decisivamente na
regulamentação dos trabalhos da~Constituinte~e no resultado de votações
importantes, como a da duração do mandato de Sarney (estendido para
cinco anos), a questão da reforma agrária (que manteve a distribuição
desigual da terra), bem como o papel das Forças Armadas.

Fernando Henrique Cardoso e Lula lembram o cenário da disputa pelo poder
e as grandes negociações no seio da~Assembleia Nacional~Constituinte,
composta por 559 membros, sendo 487 deputados e 72 senadores.

Paulo Paim comenta que a reforma agrária trouxe muita disputa e que foi
aprovada a meio termo, num cenário bastante conturbado, com direito a
muitas tapas e ``microfones voando''.~O então líder sindical Vicentinho
lembra que houve alguns avanços trabalhistas como a definição da jornada
máxima de 44 horas semanais. ``O que foi uma conquista para os
metalúrgicos em 1986, se expandiu para toda a classe trabalhadora em
1988'', recorda ele. ``Foi um avanço'', resume Jair Meneghelli.

O livro também aborda detalhadamente o porquê de o PT haver votado
contra a redação final. O ex-presidente Lula justifica a decisão do voto
do partido. ``Neste instante a Constituição privilegiou o capital em
detrimento do trabalho, apesar dos avanços nos direitos dos
trabalhadores'', diz ele, à época líder do PT. Em seguida, recorda que
houve um embate interno do partido se assinaria ou não a Constituição e,
ao final, acabaram por assinar. ``A assinatura era silenciosa, era muito
fácil. Mas o voto contra era público'', lembra Paim.

A íntegra de todas as entrevistas que compõem o filme ``Constituinte:
1987-1988'' (Fernando Henrique Cardoso, Luiz Inácio Lula da Silva, Marco
Maciel, Mauro Benevides, Nelson Jobim, Fernando Lyra, Paulo Paim,
Vicentinho, Cristovam Buarque, Roberto Freire, José Genoíno, Egídio
Ferreira Lima, Maurílio Ferreira Lima, Marcos Terena, Jair Meneguelli e
o cientista politico David Fleischer) foram transformadas nesta obra que
resgata momentos históricos do parlamento brasileiro.

A Constituição Brasileira inaugurou um novo período
jurídico-constitucional de restauração do Estado Democrático de Direito,
um novo reordenamento constitucional para o País. Após longos 24 anos, o
Brasil respira o novo regime democrático por meio da Constituição de
1988, que contribuiu para o aprimoramento das instituições sociais,
políticas e econômicas.

A Carta Magna representa a ampliação das liberdades e dos direitos civis
para a pessoa, o cidadão, o indivíduo, o ser humano, garantindo
conquistas para crianças, adolescentes, idosos, mulheres, negros e
índios. Houve em todo o território nacional uma grande participação
popular antes e durante o processo de elaboração da Constituição
Federal.

Por fim, lembramos o registro histórico do dia da promulgação
da~Constituinte, com as célebres palavras de Ulysses Guimarães:
``falando ao Brasil, declaro promulgado o documento da liberdade, da
fraternidade, da democracia, da justiça social do Brasil. Que Deus nos
ajude que isto se cumpra''.

\textbf{FERNANDO HENRIQUE CARDOSO}

*Sociólogo formado pela Universidade de São Paulo (USP), onde também
lecionou, Cardoso exerceu o cargo de senador entre 1983 e 1992. Eleito
pelo PMDB, fundou o PSDB (1988) em meio ao processo constituinte
(1987-88). Ocupou ainda os cargos de ministro de Estado e das Relações
Exteriores (1992-93), ministro da Fazenda (1993-94) e presidente da
República Federativa do Brasil (1995-2002).

\textbf{Presidente Fernando Henrique Cardoso, vamos mergulhar na
história e falar sobre a abertura da Assembleia Nacional Constituinte,
em 1987. No dia 1º de fevereiro é instalada a ANC. O governo do então
presidente Sarney (PMDB-MA) sofria várias críticas e a Folha de S. Paulo
noticiava: "Crise marca a abertura da Constituinte". Como foi a abertura
dos trabalhos?}

\textbf{Fernando Henrique Cardoso:} Foi muito confuso. Havia ali um
momento de muita dificuldade. Primeiro porque o governo era de
transição. Havia morrido o Tancredo Neves (PMDB-MG), que fora eleito
pelo Congresso, e o Sarney ainda não tinha muito controle sobre todos os
aspectos do governo. Segundo porque o Sarney entrara recentemente no
PMDB para ser presidente. Até então ele era do outro lado, então havia
certa desconfiança e isso refletia dentro do Congresso. Terceiro, porque
havia um espírito libertário forte, cada Constituinte queria fazer tudo.
Não queriam aceitar disciplina, um ponto de partida, um documento de
base.

Era um clima bonito, de aspiração democrática. E havia os interesses
políticos, sociais, medo dos setores econômicos do que a Constituinte
iria fazer. A Constituinte era soberana -- pelo menos era essa a
aspiração -- e os outros poderes se sentiam constrangidos, limitados
pela possibilidade de a Constituinte tomar decisões que não fossem do
agrado deles. Foi assim até que o Ulysses Guimarães (PMDB-SP), a grande
figura, firmasse a sua posição e desse um rumo. Foi muito difícil.

\textbf{O senhor foi relator do Regimento Interno. Como se deu a
arquitetura, a negociação do Regimento e sua aprovação, em meio às
disputas partidárias e ideológicas no âmbito da ANC?}

\textbf{FHC:} Na verdade a decisão de me colocar como Relator do
Regimento Interno foi do Ulysses Guimarães, que tinha todo o poder na
Câmara e no Congresso. E eu era líder do PMDB no Senado e tinha um
relacionamento fluido com ele. Eu chamei imediatamente o deputado Nelson
Jobim (PMDB-RS) para relator-substituto, porque era um jovem advogado,
muito ligado à OAB, que tinha posições muito claras sobre a questão da
Constituinte.

O Regimento Interno era algo difícil de fazer. Organizar o processo
legislativo e, de alguma maneira, pré-ordenar a Constituição. Diante
daquela aspiração de todos os parlamentares, que não aceitavam que
ninguém botasse uma decisão em cima deles, nós tínhamos que imaginar um
processo de discussão que permitisse pouco a pouco chegar a uma
Constituição.

Ao criar as comissões e dar os nomes a elas, dividiu-se a Constituição.
Cada qual cuidou de um aspecto da Constituição. Nas comissões entraram
todos os constituintes, cada um ficava pelo menos numa comissão. E nós
pegamos de Portugal a ideia de uma Comissão de Sistematização, que seria
a mais importante, pois centralizaria os trabalhos. Depois houve debates
importantes sobre a aceitação ou não das emendas populares, a existência
ou não de um líder da Constituinte. A discussão sobre o regimento levou
dias e eu levei muita crítica, muita paulada. Diziam ``o senador quer
botar disciplina aqui, limitar o que a gente pode falar''. As pessoas
não estavam acostumadas. Tem que ter um processamento. Mas graças à mão
forte do Ulysses nós aprovamos. Depois é que as diferenças ideológicas
se apresentaram com mais força.

Na composição das comissões já foi uma briga. Porque a decisão estava
nas mãos do líder da Constituinte que nós elegemos, o Mário Covas
(PMDB-SP), e só ele poderia determinar quem seria de que Comissão. A
briga política e ideológica foi muito grande.

\textbf{O senhor foi membro e relator adjunto da Comissão de
Sistematização, que tinha o poder de definir o texto básico a ser
submetido ao plenário da ANC. No processo de definição desta comissão
houve vários momentos de tensão. Quais foram os mais complicados?}

\textbf{FHC:} O presidente da Comissão foi o senador Affonso Arinos
(PFL-RJ), que era um homem respeitado por todos nós. O vice-presidente
era o Aluízio Campos (PMDB-PB). Os dois eram pessoas de idade avançada.
Não tinham o pulso necessário para fazer aquilo funcionar. E as brigas
estavam começando. Eu fiz um discurso forte dizendo: "não dá".

Os que jogavam contra a Constituinte usavam, para criticar, o fato de
que a comissão não avançava. Em função disso houve uma intervenção e o
senador Jarbas Passarinho (PSD-PA) e eu fomos eleitos vice-presidentes
executivos da comissão. Nós tocamos a Comissão de Sistematização. A
disputa principal foi entre parlamentarismo ou presidencialismo. O
Affonso Arinos era parlamentarista, convenceu a muitos de nós, inclusive
a mim, e o parlamentarismo ganhou na Comissão de Sistematização.

Eu acho que o presidente da República, o Sarney, no fundo era
parlamentarista, mas tinha um pouco de medo do que poderia acontecer,
então não aceitou o parlamentarismo. O relator-geral da Constituinte foi
o deputado Bernardo Cabral (PMDB-AM), um homem que variava de posição
sobre o parlamentarismo. No texto que apresentamos ganhou o
parlamentarismo. E depois, quando o Cabral apresentou o texto dele,
também foi parlamentarista. Isso levou a uma crise muito séria e foi
preciso uma intervenção do presidente da República.

Mas não foi só isso. Houve briga no que diz respeito ao papel das Forças
Armadas. A Comissão Affonso Arinos tinha uma proposta que dizia que as
Forças Armadas tinham um papel no que dizia respeito apenas as defesas
externas do Brasil. E não podia haver qualquer referência à política
interna, porque temiam ações intervencionistas. Eu apresentei a proposta
da Comissão Affonso Arinos na Comissão de Defesa Nacional, mas perdemos.

Depois, na Comissão de Sistematização, eu disse que nós havíamos perdido
e, além do mais, não era muito sensato, porque em todas as eleições nós
chamávamos o Exército e não poderíamos mais chamar? Então ficou uma
coisa híbrida. O presidente do Congresso e o presidente do Supremo
Tribunal Federal podem chamar o Exército Nacional para assuntos
específicos. Hoje as Forças Armadas estão aí e não intervêm na política,
mas havia esse temor na época.

E havia vários problemas de natureza econômica e social. Tudo isso
provocou brigas na Comissão de Sistematização. Eu era ao mesmo tempo
vice-presidente da Comissão de Sistematização, relator-adjunto do
Bernardo Cabral e líder do PMDB. Então eu tinha muitas
responsabilidades.

\textbf{Na Comissão de Sistematização formaram-se vários grupos
suprapartidários. Grupo centrista com cinco partidos e 32 parlamentares
-- entre eles os deputados} Roberto Cardoso (PMDB-SP), \textbf{José
Lourenço (PFL-BA), Fernando Lyra (PMDB-PE), Carlos Santana (PMDB-BA).
Havia o grupo do consenso esquerda positiva, do qual o senhor fez parte
com o deputado Euclides Scalco (PMDB-PR). No grupo conservador havia o
Israel Pinheiro (PMDB-MG). Como se dava a engenharia desses grupos? Por
que a questão partidária ficou de lado e os grupos eclodiram no seio da
ANC e as discussões deixaram de ser entre partidos, mas entre grupos? }

\textbf{FHC:} Acontece o seguinte, o grande partido era o PMDB, ele era
majoritário. Mas ele não tinha coesão interna. Então esses grupos
representam tentativas de fazer funcionar independentemente dos
partidos. O José Richa (PMDB-PR) era muito importante, porque fazia a
ponte entre a esquerda construtiva e os mais conservadores. O Mário
Covas que era a figura principal na Constituinte, o Luiz Henrique
(PMDB-SC), eu, o Nelson Jobim, o Miro Teixeira (PMDB-RJ) e vários outros
fazíamos uma mediação. O tempo todo.

Se um grupo do setor muito conservador se colocava contra a redução da
jornada de trabalho, e a esquerda mais vocal, mais ofensiva, dizia
``vamos reduzir para trinta e quatro'', saia trinta e oito. Mesmo em
questões mais delicadas como a reforma agrária. Nós fazíamos o que dava
para fazer avançando. E, no fundo, o Ulysses dava respaldo a esse setor
que fazia avançar. Foi assim que funcionou a Constituinte.

\textbf{Com todas as disputas, o resultado da Comissão de Sistematização
foi um resultado progressista, inclusive aprovando o parlamentarismo.
Mas o Centrão, antevendo o que estava prestes a ocorrer na primeira
sessão simultânea da Comissão de Sistematização, e do Plenário da ANC
promove a virada regimental com o argumento de que a maioria não poderia
ficar à mercê das decisões de apenas 93 Membros dos líderes O que, de
fato, aconteceu? Aquilo foi um balde de água fria em todo o trabalho
construído em praticamente um ano.}

\textbf{FHC:} O que aconteceu foi uma disputa de poder. O presidente
Sarney e seus mais próximos aliados ficaram temerosos, no fim do
mandato, do parlamentarismo. Então o Centrão foi criado, e aí, ao redor
disso, se colocaram as forças econômicas. Nessa época os setores
empresariais eram muito ativos e tinham muito medo de avanços nos
direitos de propriedade, e a reforma agrária foi muito discutida sobre
esse ângulo. Então eles se organizaram ao mesmo tempo para dar
sustentação ao presidente da República e evitar que houvesse avanços que
eles consideravam excessivos. Então fizeram um bloco que de fato ganhou
no primeiro round.

Eles mudaram a forma de votar de tal maneira que o plenário podia anular
mais facilmente o que tinha sido decidido pela Comissão de
Sistematização. Mas foi uma vitória de Pirro, porque depois que eles
ganharam essa eleição, perderam todas as demais.

Analisando o conteúdo vemos que eles não conseguiram mexer muito no que
havia sido feito. Nunca alguém analisou a Constituinte sob esse aspecto.
Houve uma vitória política do pessoal do governo, que estava ao redor de
um mandato. Vitória do temor das forças econômicas, do temor de uma
virada que não fosse conservadora, que fosse progressista. E ganharam.
Mas depois não. No embate diário nós ganhamos quase tudo. Mas perdemos
nas questões das eleições. Na reforma agrária perdemos relativamente,
porque a reforma foi sendo feita depois.

\textbf{A movimentação do Centrão foi fundamental para o governo
garantir a vitória do mandato do presidente Sarney de cinco anos e o
presidencialismo?}

\textbf{FHC:} O que se dizia na época é que foi por pressão, envolvendo
inclusive doações de canais de televisão, rádios. É difícil comprovar,
mas o que se dizia abertamente era isso, que havia uma organização muito
grande do governo para puxar para o lado dele. E aí esqueceu-se de PMDB,
PFL, o que fosse. Ficou governo e não-governo, mas só mesmo nessa
questão do mandato do presidente. No resto a Constituição tem um espaço
bom de avanço.

\textbf{O senhor afirmou, em 25 de março, após vitória do mandato de
Sarney: ``voltamos à situação de pré-Nova República. Foi um retrocesso
mesmo. Estamos sem alternativas senão recomeçar a luta''.}

\textbf{FHC:} No começo houve muita tensão, até do presidente Sarney
comigo, porque no Regimento Interno pusemos uma cláusula que dizia que a
Constituinte era soberana e tinha poder de bloquear decisões que
contrariassem seu espírito. Disseram que havíamos feito alguma coisa
contra o Executivo. Não esqueça que antes da Constituinte houve uma
discussão enorme sobre se faríamos a Constituinte pura ou uma eleição
normal. Então era difícil acomodar tudo isso.

As forças mais conservadoras foram se reagrupando. Quando eu dei essa
declaração foi em função disso. Deram um passo para trás. Mas a verdade
é que nós fomos lutando e voltando, pouco a pouco, a ganhar na batalha
diária no Congresso. Voltamos a ganhar a maior parte dos pontos que
foram negociados.

\textbf{E a partir desse momento da vitória do mandato de Sarney, do
sistema de governo, o senhor afirmou: ``Nós não estamos dispostos a
respaldar mais o governo Sarney. Até agora houve uma situação de
ambiguidade. Agora é uma situação de ruptura. Vamos assumir uma posição
crítica e de independência''.}

\textbf{FHC:} É. Isso resultou no PSDB. Foi o começo do que terminou na
criação do PSDB. Não dava para apoiar um governo que não aceitava uma
decisão democrática. É claro que naquele embate nós estávamos todos
exagerando. O Sarney foi bastante democrático. Ele queria um mandato,
achava que tinha direito aos seis anos. O Tancredo foi posto por seis
anos. Então ele acha que abriu mão de um ano, para cinco.

Mas nós achávamos que não, que ele havia imposto pela vontade
presidencial o não ao parlamentarismo. Então realmente ficamos todos
muito frustrados e muito irritados. Mas recomeçamos e, no final, a
Constituição saiu boa.

\textbf{Foi aí que começou a surgir a ideia de formar um novo partido, o
PSDB?}

\textbf{FHC:} A questão central, que mais nos irritou, foi a da reforma
agrária, porque era um programa do PMDB. E na hora da votação as pessoas
não acompanharam o programa. É verdade que acabou ficando uma coisa
razoável. Está posto lá que o direito de propriedade não é ilimitado,
que tem que cumprir a função social da propriedade. Também foram postas
certas garantias para evitar a arbitrariedade na confiscação de terras,
ficou equilibrado. Mas, no começo, deu a impressão de que estavam
traindo o programa do partido.

\textbf{O clima entre o governo Sarney e a ANC, que já era complicado,
tornou-se pior. O presidente Sarney, em junho e em julho de 1988 fez
vários ataques ao Congresso Nacional. E o Ulysses respondeu. Essa
relação política do governo ficou a partir daí permanentemente
conturbada?}

\textbf{FHC:} Havia razões programáticas, porque nós achávamos que as
coisas não estavam indo para o lado do programa do PMDB. E havia razões
políticas, sobre quem iria mandar, porque contávamos com vários
potenciais candidatos a presidente da República. Havia o Ulysses,
obviamente, o Mário Covas, que estava surgindo como um nome possível.
Então as questões políticas começam também a se complicar. E o Sarney
também tinha muita dificuldade, porque se elegeu indiretamente e não foi
muito natural. Ele era o presidente do PDS, o partido que sustentava o
regime militar. Quando Ulysses fez o pedido para as Diretas Já ele foi
para à tribuna do Senado dizer ``não''. Então para nós era complicada
aquela situação. E o Sarney não tinha a maioria. O Tancredo me designou
para ser líder no Congresso. E quando o Tancredo morreu , logo depois
que o Sarney tomou posse, eu disse que não havia mais sentido, e ele me
disse ``pelo amor de Deus, não vá embora, ou acreditarão que estou aqui
para corromper o monte''.

Fiquei um ano como líder do governo. Ao mesmo tempo havia tentativas de
me tirar, de botar outro líder, mas nunca tiveram força para fazer isso,
só no finalzinho quando se botou outro líder. Mas o Sarney, naquele
momento, teve que se afirmar. E o Ulysses, então presidente da Câmara,
convocou um jantar na casa dele, convocando todos os ministros que eram
do PMDB, ou seja: todos, menos o Francisco Dornelles (PFL-RJ). Era uma
tentativa de mostrar que o poder estava ali. Então houve uma disputa
permanente de quem detinha o poder: se era o Ulysses, que era o chefe do
partido e tinha o Congresso, ou se era o Sarney, o presidente da
República. No meio daquilo tudo, o Sarney avançou com a democracia. Uma
coisa que eu sempre digo: o Sarney foi o primeiro que convidou o João
Amazonas, do PCdoB, para ir tomar um café lá no Palácio da Alvorada.
Quer dizer, o Sarney teve uma posição respeitável. Na briga não, nós
queríamos mais. Mas ele foi um presidente democrático.

Havia ainda a pressão militar por trás. A pressão militar só acabou em
meu governo e no governo Itamar foi diminuindo. Mas no tempo do Sarney
nós tínhamos essa preocupação e, quando havia uma crise, pode ver nos
jornais, aparecia uma fotografia do Sarney ao lado dos militares para
mostrar que o poder é uma coisa mais complicada do que o Congresso.
Havia uma disputa real. Havia também candidatos, mas era uma disputa
real sobre o papel do Executivo e do Congresso. Isso era muito vivo e o
presidente Sarney não possuía muita força política, só passou a ter
depois do Centrão. Foi quando ele deu as cartas.

\textbf{Presidente, qual a importância das Emendas Populares, dessa
efervescência do povo na Assembleia? }

\textbf{FHC:} Foi muito grande. Inclusive foi uma coisa nova, porque não
só podiam fazer a emenda como também podiam defendê-la do Parlamento.
Quer dizer, alguém que não é parlamentar vai ao Parlamento, assume a
tribuna e defende uma emenda. Além disso houve uma mobilização popular
enorme. Se alguém for estudar esse período olhando o que está arquivado,
vai verificar que o Brasil todo sonhou naquele momento.

Todo mundo possuía ideias, todo mundo queria propor coisas. A sociedade
toda foi para lá. Quando eu fui senador, ainda no regime militar, no
governo Figueiredo, o Senado era vazio, a Câmara era vazia. Ali começou
uma movimentação enorme e a Constituinte teve um papel fundamental
nisso. As Emendas Populares eram um elo entre o Congresso e o País. Pelo
menos as pessoas se sentiam participando.

\textbf{Acompanhando todo esse processo contra o regime militar, pela
redemocratização, a anistia, e como presidente da República, o senhor
considera que o período da Assembleia Nacional Constituinte foi o
período mais rico do Parlamento brasileiro?}

\textbf{FHC:} Eu não tenho dúvidas. Mesmo constatando que no meu período
de presidente houve muita riqueza de planos, reformas, discussões e tudo
mais, aquele foi o momento crucial do Brasil. O Brasil apontou para
outro rumo na Constituinte. O que aconteceu na década anterior, na luta
contra o regime militar, greves, Diretas Já, criação de muitas
organizações da sociedade civil, tudo aquilo se consubstanciou na
Constituinte. A Constituição nova abriu espaço para a participação
efetiva. Por exemplo: era difícil apelar ao Supremo para várias medidas.
Nós demos acesso a vários sindicatos e a partidos, que podem contestar a
legalidade das leis. Separamos a Defensoria da União e a Consultoria da
República foi dividida: a Advocacia da União para um lado e a
Procuradoria para outro. O Ministério Público passou a zelar pela lei em
nome da sociedade e não do Poder Executivo. São avanços muito
importantes. Eu acho que a Constituição é um marco.

\textbf{Sobre a criação do PSDB, o senhor foi, junto com o Mário Covas,
o grande ideólogo de criação da sigla. O senhor imaginava, naquele
momento, que seria presidente do País?}

\textbf{FHC:} Não. Nem de longe. Naquele momento eu era líder do Senado
e comecei a dar declarações que era preciso criar outro partido. O Covas
era Líder Constituinte. Eu fui muito vocal, falava muito, falava muito
de um partido mais coerente, mais social democrático, com ideias. Quando
eu fui me despedir do Ulysses dizendo que eu sairia do PMDB -- eu era
muito amigo do Ulysses, gostava imensamente dele e o respeitava -- ele
disse: ``mas você também? Você podia ser presidente do Senado''. Eu
disse: ``mas eu já podia antes. Eu não quis e não quero. Realmente eu
cansei. Não estou saindo por razões pessoais, mas por razão de
posição''. Ele não gostou muito, mas não tinha o que fazer.

Acreditávamos que devíamos criar um partido que não fosse uma geleia
geral. Derrubado o regime militar, nós precisaríamos de um partido com
mais coerência. O Mário foi muito importante, porque não era formulador,
mas tinha mais pressão política. Ele hesitou, porque o grupo dos
``autênticos''do PMDB, que já tinha outro nome àquela altura, não queria
que o Mário saísse, mas que fosse candidato à Presidência da República
pelo PMDB. Mas a maioria não apoiaria o Mário, então pressionaram. O
Mário no último momento decidiu pelo PSDB. Ele tinha visto que não havia
mais saída.

Sem o Mário nós não teríamos feito o PSDB, porque nós precisávamos de um
nome para ser candidato à Presidência da República e ele teve uma boa
votação. 14,15\% é muito. Chegou perto do Brizola e do Lula. Eu nem
sonhava com a Presidência da República. E eu queria era isso, um partido
que organizasse o Brasil.

\textbf{O Brasil, no golpe de 1964, entrou num processo de luta pela
reabertura política. Com o MDB e a eleição de vários senadores, a luta
pela Anistia foi mais forte. E ela veio, enfim. Depois, tivemos a
eleição do Tancredo, o governo Sarney, a primeira eleição direta em
1989, o impeachment, Itamar. Então o senhor é eleito pelo voto direto e
tem oito anos de governo. Depois o governo Lula, por mais oito anos e,
agora, a Presidenta Dilma. A democracia está avançando?}

\textbf{FHC:} A democracia é, por um lado, um processo institucional,
regras estabelecidas e a Constituição que regula. Por outro lado, é um
processo social que vai ter que ser reiterado permanentemente. Nós temos
avançado. Hoje ninguém vai discutir a ideia de que os presidentes sejam
eleitos. Não vejo pela frente risco de um golpe militar. Mas o
sentimento democrático é menos forte que as instituições. Por exemplo: o
ponto de partida de qualquer democracia é a igualdade formal perante a
lei. As pessoas são desiguais de riqueza, de cor de pele, de
inteligência, de capacidade, são desiguais em tudo. Mas perante a lei
tem que ser tudo igual, cidadão. Nós não somos.

O presidente Lula disse recentemente que ``o presidente Sarney, meu Deus
do céu... fez tanta coisa...''. Como se estivesse acima da lei. Custa na
nossa mentalidade, patrimonialista, que é hierárquica, aceitar a
igualdade perante a lei. Você não consegue botar na cadeia certo tipo de
gente porque tem posição social. Então temos que avançar muito.

Eu lembro que o Joaquim Nabuco andou na Inglaterra e nos Estados Unidos
como diplomata. Ele disse que a Inglaterra é um país onde tem monarquia,
é cheio de diferenças sociais, mas se a Justiça Inglesa chamar o mordomo
do duque e o duque, não irá distinguí-los. Essa igualdade está
assegurada. Nós ainda não asseguramos isso, mas é um processo. Pouco a
pouco vamos avançando. A política nunca está resolvida, está sempre em
processo.

\textbf{Hoje, quando o senhor olha para trás, após ser professor,
senador e presidente, acha que valeu a pena? O País tem melhorado?}

\textbf{FHC:} Eu não tenho dúvida. Tenho 81 anos e quando eu nasci, isso
aqui não se compara com o que é hoje. Mudou materialmente e mudaram
também as instituições. Quando eu nasci, uma em cada três crianças ia
para a escola. E hoje não é mais assim. Mudou para melhor.

Podem dizer: "ah, mas ninguém leva a sério o Congresso hoje". Não! Não é
bem isso, houve um processo de democratização. Podem gostar ou não de
quem esteja lá, mas não importa. Representa outras camadas. Houve uma
abertura social no Brasil. Uma mobilidade social muito intensa,
garantias pessoais... Falta muito. Tem violência, tem desigualdade como
mencionei, tem pobreza. Mas nós avançamos. Valeu a pena.

\textbf{LUIZ INÁCIO LULA DA SILVA}

*Nascido no semi-árido pernambucano, Lula inicia sua trajetória política
no Sindicato dos Metalúrgicos de São Bernardo do Campo e Diadema, estado
de São Paulo. Liderança popular na oposição ao regime ditatorial
(1964-85), fundador do PT (1980) e da CUT (1983), tornou-se o primeiro
operário a assumir a Presidência da República Federativa do Brasil
(2003-2010).

\textbf{**A entrevista com o então presidente Lula foi realizada pela
assessorial de comunicação do Governo Federal.}

\textbf{O senhor tinha 43 anos e exercia o seu primeiro cargo público
quando chegou a Constituinte. Compareceu a 95\% das sessões, o que foi
uma das médias mais elevadas de comparecimento, e aprovou sete de 41
emendas apresentadas. Vinte anos depois, olhando aquela experiência, o
senhor faria alguma coisa diferente, um voto ou uma conduta durante a
Constituinte?}

\textbf{Luiz Inácio Lula da Silva:} Primeiro é importante lembrar que o
PT possuía, na época, apenas 16 deputados e que nós chegamos na
Constituinte com uma proposta de Constituição pronta e acabada, que se
tivesse sido aprovada tal como eu queria, certamente seria muito mais
difícil de governar o País do que é hoje.

Nós chegamos também com uma proposta de regimento interno pronta.
Acontece que nós só possuíamos 16 deputados, mas fazíamos barulho como
se tivéssemos 50 ou 60. Mas a chance de aprovar todas as coisas que nós
queríamos estava muito distante. Fato concreto é que o PT não aprovou a
Constituição. O PT votou contra a Constituição e há uma confusão que de
vez em quando dizem que o PT não assinou. O PT assinou.

\textbf{Como foi essa discussão na qual vocês resolveram assinar de
última hora? }

\textbf{Lula:} Nós não concordávamos com a questão dos direitos sociais,
porque embora houvesse um avanço extraordinário, a verdade é que tudo
ficou para ser regulamentado pela Lei Ordinária e isso nós já havíamos
visto na Constituição de 1946, na qual o que ficou de ser regulamentado
teve muita dificuldade de ser regulamentado e, ao mesmo tempo, achávamos
que poderíamos ter regulamentado algo.

Como a maioria não quis, nós nos sentimos obrigados a votar contra. Mas
na hora de votar houve uma discussão dentro do PT: ``assina ou não
assina?''. Eu era o líder da bancada e disse: ``nós temos que assinar.
Nós participamos, trabalhamos aqui, votamos, ganhamos e perdemos, dirá a
quantidade de horas de reunião que tivemos aqui, agora temos que deixar
nosso nome na história e assinar''. E assinamos a Constituição.

O que é que eu penso da Constituição: eu penso que nos fizemos uma
Constituição extremamente avançada, uma Constituição que foi menos
sabedoria dos constituintes e mais sabedoria popular, como jamais houve
na história desse País. Eu lembro das inúmeras pessoas que circulavam
por dentro da Câmara do Congresso Nacional, fazendo reunião com todos os
líderes, fazendo e pressão. E nós conseguimos retratar na Constituição
um pouco da cara do que a sociedade pensava naquele momento, sobretudo a
sociedade organizada. E acho que isso foi extremamente importante para o
País, porque ela está hoje balizando e garantindo que nós tenhamos o
maior período de democracia contínua no Brasil.

\textbf{O senhor não acha que a Constituição de 1988 envelheceu muito
rapidamente? Ela tinha apenas 20 anos e mais de 50 emendas.}

\textbf{Lula: P}enso que a Constituição sofreu algumas mudanças que eu,
particularmente, não tomaria a iniciativa de fazer nunca. No capítulo de
ordem econômica, por exemplo, aquela mudança de capital nacional e
capital internacional, a questão do monopólio da Petrobras, que foi
quebrado, eu não faria isso. Entretanto, acho que a Constituição sofreu
modificações que as exigências políticas exigiram até agora. A década de
1990 foi um período em que a queda do Muro de Berlim e o Consenso de
Washington fez com que os políticos daquele instante achassem que ser
moderno era abandonar até coisas que interessavam a soberania nacional.
Hoje isso já mudou um pouco. Mas penso que as mudanças que aconteceram
se tornam pequenas diante da grandiosidade da nossa Constituição. Hoje
eu diria o que disse Ulysses Guimarães (PMDB-SP) num pronunciamento às
três horas da manhã no dia 5 de outubro: ``é a Constituição Cidadã''.

\textbf{Então o senhor acha que embora os setores conservadores do País,
a partir da emergência do Centrão, tenham obtido importantes vitórias,
como na questão da terra, ao colocar na balança o senhor acha que houve
mais avanço que conservantismos?}

\textbf{Lula:} É importante lembrar por que que surgiu o Centrão. Ele
surgiu porque nós estávamos muito articulados. Fazíamos muitas reuniões
com o Mário Covas, com o Bernardo Cabral e eu penso que os setores
progressistas e os movimentos sociais foram conquistando muitas coisas.
De repente, como a gente queria cada vez mais, eu acho que os setores
mais conservadores se organizaram e constituíram o Centrão, passando
outra vez a dar a tônica do que viria a ser o relatório final da
Constituição. Mas de qualquer forma nós já havíamos avançado um pouco e
acho que não retrocedemos tanto na questão dos direitos sociais.

\textbf{O senhor disse que não mudaria a questão do monopólio do
petróleo ou o conceito da empresa nacional, por exemplo. O que o senhor
acha que não deu certo na Constituição?}

\textbf{Lula:} Se tivesse que dizer hoje o que mudar na Constituição, eu
diria que nós temos que ter uma Constituição mais presidencialista, já
que o regime é presidencialista e ela é muito parlamentarista.

\textbf{O senhor votou contra o mandato de cinco anos. Já nas eleições,
afirmou que quatro eram muito pouco. Esse era um voto que o senhor teria
dado diferente?}

\textbf{Lula:} Eu diria diferente. Se você perguntar para mim qual é o
modelo de mandato, o tempo de mandato, eu diria que poderíamos ter cinco
ou seis anos sem reeleição. Entretanto, hoje, com a experiência que eu
tenho, posso dizer que quatro anos é muito pouco. Num país que tem
eleições a cada dois anos você tem muito pouco tempo para cumprir um
programa de governo. Num país que você tem uma estrutura de fiscalização
que não existia na década de 1950, na década de 1960. Hoje você tem um
Tribunal de Contas e um Ministério Público muito mais fortes, muito mais
atuantes. Tem as questões ambientalistas muito mais fortes e exigentes.
Uma coisa que o presidente decidia e fazia, hoje ele decide, manda para
o Congresso, depois tem o julgamento do Tribunal de Contas, tem o
Ministério Público. Quando parece estar tudo resolvido, tem uma ação
popular, tem o Poder Judiciário.

Esta é a grandeza da democracia brasileira, é tudo mais demorado, mas
quando as coisas acontecem, acontecem de verdade. Por isso é que eu acho
que a gente não tem que temer essas dificuldades que a gente enfrenta.
Isso é um processo de construção democrática. E a gente só aprende a
fazer democracia vivenciando ela todo o santo dia, enfrentando
obstáculos, vencendo alguns, perdendo outros. Eu aprendi muito com as
três derrotas que tive para a vaga de presidente da República. Muito. Eu
fico imaginando se tivesse chegado à Presidência em 1989 com um partido
inexperiente. Em vez de programas de governo, a gente muitas vezes fazia
uma pauta de reivindicação, como se nós nunca fôssemos chegar ao
governo, sabe? Todo mundo com muita vontade, mas todo mundo muito novo,
muito inexperiente. De lá para cá, nas três derrotas, nós elegemos
prefeitos, governadores. Aprendemos. Em 2002 eu cheguei infinitamente
mais preparado, muito mais calejado do que eu chegaria em 1989.

\textbf{O senhor foi Constituinte ao lado do Covas (PMDB-SP), José Serra
(PMDB-SP), FHC (PMDB-SP). Na época, a democratização da política
brasileira era de partidos progressistas à esquerda e conservadores à
direita. Isso foi uma evolução para o País?}

\textbf{Lula:} Naquele tempo eu estava mais na esquerda. O Mário Covas,
o Fernando Henrique Cardoso, o Serra, quando deixaram o PMDB, estavam
mais ao centro do que eu tinha à direita. Obviamente que há evolução.
Uns evoluem para melhor, outros não evoluem. Esse é o mundo em que
vivemos e é bom que seja assim. Hoje estou mais maduro e com a
responsabilidade de governar o País. Embora continue tendo a mesma
vontade que tinha antes, a pessoa, na hora de fazer as coisas, tem que
medir a correlação de força política, a situação econômica do País, as
possibilidades do País. Amadurece, quer queira, quer não. Na Presidência
da República se você tiver compromissos de origem e não quiser traí-los,
consegue ficar maduro e fazer as coisas acontecerem como estão
acontecendo hoje no Brasil.

\textbf{Então eles estão mesmo mais maduros?}

\textbf{Lula:} Eu acho que todo mundo está mais maduro. Sobretudo quando
a gente completa 60 anos: cada ano vale cinco daqui para frente, então
você vai ficando mais maduro, sabe? Eu lembro de uma discussão eu,
Genoíno (PT-SP) e Plínio de Arruda Sampaio (PT-SP). Fomos conversar com
Nelson Carneiro (PMDB-RJ) sobre se entrava Jesus Cristo na Constituição
ou não. Se entrava Deus na Constituição. E o Genoíno dizia: ``Não. Nós
temos que defender um estado laico, a Constituição tem que ser laica''.
E lá fomos nós conversar com o Nelson Carneiro. Eu, o Plínio e o
Genoíno. O Plínio era a favor, o Genoíno era contra e eu era o mediador.
Quando chego lá o Nelson Carneiro olhou para as nossas caras e disse:
"escuta aqui, vocês acham que com a idade que eu tenho eu vou brigar com
Deus? Eu vou colocar Deus e acabou!".

\textbf{O senhor vinha da vitória da linha sindicalista, era seu
primeiro cargo eletivo e foi ali que o senhor conheceu a elite política.
Com quem o senhor mais aprendeu na Constituinte?}

\textbf{Lula:} Primeiro eu queria lembrar: sabe que eu nunca tive
vontade de ser deputado?! Nunca quis ser deputado. Eu não queria. Só
queria ser Constituinte. Eu não queria ser deputado sabe por quê? Porque
para quem vem de fazer assembleia na porta de fábrica com 20, 40, 80 mil
trabalhadores, viajar esse País falando com trabalhadores rurais,
falando com sindicalistas na porta de fábrica... vir para dentro do
Congresso é quase que colocar uma mordaça na boca! Porque você perde a
essência da tua política que é de falar com os trabalhadores e as
massas. Você tinha que fazer discurso no Plenário, muitas vezes com o
presidente da mesa cochilando.

Lá no Plenário havia 50 deputados, cada um numa reunião dentro do grupo,
cada um num grupo de 10 e um ou outro amigo ainda prestava atenção no
que você estava falando. Só no auge das grandes votações é que fazia
algum sentido você falar, mas também era um jogo de cartas marcadas.
Porque quando se construía a maioria você já sabia que a maioria iria
ter "tantos votos", nós vamos ter "tantos votos". Muitas vezes nós
tínhamos menos, porque nós dividíamos. Lembro que numa votação quando se
foi discutir uma questão que envolvia os negros e eu lembro do discurso
que a Benedita da Silva (PT-RJ) fez...

\textbf{Foi a proibição de relações diplomáticas com Países que
praticassem a discriminação racial?}

\textbf{Lula:} O discurso da Bené foi uma coisa exuberante, de muita
gente chorar em Plenário. Se tivéssemos colocado em votação naquele
momento, certamente nós teríamos ganho. Mas como havia outros deputados
negros e que vieram falar, se inscreve um, se inscreve outro, o que
aconteceu? Esfriou o ânimo dos Constituintes e, quando se colocou em
votação, nós perdemos. Lembro perfeitamente do discurso do Alceni Guerra
(PFL-PR) sobre a questão da licença-paternidade. Era algo que,
teoricamente, cinco minutos, antes não passaria. Mas ele fez um discurso
brilhante e nós aprovamos o auxílio-paternidade. Então teve essa coisa
maravilhosa. Ulysses Guimarães (PMDB-SP), por exemplo. Às vezes falo que
todos os presidentes da Câmara deveriam ser igual ao Ulysses Guimarães,
porque o Ulysses às vezes enrolava a gente uma semana, quatro dias... A
gente ia lá e nunca tinha votação. Quando ele articulava o que ele
queria votar, ele sentava naquela cadeira e a maioria aprovava as
coisas.

\textbf{Mesmo não gostando o senhor aprendeu, não foi?}

\textbf{Lula:} Aprendi porque aquilo foi uma escola extraordinária e a
convivência de adversários do mesmo espaço político... é como se você
estivesse numa sala com um monte de adversários. Mais dia, menos dia,
você começa a conversar. Então o Congresso Nacional foi uma coisa
atípica. E você se relaciona com extrema direita, com direita, com o
centro, com extrema esquerda, porque são todos cidadãos brasileiros
acreditando em coisas diferentes, mas muitas vezes objetivando construir
um país melhor e que convivem democraticamente ali.

Eu acho que aquela é uma área de onde se tiram lições de vida. Aprendi
muito. Mas confesso que não gostei. Não voltaria a ser deputado. Tanto é
que depois que eu perdi as eleições em 1989 o pessoal queria que eu
fosse candidato e recusei, também porque queria dar uma lição no PT.
Naquele tempo se dizia que só tinha força no partido quem era deputado.
E eu resolvi dizer que mesmo sem deputado a gente tem força.

\textbf{Volta e meia surge uma proposta de Revisão Constitucional,
Assembleia, de Constituinte. O senhor vê clima para isso?}

\textbf{Lula:} Eu acho que de vez em quando, quando a maioria entender
que é preciso adequar a Constituição -- afinal de contas, em 20 anos a
sociedade evoluiu muito -- você pode mudar. O que não se pode mudar são
os princípios fundamentais da Constituição, mas você fazer uma Emenda
Constitucional para melhorar, por exemplo, a questão do tributo, é uma
coisa impressionante.

Eu lembro de quantas vezes a gente brigou, de quantas guerras, de
quantas noites acordado, só para determinar que os juros reais seriam a
12\%. Hoje os juros reais estão bem menos que isso. Ao passo que se a
gente tivesse, quem sabe, brigado menos, a gente não teria perdido tanto
tempo no capítulo dos juros. Mas é que naquele momento ele era
extremamente importante. Para alguns Constituintes era quase que sagrado
e eu confesso que valeu a pena. Acho que eu vou carregar na minha
biografia o direito de ter sido deputado constituinte, de ter
participado e trabalhado como nunca na vida. Acho que o Congresso nunca
trabalhou tanto como naquele período. A gente ficava lá a semana
inteira, de segunda a domingo, às vezes ficava quinze dias sem ir para
casa, trabalhando.

Do dia 3 ao 5 de outubro de 1988 ela foi gestada de forma grandiosa. Eu
muitas vezes ficava nervoso porque a gente ficava trabalhando na
Constituição o dia inteiro, reunião às 9 horas, reunião às 11, reunião
às 3 da tarde, reunião às 5 da tarde, reunião do partido às 19h para a
gente aprovar. Chegava na bancada do PT, a briga estava mais forte,
porque as pessoas não concordavam. Quando você chegava, ligava a
televisão, o que é que estava passando? Gente que não tinha feito nada,
que nem estava na Constituinte dando palpite. Aí você falava: ``o que eu
ganhei no dia de hoje? Nada!''. Mas eu penso que foi um momento
grandioso para o Brasil. Graças a Deus eu passei por isso e estou muito
agradecido.

\textbf{Marco Maciel}

*Advogado e professor universitário, filiou-se à ARENA ainda na
juventude, Maciel iniciou sua carreira política como deputado federal
(1971-1979), presidindo a Câmara nos últimos dois anos. Foi indicado
como governador de Pernambuco (1979-1982) e senador (1983-1994), havendo
se licenciado para assumir o Ministério da Educação (1985-1986). Foi
eleito vice-presidente da República (1995-2002) e senador (2003-2010).

\textbf{Marco Maciel, a Frente Liberal liderada pelo senhor foi criada
como uma dissidência do PDS e indicou José Sarney (então no PDS,
posteriormente no PFL) como vice na chapa de Tancredo Neves (PMDB-MG). A
articulação da Frente já tinha em vista esta indicação?}

\textbf{Marco Maciel:} Na verdade o que nós fizemos foi algo mais amplo,
de grande impacto político. Nós queríamos dar um novo rumo ao processo
sucessório, porque Paulo Maluf (PDS) se lançara candidato e empolgara
grande parte do então PDS. Como atitude de reação nós organizamos um
grupo que proclamou que não compareceria à Convenção para a escolha do
Maluf. Começamos a constituir, portanto, uma dissidência que a imprensa
batizou de Frente Liberal. Isto não foi proposta nossa. Foi a imprensa
que disse: ``vocês estão fazendo um jogo de abertura'', então deram o
nome de Frente Liberal.

\textbf{Depois vocês terminaram criando o PFL.}

\textbf{Maciel:} Exatamente. Houve na sequência a conversão da Frente
Liberal no Partido da Frente Liberal. O passo mais lógico era também
fazer um entendimento com o PMDB, que se deu primeiramente no
apartamento do Sarney. Um dia eu fui à casa do Sarney e me encontrei com
Ulysses Guimarães, que era o presidente do PMDB. Então nós acertamos que
nos entenderíamos e, como consequência, deveríamos aprovar um documento
para que a sociedade soubesse por que nós, embora tivéssemos sido
adversários no passado, estávamos agora unidos.

Nós estávamos fazendo essa coligação pensando no País, em mudar o eixo
da sucessão, evitando a eleição de Paulo Maluf e criando condições para
eleger uma chapa comprometida com as mudanças que o País esperava,
inclusive a realização da Constituinte. Então, o que se estabeleceu,
logo no início, foi que a candidatura à Presidência ficaria com alguém
do PMDB. O Tancredo Neves brotou candidato naturalmente porque era tido
como uma pessoa muito conciliadora.

Ulysses era um grande líder, mas não era tão conciliador. Às vezes
radicalizava em certos pontos e algumas pessoas reagiam. Mas Tancredo
era uma certa unanimidade nacional. Ficou estabelecido ainda que a
Vice-Presidência seria indicação da Frente Liberal, que, ao final,
terminou sendo José Sarney. Ele, inclusive, era o presidente do PDS, mas
deixou a Presidência do PDS e passou a integrar a nossa coligação.

\textbf{E o que foi feito a partir da escolha dos nomes de Tancredo e
Sarney? }

\textbf{Maciel:} A partir daí nós começamos a visitar os estados para
mobilizar, porque a eleição era indireta, através de um colégio
eleitoral. Eu visitei muitos estados com o Tancredo e ele sempre dizia:
"Eu sou candidato por um processo indireto, para destruí-lo", como
querendo dizer assim: "Com minha eleição pelo colégio eleitoral, se fará
uma nova Constituição, que acabará com esse negócio de eleição
indireta". Noutro momento Tancredo me disse assim: "Olhe, Marco, Maluf
está dizendo em todo canto que tem um programa de governo e que nós não
temos". Ele tinha feito em São Paulo com uma pessoa especializada em
fazer programas de governo e era uma coletânea com uns 40 volumes. Aí
Tancredo sugeriu que nós fizéssemos um texto básico, que se intitulou
``Compromisso com a Nação'' Este foi o pacto constitutivo da chamada
Aliança Democrática, que reunia o PMDB e a Frente Liberal.

Esse documento, que foi distribuído para o País todo, inclusive em
comício, foi assinado no auditório da Câmara dos Deputados por quatro
pessoas: Ulysses e Tancredo pelo PMDB, Aureliano e eu pela Frente
Liberal. Era um texto relativamente curto, quatro laudas, se isso. Na
hora em que foi assinado eu disse: "Doutor Tancredo, eu sou o mais novo
dos membros. O senhor, o Ulysses, o Sarney são pessoas com bem mais
idade do que eu". E ele disse: "Não. Você coordenou, então você deve
assinar". Depois esse pacto foi divulgado no País todo. Foi ele que
tornou possível viabilizar a eleição de um candidato comprometido com a
abertura política, com a Constituinte, com uma série de princípios
básicos que estão elencados no ``Compromisso com a Nação'' e que tornou
possível, consequentemente, o retorno das eleições diretas, o pluralismo
partidário, a Lei da Anistiae a pluralidade sindical.

Alguém poderá perguntar: "e a censura?" A censura, na realidade, não
constava de lei. O que se estabeleceu foi que, tão logo Tancredo tomasse
posse, ele mandava cessar a atividade de censores. A bem da verdade,
aconteceu um pouco antes, porque o próprio Figueiredo achava que não
tinha razão de ter censores nos jornais. Tudo isso criou, então, um
clima para a Constituinte, que foi um marco muito decisivo para que nós
voltássemos ao chamado Estado Democrático de Direito.

\textbf{Em 1986, um ano antes da Constituinte, tivemos eleições para a
Câmara dos Deputados. Em 1985, já sabendo-se que haveria a ANC em 1987,
havia uma discussão sobre se ela deveria ser exclusiva ou congressual.
Esse debate continuou até a abertura da Assembleia. Quando o ministro
José Carlos Moreira Alves (STF) abriu, José Genoíno (PT-SP) e Roberto
Freire (PCB-PE) se pronunciaram pedindo a impugnação de 23 senadores
eleitos em 1982, entre eles, o senhor. Eles diziam que vocês não
poderiam participar porque eram senadores do período ditatorial. Foi um
debate muito acalorado, muito forte, que tomou tempo. Como o senhor via
esse debate na época?}

\textbf{Maciel:} Na época essa questão se colocou, mas logo se
esclareceu que, na verdade, a Constituinte não era exclusiva, quer
dizer, não se iria convocar eleições gerais no país para eleger um
Congresso Constituinte. Ou seja: participariam dela os deputados eleitos
ou reeleitos e os senadores que foram eleitos na legislatura anterior e
cujos mandatos permaneciam. Então não se tinha, àquela ocasião, outro
caminho a não ser referendar essa solução e dar ao parlamentar o Poder
Constituinte também, ainda que não seja o Poder Constituinte originário,
posto que era um Poder Constituinte derivado, no exercício do mandato.

Um outro caminho seria muito mais complicado e, talvez, retardasse muito
mais até que se fizesse uma eleição, etc. E, de alguma forma, já eram
pessoas que estavam ali, que tinham participado do processo sucessório
em 84, quando se fez a chamada Aliança. Então já havia muitas pessoas
que tinham um certo protagonismo nesse processo, o que permitiu fazer
com que logo se começassem os trabalhos.

Alguém pode dizer que demorou quase dois anos, mas o processo seria
muito mais longo se fosse uma Constituinte exclusiva. Se chegassem de
uma hora para outra pessoas vindas de diferentes partes do País, que não
haviam participado do processo anterior, então tudo seria uma coisa de
maior complexidade. Então o que se fez foi algo semelhante ao que
aconteceu com a Constituição de 1946.

Quando Getúlio foi deposto, em 1945, foram convocadas as eleições para a
Câmara e o Senado. De 1937 até 1945 não havia Congresso e os eleitos
passaram a ser constituintes originários. Fizeram a Constituição de
1946, que durou até 1964, e foi uma Constituição que teve conquistas,
estabilizou os partidos políticos. Foi um momento de grande afirmação
democrática no país e liberdades. Mas quero fazer agora um raciocínio:
hoje nós já vamos fazer 21 anos da Constituição de 1988. A Constituição
de 1946 não chegou a vigorar todo esse tempo, vigorou somente 18 anos, o
que prova que a experiência se consolidou para ninguém questionar mais
no Brasil que vivemos sob o chamado Estado Democrático de Direito e
consequentemente vivemos sob a égide da lei, do devido processo legal. A
sociedade logo percebeu também que esse era o melhor caminho.

\textbf{Como foi a discussão sobre o sistema de governo presidencialista
ou parlamentarista na Constituinte?}

\textbf{Maciel:} Uma pessoa que teve papel importante foi Humberto
Lucena (PMDB-PB), que era o presidente do Senado à época. Ele era
presidencialista e inclusive fez uma emenda presidencialista. Quando o
doutor Ulysses me comunicou que tomaria o caminho do parlamentarismo eu
já tinha sinalizações de que isso poderia acontecer.

\textbf{Porque Ulysses mudou de posição?}

\textbf{Maciel:} Eu acho que ele foi procurado pelo pessoal do PSDB, por
outras forças que veem com simpatia o parlamentarismo. E aí eu disse
para o doutor Ulysses: ``Vou respeitar a posição do senhor, mas vou
subscrever a Emenda Humberto Lucena e trabalhar para que ela seja
acolhida'', o que, ao final, foi o que aconteceu. Eu acho que podemos
ter um parlamentarismo no futuro no Brasil. Mas não adianta se nós não
fizermos a reforma política, se não fizermos verdadeiros partidos
políticos, com propostas claras.

No Brasil você tem muitos partidos, muitos dos quais você não é capaz de
entender o que priorizam. Eu, às vezes, digo brincando que aqui no
Brasil a gente tem maioria, minoria e ``unoria'', quando o partido só
tem um representante no Congresso. Então nós precisamos aprofundar nesse
campo, no sentido de dar um caráter programático aos partidos e para
isso tem que ter lei de fidelidade partidária também. Salvo nas questões
de consciência, se a pessoa adere a um programa, tem que votar de acordo
com esse programa.

\textbf{Sobre a questão da duração do mandato de Sarney, a Comissão de
Sistematização aprovou quatro anos e houve até comemoração. O pessoal de
Mário Covas foi para o Restaurante Piantelo. E o bloco do PFL, que o
senhor defendia havia 4 anos, foi para casa do deputado Saulo Queiroz
(PFL-MS).} \textbf{Disseram que foram comer um Pato a Zé Lourenço
(PFL-BA). O senhor e o Aureliano Chaves (ministro de Minas e Energia,
PFL) não foram, mas houve essa comemoração dos dois lados. A razão dessa
vitória era a campanha de Aureliano Chaves para presidente? O PFL já
trabalhava com essa perspectiva?}

\textbf{Maciel:} Nós pensamos na candidatura de Aureliano antes daquele
entendimento com Tancredo, àquela época. No PDS havia mais quatro
propostas. Havia o Hélio Beltrão (ex-presidente da Petrobras), o Mário
Andreazza (ex-ministro dos Transportes). Este último foi até o fim, foi
à Convenção com Maluf. É lógico que Maluf derrotou porque era um
político mais hábil e mais tarimbado. Andreazza nunca tinha disputado
uma eleição, era um coronel da reserva, não tinha experiência política.

O meu nome também foi lançado e cheguei a fazer algumas viagens porque
muita gente estimulava. Em alguns lugares até a recepção era muito
grande. Então o Aureliano surgia como o nome mais forte e, eu,
inclusive, cheguei a fazer umas viagens também com o Aureliano Chaves.
Passado esse episódio, se pensava que o partido pudesse retomar a
candidatura de Aureliano Chaves, mas aí vieram outras questões e o
próprio Aureliano foi ser ministro de Minas e Energia e, no final,
terminou com uma votação muito pouco significativa porque haveria uma
grande dispersão de candidaturas, em função de um grande número de
partidos. Depois não quis mais disputar nada. Praticamente encerrou a
vida pública dele.

\textbf{Quando aconteceu a vitória, no plenário, dos 5 anos para Sarney,
saiu uma matéria no jornal Folha de S. Paulo dizendo que a vitória
esmagadora dos cincoanistas confirmava que quem manda realmente no
partido é o ministro das Comunicações, Antônio Carlos Magalhães
(PFL-BA), responsável pela distribuição de concessões de rádio e TV, e
não o presidente nacional do partido. Quem realmente mandava no PFL?
Antônio Carlos ou Marco Maciel? }

\textbf{Maciel:} Nesse processo o Antônio Carlos apoiara Andreazza. Ele
foi para Convenção, diferente de nós que fizemos a chamada Frente
Liberal. Antônio Carlos só passou para o PFL mais de um ano depois. Eu
acho que foi em dezembro de 1985 que ele me procurou. Eu fui com ele ao
presidente Sarney, num sábado à tarde, no Palácio da Alvorada.
Reunimo-nos ali naqueles jardins, ali atrás e conversamos um pouco. Foi
quando ACM disse: ``estou achando que não devo continuar no PDS, eu
gostaria de passar a integrar a aliança que apoiava Sarney" e que era o
PFL. A partir daí ele veio para as nossas hostes.

\textbf{Eles já não estavam apoiando o Tancredo? }

\textbf{Maciel:} É verdade, mas ele não saira do PDS. Tancredo nos
comícios disse assim: ``Estou aqui com os meus colegas, meus
companheiros do PMDB, com meus companheiros do PFL e os dissidentes do
PDS. "Os dissidentes'' do PDS, no caso, era Antônio Carlos. Ele não foi
para o PFL no primeiro momento. No entendimento que ele fez com
Tancredo, Tancredo o nomeou ministro das Comunicações. Quer dizer, o
nomeou, mas na realidade, nós temos no decreto de nomeação de ministro
duas cópias: uma assinada por Tancredo e outra por Sarney. Como Tancredo
não chegou a tomar posse, eu tinha o Decreto já assinado de véspera,
porque era de praxe, no cerimonial, que os decretos fossem assinados na
véspera, e não no dia da posse. Tancredo o havia nomeado ministro das
Comunicações e, a partir daí, no fim do ano ele vem finalmente para o
Partido da Frente Liberal, onde ficou até falecer.

\textbf{Em julho de 1988, o presidente Sarney faz um pronunciamento em
rede nacional de televisão e diz que a Constituição deixaria o país
"ingovernável". Preocupava o governo o tema da Reforma Tributária,
porque tinha o processo da descentralização dos recursos e impunha
perdas à União. No outro dia Ulysses Guimarães vai à televisão e procura
desmentir Sarney. Como o senhor viu esse momento crucial? }

\textbf{Maciel:} Esse ponto de vista era exposto pelo então ministro da
Fazenda, Maílson da Nóbrega. Ele achava que a Constituinte estava
fazendo muitas concessões, atribuindo muitos direitos e poucos deveres e
que, uma vez promulgada, poderia levar o País a um retrocesso, a uma
ingovernabilidade. E, também, a conflitos federativos. Cada vez mais há
uma concentração de poderes na União, em detrimento dos Estados e
Municípios. Uma Federação geralmente é uma associação de Estados e de um
Distrito Federal. No Brasil nós avançamos, além de consideramos que a
Federação é constituída pela União, pelo Distrito Federal, pelos
Estados, incluímos também os municípios. Mas em que pese em ter havido
essa enorme descentralização, os mecanismos não expressam essa intenção
do Legislador Constituinte. E daí muitos prefeitos vão ao meu gabinete
ou ao gabinete de qualquer outro senador ou deputado reivindicando
coisas porque eles acham que, da forma como é o centralismo tributário,
os governantes estaduais e sobretudo os municipais não tem condições de
executar os seus programas de trabalho.

Então, vem a questão mais uma vez referente à Reforma Política: a gente
precisa fortalecer a Federação e as instituições republicanas que, a meu
ver, estão em crise também. Isso não é novidade.

\textbf{Falando um pouco da questão do Centrão e da virada regimental
que ajudou o grupo em torno do PFL a aprovar os cinco anos e zerou tudo,
como se deu a formação desse bloco e qual o papel de Ricardo Fiúza
(PFL-PE) nessa articulação? }

\textbf{Maciel:} Ele tem um papel muito importante, como também o de
Luís Eduardo Magalhães. Um dia, fim de tarde, estava no meu gabinete e
Luís Eduardo me telefona. Disse: ``Olha, eu estou querendo lhe falar um
assunto e eu não queria tomar uma decisão sem antes lhe falar. Será que
você poderia me receber?''. Eu disse: ``Passe aqui''. Ele levou mais
umas duas ou outras três pessoas, e expuseram a história da criação do
Centrão. No Parlamento sempre a proporcionalidade é que pesa, então
ficou claro para eles que se organizassem o Centrão, teriam um peso
maior nas deliberações finais da Constituinte porque eles entendiam que
estavam sendo ultrapassados por grupos que eram minoritários e que, ao
final, estavam tendo uma participação maior do que o cacife que
dispunham. Aí foi criado o Centrão.

Inclusive eu assinei o documento do Centrão. Não naquele momento, porque
era um momento inicial, mas depois eu assinei também o documento.

\textbf{A estratégia deu certo?}

\textbf{Maciel:} Alterou-se um pouco o eixo, porque se dizia que o eixo
estava muito à esquerda. Se bem que uma vez uma jornalista me procurou,
isso já no fim da Constituinte, e eu disse que o problema não estava
entre esquerda e direita, mas em quem se organizou e quem não se
organizou. Quem se organizou levou a melhor. Quem não se organizou,
perdeu. Os banqueiros perderam porque não se organizaram. O pessoal de
Ronaldo Caiado, que naquela época presidia a União Democrática
Ruralista, ganhou as paradas por quê? Porque estavam organizados.
Fizeram uma votação. Botaram a tropa no Plenário, conseguiram fazer uns
acordos políticos e conseguiram a maioria. Então, na verdade, o Centrão
teve esse papel -- em linguajar do judiciário -- de chamar o feito à
ordem, discutir como é que o jogo seria feito. Então, sob esse aspecto o
centrão teve um sentido positivo, mesmo porque não alterou basicamente
muita coisa.

Muita coisa já havia sido votada. O Centrão ordenou um pouco o processo.
Vou fazer uma afirmação que não tem nenhuma fundamentação empírica: eu
diria que o Centrão, talvez sem querer, contribuiu para terminar mais
cedo a Constituinte, porque as grandes questões foram logo elucidadas. O
Centrão provocou um certo tremor de terra, um certo terremoto no
Congresso que ajudou, talvez, a que as coisas começassem a ter uma
definição.

\textbf{O senhor disse que boa parte da Constituição estava aprovada
neste momento. O que o senhor pensa sobre o texto constitucional?}

\textbf{Maciel:} Eu quero fazer uma observação de natureza formal e não
material: A Constituinte terminou produzindo um texto muito longo. Os
constitucionalistas geralmente dizem que as constituições podem ser
divididas entre sintéticas e analíticas. Vou dar um exemplo de uma
sintética: Eu citei a Constituição de 1891, a primeira Constituição
Republicana, foi uma Constituição sintética, só tinha 91 artigos e 8 ou
9 disposições transitórias, então era uma Constituição mais ou menos
concisa. A mesma coisa eu poderia dizer com relação à Constituição do
Império, que foi uma Constituição outorgada, mas de toda maneira não era
uma Constituição longa e acolhia um poder novo, que era o chamado Poder
Moderador. A Constituição de 88, ao final, terminou sendo uma
Constituição analítica. Ela incorporou no seu texto matérias que não são
constitucionais.

\textbf{Isso ocorreu em função do processo ditatorial?}

\textbf{Maciel:} Em função da abertura, todo mundo queria colocar na
Constituição coisas que achavam relevantes. Era como se pensassem assim:
``Agora temos liberdade, agora podemos colocar tudo''. Muita coisa não
era para estar em Constituição. Eu citei o exemplo agora dos 8\% de
juros. Aquilo não era para matéria de Constituição, isso é matéria para
o Banco Central. No máximo uma lei ordinária, mas nunca estar na
Constituição, porque se há uma alteração de conjuntura econômica, você
vai ter que reformar a Constituição, então se gera uma certa
ingovernabilidade e terminou a Constituição sendo enorme. Se você fizer
uma decantação, vai observar que tem mais de 1.200 comandos e então é
muito difícil você decorar a Constituição brasileira. Por exemplo, em
alguns países da Europa, que tem constituições mais concisas, dá quase
para decorar. A nossa é impossível decorar e já tem 63 emendas. Para
vinte anos de Constituição é muita emenda. Pense bem: e quando fizer 100
anos? A Constituição americana é de 1787, só tem 26 emendas.

\textbf{Para alguns especialistas, os países da América Latina que
passaram por processos ditatoriais têm incertezas tão grandes que é até
compreensível que as constituições sejam detalhadas.}

\textbf{Maciel:} Eu não quero radicalizar, mas acho que a Constituição
muito longa termina sofrendo muitas emendas, sendo muito questionada.
Termina não sendo uma boa técnica legislativa porque a "boa
constituição" é aquela que a pessoa leia como se lê um romance, que
tenha uma lógica, um encadeamento. Se você pega a Constituição de 1946 e
compara com a Constituição de 88 vai verificar que a de 1946 tem uma boa
técnica, entendeu? Até o encadeamento das disposições etc, é uma coisa
muito diferente, como se a história tivesse começo, meio e fim.

\textbf{MAURO BENEVIDES}

*Profissional de letras, iniciou sua carreira política no PSD sendo
eleito vereador de Fortaleza (1956-1959) e deputado estadual no Ceará
(1959-1975). O bipartidarismo (1966) imposto pelo regime militar
(1964-1985) levou-o ao MDB, partido pelo qual foi senador (1975-1982 e
1987-1994) e deputado federal (2007-2014).

\textbf{A Assembleia Nacional Constituinte foi instalada no dia 1º de
fevereiro de 1987. O dia da instalação foi de festa cívica. Ao mesmo
tempo, o jornal Folha de S. Paulo exibia num de seus cadernos a manchete
``Crise marca a instalação da Constituinte". Por que foi difícil o
início dos trabalhos?}

\textbf{Benevides:} Naturalmente, no primeiro momento, o presidente José
Sarney entendeu que ofereceram à Assembleia Constituinte um projeto
perfeito e acabado. Uma comissão de juristas, integrada inclusive por um
que seria Constituinte, Affonso Arinos (PFL-RJ), foi oferecida.
Portanto, uma mesa da Assembleia com um projeto que consubstanciava
naquele momento, no entender dos juristas, algo representativo para o
acolhimento das aspirações do povo brasileiro, após de tantos anos de
arbítrio instalado em nosso território.

Então houve uma recusa para que nós passássemos a trabalhar nesse
documento confeccionado por juristas, elaborado por juristas, mas que a
Assembleia não entendeu como indispensável ao processo de elaboração
constituinte. Por isso surgiu um pequeno incidente, superado logo depois
do pronunciamento do ministro José Carlos Moreira Alves (STF), que, como
presidente do Supremo, instaurou a Assembleia naquele 1º de fevereiro de
1987, numa sessão memorável.

A partir desse momento iniciamos o nosso trabalho com a eleição da mesa
da Assembleia Nacional Constituinte, para a qual se elegeu o Ulysses
Guimarães numa unanimidade consagradora. E em função do prestígio do
Ulysses, eu me vi também eleito primeiro vice-presidente da Assembleia,
porque se ele era deputado e passaria para o cargo de presidente, era
normal que um senador pudesse ser escolhido para exercer a primeira
vice-presidência. Pude com isso coadjuvar o trabalho notável que o
Ulysses Guimaraes realizou, que foi sem dúvida a maior expressão da
Constituinte e a quem se atribui merecidamente a condição de
reconstrutor do Estado Democrático de Direito.

\textbf{A relação política do governo Sarney (PMDB-MA) com a Assembleia
Nacional Constituinte foi conturbada, visto que os temas da duração do
mandato de Sarney e o sistema de governo tiveram destaques na mídia.}

\textbf{Benevides:} Foi questionado, no momento, se seriam 6 anos, 4
anos ou se seriam 5 anos (de mandato). Mas eu acredito que o grande
empecilho que se registrou não foi exatamente a fixação da duração do
mandato. Foi, sobretudo, o fato de que o presidente José Sarney, diante
de conquistas sociais que estavam sendo inseridas no texto em
elaboração, entendeu que aquilo exorbitava a potencialidade do Tesouro
Brasileiro, que seriam gravames que atingiriam seriamente, digamos, "o
poder de cumprimento" daqueles encargos por parte do Poder Executivo.

A partir desse momento, sobretudo quando se aprovou a
licença-paternidade, de autoria do deputado Alceni Guerra (PFL-PR),
houve uma mobilização para que o presidente (da Câmara) Ulysses
respondesse ao presidente da República, defendendo naquele momento o que
era fundamental para a soberania do Poder Constituinte. Assim foi feito
através de uma cadeia de radio e televisão e permaneceu até o final o
nosso trabalho nessa diretriz, que foi fundamental e perseguida
obstinadamente por todos nós. Eu na condição de senador e os outros
companheiros que integravam a Câmara dos Deputados, a qual pertenço hoje
com muita honra.

\textbf{O presidente Sarney desferiu vários ataques ao Congresso
Constituinte. No dia 15 de julho de 1987 a imprensa noticiou: "Ulysses
Guimarães (PMDB-SP) rebate as críticas de Sarney contra a Constituição".
Segundo ele, as populações não estavam aqui, no Distrito Federal, no
Palácio do Planalto, na sala do presidente e muito menos na mesa de
Getúlio Vargas, onde o presidente resolve os problemas. As críticas do
presidente Sarney eram infundadas?}

\textbf{Benevides:} Não. Ao longo do trabalho da Constituinte houve
esses desencontros de formulações, mas sempre as pessoas procuravam
interferir para que não surgissem entre o Legislativo Constituinte e o
Executivo algo que se contrapusesse aos objetivos maiores do que nós
estávamos incumbidos por força do poder originário, aquele que emana do
próprio povo na manifestação soberana da ajuda, dado então naquele ano
de 1986. Esses desencontros surgiam e se esvaziavam em função da grande
preocupação, dedicação e do nosso esforço para a tarefa da Constituinte,
para legar ao País algo que consubstanciasse, naquele momento, as
aspirações mais justas e legítimas da comunidade.

\textbf{O regimento da ANC acolheu o pedido do Plenário Nacional em prol
da participação popular na Constituinte. O direito de apresentar 122
emendas e mais de 12 mil assinaturas foi uma grande vitória, bem como a
participação de todos os grupos sociais e patronais nos corredores da
Assembleia Nacional Constituinte?}

\textbf{Benevides:} Perfeitamente. A participação popular foi
significativa. Eu recebi algumas dessas Emendas Populares a pedido do
presidente Ulysses Guimaraes. Aquelas lideranças que traziam documentos
com 200 mil assinaturas, 300 mil assinaturas. Eu recordo de algo que era
da criança e do adolescente, que cheguei a receber e, depois, nos
incumbíamos de fazer a constatação daquelas assinaturas para que
entendêssemos a ponderabilidade daquelas manifestações. Elas valeram
substancialmente porque, aceitas pela Assembleia, foram distribuídas por
assunto para as comissões temáticas e, naturalmente, a manifestação
conclusiva era do plenário com a Comissão de Sistematização, que tinha
como expressão maior o grande Affonso Arinos. Ele a presidiu e dirimia
aquelas dúvidas numa sequência de entendimentos dos quais era partícipe
maior o próprio presidente Ulysses Guimarães, que conduzia os
entendimentos e buscava nas lideranças um consenso para que nós
tivéssemos um rendimento bastante razoável e acolhêssemos aquilo que
pudesse, no momento, se ajustar às aspirações mais justas e legítimas da
coletividade.

\textbf{Durante o processo, vários embates entre os progressistas e os
conservadores. Pontos polêmicos como sistema de governo, reforma
agrária, definição de empresa nacional, papel das Forças Armadas,
reforma tributária, estabilidade no emprego, questões de ordem
econômica, entre outros. Como foi lidar com várias frentes partidárias e
depois os grupos suprapartidários? Foi possível o diálogo entre
opostos?}

\textbf{Benevides:} Bom, nós tivemos uma grande dificuldade para
estabelecer os diálogos com as corrente que ideologicamente se
digladiavam no plenário. O grande momento realmente que tivemos foi
quando o Centrão pela manifestação do seu líder maior que era o saudoso
deputado Cardoso Alves (PMDB-SP), fez algumas imposições ao próprio
presidente Ulysses, da tribuna da Assembleia, e isso compelia a mesa a
aceitar algumas daquelas ponderações que só seriam viabilizadas caso se
processasse uma ponderação do regimento da Casa. A elaboração do
primeiro regimento coube ao então senador Fernando Henrique Cardoso
(PMDB-SP).

Depois de composta a mesa da Assembleia, entretanto, a alteração
pretendida teria que ser a cabo dos membros da mesa. Foi então que o
presidente Ulysses Guimarães, entendendo que eu possuía condições de
compatibilizar aquelas divergências que transbordavam nos debates -- uns
debates mais acesos, outros mais compreensivos -- entendeu de conferir a
mim a tarefa de promover a reformulação do regimento, pedido do Centrão.
Mas os membros daquele grupo não participariam dos trabalhos de
elaboração da Constituinte, o que praticamente inviabilizava o nosso
projeto ou então delongaria exageradamente aquela tarefa a que fomos
incumbidos pela manifestação do próprio povo brasileiro.

\textbf{Depois da criação do Centrão, no dia 10 de novembro de 1987, e
da aprovação de sua proposta de mudança, como se deu a organização da
qual o senhor falou? O que de fato aconteceu? Houve vitórias e derrotas
de lado a lado?}

\textbf{Benevides:} Não. O que houve foi o seguinte: nós buscávamos
consensuar aquelas opiniões divergentes. Foi um trabalho exaustivo. Nós
também tivemos que utilizar toda a sistemática de persuasão para
garantir que, com a reformulação regimental, naquilo que fosse possível,
não se aceitaria exclusivamente a imposição consequente do Centrão, mas
que se fizesse aquela concessão possível, porque muito mais importante
do que divergências ocasionais seria, sem dúvidas, a elaboração da Carta
a que nós estávamos incumbidos. E isso efetivamente ocorreu.

\textbf{O que de fato aconteceu nessa questão do Centrão?}

\textbf{Benevides:} Não o Centrão, por exemplo, ele queria que nós
tivéssemos que aceitar os destaques. Tivemos que inovar para aceitá-los.
E, muito mais do que isso, que nós determinadamente só admitíssemos a
aprovação de qualquer matéria com 280 votos. Sem isso, sem esse quórum
qualificado de 280 votos, nenhuma matéria seria considerada aprovada.
Então lógico que nós tivemos que gestionar para que nós seguíssemos os
trabalhos e soubéssemos fazê-lo com a maior tranquilidade, fazendo
concessões aos grupos conflituosos naquele momento. Nós buscávamos
sobretudo a elaboração desse trabalho, no esforço extraordinariamente
levado ao efeito, que resultou na elaboração de uma carta que o Ulysses
Guimarães entendeu de considerar a ``Carta Cidadã''.

\textbf{No dia 22 de março o presidencialismo foi aprovado com 5 anos
para Sarney. Houve compra de votos, obtenção de cargos no governo e
barganhas? O que foi feito para garantir a vitória do governo com uma
votação de 344 a 212?}

\textbf{Benevides:} Não. A divergência numérica ocorreu preponderando a
fixação em 5 anos com entendimento próprio do presidente Sarney. Nós
chegamos a esse entendimento e ele se processou naturalmente por uma
compreensão daqueles que pensavam contrariamente. Uns queriam 6 e outros
queriam 4, até que se chegou-se à média de 5 anos -- o que prevaleceu.

\textbf{Em julho de 1988 tivemos um momento muito tenso em relação à
Assembleia, envolvendo o Dr. Ulysses, o senhor e também o presidente
Sarney. Em cadeia nacional, Sarney discursou durante 28 minutos contra o
texto constitucional e afirmou que, se aprovado sem modificações, o
texto levaria o País à ``ingovernabilidade''. Ulysses o contradisse no
dia seguinte, também em rede nacional, afirmando que "a Carta seria,
sim, uma guardiã da governabilidade". Nesse momento crítico, o que de
fato aconteceu?}

\textbf{Benevides:} Este foi, talvez, o momento mais delicado do
relacionamento entre Constituinte e o Poder Executivo, naquela época nas
mãos do presidente José Sarney, que tinha realmente a incumbência de
decidir os destinos nacionais. A manifestação do presidente José Sarney
foi discutida numa reunião que se instaurou na residência oficial da
Câmara dos Deputados, que era do presidente da Assembleia Constituinte,
o Ulysses Guimarães. Nós sugerimos ao presidente da Constituinte que,
com a autoridade encarnava admiravelmente, respondesse ao presidente da
República através da televisão, numa cadeia que foi requisitada pela
Assembleia Constituinte. Não precisava ser um diálogo que distanciasse
os dois poderes, mas que ele mostrasse à sociedade que as inovações
jamais comprometeriam a governabilidade do País.

Até porque se algo fosse inserido a inconsequência seria daqueles
segmentos sociais que reivindicavam num momento de expectativa nacional.
Eles reivindicavam o atendimento daquelas postulações que eram tão
viáveis que em nenhum momento a Nação deixou de cumprir todos aqueles
compromissos, significando claramente que nós não estávamos realizando
um trabalho irresponsável, estávamos absolutamente conscientes do nosso
encargo, das nossas responsabilidades e como as conquistas se ajustariam
à realidade daquela época cognominada seguidamente de
``ingovernabilidade do País''. Nós, da Constituinte, não queríamos
atentar contra a governabilidade por um motivo muito claro: nós
queríamos viabilizar no País aquelas conquistas que se tornaram
presentes ao longo da elaboração do processo Constituinte. Foi isso o
que ocorreu. Os episódios foram gradualmente superados e proclamada a
Carta, numa fotografia histórica. O presidente Sarney me tendo ao seu
lado, ele ao lado do presidente Ulysses e eu como vice-presidente ao
lado do presidente Sarney. Ele jurou cumprir a Constituição, o que ele
fez durante o seu mandato, recebendo, ao final, os aplausos e
conseguindo ainda todos os espaços que posteriormente lhe garantiram
inclusive a eleição de senador pelo estado do Amapá. Em 1990 ele
conquistou o mandato de senador e, em 1991, passou a exercê-lo com
proficiência, com dedicação, alçando-se por duas, três vezes, agora à
Presidência do Senado e, consequentemente, à chefia do Poder Legislativo
Brasileiro.

\textbf{Por que as medidas provisórias representam para o Texto
Constitucional um calcanhar de Aquiles?}

\textbf{Benevides:} Porque naquele momento estávamos vivenciando, como
ainda agora, algo complexo na tramitação de matérias do governo. Este se
queixava que depois da extinção do famigerado decreto-lei, precisava
buscar algo que assegurasse uma maior celeridade ao processo
legislativo.

Então houve excesso de todos os presidentes. Não foi só do presidente
Lula (PT). Todos os presidentes, a começar por José Sarney (PMDB),
Fernando Collor (PRN), Itamar (PMDB), Fernando Henrique Cardoso (PSDB) e
o presidente Luís Inácio Lula da Silva. Todos foram utilizadores
exacerbados da medida provisória. Sofre aqui e ali uma limitação, mas,
no fim, ela continua prevalecendo. Até porque entenderam os presidentes
da República que sem um instrumento capaz de agilizar o processo
legislativo realmente o poder executivo ficaria desarmado para aprovar
políticas públicas. Significaria um impulso para favorecer o povo
brasileiro.

\textbf{Em defesa dos fundos constitucionais o senhor atuou durante a
ANC no desenvolvimento do Nordeste, o senhor lutou pela a aprovação
dele, onde o próprio gabinete do senhor serviu como infraestrutura para
o trabalho. Como se deu essa articulação em defesa do Nordeste?}

\textbf{Benevides:} Eu naturalmente trazia para a Assembleia
Constituinte aquela experiência que adquiri como presidente do banco.
Visualizara a necessidade de garantir recursos substanciais para atender
as áreas subdesenvolvidas do País ou em desenvolvimento. O Nordeste, o
Norte e o Centro-Oeste se ressentiam de instrumentos capazes de
impulsionar uma expectativa de desenvolvimento. Daí surgiu a emenda da
qual eu sou o primeiro signatário, que foi subescrita por vários
parlamentares, todos instituídos no Fundo Constitucional de
Financiamento do Nordeste (FNE), no Fundo Constitucional de
Financiamento do Norte (FNO) e no Fundo Constitucional de Financiamento
do Centro-Oeste (FCO). Tais fundos foram constituídos com o recurso do
Imposto sobre Produtos Industrializados (IPI) e do Imposto de Renda
(IR).

Foi uma vitória extraordinária que me identificou com o comércio
brasileiro. Daí porque diversas vezes tenho recebido manifestações
sensibilizadoras desse segmento, que é aquele que detém o maior número
de empregos diretos e, consequentemente, pelo menos em relação ao meu
estado, a maior contribuição para os cofres do tesouro estadual.
Portanto nós conseguimos incluir os fundos e eles estão aí atingindo os
objetivos com os bancos de desenvolvimento, como o Banco da Amazônia
(BASA) na região Norte, o Banco do Nordeste do Brasil (BNB) na região
Nordeste e com o Banco do Brasil no Centro-Oeste funcionando também como
banco de desenvolvimento. Os fundos estão sendo aplicados, os recursos
estão sendo transferidos e nós acreditamos que essa foi uma das grandes
conquistas da Assembleia Nacional Constituinte e ainda aquelas outras
que estavam intrinsicamente vinculadas aos aspectos institucionais do
nosso próprio País.

\textbf{Vinte anos após a promulgação da Constituinte ainda temos muito
a fazer?}

\textbf{Benevides:} Acredito que sim, porque aprovaram mais de 60
emendas até agora. Nós, Constituintes, tivemos a previsibilidade de
admitir que a Carta não era perfeita. Previmos que em 5 anos nós
deveríamos realizar o processo revisional. Nesse momento eu não estava
presidindo mais o Senado. E o Congresso estava como líder da maioria no
Senado, por extensão, no Congresso. Conduzimos o processo revisional
tendo como Relator o deputado Nelson Jobim (PMDB-RS), que viria a ser
ministro da Justiça e ministro do Supremo Tribunal Federal.

Entendemos que a Carta deve ser alterada naquilo que seja fundamental.
Mas mencionar, como as vezes se escuta, que é preciso uma Assembleia
Nacional para discutir, por exemplo, a reforma tributaria e a reforma
política, isso é uma incongruência jurídica que não pode ser admitida,
porque uma Constituinte implica uma ruptura institucional e nós não
queremos mais que o Brasil vivencie um quadro de ruptura institucional.

\textbf{NELSON JOBIM}

*O advogado, jurista e professor iniciou a carreira política na eleição
para a Assembleia Nacional Constituinte, sendo eleito deputado federal
(1987-1994). Foi ministro da Justiça (1995-1997), ministro do Supremo
Tribunal Federal (1997-2006), ocupando a Presidência do STF (2004-2006),
e ministro da Defesa (2007-2011).

\textbf{A instalação da Assembleia Nacional Constituinte aconteceu em 1º
de fevereiro de 1987. O tema da soberania nacional foi bastante
debatido. Por quê?}

\textbf{Jobim:} Não era bem a soberania nacional. O que mais se debatia
era, além da soberania nacional, a soberania da Constituinte. Porque no
momento anterior havia a sustentação da Constituinte Exclusiva, pela
OAB, e, no final, quando o presidente Sarney mandou o projeto de emenda
constitucional que convocou a Assembleia Constituinte, o deputado Flávio
Bierrembach (PMDB-SP) que fora relator da emenda constitucional em 1985,
fez um projeto substitutivo da emenda constitucional dando absoluta
autonomia à Assembleia Constituinte. Ele acabou derrotado e foi aprovado
aquele modelo simbiótico, um modelo tipicamente brasileiro. No Brasil
você não encontra rupturas. O Brasil é um País de transição. Quando um
regime se supera, de dentro do próprio regime surgem as fórmulas de
superação do regime. E foi o que aconteceu. Ou seja, se você falar
tecnicamente e a rigor, ortodoxamente o problema jurídico, você verifica
o seguinte: os deputados e senadores eleitos, uns em 1982 e outros em
1978, aprovaram uma emenda constitucional convocando uma assembleia
constituinte e dando poderes a essa assembleia constituinte para fazer
uma nova Constituinte com lei absoluta em sessão unicameral.

Se pegarmos isso em termos técnicos encontraremos o seguinte: o poder de
fazer a nova Constituição, o chamado constituinte originário, ele veio,
foi outorgado pelos constituintes derivados, que votaram em Emenda
Constitucional, que convocou a Constituinte em 1985... Isso já cria um
``frisson'' dentro da dogmática constitucional, porque os constituintes
de 1987-1988 eram nada mais nada menos que os deputados e senadores
eleitos sob a égide da Constituição, da emenda constitucional de 1969.
Tanto é que eu apresentei, na bancada do PMDB, dois projetos que eu
chamava de decisões constitucionais. Um era para que nós não jurássemos
a Constituição de 1969, já que nós iriamos fazer uma nova constituição
e, portanto, não poderíamos jurar a permanência da Constituição de 1969.
E apresentei um outro projeto dentro da bancada que era para definir a
autonomia da Constituinte. Inclusive dava regras ao decreto lei, mudava
o sistema do decreto lei, estabelecia uma série de restrições ao Poder
Executivo para o funcionamento da Assembleia Constituinte. Depois o
doutor Ulysses achou que aquilo poderia gerar radicalizações e acabei
afastando. Mas o texto ficou lá e eu não apresentei formalmente --
apresentei na bancada do partido.

E a coisa começou no dia 1º. Na posse foi suscitado um problema de forma
indireta pelo Plínio de Arruda Sampaio (PT): o problema do terço do
Senado que havia sido eleito em 1982 -- porque na eleição de 1986 o
senado se renovou em dois terços, mas o terço do senado que continuava
era os eleitos de 1982. E o Plínio, que era líder do PT, suscitou uma
questão de ordem para o presidente da Constituinte daquele momento, que
era o presidente do Supremo, o José Carlos Moreira Alves. E o José disse
``olha, a emenda constitucional disse que os deputados e senadores
reunir-se-ão a partir de 1987 em fevereiro em sessão unicameral. Logo,
deputados e senadores são todos, não só os dois terços do Senado que
foram eleitos em 1986, mas também os eleitos em 1982. Durante todo esse
período e durante todo o desenvolver do processo constituinte sobrou uma
tensão: uma tensão do Executivo com a Constituinte -- e é exatamente
esse o problema da autonomia, da soberania.

\textbf{A ANC foi instalada numa conjuntura de crise de sustentação do
governo Sarney e marcada pela crise do PMDB?}

\textbf{Jobim:} A crise do PMDB e a crise do governo Sarney eram:
primeiro, o Plano Cruzado fracassara, inclusive houve toda aquela
afirmação do Brizola (PDT-RJ) na época, de que havia um estelionato
eleitoral, que o PMDB ganhou as eleições de 1986 e a reforma do Plano
Cruzado só se deu depois das eleições. Quando começou o processo
Constituinte se estabeleceu, dentro do PMDB, uma disputa. E essa disputa
era a seguinte: de um lado o presidente Sarney com o grupo "sarneyista"
do PMDB, grupo dirigido e liderado por Carlos Santana (PMDB-BA), o líder
do governo na Câmara. Do outro lado o doutor Ulysses e seu grupo, o qual
integrei. Eu fui do segundo time.

O conflito se deu primeiro na eleição do líder do PMDB na Constituinte.
Depois que o doutor Ulysses disse que seria o presidente da Constituinte
e da Câmara -- e ele tinha razão, porque se não a Constituinte iria
ficar sem infraestrutura administrativa e iria ser uma dificuldade
enorme, com o presidente da Constituinte pedindo ao presidente da Câmara
ou do Senado que desse estrutura --, então o doutor Ulysses, com a dupla
eleição, ficava como presidente da Câmara e da Constituinte. Dessa forma
a estrutura da Câmara ficou servindo à Constituinte. Na eleição para
presidente da Câmara, que antecedeu a eleição da Constituinte, o doutor
Ulysses teve uma disputa com o Fernando Lyra (PMDB-PE), que decidira ser
candidato a presidente da Câmara.

\textbf{Fernando Lyra fez várias críticas. Chegamos a entrevistá-lo e
ele disse que o doutor Ulysses "rasgou" o regimento interno, que ele não
poderia ser presidente da Constituinte e presidente da Câmara dos
Deputados.}

\textbf{Jobim:} Não é verdade. A discussão na época não era essa, mas
era se ele poderia ser presidente da Câmara por haver sido presidente da
Câmara na legislatura anterior. E o entendimento que se deu e que se dá
até hoje é de que não pode haver eleição para presidente da Câmara e do
Senado dentro da mesma legislatura. Então se entendeu que o sujeito que
é presidente no primeiro biênio de uma legislatura não pode ser reeleito
para o segundo biênio. Mas, quando se abre uma legislatura nova, mesmo
que ele tenha sido presidente da Câmara do segundo biênio da legislatura
anterior, ele está numa legislatura nova. Então isso que foi levantado
por Fernando Lyra e por outros foi resolvido, porque na verdade era uma
legislatura nova e os novos legisladores não ficavam vinculados às
decisões tomadas pelos antigos. Esse assunto foi vencido.

O problema do Fernando Lyra se deu em consequência dessa disputa, porque
o Lyra desenterrou um anteprojeto, um esboço de anteprojeto para a
Constituinte. E esse esboço de anteprojeto fora feito pela Assessoria da
Câmara a pedido do doutor Ulysses, em 1986, e que mais ou menos copiava,
ajustava o regimento interno que regeu a Constituinte de 1946 e criava
uma grande comissão. Na constituição de 1946 houve uma grande comissão,
a Comissão Nereu Ramos, que elaborou um projeto de constituição e a
Constituinte de 1946 votou em cima desse projeto. Em 1987 havia um
problema: o governo não tinha como mandar um projeto de constituição
porque era um governo fraco. Havia dois modelos de fazer constituição no
Brasil. O primeiro é o Executivo mandando um projeto, como fizeram
Deodoro da Fonseca para a Constituição de 1981 e Getúlio Vargas para a
Constituinte de 1934. O outro modelo foi o adotado em 1948, após a
derrubada do Getúlio: o presidente do Supremo Tribunal assumiu o
governo, mas não tinha condições de mandar projeto, então fizeram uma
comissão dentro da própria Constituinte e essa comissão fez um projeto
de constituição.

A Assessoria da Câmara fez isso para o doutor Ulysses pressupondo que o
presidente Sarney não possuía condições políticas de mandar um projeto
de constituição. E de fato não mandou, só mandou aquela comissão do
Affonso Arinos (PFL-RJ) como sugestão, como elemento de pesquisa, mas
não como projeto propriamente dito. O que aconteceu foi que, na eleição,
o Lyra usou esse desenho, esse anteprojeto, um rascunho de regimento
interno para a Constituinte, dizendo assim: "Olha, vocês estão vendo o
que o velho quer fazer. O velho quer pegar o grupo do 'poar''' -- que
era um grupo ligado a ele -- ``fazer a Constituição e deixar vocês em
segundo time''. E, com isso, criou a figura dos constituintes de
primeira categoria, que eram os membros da comissão, e os de segunda
categoria.

Isso deu alguns votos. O Fernando Lyra acabou se elegendo, mas queimou a
possibilidade de fazer o regimento interno que fosse do modelo de 1946.
Daí tivemos que inventar quando o doutor Ulysses nomeou o Fernando
Henrique (Cardoso, PSDB) como relator do regimento interno. E a briga do
PMDB era a briga Ulysses de um lado e Sarney do outro. Isso ficou muito
claro na luta da liderança do PMDB na Constituinte. O Ulysses tinha um
candidato para ser líder do PMDB na Constituinte e que era o líder do
PMDB na Câmara, que era o Luíz Henrique da Silveira (PMDB-SC), que viria
a ser o governador de Santa Catarina e também senador.

O Luís Henrique candidata-se à liderança e o Mário Covas também. A
votação era uninominal. Os deputados e senadores votavam em conjunto na
bancada em comum. Os senadores do PMDB votaram no Mário Covas e a parte
`'sarneyista'' da câmara votou também no Mário Covas, que derrotou o
Luís Henrique com os votos do Sarney. Mas surge uma consequência também
nisso: o Mario Covas que, digamos, acabou se elegendo com os votos do
Sarney, não estava vinculado com o Sarney, mas foi eleito com os votos
do Sarney, então a imprensa repercutiu esse fato. Então o Mário, quando
compôs as comissões e subcomissões para a elaboração da Constituinte,
privilegiou a esquerda do partido.

\textbf{Na partilha dos cargos, dos 132 cargos existentes coube ao PMDB
15 das 24 presidências e 21 das 32 relatorias. Nas disputas internas
partidárias e ideológicas foi um longo caminho. Como se deu a regra do
jogo? A questão partidária exercia influencia na definição da agenda?}

\textbf{Jobim:} A questão partidária era a disputa de Sarney. Ou melhor:
a disputa entre progressistas e conservadores, porque o PMDB era um
conjunto de liberais, esquerda e gente de direita. O Mário Covas, que
era o líder e que tinha o poder de indicação dos presidentes das
subcomissões, dos subrelatores, dos relatores das comissões e dos
presidentes das comissões, de acordo com o regimento interno feito por
Fernando Henrique com a minha ajuda, o Mário privilegiou a esquerda do
partido. Tanto é que a Comissão de Sistematização -- que nada mais era
do que a reunião de 40 e poucos deputados e senadores que ficavam fora
do processo inicial e eram escolhidos como formadores de opinião e que a
eles se agregariam os relatores, vice-relatores e presidentes do período
das comissões -- este conjunto ficou à esquerda do plenário. E a
condução do PMDB, a condução do processo constituinte na Comissão de
Sistematização, é que provocou a criação do Centrão, porque o Centrão se
opôs, junto com o governo, evidentemente, aos resultados finais da
Comissão de Sistematização.

\textbf{Nessa arquitetura, como se deu a negociação do regimento interno
para a aprovação em meio às disputas partidárias e ideológicas?}

\textbf{Jobim:} O primeiro problema foi sobre como nós iriamos construir
o regimento interno. Nós não podíamos copiar um modelo, porque o Brasil
teve dois modelos: um que era o de 1946, a criação de uma grande
comissão, que havia sido queimado na disputa entre Ulysses e Fernando
Lyra. Não se podia nem falar em criar uma comissão para elaborar um
anteprojeto. E o outro modelo era um projeto executivo. Também não se
podia falar nisso porque o presidente Sarney não tinha condições
políticas naquele momento de mandar um projeto.

Então nós tivemos que inventar. E nós rigorosamente inventamos um modelo
que começou do zero. Criou-se oito comissões compostas de 21
subcomissões. Cada subcomissão com uma tarefa. Terminados os trabalhos
das subcomissões, reunia-se tudo num conjunto e a comissão votava
naquele conjunto de trabalhos. Num momento você tinha, de início 24
textos correspondentes às 24 subcomissões e, em seguida, esses 24 textos
viravam oito textos, porque cada texto cumpria uma comissão. Depois
desses oito textos virou um texto só que foi para a Comissão de
Sistematização. E tudo isso foi uma negociação complicadíssima, porque
havia um fato político.

Era mais fácil você aprovar um texto na Constituição do que você votar
na lei. Então todos os setores que tinham interesse de preservar, e
pretensões de espaço constitucional, eles procuravam colocar dentro da
Constituição.

\textbf{A Comissão de Sistematização, integrada por 93 dos 559
constituintes, teve o poder de definir o texto que seria submetido ao
plenário. Quais os momentos de maior agonia, perplexidade e
agravamento?}

\textbf{Jobim:} Um momento grave foi o final. Primeiro houve as
disputas, a questão do petróleo, da reforma agraria etc, que eram temas
dentro da Comissão de Sistematização. Mas houve um grande problema que
eu chamei de ``crise do regimento''. O regimento interno, do qual fora
relator o Fernando Henrique, e eu trabalhei com ele nisso, nós prevíamos
o regimento interno do seguinte modelo: primeiro não seriam admitidas
emendas substitutivas globais; segundo não haveria a figura do destaque
para votar separado. O que aconteceu?! Exemplifico e isso foi percebido
pelo Centrão, principalmente pelo Gastone Righi (PTB-SP), que foi quem
mais sustentou esse assunto: cada subcomissão possuía 21 membros.
Apresentava-se um texto qualquer e, para ser aprovado, precisava de
maioria simples: ter 11 votos. Aprovado um texto "x" com 11 votos, esse
texto na subcomissão era reunido, na comissão de 63 membros, com os
outros textos das subcomissões. E estava lá dentro do texto o "x". Mas
para tirar o "x", se alguém quisesse suprimir aquele "x", precisava ter
maioria. Maioria de 63 é 32. Alguém entrava com uma emenda supressiva
daquele texto ``x'' que entrou com 11 votos. E se essa emenda ou
destaque supressivo apresentado no seio da comissão tivesse 31 votos não
era aprovado. Logo, 11 era maior que 31. Depois esse texto "x" sobrevive
à comissão e vai para Comissão de Sistematização, composta por 93
deputados. Também nessa precisava-se da maioria, 47, para excluir o
texto que foi aprovado por 11. Então 11 era maior que 47, porque os 41
não conseguiriam derrubar os 11. Por isso surgiu a crise do regimento.

E essa crise era apresentar emenda substitutiva global e apresentar
"DVS" -- Destaque para Votar em Separado. Ela analisava o texto "x" fora
daquela comissão e o autor do texto teria o ônus de buscar a maioria
para colocar o "x" de volta. Entendeu o jogo?

E um segundo momento grave foi relativo às emendas substitutivas
globais. Terminada a Comissão de Sistematização houve a paralização dos
trabalhos por conta da crise do regimento e houve a alteração do
regimento interno. O Centrão conseguiu introduzir o destaque para votar
em separado e as emendas substitutivas. O bloco fez oito emendas
substitutivas dos oito capítulos da Constituição. E surgiu um problema
político que foi discutido no seio da liderança do PMDB, com o Mário
Covas: "vamos aprovar o texto da Comissão de Sistematização ou vamos
fazer uma guerra de guerrilha?

Avaliamos politicamente o problema e resolvemos aprovar as oito emendas
do Centrão e, depois, durante todo o processo constituinte, tentar
recompor a emenda da Sistematização via destaques, emendas supressivas e
emendas substitutivas para incluir no texto do Centrão. E foi isso que
fizemos.

Na primeira sessão o Mário Covas encaminha para votação, para a
aprovação como texto base, a emenda substitutiva do Centrão. E isso
prejudicou a emenda da Sistematização, mas nós já havíamos feito na
Assessoria e eu trabalhava muito nisso. Fizemos vários destaques que
entendemos necessários para, tendo como base o texto do Centrão,
trazermos de volta aquilo que havíamos aprovado na Comissão de
Sistematização. Por isso que o processo constituinte levou tanto tempo.
Foi uma guerra de guerrilha. E por que eu sustentava que tínhamos que ir
para guerra de guerrilha? Porque quando o Centrão fez as emendas
substitutivas dele, globais, ele chamava os constituintes que tinham
votos para que botassem os seus textos ali dentro.

Houve um caso curioso envolvendo um deputado do Paraná que era pastor
evangélico e tinha uma gravadora. Ele queria, na discussão durante todo
o processo da Comissão de Sistematização, que o ``copyright'', o direito
da propriedade intelectual, não abrangesse composições religiosas, para
que a gravadora dele pudesse lançar discos de músicas religiosas sem
pagar direitos autorais para os compositores. E o Centrão botou essa
coisa, excepcionou os direitos autorais para produções religiosas. Com
isso conseguiram o voto desse deputado pastor. Repetiram isso com
vários. Eu dizia "olha, nós vamos perder". Eles solidarizavam com uma
série de mecanismos. E nós resolvemos votar. Votamos no texto básico do
Centrão. Significava que o trabalho da Comissão de Sistematização não
foi o texto básico do Plenário. O texto básico foram as emendas
substitutivas do Centrão. E durante todo o período até o final, nós
fizemos destaques e emendas para tentar recuperar o texto aprovado na
Comissão de Sistematização.

\textbf{Na sistematização, após vários embates, várias frentes, sistema
de governo, reforma agrária, definição de empresa nacional, o papel das
Forças Armadas, estabilidade no emprego, questões de ordem econômica,
entre outros, surgiram vários grupos. Grupo centrista com 32
parlamentares, José Lourenço; grupo do consenso, com Fernando Henrique.
Como é que se dava essa articulação entre os grupos? Os partidos
deixaram de ter proeminência?}

\textbf{Jobim:} Não. Não trabalhávamos por sigla de partido, mas por
conjuntos. E as alianças eram diferentes e, em determinados temas, você
se aliava por conjunto. Funcionava através de reuniões na liderança do
PMDB, presidida pelo Mário Covas até que ele passou para o PSDB. Lá na
liderança nós negociávamos toda a votação da tarde. Negociava com o PFL,
PDT... E cada grupo, cada partido desses, viabilizava e era veículo de
interesse de outros que estavam atrás da reforma agrária. Havia o PFL, o
UDR veiculando suas pretensões via PFL... Também negociamos com gente do
PMDB, que era o caso do Cardoso Alves (PMDB-SP), conhecido como
Robertão. O Robertão, por exemplo, viabilizava nas discussões de
votação. Nos acordos nós levávamos em conta essa situação toda. Havia
grupos de interesse que veiculavam as suas pretensões via partidos e
grupos. E não havia segurança de que a tua bancada votaria contigo. Um
exemplo: o PMDB teve na Constituinte acho que 230 deputados e senadores
de números redondos, mas 160 votavam junto com Mário Covas e os demais
com o Centrão.

\textbf{E o diálogo seguia desta forma?}

\textbf{Jobim:} Sim. O diálogo era feito com grupos e basicamente dentro
da liderança do PMDB, comandada pelo Mário Covas. Então ali você fazia
os debates, os acertos, as discussões. Tanto é que a gente inventou,
dentro do processo, as chamadas emendas de fusão ou emendas de
transação, porque às vezes você não conseguia fazer acordo nenhum. Quer
com o texto da Sistematização, quer com o texto das emendas que
existiam, quer com as emendas do Centrão. Então você tinha que criar uma
forma de formar maioria. É você fazer um texto longo e aprová-lo na
hora.

\textbf{Sobre as emendas populares: o regimento interno, na elaboração
junto com Fernando Henrique, acolheu o pedido da plenária nacional em
prol participação popular. Qual foi a importância das emendas
populares?}

\textbf{Jobim:} Não houve. É mistificador dizer que elas tiveram
relevância, porque elas se perderam no meio daquele conjunto todo. O
fato de a emenda ser popular não lhe dá mais autoridade do que as outras
emendas. Não dava a elas tratamento diferenciado. Quando fomos para a
Constituinte se falava muito na sociedade civil organizada. Nos demos
conta de que não possuíamos sociedade civil organizada, mas grupos de
interesse organizado, que queriam pegar pedaços do Estado para si. A
grande intenção era dos grupos de interesse muito organizados. Não
possuíam representação popular propriamente dita, mas eram grupos de
interesses muito organizados que queriam cravar na Constituição os seus
interesses. E foi isso o que aconteceu. A gente viu isso muito bem no
Poder Judiciário, no Ministério Público. Todo mundo queria colocar a sua
autonomia ou um nicho de status para ficar com eles.

\textbf{Qual o peso das emendas populares?}

\textbf{Jobim:} Quem viveu sabe perfeitamente que a emenda popular não
teve tratamento diferenciado de outras emendas. Passava pelo processo de
negociação e sabíamos que a emenda popular não fora produzida pelo povo,
mas algo organizado por um grupo de interesses que colhia assinaturas na
avenida Copacabana. Era a mesma coisa dos deputados que assinavam nos
corredores. Não foi uma coisa que foi debatida pela população, mas
apenas um grupo de interesses que colhia assinaturas. E tentavam com
isso legitimar o que era legítimo. Mas na discussão dentro da
Constituinte elas tinham o mesmo tratamento.

\textbf{O presidente Sarney fez vários ataques ao Congresso
Constituinte. A Comissão de Sistematização aprovou o parlamentarismo e a
Constituinte estava demorando a definição da elaboração do mandato.
Sarney ocupou a cadeia de televisão e rádio dizendo que a Constituinte
iria colocar o País na ingovernabilidade. Depois Ulysses rebateu as
críticas. Isso no mês de julho de 1988. Como foi esse período de
tensão?}

\textbf{Jobim:} Na verdade o pessoal que compôs as lideranças na
Constituinte não possuía muita experiência executiva. Eram todos
parlamentares, professores, profissionais liberais, mas nenhum de nós
havia passado pelo Executivo. E também era um desaguar de esperanças da
Constituinte, de promessas, de desejo. E não fazíamos conta de como
pagar essa conta. E o governo também ficou silente disso durante um
período.

O problema do governo naquele momento não era discutir isso, mas o tempo
do mandato do presidente Sarney e também o presidencialismo e o
parlamentarismo. Lembre-se bem que o Sarney teria 6 anos de governo e
ele queria reduzir para cinco, enquanto o PMDB queria reduzir para
quatro. O Mário Covas e o PMDB queriam quatro.

Durante o período em que se discutiu esses assuntos o governo direcionou
a sua atenção para os cinco anos do presidente Sarney e para a questão
do presidencialismo. Eu lembro que haveria uma reunião na casa do doutor
Ulysses, que era o presidente da Câmara, e houve uma negociação política
para tentar resolver esse problema. Quem comandou isso foi o doutor
Ulysses, evidentemente, mas quem viabilizou os diálogos foi o senador
José Richa (PMDB-SP), que fora governador do Paraná e era muito amigo do
Sarney. O José Richa foi conversar com o presidente Sarney para tentar
fazer uma proposta de regimento com ele. O presidente fez a proposta que
o Richa nos trouxe para submeter ao PMDB: ele aceitava o parlamentarismo
no modelo que fora desenhado, um parlamentarismo com o Executivo mais
forte, à lá Constituição Francesa. Aceitava o parlamentarismo e,
promulgada a Constituição, ele indicaria o primeiro-ministro e o
ministério. E esse ministério só poderia ser suscetível de voto de
confiança dentro de seis meses ou 12 meses, algo assim. Ele queria um
período de estabilidade, mas também que o mandato do presidente fosse de
cinco anos. Mas o líder o Mário Covas não aceitou. Disse que o mandato
teria que ser de quatro anos.

Então nós acabamos indo para a votação, perdemos o parlamentarismo e
perdemos os quatro anos. Ganhou os cinco anos e ganhou o
presidencialismo. Lá o presidencialismo teve uma aliança que não teve
nada de partidária. Tanto é que na emenda presidencialista os dois nomes
maiores eram o senador Humberto Lucena (PMDB-PB) e o Vivaldo Barbosa
(PDT-RJ).

De novembro de 1987 a janeiro de 1988, period de composição das
comissões, subcomissões e relatorias, ficou nítida a composição
progressista. O resultado foi um anteprojeto à esquerda do plenário.
Quando houve a questão da virada regimental o senhor colocou que a
vitória do Centrão na disputa do novo regimento, o congresso
constituinte restaurou o processo democrático. E disse: se a esquerda
não tem a maioria, o que vamos fazer?!

\textbf{Jobim:} Numa coisa eles tinham razão: que 11 votos não poderia
ser maior que 250. Com apenas 11 votos se aprovava um texto na
subcomissão. Para ser excluído, esse texto precisaria de metade dos
votos da comissão e, posteriormente, metade dos votos da Comissão de
Sistematização e metade dos votos do plenario, de modo que 250 votos do
plenário, por não constituírem mais da metade, teriam menos força que os
11 votos que colocaram o texto lá na subcomissão. Então eles tiveram
razão em relação ao ``DVS'' -- Destaque para Votar em Separado.

Mas a esquerda ficou perplexa, tomou um susto. Havíamos dominado todo o
proceso inicial até a Comissão de Sistematização, onde fizemos um texto
à esquerda do plenário. Isso deu origem ao Centrão e à crise. E tivemos
que ceder porque não possuíamos a maioria: cedemos à emenda substitutiva
e ao DVS. Votamos e aprovamos a emenda do Centrão e fizemos um árduo
trabalho no texto da sistematização, apostando no fato de que no
plenario havia mais condições de vencermos em guerra de guerrilha que
numa batalha. E optamos pela guerrilha ao invés da batalha global.

Em 22 de março de 1988 o presidencialismo é aprovado com os 5 anos de
mandato para Sarney. Sobre as disputas do mandato e do sistema de
governo, o que foi feito para o governo garantir a vitória por 344 a
212?

\textbf{Jobim:} Eu darei um exemplo. Durante o processo de levantamento
dentro do PMDB para ver quem votava por 4 anos e quem votava por 5 anos,
um colega do PMDB lá do Sul veio falar comigo. Ele exibiu uma série de
cartas e manifestações do pessoal que havia eleito ele naquela região. E
o grande interesse era o asfaltamento de uma estrada. Eram cartas de
prefeitos de vários partidos, carta do diretório do PMDB local, carta do
Clube dos Dirigentes Lojistas\ldots{} E o Governo Federal garantiu o
asfaltamento desde que ele votasse pelos 5 anos. Ele me disse que não
tinha condições de votar pelos 5 anos porque o que o pessoal queria era
a estrada. Ele votou pelos 5 e fizeram a estrada. A luta foi essa. Não
vejo como corrupção. O governo ofereceu resultados para obter o voto
pelos 5 anos.

E vamos deixar bem claro: a discussão sobre 5 ou 4 anos estava atrelada
à sucessão presidencial. Se o mandato de Sarney fosse de 4 anos a
eleição seria em 1988. E quem seriam os principais candidatos? Os
personagens com mais disposição no processo Constituinte. Mas a eleição
foi em 1989 e quem venceu foi o Collor (PRN), que não tinha nada a ver
com a Constituinte.

Então sinto que na verdade o debate dos 4 ou 5 anos estava relacionado à
virtualidade de ganhar uma eleição e à ideia de que quem tivesse uma
exposição maior durante todo o processo teria mais chance na candidature
à Presidência da República.

A disputa entre parlamentarismo e presicencialismo também foi
influenciada pela temática da eleição?

\textbf{Jobim:} Também. Tanto que houve a proposta real feita pelo
Sarney a nós da maioria do PMDB: a de ser aprovado o parlamentarismo em
troca dos 5 anos de mandato. E não aceitamos. Porque o que estava por
trás era a eleição. E o que era o mais importante? O mais importante era
o periodo do mandato.

Sobre a ordem econômica, como se deu o debate histórico da definição da
empresa nacional?

\textbf{Jobim:} Aquilo foi pura e simplesmente a reprodução da lei de
reserva de informática. E algo precisa ficar claro: o PMDB era um
partido que em seu bojo havia gente da esquerda, liberais e gente da
direita. Então o PMDB nunca debatia muito a questão da ordem econômica,
tanto é que a ordem econômica acabou sendo uma ordem econômica que vinha
de trás e que depois foi reformada em 1995 e 1996 no governo Fernando
Henrique, passando pelo grande debate emocional de 1993, que foi a
revisão constitucional. Mas quando votamos na ordem econômica o fizemos
olhando para trás. Não estávamos vendo que o mundo estava mudando.
Fechávamos a economia, estatizávamos as coisas, criávamos as figuras de
privilegiamento das indústrias brasileiras -- e as industrias queriam,
as empresas queriam... porque as empresas brasileiras de capital
nacional possuíam uma variável econômica de competitividade maior que as
empresas brasileiras de capital estrangeiro. Esse debate é curioso. Mas
acabamos aprovando e enfiando na Constituição o que havia da antiga Lei
de Reserva de Mercado da Informática. No governo Fernando Henrique, em
1995, derrubamos a diferença de empresa brasileira de capital nacional
para não ter um conceito de empresa brasileira.

Com a fundação do PSDB o senhor assumiu a liderança do PMDB. Como foi
esse processo?

\textbf{Jobim:} Sou advogado e participei da Constituinte, conhecia o
debate constitucional e possuía certa formação no tema. Também possuía
formação em lógica matemática. Naquela época não existia laptop.
Controlávamos os votos com base em fichas, no papel. Eu era um
trabalhador, gostava disso. E essas foram as minhas tarefas. Eu produzia
textos. Ao surgir um problema, discutia-se -- mas eu não participava
ativamento do processo decisório do conteúdo. Quem discutia eram o
doutor Ulysses Guimarãesm o Mário Covas, Nelson Carneiro, o Richa, o
Celso Furtado e alguns governadores que às vezes participavam, como o
Pedro Simon e o Miguel Arraes. Eram o núcleo decisório do PMDB. E havia
os executores. Então eu não estava no nível da definição, mas no da
estratégia e da operação.

Houve um problema na Constituinte sobre o texto do repouso semanal, uma
regra antiga que existia desde a Consolidação das Leis do Trabalho (CLT)
de Getúlio Vargas. O pessoal da esquerda, o Plínio de Arruda Sampaio
(PT-SP), que viabilizava as pretensões e os interesses dos sindicatos,
queriam um texto dizendo: ``repouso seminal remunerado, obrigatoriamente
aos domingos''. Mas o Centrão queria ``repouso seminal remunerado na
forma de convenção coletiva'', argumentando qeue determinadas empresas
não podiam parar no domingo, como as de siderurgia. Acontece que nem a
esquerda e nem o Centrão possuíam votos suficiente para aprovarem seus
textos, então precisavam encontrar uma saída. E essa era a mnha tarefa.
Eu apresentava os textos ao doutor Ulysses e a conversa era: ``com esse
texto como está, conseguiremos quantos votos?'', ``ah, a gente consegue
100 votos''. E a gente buscava uma redação em que começava a ganhar
espaço para ambiguidades, para conseguir aprovar ele com maioria. O
texto ambíguo era o texto que formava maioria.

Havia duas forma de não enfrentar o problema: transformar o texto em
ambíguo ou jogar a discussão do problema para uma discussão adiante.
Todo mundo enviava para lei ordinária e para lei complementar aquilo em
que não coneguiu fazer acordo. E a Constituição teve muito disso, se não
a gente não aprovaria nada. No caso citado eu usei outra técnica. Fomos
negociar com o Centrão com o pessoal da esquerda. Mas a esquerda nao
abriu mão e o Centrão disse ``de jeito nenhum''. Não tínhamos condições
de aprovar coisa alguma, mas precisávamos colocar alguma coisa sobre o
repouso seminal. E inventei a seguinte frase: ``repouso semanal
remunerado, preferencialmente aos domingos''. Manteve o domingo que a
esquerda queria, com o ``preferencialmente''reduzindo a obrigação.
Atendeu parcialmente aos dois. Esta era a minha participação. Eles
decidiam a solução política e eu ia construindo junto com os outros para
esse troço ser aprovado. Estava no nível operacional e tático para
produzir.

O Mário Covas tinha como vice-líder o Euclides Scalco (PMDB-PR). E o
PSDB surgiu por três pontos: a disputa por São Paulo, com Covas,
Fernando Henrique e José Serra de um lado; a briga do Richa com o Álvaro
Dias no Paraná; e o problema em Minas entre o Pimenta da Veiga e o Nildo
Cardoso. O PSDB nasceu das divergencias nesses três estados, não havia
nada de ideologia. O Mário Covas saiu para o PSDB e eu me tornei líder
do PMDB no final da Constituinte. Sobrou para mim. Quem deveria ser o
líder era o Antônio Brito (PMDB-RS), mas ele saiu para disputar a
prefeitura de Porto Alegre em 1988, que ele perdeu. E eu não possuía
autoriade política para ser líder, tanto é que pedi para ter como vice o
Nelson Carneiro, para me dar uma estatura, porque eu ainda era um
deputado de primeiro mandato. O meu único diferencial é que eu sabia de
tudo relativo aos problemas, onde estavam as urgências, quais as
dificuldades. Mas tudo isso o Brito também sabia.

Quais foram as maiores ambiguidades da Constituinte?

\textbf{Jobim:} Não dá pra definir muito bem. Aquilo foi o que se deu
para produzir nas circunstâncias do momento. As críticas que fazem os
autores de Direito, sobre a Constituição ter 40 ou 50 emendas, é porque
depois o processo real foi ajustando a Constituição a uma situação real.
Foram a mudança na ordem econômica, que se deu em 1995, no governo
Fernando Henrique, em que trabalhei conduzindo e redigindo aqueles
textos de mexer na ordem econômica; depois vieram as alterações de ordem
social, os ajustes na Previdência Social e os problemas das reduções do
plano de aposentadoria, que na época da Constituinte todo mundo queria
dar tudo.

Se pegarmos a definição do papel das Forças Armadas na Constituição
anterior, a de 1969, e colocarmos ante a Constituição de 1988, ficou
inalterada?

\textbf{Jobim:} Não. Houve uma mudança importante. As Forças Armadas
foram eleitas, na tradição constitucional brasileira, como preservadoras
da Lei e da ordem. Então as Forças Armadas tinham como tarefa e função a
preservação da soberania. Acontece que, com isso, interpretou-se que os
militares tinham recebido diretamente da Constituição uma tarefa, sem
intermediários, que é a preservação da lei e da ordem.~ E com isso se
sentiram legitimados a intervir por vontade deles, porque supostamente a
Constituição autorizava. Em 1988 nós alteramos. É função militar a
garantia da lei e da ordem, mas quando determinada por algum chefe dos
poderes. Uma coisa é a ação militar fora do território, a defesa da
soberania contra o ataque externo. E outra coisa é a ação dos militares
dentro do território por questões de desordem interna.

\textbf{Maurílio Ferreira Lima}

*Advogado, foi deputado federal pelo PMDB por cinco legislaturas
(1967-1971, 1983-1995 e 1999-2003). Na Assembleia Nacional Constituinte
integrou a Subcomissão dos Direitos Políticos, dos Direitos Coletivos e
Garantias, dentro da Comissão da Soberania e dos Direitos e Garantias do
Homem e da Mulher; presidiu a Subcomissão dos Negros, Populações
Indígenas, Pessoas Deficientes e Minorias, dentro da Comissão da Ordem
Social; e foi suplente na Comissão de Sistematização.

\textbf{A instalação da Assembleia Nacional Constituinte se deu em que
contexto político? }

\textbf{Maurílio:} É preciso entender o contexto dentro do qual a
Assembléia Nacional Constituinte se iniciou. Em 1964 houve um golpe
militar que deveria ser temporário. A partir de 1969, com o Ato
Institucional número 5, o país foi completamente amordaçado. Durante
esses anos de amordaçamento até a anistia e a luta pela Assembléia
Nacional Constituinte, a gente vivia um mundo de sonhos e de utopia. O
mundo se dividia em duas facções absolutamente antagônicas que não eram
apenas adversárias, mas se consideravam inimigas, esquerda e direita.

A sociedade não se divide dessa forma. A sociedade tem vários matizes e
a área maior é a área cinzenta onde as pessoas pensam como querem. A
Assembléia Nacional Constituinte, nesse clima, se dividiu com uma
esquerda que queria fazer nas leis aquilo que não existia na sociedade.
Não tem lei que mude a sociedade, a sociedade muda a partir da sua
população, da sociedade civil organizada, dos sindicatos, da igreja, dos
capitalistas, dos banqueiros, dos bancários, dos padres, dos ateus e,
hoje, das lésbicas, dos homossexuais, dos casais bissexuais. Se você não
tem a sociedade querendo mudar não adianta fazer uma lei. Mas na
Constituinte nós imaginávamos que bastava botar na lei: ``acabou a
fome'' e teria comida para todo mundo.

\textbf{O discurso do ministro Moreira Alves. Por que ele seria oportuno
apenas no Chile?}

\textbf{Maurílio:} O Chile é um país diferenciado na America Latina. A
America Latina praticamente se divide em alguns blocos muito estanques.
Você tem o Brasil, que foi preservado na sua grandiosidade pela
colonização portuguesa, e você tem uma mistura, porque o português
gostava muito das crioulas, dormia muito com as crioulas e com isso
ficou um país miscigenado. E hoje você tem um Brasil onde tem mais
pessoas de cor, pessoas mais próximas da senzala, do que da casa grande,
mas um país sem grandes conflitos sociais. Se você vai na America Latina
e na Andina, é diferente.

Você vai Bolívia, Equador, Peru e o povo não fala língua de branco. Quem
fala espanhol é o branco, é o rico, é o poderoso, é elite. O povo fala
ketchua, o aramaico, o povo fala as línguas que há milênios existem
naquela região. E o índio tem uma cara diferente daquele que é
descendente do espanhol. Então se você botar num quarto um índio e um
branco você já sabe distinguir. No Brasil é muito difícil você
distinguir as pessoas, nem mais pela roupa, porque roupa no Brasil hoje
se tornou tão barato que até as empregadas domésticas saem de casa hoje
mais elegantes do que as patroas.

Mas na assembléia constituinte havia essa disposição: vamos botar na
lei. Por isso que a Assembléia Constituinte votou uma constituição que é
uma das mais extensas do mundo. E as pessoas dizem: como é que os
Estados Unidos têm poucos artigos e, no Brasil, você passa quatro ou
cinco noites lendo a Constituição e não termina de ler, muito menos de
entender. É consequência disso, de cada um querer botar um pontinho lá.
A escolha dos constituintes já foi um processo que mobilizou muito o
país. Os bancários se reuniam, indicavam um candidato, os metalúrgicos
se reuniam, indicavam um candidato.

\textbf{Quando você foi candidato em 1986, sabia que seria candidato a
constituinte?}

\textbf{Maurílio:} Sabia, mas não era ligado a nenhum sindicato. E fui
para a Constituinte dentro desse clima que falei. Eu achava como toda
esquerda achávamos que salvaríamos o mundo. Por que eu teria o direito
de botar no papel tudo aquilo que eu queria: que o salário mínimo tinha
que ser suficiente para pagar o aluguel, para botar os meninos na
escola, para comprar tênis, para comprar jeans, para comer três vezes
por dia, para viajar de férias. Eu achava que tudo isso existia. Como
meu amigo, por exemplo, Fernando Gasparian (PMDB-SP), um dos homens mais
dignos desse país, achava que bastava botar na lei que os juros tinham
que ser tabelados em 12\% por cento -- ``fora disso seria usura''. Ora,
isso não existe. Mas naquela época a gente acreditou nisso e fez uma
Constituinte que eu digo que foi um ``porre cívico''. O que é um porre
cívico? É a gente imaginar que mudava tudo no papel. Papel não muda
nada.

\textbf{Maurílio, pensava-se que o regimento interno estaria elaborado
logo em fevereiro, primeira semana seguinte, ele só veio ficar pronto em
março. Por que essa demora dos 559 constituintes? Assim que Ulysses
Guimarães assumiu a presidência, falou que tinha um projeto de resolução
provisório num regimento. Esse regimento provisório foi descartado pelos
constituintes. Então levou varias reuniões e negociações. Por que a
demora da elaboração do regimento?}

\textbf{Maurílio:} É porque as pessoas entendem que uma casa legislativa
é como uma máquina, um liquidificador. Você chega, bota laranja no
liquidificador, aperta um botão e sai o suco. E questionam: por que é
que o congresso não vota tal lei? Porque é complicado. Complicado por
quê? Uma lei de greve é complicada, porque: bombeiro pode fazer greve? A
polícia pode fazer greve? O médico no plantão pode fazer greve? O
exército pode fazer greve? Há toda essa confusão. Demora a juntar todas
as opiniões e, numa casa política, qualquer coisa só acontece quando tem
uma maioria para votar. Então eu quero ressaltar a posição do doutor
Ulysses (PMDB-SP). Se não fosse Ulysses Guimarães não teríamos feito a
Constituição. Porque ele sentava naquela cadeira e a gente ficava
impressionado: será que ele não faz pipi?! Porque ele sentava às duas da
tarde e levantava às dez da noite. A gente ia ao banheiro umas dez vezes
ele ficava lá. Ou usa fralda descartável, não sei. Era um negócio
realmente impressionante. Se o doutor Ulysses tinha um artigo para
votar, fazia o seguinte: "quem está a favor fique do jeito que está". E
você diz: "Doutor Ulysses, não ouvi". Você não ouvindo faz confusão,
avalie se ouvisse. Passa ao artigo seguinte.

Ele fazia as votações dessa maneira, porque se não fizesse assim, não
havia votação. Há um deputado que faleceu, Amaury Muller (PDT-RS), do
Rio Grande do Sul, que levantou essa questão curiosa. "Presidente, eu
estou aqui sentado na sua frente, não entendo o que o senhor diz". Ele
disse: "meu filho, o senhor sem entender faz uma confusão desgraçada,
avalie se entendesse. Não é para entender não. Ta aprovado!". E assim as
coisas foram caminhando. E se ele não fizesse desse jeito a gente não
avançava para canto nenhum. Porque a gente sentava ali no plenário e, em
cima, onde ficam as galerias, que hoje é cercado de vidro para impedir a
esculhambação -- porque era uma esculhambação, nós fazendo escolhas para
o País e um sujeito jogando cigarro, dinheiro, esculhambando. Isso não
existe em parlamento nenhum do mundo. Uma democracia é baseada na
representatividade, da proporcionalidade.

\textbf{Mas nós ainda não éramos uma ``democracia''}.

\textbf{Maurílio}: Não. Estávamos saindo de uma ditadura e queríamos
fazer o melhor dos mundos. Nessa época ainda não havia essa historia de
terceirização, inventada pela globalização. Nessa época havia os
contratados pela CLT (Consolidação das Leis do Trabalho) e os
contratados pelo regime jurídico do funcionalismo federal. Então,
lógico, eram dois tratamentos diferentes. Quem era contratado pela CLT
era regido pela lei trabalhista. Na lei trabalhista você só se aposenta
com trinta e cinco anos de contribuição e naquele salário maior do INSS
(Instituto Nacional de Seguro Social). No serviço publico você se
aposenta com o salário integral. E lá estava e se aproximou um sujeito:
``como é que pode o país viver nessa esculhambação? Eu estou trabalhando
ao lado de alguém, mas ele tem direito a uma aposentadoria integral e eu
tenho direito só a aposentadoria do INSS''.

E inventaram a palavra mais indecente e mais imoral que conheci na vida,
essa tal de isonomia. Sala igual para quem trabalha igual. Isso não
pode. É uma esculhambação. Imagine um professor -- não quero esculhambar
com Porto Velho -- de Porto Velho, lá no interior do Brasil, lá nos
confins do Brasil, e um professor de Campinas e da Universidade de São
Carlos. O sujeito que está estudando em Campinas e na Universidade de
São Carlos geralmente tem mestrado, tem doutorado, fala inglês, passa o
dia na internet, lê a revistas, se informa. O ser que está lá em
Roraima, em Porto Velho, não tem acesso a esses meios. Ele é professor,
então tem que ganhar igual ao outro. Não, meu amigo. Os professores
exercem as suas atividades em locais diferenciados. Então, essa palavra
isonomia é a maior indignidade contra o País, porque é aquela que diz
igualdade. Igualdade é historia da revolução francesa. Tratar igual as
pessoas é fazer a maior injustiça, porque as pessoas são desiguais. Se
as pessoas são desiguais eu tenho que tratá-las desigualmente. Se não eu
estou praticando uma grande injustiça. Eu posso tratar um deficiente
físico igual a um sujeito que está se preparando para as Olimpíadas? Não
posso. Eu posso atribuir a alguém que é deficiente mental os mesmos
direitos de quem tem a cuca em ordem? Não pode. Então se criou algumas
palavras que só servem para perpetuar a injustiça e a sacanagem.
Isonomia, igualdade, isso não existe. Você tem que tratar as pessoas
desigualmente porque elas são desiguais.

\textbf{Está na Constituição que todos são iguais perante a lei.}

\textbf{Maurílio:} Isso é relativo. Isso é relativo porque colocaram na
Constituição e havia exceção. Está caindo. Um sujeito se forma numa
faculdade safada, tem um título de advogado e não passa em nenhum exame
da OAB do Brasil. Só porque tem um título dá um tiro em alguém e quer
ficar num quartel folgado, com televisão de plasma ou LCD, entendeu?
Recebendo a namorada, transando a hora que quer. E o sujeito que rouba
uma galinha vai embora levar porrada no (Presídio) Aníbal Bruno. Por que
o outro tem prisão especial? Porque conseguiu fazer um curso fajuto. O
doutor é tão bandido quanto o bandido que assalta. Então são essas
igualdades que eram atenuadas também por exceções legais e, felizmente,
estamos caminhando para um País. Eu sou muito otimista. O Brasil de hoje
é muito melhor que o Brasil de dez anos atrás.

Daqui a dez anos o Brasil será muito melhor que o de hoje. As pessoas
pensam que será pior. Não. Será melhor, porque as pessoas participarão
mais, exigirão maior transparência. Você não viu o Sarney? Foi para a
televisão pedir desculpas porque disse que não sabia que recebia o
auxílio-moradia. Eu acredito na historia dele porque Sarney é um homem
muito rico e o salário de senador é ridículo. É grande para quem ganha
salário mínimo. Senador ganha dezesseis mil e quinhentos reais. Recebe
líquido doze mil. Ninguém em Brasília que ganha doze mil reais mora no
Plano Piloto. É de Taguatinga para dentro, para o meio dos bandidos.
Sarney, por ser um homem muito rico -- não sei como ele enriqueceu --
ele não vai olhar o cheque dele, ele nem olha aquilo.

É como o Henrique Meireles. Achas que Henrique Meireles quer saber do
salário dele de Banco Central, um salário de menos de dez mil reais?
Isso não existe para ele. Ele quer nem saber. Dois ou três anos depois
diz: "vê aí meu saldo se tem algum dinheiro", enquanto o pobre aqui tem
que pagar aluguel, tem que pagar a escola do menino, cortaram a luz
porque não tem dinheiro para pagar. O Sarney quando diz: "Eu não sabia",
ele não sabe porque não olha o contracheque dele, pois ele não precisa
olhar. O trabalhador, não: "descontaram dois reais aqui do café, eu não
tomei o café", e vai atrás para receber.

\textbf{FERNANDO LYRA}

*O advogado iniciou sua vida pública como deputado estadual pelo MDB
(1967-1970) e, em seguida, deputado federal (1970-1990), havendo se
licenciado para assumir o Ministério da Justiça (1985-1986). Integrou o
grupo dos "autênticos" do PMDB, partido do qual se desfiliou para
ingressar no PDT, pelo qual disputou, em 1989, a Vice-Presidência da
República na chapa de Leonel Brizola. Já filiado ao PSB, voltou à Câmara
Federal em 1993, permanecendo até 1998. Presidiu a Fundação Joaquim
Nabuco (2003-2011). Faleceu em 2013.

\textbf{Em julho de 1985, em pleno processo de transição democrática, o
presidente José Sarney convocou uma comissão provisória de estudos
constitucionais. O grupo ficou conhecido como Comissão Affonso Arinos,
``os notáveis''. Meses depois a Folha de S. Paulo publica artigo de
vossa autoria entitulado ``Constituinte: um primeiro passo''. O senhor
colocou que ``a comissão desempenha uma missão das mais patrióticas e
cuja importância é tão grande quanto a qualidade do seu produto''. Em
dezembro o jurista Miguel Reale, integrante da comissão, se queixava das
crtíticas, que diziam que o grupo fazia reuniões e discutia por horas,
mas não se chegava a qualquer conclusão. Qual foi a importância da
Comissão Affonso Arinos para o processo de transição democrática?}

\textbf{Lyra:} A ideia da Comissão foi do doutor Tancredo Neves
(PMDB-MG). Ele ainda candidato, nós conversamos muito, inclusive
chegamos a convidar o ex-ministro Affonso Arinos (PFL-RJ) para ser o
presidente da Comissão. E convidamos algumas personalidades
importantíssimas no contexto. Mas, em função do desaparecimento de
Tancredo, somente em junho é que foi instalada a Comissão.

E essa comissão teve um trabalho muito bom. Mas poderia ter sido melhor
se ela fosse mais enxuta, no termo popular -- com menos componentes e
mais ideias para o processo. Acontece que com o desenrolar da
Constituinte houve alguns problemas políticos sérios. E esses problemas
políticos fizeram com que a comissão não tivesse a participação que
deveria ter.

Na minha opinião deveríamos ter uma Constituinte na qual os deputados e
vereadores fossem eleitos para a Constituição. Mas não foi assim. Os
deputados que participaram das eleições de 1986, eles não tinham como
meta a Constituinte. Dos deputados federais eleitos em 1986, eu não me
lembro de algum que se elegeu como deputado Constituinte. Elegeu-se como
deputado federal. E mais grave foi o Senado da República: um terço do
Senado Constituinte era biônico. Foram nomeados biônicos.
Consequentemente não possuíam qualquer legitimidade para fazer uma
Constituinte na nova etapa do processo político que abarcou uma
democracia em curso. Então começaram também os problemas internos dentro
da Constituinte.

Nós conseguimos uma coisa excepcional. Eu lembro bem que convidei o
senador Mário Covas para ser o candidato a líder. Ele resistiu e eu
disse ``você tem que ser o candidato a líder''. E conseguimos que ele
aceitasse e ele se elegeu. E a grande contribuição do PMDB naquela
oportunidade, na Constituinte, foi a liderança de Mario Covas, um homem
da melhor categoria, do melhor passado.

\textbf{Houve resistência?}

\textbf{Lyra:} Havia resistência, mas resistências sem maiores
importâncias. Porque cada liderança queria ser o líder da Constituinte,
cada personalidade política importante queria ser o líder. Mas nós
conseguimos eleger Mario Covas, que foi um avanço extraordinário. Mas,
na briga à qual eu me referia há pouco, a Constituinte teve um problema
gravíssimo, que teve conseqüências muito serias, que foi o mandato do
presidente Sarney. O presidente Sarney teria, por direito, seis anos de
mandato. Por quê? Porque ele foi vice de Tancredo Neves e Tancredo não
conseguiu se empossar -- faleceu antes. E o Sarney assumiu. Então ele
tinha direito a seis anos. Mas começou o movimento, principalmente
quando reformou-se o governo todo. E o ministério de Tancredo, do qual
eu fiz parte, saiu em fevereiro de 1986.

Aconteceu que a maioria ou a quase maioria do PMDB queria quatro anos
para Sarney, em vez de seis. E foram para as ruas por Diretas Já. Foi o
inicio das Diretas, na Constituinte, quando fomos às ruas pregando
eleições diretas para a Constituinte, que a Constituinte decretasse a
eleição direta dos quatro anos do presidente Sarney, que seria em 1988.
Conclusão: vai para lá, vem para cá, chegou-se ao acordo de cinco anos
para o presidente Sarney. A primeira eleição de presidente da República
após a ditadura -- eleição direta -- foi realizada isolada de tudo. Foi
só candidato a presidente da República. Ninguém sabe, hoje -- eu não sei
e eu participei presidente da Constituinte, participei de eleição
posterior, inclusive como candidato a vice-presidente de Brizola
(PDT-RJ) -- e não sei o nome do partido que elegeu Pedro Collor e
Fernando Collor. Para ver como a coisa foi totalmente distorcida.

Com a eleição do Collor, toda aquela nossa luta da Constituinte viveu um
problema sério. Porque teve que abrir o impeachment e teve que assumir
um vice. Houve muitas consequências para o processo político brasileiro.

\textbf{Em setembro a comissão entregou ao Sarney o anteprojeto de
constituição. O senhor participou da cerimônia e questionou o Affonso
Arinos se ele sugeriria ao presidente que enviasse o projeto ao
Congresso. Por que o Sarney decidiu não enviar o anteprojeto para a
Constituinte?}

\textbf{Lyra:} Vem um problema sério. Quando o doutor Tancredo pensou na
Comissão Constituinte, pensou exatamente em como a sociedade brasileira
como um todo, através de pessoas escolhidas e qualificadas, de todos os
seus setores, tivesse condições de contribuir para a Constituinte. Como
o presidente Sarney não enviou o relatório ou o anteprojeto da Comissão
e o presidente da Câmara e da Constituinte, Ulysses Guimarães, também
rejeitou aquele trabalho extraordinário feito pela Comissão da
Constituição. Rejeitou porque ele não foi ouvido, muito pelo contrário.
Criou-se um problema sério. Quase deram um golpe na Constituinte,
criando o Centrão. Por que o Centrão? Porque a esquerda, da qual nós
fazíamos parte, crescera e tinha dado uma contribuição muito forte à
Constituinte. Conclusão: não tivemos o trabalho elaborado pelos juristas
e pelos componentes da Comissão aproveitado. Não tivemos o trabalho
aproveitado e tivemos um retrocesso com a formação do Centrão. O
objetivo era só anular os avanços. A tentativa foi anular os avanços na
área social e econômica da esquerda brasileira.

\textbf{O senhor defendia que a Assembleia Constituinte fosse exclusiva,
não congressual. Por que ela não foi exclusiva?}

\textbf{Lyra:} Ela não foi exclusiva por circunstâncias políticas. A
eleição do doutor Tancredo, da qual eu participei ativamente, foi uma
eleição consensual. Apesar de haver a oposição do Paulo Maluf, a eleição
de Tancredo foi consensual. Várias pessoas que antes eram PDS criaram
outros partidos, como a Frente Liberal, para, exatamente, participar do
governo de Tancredo. Então ficaria muito difícil para o presidente
Tancredo lançar aquilo que nós pregávamos, que era a Constituinte livre
e soberana, exclusiva, para que os eleitos fizessem a constituinte.
Porque no mundo todo foi assim. Quando abriu o processo democrático eu
fui para Portugal. Porque quando abriram o processo democrático, a
primeira coisa que se fazia era exatamente convocar uma nova
Constituinte para fazer uma nova Constituição. Aqui não. Foi aproveitado
o Congresso Nacional. E o Congresso Nacional, em que pese ter a
obrigação de acompanhar o processo como um todo, não estava preparado
para ser um Congresso Constituinte. Eu volto a afirmar: ninguém defendeu
essa ou aquela questão, na eleição, em função da Constituinte.

\textbf{Em 1987 houve duas disputas. A da Presidência da Assembleia
Constituinte foi vencida por Ulysses ante o deputado Lisâneas Maciel
(PDT-RJ): 425 votos a 69. Mas a sua disputa foi para a Presidência da
Câmara Federal, vencida novamente por Ulysses Guimarães: 299 a 155
votos. Embora tenha perdido, você foi bem votado. O que representou a
candidatura de Fernando Lyra para a Presidência da Câmara Federal?}

\textbf{Lyra:} Eu fui ministro até janeiro, fevereiro de 1986 e me elegi
como deputado federal -- inclusive com uma excelente votação em
Pernambuco -- e quando eu cheguei na Câmara, de volta a um mandato
eletivo, procurei verificar quem seriam os candidatos do PMDB, que era o
voto majoritário no Congresso, que teriam condições de ser candidatos a
presidente da Câmara. E analisei líder a líder, pessoas a pessoas.
Examinei bem. E cheguei à conclusão que eu possuía condições de
concorrer bem ao processo. E começou um movimento para a reeleição do
presidente Ulysses Guimarães à Câmara. E houve uma consulta à Comissão
de Justiça da Câmara, que se pronunciou textualmente. Isso inclusive
está no livro do Raimundo Faoro, editado há pouco, o "Democracia
Traída". Ele ratifica o que eu estou falando.

Estou dizendo que Ulysses Guimarães não poderia ser candidato à
reeleição porque era proibido àquela época pelo regimento e pela
Constituição. Não podia ser candidato. Ele poderia ser candidato a
presidente da Constituinte, mas jamais a presidente da Câmara dos
Deputados. E por isso eu fui candidato. Mas Ulysses manteve a
candidatura dele, violando a Constituição. Para mim é imperdoável, em
que pese eu ter uma admiração grande por doutor Ulysses, que foi um
homem, um líder, que deu a maior contribuição à abertura política. Mas
esse fato em si é um fato sobre o qual, quando falo, fico constrangido.
Mas é verdadeiro: doutor Ulysses violou a Constituição para se eleger. E
por que ele fez isso? Porque ele reeleito, não era apenas presidente da
Câmara, presidente da Constituinte e presidente do PMDB. Ele era o
vice-presidente. De fato e de direito. Porque era o presidente da
Câmara.

Essa minha derrota da Câmara não foi uma derrota normal. Vou dar um
exemplo a você. Eu teria entre os 25 deputados federais de Pernambuco
daquela época, se não me engano, 19 ou 20 votos. Eu fiquei com quatro ou
cinco votos. Por quê? Porque havia toda a pressão de um governo novo,
onde havia vários interesses. Foi não para me derrotar, mas para eleger
Ulysses. E teve votações incríveis que eu vi. Pessoas que não poderiam
votar jamais em alguém que violou a Constituinte, mas votaram. Eu
confesso que eu não me sinto derrotado da eleição. O grande problema foi
que a eleição de Doutor Ulysses não foi uma eleição autêntica. Ela
violou a Constituição daquela oportunidade.

\textbf{O jornalista Carlos Chagas colocou em sua coluna: ``Lyra não faz
concessões e acusa a sua direção, isto é, Ulysses Guimarães: não escuta
companheiros, não abre debate, não elabora qualquer estratégia para
remover o lixo autoritário incrustado na atual Constituição''.
Pergunto-lhe: era muito poder nas mãos de Ulysses Guimarães? Após a
disputa houve retaliação por parte do Ulysses para/com a sua atuação na
Assembleia Nacional Constituinte?}

\textbf{Lyra:} Não houve retaliação porque não tinha motivo algum. Pelo
contrário. Ele venceu as eleições, me derrotou fragorosamente, por 140
votos. O problema todo não foi a eleição, foi a forma de ele se eleger.
Confesso a você que eu não tenho mágoa, não. Apenas relato porque tenho
a obrigação, inclusive título do livro que eu lancei há pouco, de dizer
o que eu sei. Disso eu sei exatamente o que aconteceu. Em função disso é
que eu fiquei com o pé atrás e muito preocupado, porque o doutor Ulysses
teve uma influência muito grande no governo Sarney, porque ele era tudo
isso que a gente falou. Por exemplo, o ministério da Fazenda. Doutor
Tancredo dizia e disse a mim: ``Ministério da Fazenda é onde o
presidente da República tem que colocar alguém que é como se fosse ele
lá dentro''. E botou Francisco Dornelles (PFL-RJ), que era, além de
sobrinho, político, como sendo o ministro da Fazenda. E ele me escolheu
ministro da Justiça porque ele me dizia que o ministro da Justiça é o
como ele foi no governo de Vargas. No governo de Getúlio Vargas e no
governo de João Goulart, ele foi um coordenador político.
Consequentemente eu seria também um coordenador político. O que houve
foi que Ulysses trouxe Funaro de São Paulo, porque São Paulo não poderia
ficar sem o ministério da Fazenda.

Quem indicou o Funaro foi o presidente Ulysses Guimarães. Quem tinha
influência absoluta no governo era Ulysses Guimarães. Eu me lembro bem:
eu estava concedendo uma entrevista ao Ricardo Sette para a Playboy, no
final de 1985, mais ou menos no início de dezembro e, na hora que nós
estávamos falando, saiu uma reportagem de Pedro Simões saindo da casa do
presidente Ulysses Guimarães em Brasília e dizendo que era importante a
renovação e a mudança do ministério a partir de fevereiro. Mas não era
isso, porque sairíamos de qualquer maneira até maio para sermos
candidatos a deputados, senadores etc. Foi Ulysses quem fez com que nós
nos antecipássemos, porque o plano econômico era de Funaro, que Ulysses
colocou lá: o Plano Cruzado. Então a coisa era toda engrenada para que
Ulysses continuasse a ser exatamente o que foi: o grande operador do
Governo Sarney.

\textbf{Na Assembleia Nacional Constituinte, após a instalação das
comissões e subcomissões, o Ulysses e o Bernardo Cabral (PMDB-AM)
rejeitaram a proposta do senador José Richa (PMDB-PR) de suspender os
trabalhos da Constituinte para que se discutisse demandas conjunturais
referentes à crise econômica e política. A crise do governo Sarney
atrapalhou o progresso da Assembleia Nacional Constituinte?}

\textbf{Lyra:} Não digo que a crise atrapalhou. O grande problema é que
nós teríamos que ter um governo para que seu objetivo fundamental fosse
uma nova Constituição, para que através dela nós pudéssemos fazer aquilo
que doutor Tancredo sonhou em fazer. Eu me lembro bem: ``não se paga a
dívida com a fome do povo brasileiro''. Doutor Tancredo disse muito
isso, enfaticamente, e outras coisas mais. Mas a verdade é que o Governo
do presidente Sarney, do qual eu fiz parte durante 12 meses, foi como se
não tivesse acontecido nada antes. E foi o governo que fez a transição
democrática. Então a Constituinte era fundamental. Quando o nosso grande
companheiro José Richa pediu exatamente isso, é porque ele viu que a
Constituinte ao invés de dar, estava tirando condições de o governo,
àquela época, fazer alguma coisa. Então ficou muito confuso o quadro
daquela oportunidade. Entre o que era Congresso, o que era Constituinte,
se era legítimo, se não era legítimo. Foi uma discussão muit séria e
profunda.

\textbf{Sobre a formação dos grupos -- progressistas, conservadores,
Centrão --, como ficaram os partidos? Como se deram os acordos nos
momentos de votação?}

\textbf{Lyra:} Sobre o problema partidário eu tenho uma teoria, que
sempre foi a minha opinião: é a de que nunca existiu um partido no
Brasil e nem existirão jamais. Para mim é um faz de contas. E já na
Constituinte várias pessoas liberais votavam conosco nas teses de
esquerda e companheiros nossos conservadores votavam com o Centrão na
votação de teses conservadoras. Então os partidos não funcionavam como
expressões da Constituinte. Os partidos funcionavam, sim, na questão das
eleições das lideranças, na eleição da mesa da Constituinte. Mas jamais
na ação de cada deputado. Isso não existiu.

\textbf{Durante os 20 meses de Assembleia Nacional Constituinte o PMDB
viveu a dúvida de ``ser ou não ser governo''. A questão partidária
exercia ou não influência nas votações em plenário?}

\textbf{Lyra:} O PMDB sempre foi dividido. Quando eu cheguei lá, em
1971, formamos eu, Francisco Pinto e Alencar Furtado, o grupo dos
autênticos, que depois cresceu muito. Em Pernambuco entrou Marcos Freire
e outros companheiros. Fizemos um grupo tido como radical, mas era um
radical dentro da constitucionalidade da época. Desde aquela época
estamos lutando. Tivemos candidato contra Ulysses, internamente. Nós
lutávamos sempre e essa coisa não parou. Até que hoje, 2009, o PMDB é
uma capitania hereditária. O que é capitania hereditária? Cada estado
tem o seu dono. O PMDB daqui não tem nada a ver com o PMDB de lá e nem
de acolá. Não existe partido nacional. Quem é o líder do partido do
PMDB? Eu não conheço.

\textbf{O senhor lembra de alguma disputa entre grupos envolvendo
diretamente você?}

\textbf{Lyra:} Eu lembro bem do episódio sobre o qual me referi há
pouco, sobre o mandato do presidente Sarney. Foi uma luta interna muito
forte. Porque estava em jogo se haveria mais um ano de Governo ou dois
anos. Foi uma disputa seríssima e cada um votou de acordo com a sua
opinião. Não houve nenhuma posição partidária. Cada um tinha a sua forma
de pensar e votou com o seu pensamento.

\textbf{O mandato presidencial foi uma derrota do campo progressista?}

\textbf{Lyra:} Foi. À época foi. Porque nós queríamos eleição após 4
anos de mandato. Tanto que fizemos a campanha das Diretas Já. Realizamos
um comício em Caruaru pedindo realização de eleições diretas após quatro
anos de Sarney, que seria em 1988.

\textbf{Durante o primeiro turno no plenário o Centrão obteve a vitória
da reforma regimental e das procrastinações, que permitiram que eles
reunissem assinaturas. Como foi a movimentação do Centrão para a coleta
de assinaturas? A Folha de S. Paulo noticiou que a estrutura
governamental amplamente foi utilizada para a coleta de assinaturas em
favor do Centrão. O senhor lembra desses fatos?}

\textbf{Lyra:} Eu lembro que houve um movimento muito grande de
procrastinação, porque havia algumas teorias que eram naturais que
viessem para a votação na Constituinte. E o Centrão, a direita, não
queriam. Então restou procrastinar. Hoje eu não sei exatamente quantos
têm, mas vejamos quantos artigos ainda hoje não foram devidamente
atualizados, regimentados. Há centenas de artigos aí que nunca tomaram
conhecimento dessa situação. Enquanto nós imaginávamos para o futuro, o
pessoal pensava na permanecia da questão. A maioria do Centrão não
pensava no Brasil, numa nova era, em mudar. Não. Queriam que continuasse
mais ou menos a mesma coisa, contanto que eles fossem governo.

\textbf{Como foram as discussões em torno dos temas de recursos
minerais, reforma econômica e mandato presidencial?}

\textbf{Lyra:} As discussões do setor técnico eram discussões que, eu
confesso a você, eu não participava ativamente, porque eu não possuía
conhecimento. Eu participei firmemente da questão política. E há algo
muito interessante e incrível na constituinte: é que toda a formação em
política da Constituição foi destinada para ser um regime
parlamentarista. Quando chegou na hora ``H'' do voto, o presidencialismo
ganhou. A medida provisória é uma medida de regime parlamentarista,
jamais do presidencialismo. É visível como a coisa foi difícil na
Constituinte. Tanto que colocou-se depois em plebiscito e o
parlamentarismo perdeu, já em função do novo presidente, da questão que
estava em jogo, que era o poder.

\textbf{As vitórias progressistas na Constituinte só começaram a
acontecer quando o relator e o líder do PMDB adotaram posições
coincidentes. Como isso repercutiu entre os conservadores?}

\textbf{Lyra:} Aconteceu da mesma forma quando do problema do Centrão.
Em algumas circunstâncias se uniram a nós alguns liberais e pessoas que
são ``do outro lado'', mas votaram conosco em algumas questões que eram
muito avançadas no processo político brasileiro. Mas há coisas
incríveis. Não da Constituição, mas faz parte do processo: a Lei de
Segurança Nacional de hoje é a mesma da ditadura. E nós como ministros
-- e José Paulo Cavalcanti Filho como Secretário Geral -- fizemos uma
lei em defesa, uma lei democrática. E está na gaveta do presidente há
exatamente 22 anos. E ninguém sabe onde está. E era exatamente uma lei
democrática, não uma lei como a que ainda está em vigor.

\textbf{A nossa Constituição foi a possível de ser feita ou a ideal?}

\textbf{Lyra:} O que nós fizemos foi o possível naquela oportunidade.
Como nós estávamos ansiosos demais por várias questões sociais e
econômicas, houve um exagero na Constituinte de detalhar determinadas
coisas, que tiveram que ser reformuladas depois. A Constituinte deveria
estar, caso ela fosse livre e soberana como nós imaginávamos, acima das
questões menores. Porque as questões menores não se resolvem pela
Constituição, mas pelos decretos, pela continuidade do regimento
administrativo. Lembro muito bem de um companheiro nosso, Fernando
Gasparian (PMDB-SP), uma figura extraordinária, que criou um artigo
estabelecendo em 12\% os juros anuais. Como é que você pode numa
constituinte determinar os juros? Como? Mas era o sonho dele de evitar a
inflação. Isso é impossível de fazer. Então entramos em muitos campos
que não teríamos que entrar, em função daquela ansiedade de mudar as
coisas.

\textbf{Como foi a participação do estado de Pernambuco, dos movimentos
sociais e terceiro setor local no processo da Assembleia Nacional
Constituinte?}

\textbf{Lyra:} Pernambuco é um estado privilegiado porque sempre teve
bancadas muito participantes, muito boas. E comparandos a outros estados
nós somos privilegiados pela qualidade dos nossos representantes. Na
Constituinte, por exemplo, foi uma participação muito grande, muito boa,
efetiva, que resultou em coisas muito importantes. Eu acho que
Pernambuco sempre foi e continuará sendo um líder nos avanços
democráticos brasileiros.

\textbf{Em meio às discussões sobre os 30 anos da Lei da Anistia, qual a
sua opinião sobre a essa lei que foi tão debatida na Assembleia Nacional
Constituinte?}

\textbf{Lyra:} A anistia decorreu daquilo que eu sempre falei. Como
houve um governo de consenso, não um governo de oposição absoluta
assumindo em 1985, a questão da anistia foi tratada de forma absoluta.
Anistiou todo mundo. Tá certo? Tenho minhas dúvidas até hoje. Mas à
época não havia outra condição. A anistia ampla beneficiou muitas
pessoas que não deveriam ser perdoadas jamais. Mas foi assim em função
das circunstâncias.

\textbf{ROBERTO FREIRE}

*Advogado oriundo do movimento estudantil, Freire tenta sorte na
política pela primeira vez em 1972, sendo derrotado na disputa pela
Prefeitura de Olinda. Por Pernambuco foi deputado estadual (1975-1979) e
federal (1979-1995), além de senador (1995-2002). Mudou-se para São
Paulo, estado pelo qual se elegeu deputado federal (2011-2018). Iniciou
sua carreira política no MDB, posteriormente PMDB, se desfiliando em
1985 para ingressar no PCB, onde permaneceu até 1992, ano em que fundou
seu atual partido, o PPS.

​

\textbf{Dr. Roberto Freire, no final de 1985 travou-se um grande debate
em torno das duas espécies de Assembleia Nacional Constituinte:
exclusiva ou congressual. Quando se instalou a Constituinte, o senhor e
o deputado Plínio de Arruda Sampaio (PT-SP) levantaram, durante a
sessão, a questão da impugnação dos 23 senadores eleitos em 1982,
durante a ditadura. Contraditaram o senador eleito em 1982 Fábio Lucena
(PMDB-AM) e o deputado Gastone Righi (PTB-SP). Eles alegaram que os
senadores tinham poderes constituintes derivativos. Por que o
entendimento do PCB e dos partidos de esquerda na época era que esses
senadores não deveriam participar da ANC?}

\textbf{Freire:} A ideia de Constituinte exclusiva ou congressual não
foi grande polêmica. Havia a ideia de que fossem eleitos os
constituintes, elaborada a Constituição, promulgada e depois dissolvido
o Congresso e eleito outro. Isso seria a chamada Constituinte exclusiva,
mas não foi feito. Houve a convocação da Constituinte e, ao mesmo tempo,
a eleição do Congresso ordinário.

Como Congresso ordinário, um terço do Senado permaneceu, porque o Senado
não renova de quatro em quatro anos, mas um terço e posteriormente dois
terços. Ocorre que 23 senadores foram eleitos no pleito anterior. Entre
esses senadores havia figuras que ajudaram e muito o que chamamos de
campo democrático progressista, a esquerda. Um deles depois foi
presidente, o Itamar Franco (PMDB). Eu sempre tive uma ligação muito
forte com ele do ponto de vista político e ele ficou um pouco sentido.
Eu disse: ``não é nada pessoal, mas uma colocação do ponto de vista
político''. Eu e o Plínio de Arruda Sampaio levantamos essa questão de
ordem, mas sabíamos que seríamos derrotados. Era apenas para constar que
havia senadores que iriam participar do Congresso Constituinte mas não
foram eleitos como constituintes.

\textbf{No primeiro momento o PCB, o PT e o PCdoB formaram um bloco de
esquerda. Essa medida foi importante para a coesão no processo decisório
da Assembleia Nacional Constituinte?}

\textbf{Freire:} Não foi só esse bloco. Foi bem mais amplo do que os
partidos referenciados de esquerda. Houve o PSB e o PDT, que eram também
parte da esquerda, e um amplo setor no PMDB que foi de fundamental
importância, inclusive para que a esquerda tivesse um peso muito grande
na Assembleia Nacional Constituinte. Um dos aspectos fundamentais para
que você tivesse uma Constituição bem mais avançada foi o setor de
esquerda do PMDB, liderado por Mário Covas e Fernando Henrique Cardoso.
Inclusive foi ele o autor do regimento que criou uma forma de elaboração
que fugia a uma tradição do Brasil, das elaborações constitucionais a
partir de um projeto inicial.

\textbf{Quais foram as vantagens e as desvantagens deste novo modelo de
Constituinte?}

\textbf{Freire:} Ao escolher esse mecanismo de articulação com a
sociedade civil para a elaboração constitucional, se rompeu com uma
tradição brasileira em que o projeto foi feito muita vezes por força
externa ao Congresso. O Tancredo, ainda nas eleições, quando eleito
indiretamente pelo colégio, idealizou a comissão Affonso Arinos com
alguns juristas que elaborariam o texto constitucional. Isso eu sei
porque conversei com ele sobre a legalização do Partido Comunista
Brasileiro e ele disse ``Roberto, aguarde uma Assembleia Nacional
Constituinte''. O Tancredo era muito prudente, um democrata firme, mas
achava que nós não deveríamos ser legais antes da convocação da ANC e
pretendia convocar entregando o projeto da Affonso Arinos. Com a morte
de Tancredo, nós conseguimos a legalização em maio, antes de convocação
da Constituinte. Com o Sarney a ideia da Comissão Affonso Arinos perdeu
força, então veio essa proposta de começarmos com comissões gerais
subdivididas por temas. Com isso houve uma presença a cada dia maior da
sociedade civil. Diariamente, no Congresso, era uma avalanche de
manifestações, de grupos, de sindicatos, organizações. Uma sociedade em
ebulição, presente nos trabalhos, trazendo as suas contribuições. Isto
foi a base fundamental para que no final dos trabalhos se tivesse uma
constituição bem mais avançada. A Constituição começou com as
reivindicações mais radicais, vindas da sociedade, que cada dia trazia a
sua contribuição, com afirmação total daquilo que representava. Com as
várias comissões temáticas houve um corte no radicalismo próprio da
sociedade civil, mas, ainda assim, o avanço foi tamanho na
sistematização que foi criado o Centrão. Nossas derrotas não foram
suficiente para fazer uma Constituição conservadora e atrasada. Pelo
contrario: nós temos uma constituição muito avançada.

\textbf{Em março a liderança do PFL ficou com José Lourenço. No PDS
Amaral Neto, no PDT Brandão Monteiro, no PTB Gastone Righi, no PT Luiz
Inácio, no PL Adolfo Oliveira. A única alteração foi no PMDB: Luiz
Henrique por Mário Covas. O Dr. Ulysses apoiava Luiz Henrique e Mário
Covas era apoiado pelo conjunto progressista do PMDB. Esta foi uma
primeira disputa entre Ulysses Guimarães e Mário Covas?}

\textbf{Freire:} Não houve muito essa disputa. Ulysses pairava um pouco
acima. Acho que ele entrou mal nessa disputa. Eu não participava do
PMDB, mas tinha uma relação muito estreita. Foi uma disputa entre
setores mais avançados do PMDB, não uma disputa entre um setor mais à
esquerda e outro mais à direita. Foi uma disputa de influência dentro do
PMDB, porque Luiz Henrique tem uma trajetória no PMDB desde o grupo
autêntico, também vinculado ao setor mais progressista, mas ele era mais
ligado a Ulysses do que a Covas. Houve uma disputa meio paulista. De
fora, nós torcemos por Covas.

\textbf{Na partilha dos cargos, ocorreu um grande acordo entre PMDB e
PFL, orquestrado pelo líder do PMDB, Mário Covas. Os partidos pequenos,
tanto de direita como de esquerda, reagiram. O senhor afirmou, na época,
que o PMDB e o PFL pretendiam usar o critério da proporcionalidade para
esmagar os partidos pequenos. Como ficou o PCB na partilha?}

\textbf{Freire:} Ficou ótimo. O que eu disse depois foi atendido
plenamente. Acabou se dando um ``desequilíbrio ao contrário''. Como os
partidos de esquerda eram pequenos e em número muito maior, quando se
quebrou a proporcionalidade nós saímos em vantagem. Nas comissões
temáticas não houve proporcionalidade. Os pequenos partidos tiveram
presença em todas. Nós vencemos essa questão porque Covas e o PMDB
garantiram a nossa presença, a ponto de você ter na comissão de
sistematização pequenos partidos que não teriam acesso pela
proporcionalidade. Nós (PCB), o PCdoB e o PSB fizemos parte da comissão
de sistematização. Tanto é que vencemos quase tudo. E se fizer o
levantamento dos membros do PMDB na comissão de sistematização, em sua
grande maioria foram, depois, formar o PSDB, que era a ala à esquerda no
PMDB. No final dos trabalhos da comissão de sistematização -- isso é um
dado importante -- a direita foi toda embora, ficando apenas José
Lourenço (PFL-BA), Luiz Eduardo Magalhães (PFL-BA) e Ricardo Fiúza
(PFL-PE), só para votar e dizer que estavam discordando do que a maioria
decidia. Eles começaram a perceber que perdiam todas. Tudo isso deixou
de existir quando foi para o plenário. Veio o Centrão e começou a
desbastar, como a gente dizia. Começamos a perder em algumas propostas.
Talvez o mais significativo tenha sido a perda do parlamentarismo, que
havíamos ganho na sistematização e foi derrotado no plenário pelo
sistema presidencialista. Essa foi talvez a grande derrota da Assembleia
Constituinte.

\textbf{Após esse acordo de Mário Covas e José Lourenço, o PFL se deu
conta da vantagem dos progressistas. A revista ``Veja'' publicou uma
matéria dizendo ``esquerda ganha os cargos das comissões''. Foi uma
grande estratégia de Covas que, no primeiro momento, algumas pessoas não
entenderam. E a maioria dos relatores que o PMDB indicou era
progressistas. Na sua opinião o poder de agenda na mão dos progressistas
foi maior?}

\textbf{Freire:} Era o normal. O líder do PMDB era o Mário Covas e o
PMDB possuía mais de dois terços da Assembleia Nacional Constituinte e,
como maioria, coube a ele indicar. E não era o uso da proporcionalidade.
Como possuía o maior partido, que era o PMDB, e um líder progressista,
evidentemente haveria essa configuração nas indicações.

\textbf{O Movimento de Participação Popular garantiu as emendas
populares. Vários segmentos sociais de diversos setores estiveram
presentes. A Folha de S. Paulo publicou, no dia 21 de fevereiro de 1987,
que dois mil lobistas tentavam persuadir os constituintes. Nessa
reportagem senhor afirma que a sua crítica era à ação dos lobistas, que
não antecipava o seu voto e que os grupos econômicos sabiam de tudo o
que sairia da Constituinte. Como foi combater esses grupos empresariais?
O senhor considerava ilegítima a atuação dessas corporações?}

\textbf{Freire:} Claro. Eu não estou querendo desqualificar o lobby.
Havia lobby de todos os lados: o dos poderes econômicos tem um know-how
muito maior do que os outros, porque já estava presente antes da
Assembleia Nacional Constituinte, ao passo que muitos dos movimentos
sociais surgiram no processo de democratização, como forma organizada de
pressão no Congresso e na Assembleia Nacional Constituinte. Então o
trabalho destes muitas vezes não era tão eficiente quanto o daqueles que
possuíam uma experiência maior. Mas o lobby mais forte que tivemos foi o
do Poder Judiciário. Foi o mais forte porque era o mais estabelecido.
Foram poucas as mudanças no Judiciário. Eu lembro do papel desempenhado
por Plínio de Arruda Sampaio (PT-SP) e Egídio Ferreira Lima (PMDB-PE),
que trabalharam muito nessa questão para tentar aprofundar, ampliar,
democratizar o debate. Eu apresentei emenda para o ministro do Supremo
Tribunal Federal ter mandato, para a corte ter melhor definida a sua
competência, diminuir aquilo que vem para o Supremo de todas as formas,
com recursos extraordinários, e deixá-los cuidando mais da questão
constitucional.

\textbf{A ANC aprovou o fortalecimento do Congresso, visto que ele
perdera suas prerrogativas constitucionais durante a ditadura. Com a ANC
ele passou a aprovar projetos de lei e convocar ministros. Foi fácil
aprovar o fortalecimento do Congresso Nacional?}

\textbf{Freire:} Foi e acho que exacerbamos. Colocamos na Constituição
coisas que ficariam melhor na legislação ordinária. Mas foram postas
para impedir uma presença maior do Executivo através do seu veto.
Inclusive algumas questões trabalhistas, engessando aquilo que poderia
avançar. Uma delas é a questão da jornada de 44 horas. Quem sabe não
estivesse em 40 horas ou menos. Mas 44 horas parecia um avanço naquele
momento. Hoje começa a ser utilizado pelos setores que querem manter um
maior nível de exploração. Naquele momento era a ideia de você trazer
para o Congresso aquilo que fora usurpado durante o período ditatorial.
Mas houve uma ironia: aquilo que parecia ser um avanço, a chamada a
Medida Provisória, que é um instrumento próprio do parlamentarismo,
gerou uma distorção que existe até hoje porque perdemos o
parlamentarismo. E não é só a medida provisória. Existem muitas
disposições que eram apropriadas ao parlamentarismo , mas o
presidencialismo venceu no plenário.

\textbf{A comissão de sistematização teve 93 membros dos 559 e definiu o
texto base a ser submetido ao plenário. Dentro dela houve alguns
momentos de agravamento das tensões políticas e ideológicas. Quais foram
os principais impasses?}

\textbf{Freire:} Aquele foi um momento em que vários fatores foram
importantes, como a presença de Mário Covas na liderança do PMDB, a
indicação dos membros da comissão e a quebra da proporcionalidade com a
presença de todos os pequenos partidos. Isso deu ampla maioria para os
chamados setores progressistas, com as indicações do PMDB. Apesar haver
certo recuo em relação a algumas das propostas radicalizadas do
movimento social, no geral ela manteve os avanços e gerou grandes
embates. Talvez o maior impasse tenha sido com a UDR no capítulo da
reforma agrária. O impasse foi de tal ordem que não houve nem emenda
para colocar no texto constitucional. E surgiu -- e tive uma
participação nisso -- a chamada emenda aglutinativa, que aproveitava
todas as emendas e construía uma nova, com o apoio daqueles que tinham
assinado a emenda original. A emenda aglutinativa foi a saída para
preencher aquilo que era uma lacuna. E isso foi algo fundamental para
dar mais dinamismo aos trabalhos. O momento de mais de choque e impasse
na Comissão de Sistematização foi o da reforma agrária, mas houve
outros, como na discussão do parlamentarismo ou presidencialismo.

\textbf{Como se deu o surgimento dos grupos suprapartidários e, quando
eles surgiram, foi possível o diálogo entre opostos?}

\textbf{Freire:} Só foi possível se elaborar a Constituição porque havia
diálogo. Não se elabora uma constituição, mesmo com maioria qualificada,
sem ter o mínimo de consenso. Poderia até fazer se houvesse um PMDB
homogêneo, porque o PMDB tinha dois terços do Congresso, mas não foi o
caso. As eleições de 1986, em cima do chamado ``êxito'' do Plano
Cruzado, deram ao PMDB uma ampla maioria no Brasil. Mas o PMDB era uma
frente, não um partido. Tanto é que você tinha o Roberto Cardoso Alves
(PMDB-SP), representante do setor bem mais conservador, e os setores de
esquerda, que foram a formar o atual PSDB. Era preciso ter capacidade de
articular as várias forças políticas. Contarei um fato interessante: o
Virgílio Távora (PDS-CE) era chamado de um dos coronéis do Ceará, mas
tinha na questão da reforma agrária a posição mais avançada. Não da sua
concepção, mas do ponto de vista do texto constitucional. A esquerda
queria fazer desse capítulo uma declaração de intenções. E eu disse que
aquilo era perigoso, porque declarar o que é uma propriedade produtiva
ensejaria uma regulamentação e poderia paralisar. Portanto deveria
deixar aquilo para a lei ordinária. Colocar na Constituição apenas o
instrumento fundamental da desapropriação por interesse social. E eu
dizia: ``talvez o texto mais interessante seja o texto da própria
ditadura'', que era mais enxuto. E quem apresentou uma emenda dessa foi
o Virgílio Távora. E eu dizia: ``essa emenda é a melhor, porque a gente
não cria nenhum problema e temos um instrumento fundamental que é a
desapropriação por interesse social e pagamento de títulos da dívida
agrária. Não precisa fazer declaração de intenção''. Mas fui derrotado
na reunião das esquerdas, na CNBB (Confederação Nacional de Bispos do
Brasil), porque eu defendi que o capítulo deveria ser o mais enxuto
possível. A partir daí criou-se tanto problema sobre o que é propriedade
produtiva, se a propriedade pode ou não ser desapropriada. Gerou-se o
famoso buraco negro e não chegamos a acordo nenhum, porque o Ronaldo
Caiado, líder da UDR, desempenhou um papel muito forte articulando os
ruralistas.

\textbf{Como foram os embates entre os grupos contra e a favor da
reforma agrária?}

\textbf{Freire:} Naquele momento houve choques inclusive de
manifestantes nos corredores e salões do Congresso Nacional. Não houve
capacidade para conversar com setores que não eram da esquerda, mas que
poderiam nos ajudar numa concepção de constituição melhor. E isso se
repetiu no capítulo da reforma urbana: pessoas que não eram de nenhum
partido de esquerda, como Lúcio Alcântara (PFL-CE), que foi prefeito de
Fortaleza, possuía uma boa visão da questão urbana e ajudou muito nesse
capítulo. Era necessário romper certas barreiras, formar alianças que
fariam avançar o projeto, mesmo fora das estruturas partidárias. Se eu
fosse falar para a direita sobre reforma agrária, por exemplo, ela iria
entrar em choque. Mas se eu fosse através do Virgílio, eu poderia ter
uma capacidade maior de diálogo. Se não tivéssemos feito isso nós
teríamos perdido muito tempo, energia e teríamos confrontos maiores. Mas
foi uma preparação para os dois grandes blocos que surgiram no plenário:
o Centrão e o bloco progressista.

\textbf{A anistia se deu em 1979 e a Constituição legitimou e ampliou a
lei. Como foi a discussão e qual o papel das Forças Armadas?}

\textbf{Freire:} Na questão da anistia nós fomos derrotados. Na anistia
de 1979, e dessa eu participei como membro da Comissão de Anistia, houve
uma briga muito grande também no campo da esquerda. Defendíamos que a
anistia fosse ampla, geral e irrestrita, mas a proposta do governo
Figueiredo excluía quem havia participado do que eles chamavam de
``crimes de sangue'', da luta armada. Em 1979 só havia MDB e Arena. Mas
dentro do MDB já havia os vinculados ao PcdoB e à formação do PT.
Cristina Tavares (MDB-PE), por exemplo, foi muito atuante e só aceitava
a anistia ampla, geral e irrestrita, assim como Teotônio Vilela
(MDB-AL). Eu lembro de um debate em que disse que iriamos votar em
qualquer que fosse a anistia. Como é que eu poderia justificar votar
contra uma anistia que traria Luiz Carlos Prestes, Francisco Julião,
Miguel Arraes e Brizola?! Não seriam todos. Mas anistiando esses, eles
voltam para ajudar na anistia dos outros. Este foi o meu discurso. E
ganhamos.

Fui para a penitenciária de Itamaracá, onde vários amigos estavam
presos. Era dramático porque a anistia que nós votaríamos não os
atendia. Eu lembrava que na Espanha as anistias foram extremamente
mitigadas. Alguns votaram contra e têm dificuldade de justificar hoje.
Logo depois de aprovado, eles encontraram uma forma meio brasileira para
soltar todos os presos: por diminuição de pena do Superior Tribunal
Militar. Foi um ``jeitinho brasileiro''. Eles diminuíram as penas e
soltaram todos, mas não estavam anistiados. Na Constituinte avançamos
até os comunistas de 1935, que não eram anistiados no Brasil. E só
perdemos numa, que era a anistia dos fuzileiros navais. A Marinha fez um
drama, argumentou que era a aplicação do regimento disciplinar e não
havia perseguição política nesse caso. Mas é claro que havia. Fomos
derrotados no voto, e foi a única derrota.

\textbf{A Constituinte não foi fácil. Foram dois meses para aprovar o
regimento interno e, depois, no dia 10 de novembro, o Centrão apresenta
a sua proposta de mudança regimental, aprovada no dia 5 de janeiro. Por
que o ``centrão'' decidiu propor a mudança regimental?}

\textbf{Freire:} Na Assembleia Nacional Constituinte havia um setor que
não era de esquerda, mas era democrático e havia lutado conosco na
resistência contra a ditadura. Devíamos ter diálogo com eles porque só
seríamos maioria se contássemos com eles nesse processo. Por isso surge
o grupo do consenso, numa tentativa de articular algo suprapartidário
para formar maioria e enfrentar o Estado e os setores mais conservadores
e mais à direita. Precisávamos ter abertura suficiente para trazer
aqueles que não pensam como você, mas podem se aliar à sua luta. Às
vezes se perdia o foco por uma certa radicalização nossa. E num desses
momentos houve a formação do Centrão. Quando veio da sistematização um
texto constitucional tremendamente de esquerda, avançado e incrível,
eles alegaram que a esquerda havia tomado conta da Assembleia Nacional
Constituinte e conseguiram atrair alguns setores. Isto a direita viu,
mas alguns setores da esquerda não conseguiram perceber nem ao final dos
trabalhos. Não vamos esquecer que o PT, através de Lula, votou contra o
texto constitucional.

\textbf{O PCB e os outros partidos que votaram pela aprovação da
Constituição o que pensaram sobre o posicionamento do PT, que votou
contra?}

\textbf{Freire:} Foi completamente equivocado. Tão equivocado que
conseguimos criar a revisão constitucional após cinco anos, para vermos
se poderíamos fazer uma revisão, tentando alguns avanços que não
conseguimos inicialmente. O projeto das disposições transitórias da
revisão constitucional era um instituto da Constituição portuguesa de
1975 que a gente adotou por uma única vez, cinco anos após a
promulgação. Isso foi uma conquista nossa. Grande parte da esquerda foi
contra a revisão. É essa a contradição que eles não conseguiam perceber.
Nós avançamos.

O PCdoB fez um discurso dizendo que era um discurso ``frankenstein''.
Haroldo Lima (PCdoB-BA) afirmou que era ``uma loucura'', porque era uma
constituição cidadã, do povo, democrática e avançada e eles não admitiam
a revisão, porque imaginavam que perderiam. O pensamento de esquerda na
Constituinte era minoritário, mas avançava e conseguiu articular,
através da política, com o PMDB. Houve incompreensão de alguns setores
da esquerda sobre os avanços que estávamos fazendo. Depois, na questão
do impeachment de Collor (PRN-AL), de que participamos todos juntos, o
PT não participou quando formamos o governo que surgiu.

\textbf{O PT teve dificuldade de dialogar com os vários grupos?}

\textbf{Freire:} Alguns membros sim. Mas o Plínio de Arruda Sampaio era
uma pessoa de diálogo. O José Genoíno ainda estava na transição, afinal,
um ex-guerrilheiro convivendo no parlamento -- depois veio a se
transformar num grande parlamentar, com capacidade de diálogo. Mas não
naquele momento. O PT possuía pessoas ótimas, mas, como um todo, era
meio contraditório. Exemplo foi na comissão temática sobre educação.
Houve até briga, cenas de pugilato, na discussão sobre o ensino
religioso. Foi um debate em que se viu a força da Igreja Católica. Eu
chamei os evangélicos e disse ``vamos nos aliar contra o ensino
religioso nas escolas, pois vocês são minoria e, se aprovarem o ensino
religioso, será da religião hegemônica. A escola tem que ser laica''.
Mas não conseguimos convencê-los. E o drama é que o PT era dividido.
Nessa disputa você tinha uma grande líder do ensino religioso, a Sandra
Cavalcanti (PFL-RJ), membro dessa comissão. Eu no debate contava com
Florestan Fernandes (PT-SP). Mas havia um setor do PT que se aliava com
a Sandra. O PT é um pouco isso. E na questão do parlamentarismo, por
exemplo, nós perdemos por o PT imaginar que Lula chegaria à Presidência
da República. E ele trabalhava só para isso. Era como se eles
imaginassem que o parlamentarismo era um golpe contra Lula, porque ele
ganharia a eleição em 1989 ou depois do impeachment, em 1994. E sempre
com a ideia fixa de não participar de nenhuma aliança.

\textbf{Todos os partidos de esquerda apoiavam o parlamentarismo?}

\textbf{Freire:} Com exceção do PDT. Leonel Brizola era presidente do
partido e tinha mesma visão de que isso iria lhe retirar a possibilidade
de ser presidente em 1989. E juntou com a discussão sobre o mandato de
Sarney, que era de seis anos e ficou em cinco. O PT e o PDT queriam
quatro e embarcamos nessa. Eu me arrependo. Seria melhor a gente ter
fechado com o Ulysses, que admitia e discutia com Sarney os cinco anos
de mandato. Com o parlamentarismo talvez nós tivéssemos um País
diferente do que temos hoje.

\textbf{Voltando a essa discussão, o Ulysses aceitava os cinco anos e o
parlamentarismo?}

\textbf{Freire:} Não admitimos fazê-lo, mas existia essa possibilidade.
Sarney não é presidencialista, tendia ao parlamentarismo. Ulysses também
não era presidencialista e estava como um árbitro naquilo. O PT e o PDT,
no campo da esquerda, defendiam o presidencialismo e quatro anos para
Sarney. Queriam tirar dois anos do mandato. Sarney usou e abusou da
articulação política. Ao final ficou não com seis anos de mandato, mas
com cinco. Nós não tivemos essa capacidade de diálogo ou de buscar o
consenso, que seria os cinco anos e parlamentarismo. Essa grande
negociação nós não fizemos. Eu e Covas lamentamos não haver iniciado um
debate desses, que poderia ser vitorioso e para o País seria
evidentemente o melhor encaminhamento.

\textbf{O PT e o PDT não votaram no parlamentarismo e nem defendiam a
tese porque vislumbravam a presidência?}

\textbf{Freire:} Claro! O erro foi nosso, não deles. Eles estavam lá com
a ideia fixa. E nós falhamos, presos à ideia dos quatro anos do mandato
de Sarney. Se tivéssemos aberto discussão com um setor ligado a Sarney,
sobre os cinco anos com o parlamentarismo, era capaz de havermos
vencido, mas não o fizemos. Não sei se iria funcionar, mas lamento que
não tive, naquele momento, a capacidade de romper com os quatro anos,
que não era importante. Um ano a mais para a conquista do
parlamentarismo talvez valesse a pena. Mas ficamos com o PT e o PDT nos
quatro anos, embora eles não tivessem ficado conosco no parlamentarismo.
Perdemos o parlamentarismo e os quatro anos. E os presidencialistas
afirmaram esse sistema que causa esse grande mal a toda a América
Latina, que é uma consequência direta do absolutismo, dos monarcas, dos
reis. Todos os países mais democráticos e desenvolvidos são
parlamentaristas. A única exceção são os Estados Unidos, mas o
presidente norte americano não tem nem um terço do poder que tem um
presidente brasileiro, porque lá é um poder da federação. O Brasil é um
país unitário com um presidente imperial.

\textbf{No dia 22 de março o presidencialismo foi aprovado e saiu uma
reportagem dizendo ``Ulysses soube antes que o presidencialismo
venceria''. A pergunta é: houve compra de votos, oferta de cargos no
governo, barganhas? O que foi necessário para obter a vitória por 344
votos a 212?}

\textbf{Freire:} Pelo placar se vê que não houve compra desbragada. O
que houve foi uma articulação do governo para garantir os cinco anos. E
ele perdeu um ano. Tiramos um ano, queríamos tirar dois. É claro que ele
articulou e consolidou a sua maioria para garantir os cinco anos.

\textbf{Como foi essa articulação? Na época foi publicada uma matéria no
jornal dizendo que os ``quatroanistas'' iriam perder os seus cargos.}

\textbf{Freire:} Não foi bem assim. Houve uma articulação, como se tem
dentro do sistema presidencialista a toda hora e em qualquer votação que
o Executivo tenha interesse. Então houve a liberação de emendas --
porque o Congresso não era só Constituinte, mas também ordinário.
Naquele ano votamos o Orçamento. As pessoas estavam liberando suas
emendas tal como fazem agora. O presidencialismo no Brasil sempre usou a
liberação de emendas e, naquela época, havia a concessão de rádio e
televisão dada pelo Executivo. Houve uma distribuição razoável de rádio
e emissoras para políticos.

\textbf{Esta articulação foi legítima?}

\textbf{Freire:} Infelizmente é o mecanismo que o presidencialismo tem
para influenciar naquilo que é de interesse do Executivo. Eu não posso
dizerse é legítimo, pois não tenho conhecimento. Distribuição de rádio
havia antes e talvez tenha aumentado. Naquela época havia o Ministério
do Interior, que sempre teve muita verba para distribuir e
possibilidades de convênios com prefeituras próximas aos parlamentares.
Deve ter acontecido isso, mas eu não sei dizer quem se beneficiou
diretamente. Mas a justiça vai saber. O ``mensalão'' está aí: alguns
dizem que não existia, mas existiu e foi algo profundamente
desmoralizante do Congresso Nacional.

\textbf{Em abril de 1988 ocorre a votação sobre a ordem econômica. Como
se deu o histórico debate da definição da empresa nacional, recursos
naturais, função e monopólio da propriedade privada, nas quais o senhor
teve uma participação efetiva?}

\textbf{Freire:} Isso deve ser bem contextualizado. Nós, os comunistas
do PCB e do PCdoB, e também outros setores de esquerda, ainda vivíamos
num mundo bipolar. A União Soviética ainda existia e, naquele momento, o
PCB era um dos partidos do movimento comunista internacional -- embora
tenha avançado na opção pela democracia como uma questão central,
reconhecendo que algumas distorções e graves equívocos cometidos pelo
socialismo real foram fruto da ausência da democracia. Mas mesmo eles, o
PT, tinham como o referencial da esquerda aquilo que era o tipo de
organização social da União Soviética, a estatização como transição
socialista, tanto que reclamam muito da privatização. Todos nós da
esquerda tínhamos como referencial na discussão, na questão econômica,
um modelo de Estado e certa conotação imperialista, nacionalista na
questão das empresas nacionais e nos monopólios estatais. Mas
promulgamos a Constituição e fomos para a eleição presidencial já com o
fim -- ou com o início do fim -- daquela experiência histórica do
socialismo real. Veio em seguida um mundo que derrotou o socialismo real
do ponto de vista da economia, pela grande revolução do conhecimento
cientifico e das inovações tecnológicas. O próprio capitalismo está
enfrentando as derrotas de algumas conformações da sociedade industrial.
A crise que estamos vivendo agora é fruto da revolução tecnológica, da
integração e do processo de globalização.

\textbf{David Fleischer}

*Cientista Político norte-americano com mestrado em Estudos
Latino-Americanos. Doutor em Ciência Política pela Universidade da
Flórida, é professor emérito da Universidade de Brasília, onde iniciou
como docente em 1972. Seus estudos têm ênfase em Estado e Governo, tendo
como temas o Brasil, sistemas eleitorais, partidos políticos,
legislativo e transparência. É naturalizado brasileiro.

\textbf{A Assembleia Nacional Constituinte foi instalada no dia 1º de
fevereiro, quando havia uma crise econômica e social muito forte. Como
era a relação, naquele período, do governo Sarney com a assembleia que
foi instalada?}

\textbf{Fleischer:} A Constituinte eleita em 1986 foi uma grande vitória
do PMDB. Naquela eleição o partido elegeu os governadores de todos os
Estados, exceto o de Sergipe, e a maioria absoluta dos senadores e
deputados. O PMDB foi majoritário na Constituinte. Sendo José Sarney o
presidente da República em 1987 e também presidente do partido e com o
PMDB no comando da Câmara dos Deputados, do Senado e ocupando a
presidência da Constituinte, era de se esperar uma boa relação entre a
Constituinte e o presidente. Mas não foi bem assim.

A Constituinte seguiu um rito, ou regimento interno, muito
descentralizado, diferente da Constituinte de 1946. Em 1946 uma comissão
elaborou vários projetos enquanto o resto da Constituinte ficou à toa e
depois votaram no que a comissão elaborara. Em 1987 a Constituinte se
dividiu em (24) subcomissões, nas quais todos os constituintes
participaram. Depois os relatórios de cada três subcomissões eram
levados para uma das oito comissões temáticas.

Oito comissões viraram oito capítulos da nova Constituição. Houve
debates e redações para juntar as três partes das subcomissões para
fazer o relatório da comissão. E os oito relatórios foram para a
Comissão de Sistematização, que foi relatada pelo deputado Bernardo
Cabral (PMDB-AM). Foi produzida no final de 1987 a primeira versão do
projeto da Constituição, que foi chamado de ``monstrengo'' e juntou
essas oito partes. No fim de 1987 os setores mais conservadores da
Constituinte resolveram se rebelar contra a liderança do PMDB e
constituíram o que era chamado de Centrão. E eles obrigaram a uma
mudança no regimento interno: em vez de votar inicialmente na versão da
Comissão de Sistematização, seria votada a versão do capítulo aprovado
por essa maioria dos constituintes. Isso mudou o ritmo de votação em
1989.

\textbf{Essa virada regimental, entre novembro e janeiro, foi uma
vitória dos conservadores frente aos progressistas, visto que o
anteprojeto que a comissão de sistematização elaborou continha aspectos
progressistas e de desenvolvimento?}

\textbf{Fleischer:} Sim. Isso foi trabalhado através de um lobby feito
pelo governo Sarney e também pelo Conselho de Segurança Nacional, os
militares, que também tinham um lobby muito forte dentro da
Constituinte. O lobby militar era reconhecido por todos os que
trabalhavam na Constituinte como um lobby muito eficiente e muito bem
feito. Os dois pontos principais neste momento eram o sistema
parlamentarista e o mandato do presidente Sarney. O conselho de
sistematização ofereceu a reeleição para Sarney e um mandato de quatro
anos. Isso significaria que teríamos uma nova eleição presidencial em
novembro de 1988. Mas o Centrão obrigou a manter o sistema
presidencialista e também colocou um mandato de cinco anos para o
Sarney, provocando uma eleição presidencial em novembro de 1989.

\textbf{No processo da Assembleia Nacional Constituinte a tese do
parlamentarismo venceu, mas na votação em plenário deu o
presidencialismo e os cinco anos para Sarney. O senhor mencionou que
ouve um lobby muito forte por parte do governo. Houve barganhas, cargos,
liberação de rádios? Como foi o poder de fogo do governo Sarney?}

\textbf{Fleischer:} O poder de fogo do governo Sarney, as fichas que
eles tinham na mesa para jogar, eram justamente essas: emendas no
orçamento para implementar, cargos para nomear e, principalmente,
concessões de rádios e televisões. Quase todos os constituintes do
Centrão receberam concessões de rádios e televisões, ao ponto de o
próprio líder do governo Sarney admitir para as televisões e para os
jornais que ``o saco de bondades'' estava vazio, havia acabado, e que
não havia mais nada para oferecer.

Esses constituintes estavam exigindo cada vez mais. O governo esgotou
todas as fichas que tinha para jogar na mesa, para incentivar as
votações em favor das posições do governo. Aconteceu algo muito
interessante, aliás: àquela época havia muitos deputados eleitos pelas
igrejas evangélicas. Em vez de registrar a concessão que foi ganha em
nome da igreja, alguns deputados registraram em seu próprio nome. Muitas
das igrejas ficaram com raiva e expulsaram essas pessoas, que eram
bispos e pastores. Muitos deles não conseguiram se reeleger em 1990
porque foram desligados da igreja e estavam sem apoio para a reeleição.

\textbf{Durante a ANC houve um momento tenso. Terminados os trabalhos da
Comissão de Sistematização, o presidente Sarney foi em rede de televisão
colocar que, caso a Constituição fosse aprovada do jeito que estava, o
País ficaria ingovernável. E, no dia seguinte, Ulysses foi em rede de
televisão e rádio fazer a defesa da Constituinte. O senhor lembra desse
período?}

\textbf{Fleischer:} Essa questão foi resolvida pela Medida Provisória,
que deu um poder extraordinário, um poder legislativo ao presidente.
Supostamente deu maior governabilidade. Ela foi amplamente utilizada
pelo presidente Collor a partir de 1990, porque ele não possuía maioria
no Senado e nem na Câmara. Outro momento muito tenso foi o início dos
trabalhos da Constituinte, quando se estava elaborando o regimento
interno. Havia uma pressão muito forte, principalmente por parte dos que
chamamos de ``deputados xiitas'', os deputados novatos que queriam
começar os trabalhos antes da Constituinte, para remover o chamado
``entulho autoritário''. Diziam que a Constituinte não poderia começar a
trabalhar livremente se esse ``entulho autoritário'' não fosse eliminado
completamente.

Houve uma rebelião desses deputados e o próprio deputado Ulysses
Guimarães, presidente da Câmara, conversou com esses jovens deputados e
explicou que, se feito assim, talvez os militares intervissem e
fechassem a Constituinte. Isso porque, antes de morrer, o presidente
Tancredo Neves se comprometera com os militares de que esse entulho
autoritário não seria removido antes da nova Constituição, mas que seria
removido dentro da Constituição. Após todo o ocorrido Ulysses fez um mea
culpa e afirmou que errou ao convencer esses jovens deputados sobre essa
questão e que seria muito melhor remover esse entulho autoritário, que
tinha chegado à conclusão de que os militares não iriam reagir como
suspeitara. O regimento interno ficou nas mãos do senador Fernando
Henrique Cardoso, que era o relator.

\textbf{A Constituinte foi precedida por uma comissão chamada Affonso
Arinos. O senhor pode falar mais a respeito dela?}

\textbf{Fleischer:} Esta foi outra promessa de Tancredo Neves com o
objetivo de colher opiniões sobre a nova Constituição. A comissão foi
instituída pelo presidente Sarney em 1985 e funcionou até 1986. Ela
produziu um anteprojeto que seria distribuído para todos os
constituintes. Mas o presidente Sarney decidiu colocar na gaveta e não
distribuir. E nós, na Universidade de Brasília, sob o comando do então
reitor Cristovam Buarque, organizamos o CEAC (Centro de Estudos e
Acompanhamento da Constituinte) e a editora da universidade publicou
este relatório Affonso Arinos. Uma cópia desse livro foi distribuído
para todos os constituintes, gente do governo, gente que trabalhava com
a Constituinte.

Também convidamos renomados juristas para fazer comentários sobre esse
anteprojeto da comissão Affonso Arinos. O anteprojeto foi organizado em
oito capítulos e vinte e quatro subcapítulos. E foi justamente essa
organização que a comissão pôs em seu relatório, que Fernando Henrique
acolheu para o regimento interno.

O CEAC reuniu alunos, professores e funcionários da UNB. Contou também
com a participação de várias pessoas da sociedade civil. Acompanhávamos
todo o processo da Constituinte e publicávamos alguns boletins e livros.

\textbf{Como o senhor avalia a atuação dos partidos de esquerda? Como
eles se agrupavam, como eram feitas as alianças no processo da
Assembleia Nacional Constituinte?}

\textbf{Fleischer:} Fiz um perfil da Constituinte e apresentei no Recife
num seminário organizado na Universidade Federal de Pernambuco. Nele
mostrei que o maior partido da Constituinte não era o PMDB, mas a Arena.
Pesquisei as origens de muitos deputados e muitos deles passaram pela a
Arena, PDS e, depois, entraram no PMDB por conveniência, para se
elegerem em 1986.

Os progressistas conseguiram eleger o Mário Covas como líder do PMDB na
Constituinte, líder da maioria, contrariando o desejo de Ulysses
Guimarães de eleger Luiz Henrique Silveira. O Mário Covas, com um
discurso muito contundente, reverteu a bancada do PMDB em favor de sua
eleição.

Ao longo da Constituinte se organizou um grupo que era chamado MUP
(Movimento de Unidade Progressista), que reuniu uns cem deputados
progressistas de vários partidos para tentar organizar uma ação em comum
na Constituinte. Uma boa parte desse MUP, em junho de 1988, se
transformou no PSDB, como uma reação contra o PMDB no Centrão.

\textbf{Havia vários grupos: MUP, centrinho e outros. Como eles se
articulavam para fazer frente às propostas de defesa do status quo, do
establishment? Como eles se agrupavam e negociavam entre si? Era fácil
chegar num consenso?}

\textbf{Fleischer:} Não. Não era muito fácil. O consenso foi costurado
na Comissão de Sistematização. Mas a sistematização era um trabalho mais
burocrático, de costurar e de tirar as inconsistências de um relatório e
outro das comissões, para apresentar um relatório final, juntando todos
os relatórios das oito comissões. Houve negociações, sim, em nível de
subcomissões e comissões. Mas os conservadores eram maioria nelas, então
as articulações dos progressistas não tiveram muito efeito. Por exemplo:
a reforma política praticamente não foi aprovada. Só admitiram três
novos estados: Tocantins, Amapá e Roraima e também baixaram a idade
eleitoral para dezesseis anos. O número de deputados do estado de São
Paulo foi ampliado de 60 para 70, mas, quando os paulistas puderam
utilizar a nova cota de deputados nas eleições de 1990, o TSE (Tribunal
Superior Eleitoral) respondeu ``não'', porque não fora regulamentado. Na
área de reforma politica, reforma eleitoral e normas sobre os partidos
praticamente não houve modificações.

\textbf{Entre os vários temas debatidos nas subcomissões, comissões e
audiências públicas, como a reforma agrária, questões do capital,
empresas nacionais, direitos das minorias e direitos coletivos, quais
foram os mais tensos?}

\textbf{Fleischer:} A questão da reforma agrária foi muito tensa e
também a questão das empresas nacionais. A Constituição deu uma
conotação bastante nacionalista e teve que ser reformulada em 1995 pelo
então presidente Fernando Henrique Cardoso (PSDB-SP), para permitir um
pouco mais de participação de capital estrangeiro no Brasil e,
principalmente, para aprovar a possibilidade de privatizar empresas
estatais, que foram praticamente ``amarradas'' dentro da Constituição de
1988. Para o Brasil sair das amarras de um estado falido, sem poder de
fato desenvolver o País, foi necessário modificar a Constituição para
permitir privatizações. Mas em todas as negociações durante a
Constituinte e depois de 1995 a Petrobras foi ressalvada. A grande jóia
da coroa não foi tocada.

\textbf{Essa Constituição é cidadã ou não é cidadã?}

\textbf{Fleischer:} Foi apelidada de Constituição Cidadã pelo Ulysses
Guimarães, que saiu mostrando, erguendo-a no alto. Ele achou que
incrementaria muito à sua candidatura a presidente da República, mas em
1989 não foi bem assim.

\textbf{Como ficou a situação das Forças Armadas dentro da nova
Constituição?}

\textbf{Fleischer:} O papel das Forças Armadas foi preservado com
autonomia. Em todas as repúblicas democráticas o comandante chefe das
Forças Armadas é o presidente da República, que é o comandante supremo,
e as Forças Armadas são subjugadas ao poder Civil. Mas na Constituição
não aconteceu bem assim, porque eles tinham o direito de intervir para
``proteger a democracia''. E após a promulgação levou onze anos até que,
em 1999, Fernando Henrique Cardoso teve coragem e condições de impor o
Ministério da Defesa com ministros civis. E os ministérios militares
foram reduzidos a níveis de comando. O Brasil foi um dos pouquíssimos
Países que àquela época não possuía um Ministério da Defesa.

\textbf{Sobre a questão das emendas populares, qual foi a importância
dos movimentos populares, da sociedade civil na Constituinte?}

\textbf{Fleischer:} Os Correios abriram uma coisa muito interessante. Em
cada agencia o cidadão poderia pegar um formulário e mandar suas
sugestões para entrar na nova constituição. E houve mais de 100 mil
sugestões. Mas elas não tiveram condições de ser organizadas e
sistematizadas para serem encaixadas na Constituição. Acabaram
arquivadas no serviço de armazenamentos de dados do Senado. O caminho de
sugestões e opiniões populare existiu, mas elas não foram bem
aproveitadas. Muitos representantes da sociedade civil foram convidados
para oferecer testemunhos e depoimentos nas subcomissões e nas
comissões. Eu, como professor de Ciência Política, fui chamado para
falar na subcomissão sobre reforma eleitoral e partidária.

\textbf{Quais foram as grandes conquistas da Constituição de 1988?}

\textbf{Fleischer}: Grandes conquistas foram os artigos garantidores de
direitos fundamentais do cidadão brasileiro. Infelizmente muitos dos
direitos presentes na Constituição têm sido difíceis de efetivar. Os
direitos à habitação e à saúde, por exemplo, pois dependem de políticas
públicas.

\textbf{José Genoino}

*Cearense de Quixeramobim, iniciou sua militância na União Nacional dos
Estudantes (UNE). Ingressa no Partido Comunista do Brasil (PCdoB),
fazendo oposição ao regime ditatorial do Brasil. Com o Ato Institucional
número 5, muda-se para São Paulo e inicia a militância clandestina. Em
1970 passa a integrar a Guerrilha do Araguaia e, em seguida, é capturado
e preso pela ditadura. Solto após cinco anos, passa a lecionar História.
Fundador do PT (1981), partido pelo qual foi deputado federal (1982-2002
e 2006-2013).

\textbf{Antes da Constituinte o senhor já era deputado. O foi de 1983 a
1987 e o foi eleito também deputado constituinte. Na época o senhor era
membro do Diretório Nacional do PT. O senhor fez parte da Comissão de
Sistematização?}

\textbf{Genoíno:} Fiz parte da Comissão Sistematização e fui vice-líder
da bancada do PT com o outro vice-líder, Plínio Arruda Sampaio (PT-SP),
e com o líder e companheiro Lula (PT-SP). Integrei a Subcomissão de
Defesa do Estado, Sociedade e Segurança e a Comissão de Organização
eleitoral partidária. Eram as duas Comissões em que eu atuava
prioritariamente, mas na Constituinte atuamos em várias comissões. Atuei
em todas essas Comissões, apresentei emendas sobre todos os temas e,
além dessas duas Comissões, atuei fortemente na Comissão de Direitos
Individuais, Cidadania e Garantias dos Direitos Civis e Corretivos.

Atuei com a deputada Cristina Tavares (PMDB-PE) na Comissão de
Comunicação e fizemos grandes enfrentamentos. Foi a única Comissão que
não teve relatório, pois houve uma queda de braço e derrotaram o
relatório dela. Nós participamos da obstrução para não haver relatório
algum. E participei de outras comissões, como da Reforma Agrária. Foi
uma luta dura de plenário, que tinha um pernambucano também relator,
Osvaldinho (**), que era o Osvaldinho, né?! E várias outras comissões,
porque um deputado constituinte podia participar de todas as Comissões.
Participei da Constituinte de maneira integral. Muitas vezes eu dormi na
própria Câmara dos Deputados para entregar as emendas mais cedo, porque
as propostas eram analisadas por ordem de entrada. Eu gostava daquela
atuação. Ficava direto até a última reunião das Subcomissões e das
Subseções Temáticas, depois na Comissão de Sistematização. E no próprio
enfrentamento que nós fizemos ao Centrão. Participava das negociações do
Colégio dos Líderes Partidários, que discutia os impasses, as
alternativas e as soluções para a Constituinte.

\textbf{Em 1985 travou-se um grande debate em torno da escolha entre
duas espécies de Constituintes: a Exclusiva e a Congressual. E esse foi
o debate inicial, quando se instalou a Assembléia Nacional Constituinte,
após a abertura do ministro José Carlos Moreira Alves. O senhor,
juntamente com Plínio Arruda Sampaio, levantaram a questão da impugnação
de 23 senadores eleitos indiretamente em 1982. Qual foi o entendimento
da época? Estes senadores de 1982 não tinham poderes derivados da mesma
natureza da Assembléia Nacional Constituinte?}

\textbf{Genoíno:} Em primeiro lugar, eu participei da Comissão Especial,
que era uma Comissão do Congresso de deputados e senadores. Mesmo o PT
não tendo direito à representação formal, eu participei da Comissão. O
primeiro relatório do Flávio Bierrenbach (deputado federal de 1983 a
1986, pelo PMDB-SP), que previu uma Constituinte Exclusiva tendo apoio
do PT, esse relatório foi derrotado aprovou-se outro relatório, que foi
do Congresso Constituinte do Jovani Masini (PMDB-PR). E participei,
depois, da obstrução no Plenário do Congresso Nacional, portando
deputados e senadores na aprovação da Emenda Constitucional que convocou
o Congresso Constituinte. Portanto o PT defendeu uma Constituinte
Exclusiva.

Na instalação da Constituinte tivemos três grandes enfrentamentos. O
primeiro: quando a Constituinte foi instalada pelo ministro Moreira
Alves, presidente do Supremo, eu, sem microfone, questionei que um
presidente do Supremo que oriundo da ditadura militar não possuía
poderes para instalar a Constituinte. Que a palavra fosse concedida aos
Partidos.

Depois questionamos a ideia do Congresso Constituinte, para que a
Constituinte fosse soberana. A soberania da Constituinte significa que
ela é um poder acima dos demais poderes, tanto em relação ao Executivo,
que era o enfrentamento com o Governo de Sarney; como em relação ao
Judiciário; como em relação aos resquícios do autoritarismo do Congresso
Nacional, que eram os senadores eleitos em 1982 e que não foram eleitos
no processo da Emenda Constitucional que convocou o Congresso.

O terceiro enfrentamento foi o do Regimento Interno. O Regimento Interno
da Constituinte é o mais democrático, porque estávamos tensionando no
sentido da soberania da Constituinte, apesar de ser um Congresso
Constituinte. Esse fato foi parte da estratégia que exigiu esse
tensionamento. E com esse tensionamento nós construímos uma tática para
fortalecer as Comissões e as Subcomissões, as comissões temáticas que
produziam maioria na Comissão de Sistematização. Depois veio o Centrão
para se contrapor àquela maioria que era articulada pelo Mário Covas e
houve enfrentamento.

A Constituinte, desde a convocação até o final, foi um grande momento de
enfrentamento político. E houve dois fatos interligados à Constituinte:
foram a campanha das Diretas, derrotadas pelo Congresso Nacional, e a
participação da oposição no Colégio Eleitoral -- como é sabido, o PT não
participou do Colégio Eleitoral, mas denunciou o Colégio Eleitoral. A
Campanha das Diretas, o Colégio Eleitoral e a Constituinte foram grandes
momentos de tensionamento pela radicalização da tensão democrática,
diferente do caminho que predominou, que foi a pactuação de uma
transição negociada.

\textbf{Esse foi primeiro embate político dos partidos de esquerda PCB,
PT, PCdoB e o PSB. Foi nesse processo de embate que demonstrou-se uma
coesão do processo decisório para formação do bloco de esquerda?}

\textbf{Genoíno:} Nós formamos um bloco de esquerda que eles chamavam,
no PMDB, a ``tendência popular'': o PT, o PCdoB, o PSB. E esse bloco fez
uma articulação com o PMDB liderado por Mário Covas (PMDB-SP), Fernando
Henrique Cardoso (PMDB-SP), Pimenta da Veiga (PMDB-MG), Nelson Jobim
(PMDB-RS). E nós fizemos maioria na Comissão de Sistematização. A
esquerda teve uma articulação muito eficaz, muito ofensiva, mesmo sem
ter maioria. Mas a articulação foi uma aliança com o centro,
representado pelo PMDB, e foi o que possibilitou aprovarmos muitas
conquistas na Constituinte para essa Constituição, que é uma das mais
avançadas da história do Brasil.

Mesmo quando o Centrão veio nós fizemos uma contra-tática, um
contra-caminho para, por dentro do Centrão, garantir aquelas conquistas
que a gente já havia colocado nos diversos anteprojetos. Foram quatro ou
cinco ao todo. Eram feitos, desfeitos, negociados. E foi um processo
político muito rico.

\textbf{Em fevereiro houve um debate crucial sobre os procedimentos
regimentais que o senhor comentou. Houve um acordo entre o líder do PFL,
José Lourenço (PFL-BA), e Mário Covas (PMDB-SP) na questão das
relatorias, para que a maioria das relatorias ficassem com o bloco
progressista do PMDB. A Presidência coube ao bloco mais conservador:
PFL, partidos da obra. E os partidos de esquerda eram uma frente
pequena. O poder de agenda e decisão esteve nas mãos dos relatores, nas
mãos do PMDB. O papel dos partidos de esquerda foi irrelevante nesse
aspecto?}

\textbf{Genoíno:} Vivíamos um processo democrático em construção. A
esquerda possuía um número pequeno de constituintes, mas com muita
legitimidade e, principalmente, com alianças com os setores populares
através da participação popular na Constituinte. Muitos temas que a
esquerda defendia eram respaldados nas chamadas Emendas Populares.
Éramos respaldados do setor mobilizador da sociedade, através das
entidades, das aliança com a Igreja, com a CUT (Central Única dos
Trabalhadores), com os movimentos sociais, com a intelectualidade.

E a esquerda percebeu que não poderia ficar na Constituinte só marcando
sala. Precisávamos de uma aliança política com o PMDB. E essa aliança
foi responsável pelos avanços, que são uma conquista do trecho
constitucional em todos os capítulos. Mesmo diante da reação nós
garantimos as conquistas. Na aliança com o PMDB, toda a estratégia de
enfrentar o Centrão foi feita com a minha participação, com o Jobim, o
conhecimento do Mário Covas, o conhecimento do Lula e discussão.

No regimento, que vem após a escolha do relator, tudo foi disputado. É
importante que a disputa ela foi permanente, diária e teve um condutor,
que foi o Ulysses Guimarães. Ele conduzia aquilo muito bem para que a
corda não partisse -- em vários momentos a Constituinte correu o risco
de não produzir uma constituição, mas, graças à habilidade e à maneira
com que ele atuou nos momentos de maior tensão, nós produzimos um texto
constitucional que é um dos mais avançados da história do Brasil.

\textbf{Não foi fácil a Assembléia Nacional Constituinte aprovar o
fortalecimento do Congresso Nacional, visto que este perdera as suas
prerrogativas constitucionais desde o golpe militar de 1964. Na
Subcomissão de Defesa do Estado houve o debate sobre o conceito de
Estado Brasileiro. Como se deu esse debate, visto que saíamos do regime
militar e a Comissão era de caráter fortemente conservador?}

\textbf{Genoíno:} Foi positivo. Esse ponto nós trabalhamos com o
retrovisor, porque a referencia era expurgar, derrotar o período
autoritário da ditadura dos militares. Isso era muito forte, muito
presente, porque a Constituinte era resultado de um processo político de
enfrentamento à ditadura militar.

Agimos corretamente em não aceitar um texto constitucional prévio.
Sabe-se que a comissão tentou articular vários textos preliminares. Nós
começamos do zero e isso foi um avanço, porque possibilitou, por
exemplo, as prerrogativas do Congresso, a liberdade partidária e os
direitos e garantias individuais. Avançamos porque partimos de uma
situação nova e com a participação popular. Isso foi muito positivo para
se concretizar as prerrogativas do Congresso Nacional.

É claro que uma questão permeou a discussão das prerrogativas: o debate
sobre os cinco anos ou quatro anos do Sarney, a relação com o governo
Sarney. Acho que muitas questões avançariam se o debate dos cinco anos,
que era um debate conjuntural, não tivesse contaminado o debate mais
institucional e político do capítulo sobre as instituições políticas.
Tanto é que um dos pontos débeis da Constituição é exatamente o ponto
que trata das instituições políticas, sistema eleitoral, partido e papel
da Câmara do Senado.

\textbf{Os temas: a natureza e finalidade do Estado, Conselho de
Segurança Nacional, Estado, estado de sítio e o papel das Forças
Armadas, qual desses temas foi o mais forte, o mais difícil de se
discutir na Subcomissão?}

\textbf{Genoíno:} O papel das Forças Armadas, o famoso Artigo 142, que
trata da missão das Forças Armadas. Tentamos uma formulação que retirava
a expressão ``a lei e a ordem'' e ficava ``as Forças Armadas são
responsáveis pela defesa do território, pela soberania nacional e pelos
poderes constitucionais''. Conseguimos inclusive que o relator, Bernardo
Cabral (PMDB-AM) adotasse essa formulação do pré-projeto que foi para a
Comissão de Sistematização. Isso gerou uma crise militar. Os ministros
militares se manifestaram. Foi em decorrência dessa manifestação que
surgiu aquela frase de Ulysses Guimarães, de que ``Junta militar era a
junta de três patetas''. Foi um episódio interessante porque terminamos
a Constituinte na sexta feira com o Ulysses Guimarães vaiado pela
esquerda. Ele fez essa declaração no final de semana e, na segunda
feira, ele entra na Constituinte ovacionado pela esquerda. Ulysses tinha
essa característica.

Esse foi o ponto mais forte. Esse ponto foi muito tesionado. Acho que
acabamos produzindo uma formulação que, olhando hoje, após 20 anos,
julgo ser a mais correta. Na verdade o que estava embutido no debate era
a subordinação das Forças Armadas ao Poder Civil. Em relação ao Estado
de emergência, o Estado de sítio e ao Conselho de Defesa Nacional: não
foram tão decisivos na polêmica. O decisivo foi a questão que dizia
respeito à tutela do Poder Civil sobre o Poder Militar.

\textbf{O senhor apresentou a emenda da criação do Ministério da Defesa,
visto que as Forças Armadas iniciaram sua participação dentro do
governo.}

\textbf{Genoíno:} Acho que esse tema é importante como outros que vamos
abordar na Constituinte. Como o Brasil muda processualmente -- e temos
que confiar no processo sem perder jamais a firmeza em relação a onde
queremos chegar. Nós apresentamos a proposta e não foi muito difícil,
porque estávamos discutindo a subordinação do Poder Militar ao Poder
Civil -- e perdemos. Depois o Ministério da Defesa foi criado no governo
de Fernando Henrique Cardoso (PSDB-SP), mas não foi consolidado. E agora
o Ministério da Defesa está sendo consolidado com a Estratégia Nacional
de Defesa e com o ministro Nelson Jobim, um dos constituintes.

Na Constituinte eu e Pimenta da Veiga apresentamos uma emenda sobre o
acesso a assuntos sigilosos e ultra-secretos. Advogávamos um prazo de 25
anos. E 20 anos depois o governo Lula (PT-SP) manda para o Congresso
Nacional uma lei sobre acesso a informação prevendo os 25 anos como
período de durabilidade do sigilo para assuntos ultra-secretos -- e
propõe mais 25. Estamos propondo um sigilo eterno. E eu sou um militante
que confia nesse processo político que o Brasil vive. O País vive um
processo político desigual mas, ao mesmo tempo, avançado. E cito esse
dois exemplos porque hoje viraram realidade, mas na Constituinte parecia
uma grande utopia quando entrei com a emenda sobre o Ministério da
Defesa.

Quando abrimos o debate sobre o aborto na Constituinte foi algo que
parecia esdrúxulo. Quando abri o tema sobre a orientação sexual, a
discussão da união estável entre pessoas do mesmo sexo, foi um
terremoto. E hoje a Parada do Orgulho Gay é uma das maiores
manifestações que acontecem no Brasil e no mundo. Quando discutimos a
tutela dos direitos como prerrogativa do Estado, parecia uma revolução.
Quando apresentei a proposta de trabalhar com a idéia da soberania
popular... Eu fazia emendas assim. Uma mais radical, uma para marcar
posição e uma para negociar.

Quando apresentei o direito à rebelião da Constituinte -- eu apresentei
uma emenda que era o direito à rebelião! -- lógico que perdi. Mas passou
a emenda de participação popular. E o preâmbulo diz ``todo poder emana
do povo e pode ser exercido diretamente pelos seus representantes'',
``diretamente na forma da Constituição''. Um debate muito radicalizado.
É claro que perdi outros debates como, por exemplo, um sobre retirar a
expressão sobre a proteção de Deus. Também foi uma discussão radical.

Acho que o importante é que o Brasil foi discutido na Constituinte, o
Brasil inteiro: de temas como esse até meio ambiente, cultura,
quilombolas, reservas indígenas, o conceito de família -- que foi outro
debate radicalizado. Essa ideia de que a união estável é uma base para a
constituição da família, não apenas o casamento, também foi um debate
muito radicalizado na Assembléia Constituinte. Todos os temas eram
objetos que estávamos desenhando. Uma espécie de programa político para
o futuro. É um documento político, não pode virar uma petrificação de
interpretação constituinte. É um documento politico dos mais avançados
da história do Brasil. E acho que isso mostra como a constituinte
conduziu e foi produto desse debate, desse enfrentamento político. Havia
de tudo: da UDR ao MST, das mulheres à igreja, do movimento gay às
religiões. Havia de tudo e o Brasil se identificou com aquele processo
Constituinte.

\textbf{O Movimento nacional Pró-participação Popular na Constituinte
garantiu a iniciativa das emendas populares e não se limitou apenas às
emendas. Além do povo, vários segmentos sociais se fizeram presentes. A
Folha de S. Paulo publicou em 21 de junho de 1987: ``dois mil lobistas
procuram persuadir os constituintes''. Que lobby foi mais eficaz na
Assembléia Nacional Constituinte: o dos movimentos sociais ou dos grupos
empresariais?}

\textbf{Genoíno:} Todas as pressões foram legítimas e ainda bem que
vieram. O governo Sarney estava esvaziado com a crise, com o fracasso do
Plano Cruzado. O Judiciário vindo da sustentação constitucional à
ditadura militar e nós acumulando várias derrotas: a derrota das
Diretas, derrota do Plano Cruzado, morte do Tancredo (PMDB). Três
frustrações. Foi muito importante que a sociedade viesse para dentro (da
Constituinte). Toda a sociedade tinha seus grupos de pressão, dos mais
fracos aos mais fortes. Mas os mais ostensivos e os mais fortes foram os
da UDR e os do Poder Judiciário. Mesmos os militares, que pressionaram,
possuíam um lobby menor que o do Poder Judiciário e o da UDR.

\textbf{Comissão de Sistematização e a formação do bloco supra
Partidário: comissão integrada por 93 dos 559 Constituintes e com o
poder de definir o texto de base a ser submetido ao Plenário da
Assembléia Nacional Constituinte. Momentos de agonia, perplexidade e
agravamento dos embates políticos e ideológicos. Quais foram os
principais impasses na Comissão de Sistematização antes da criação do
Centrão?}

\textbf{Genoíno:} Os principais impasses foram no sentido de construir
textos, porque nós partimos do zero. Os textos vieram da Subcomissão e
da Comissão. Chegava na Sistematização -- e o próprio nome dizia: era
preciso dar sistemática, dar coerência, ter começo, meio e fim. Algumas
vezes eram ``mostrengos''. Fizemos vários ``monstrengos''. E a
denominação usada foi ``Frankenstein". Depois veio o ``Bebê de
Rosemary'', veio ``Mostrengo'', ``Buraco Negro''... E o processo foi
muito rico, porque foi se articulando. Como possuíamos maioria na
Comissão de Sistematização.

\textbf{Após vários embates entre progressistas e conservadores, no dia
10 de novembro o Centrão apresentou sua emenda de mudança do regimento.
No dia 5 de janeiro foi aprovado. O que de fato aconteceu? Os argumentos
do Centrão foram: a maioria ficará a mercê das decisões dos 93 membros,
dos líderes; a composição da Comissão de Sistematização foi manipulada
pelo líder do PMDB; o resultado não é compatível com a composição
ideológica do Plenário; e é necessário voltar à soberania do Plenário. O
que de fato aconteceu para criação do Centrão?}

\textbf{Genoíno:} Houve uma questão que foi decisiva: a maioria da
Comissão de Sistematização não refletia a maioria do plenário. Era isso.
Havia uma maioria do plenário que derrotaria o texto da Sistematização
e, ao derrotar o texto da Sistematização, não teríamos Constituição. Foi
um momento muito delicado eu acho que Ulysses teve um grande papel
quando o Centrão quis mudar o regimento interno para apresentarem um
projeto -- por que houve um regimento interno o qual não nos permitia
apresentar projeto de capítulo e projeto de Constituição.

Fomos para o enfrentamento. Ulysses nos chamou e disse: ``primeiro: a
Constituinte tem que formarr uma Comissão. Segundo: temos que construir
uma Comissão que tenha a maioria. Terceiro: a melhor coisa é a direita
se comprometer com a Constituição, porque se ela não se compromete com a
Constituição, isso pode levar a uma crise institucional logo após o fim
da ditadura militar'' -- e possuíamos um governo fraco, que era o
governo Sarney.

O Centrão tinha a maioria do Plenário, mas possuiam uma maioria para se
opor àquele movimento da centro-esquerda. Nós tivemos a habilidade de
perceber que aquela maioria (do plenário) não era maioria em todos os
artigos e capítulos. Após aprovarem o projeto deles, fizemos o
contrário: fizemos o que eles fizeram conosco. Como o Centrão não tinha
maioria em todos os itens -- na ordem econômica eles se dividiam, nos
direitos sociais eles se dividiam, nos direitos individuais eles se
dividiam -- nós dividimos o Centrão. E o Centrão estava fraco
politicamente por virmos de uma crise da ditadura militar, crise do
governo Sarney, crise do Plano Cruzado, crise econômica brutal.

Invés de a esquerda ficar só na marcação, fomos operar com uma tática
ampla de acabar com a unidade do Centrão dividindo por temas. Fizemos
isso discutindo aposentadoria por tempo de serviço, jornada de trabalho,
direitos e garantias individuais. A Direita veio com o tema de
criminalizar o aborto e nós -- no caso fui eu -- viemos com o tema e
legalizar o aborto e ficou a expressão genérica de ``direito à vida''.
Vários temas criaram enfrentamentos políticos. Alguns ficaram no meio do
caminho e outros foram remetidos para lei. Acho que foi um exercício
dialético da verdadeira política, pura. Por isso me revolto quando
alguns constitucionalistas olham essa Constituição como se ela fosse uma
pedra, petrificada, de uma visão constitucionalista de que não houve
gente, não houve enfrentamento, que não houve risco, ameaça de tiros,
chinela Havaiana sendo jogada no plenário, dinheiro sendo jogado no
plenário, quebra de vidro... A expressão é ``toma lá e dá cá'', ``é
dando que se recebe''. Foi a política que produziu esse belo texto
Constitucional. Foi produto desse consenso progressivo em cima do
enfrentamento. E acho que a tática que a esquerda e o centro usaram
contra o Centrão foi uma correta. Um exemplo foi a Anistia: queríamos
ampliar a Lei da Anistia -- esse tema que é discutido até hoje, que é o
direito à memória e à verdade. Nós perdemos essa votação. Tentamos
colocar o Estatuto da Terra na Constituinte e perdemos para UDR. Vários
itens.

\textbf{O tema da reforma agrária foi uma derrota da esquerda?}

\textbf{Genoíno:} A principal derrota foi a da reforma agrária. Também
perdemos no tema do crime de tortura, tema que nos era imprescindível. E
perdemos no tema da comunicação, que não conseguimos colocar o sistema
tripartite -- para ser público, privado e estatal. Foi uma guerra no
sentido literal, de bater porta, quebrar microfone, jogar sandália,
jogar sapato. Se o observador puritano visse aquilo diria que é uma
bagunça geral. Mas foi isso que produziu esse texto constitucional.
Muitas vezes houve interpretação preconceituosa em relação à política,
em relação ao Congresso. Àquela época houve muitas criticas: ``isso não
vai dar em nada'', ``a Constituinte tinha que ser feita por sábios, por
pessoas preparadas, por pessoas puras''. Essa Constituição foi produto
da luta política e de tática.

\textbf{Em janeiro de 1988 o clima entre o governo Sarney e a Assembleia
não era nada bom. O jornal Folha de S. Paulo explicitou posição
contrária ao Centrão e denunciou que o governo desenvolvia negociações
escusas em torno do mandato de Sarney e da forma de governo. Em 30 de
janeiro de 1988 a CNBB critica o quadro socioeconômico e a operação
regimental. No dia 22 de março o presidencialismo foi aprovado por cinco
anos em favor de Sarney. Houve compra de votos para obtenção de cargos
no governo, barganhas? O que houve para garantir a vitória, por 344 a
212, do governo e de quem o estava defendendo?}

\textbf{Genoíno:} Eu não reduzo o enfrentamento político a uma política
de barganha. Isso não responde as questões. Houve troca de favores,
principalmente com a concessão de rádios e televisões. Isso houve. Em
relação à compra de votos eu não tenho nenhum fato concreto. E vivi no
Congresso 24 horas por dia, sendo o primeiro a chegar e era o último a
sair, às vezes até dormia lá para apresentar as emendas. Participava de
tudo. O que houve foi disputa política e ideológica entre a
centro-esquerda e a direita. Essa disputa envolveu uma discussão sobre o
sistema de governo, entendo como um equivoco havermos aprovado o
presidencialismo. Seria melhor aprovar o parlamentarismo como sistema de
reforma das instituições, mas perdemos. Havia divergências sobre
parlamentarismo e presidencialismo e essa divergência acabou contaminada
pela disputa dos cinco anos de Sarney.

\textbf{Mas o parlamentarismo venceu em todas as comissões e perdeu no
Plenário. Por quê?}

\textbf{Genoíno:} Perdeu no Plenário porque a maioria do Plenário aderiu
ao governo, ao status quo. E o presidencialismo se adequava mais ao
status quo. O PT se posicionou pelo presidencialismo porque estávamos
diante da possibilidade de lançar a primeira campanha do Lula para
presidente da República. Mas acho que foi um erro. A Constituição estava
avançando em vários termos. O parlamentarismo era uma interrogação. Como
funcionará nas condições do Brasil? O erro foi vincular a duração do
mandato ao sistema de governo. Ao vincularmos nós nos prejudicamos,
porque foi um assunto conjuntural que contaminou um outro assunto que
não é conjuntural, que é o sistema de governo. Foi um encadeamento
equivocado. Poderíamos ter isolado o tema da duração do mandato do tema
do sistema de governo. Deveríamos ter discutido a duração do mandato nas
discussões transitórias, mas nós colocamos dentro do texto permanente.
Votamos pelos quatro anos, mas passou cinco anos e com o
presidencialismo. Poderíamos ter negociado os cinco anos pelo
parlamentarismo. O próprio Sarney toparia negociar.

\textbf{CRISTOVAM BUARQUE}

*Engenheiro pela UFPE, inicia a militância no movimento estudantil Ação
Popular, grupo de esquerda ligado à Igreja. Frente as perseguições
feitas pelo regime militar (1964-1985), foi estudar na França, onde se
tornou Doutor em Economia pela Universidade Panthéon-Sorbonne. De volta
ao Brasil, foi reitor da Universidade de Brasília (1985-1989). Pelo PT,
Buarque foi eleito governador do Distrito Federal (1995-1998) e senador
(2003-2010), havendo se licenciado do mandato para assumir o Ministério
da Educação (2003-2004). Migrou para o PDT em 2004, partido pelo qual se
candidatou sem sucesso à Presidência da República (2006) e se elegeu
mais uma vez para o Senado (2010-2018).

\textbf{Senador, já na instalação da ANC, em 1° de Fevereiro de 1987,
travou-se um grande debate em torno da escolha entre duas espécies de
Constituinte: a Exclusiva e a Congressual. A discussão teve
importância?}

\textbf{Buarque:} Total. Absoluta. Grande parte das falhas que a nossa
Constituição carrega é porque nossos Constituintes foram também
congressuais, parlamentares. Eles estavam com um olho na história, no
futuro da Constituição, e o outro na eleição. Por isso a Constituição é
tão grande e tentou-se colocar tantos interesses corporativos lá dentro.
É uma pena. A Constituinte deveria ser apenas Constituinte. E vou mais
longe: acho que quem fosse Constituinte deveria ser impedido de
candidatura a qualquer cargo 10 anos seguintes à promulgação da
Constituição. Só assim teríamos uma Constituição produzida pensando no
futuro e não na próxima eleição.

Como o senhor avalia a participação da sociedade civil organizada, que
começava a ter acesso ao Congresso Nacional?

\textbf{Buarque:} Lembro que a sociedade civil trouxe sugestões antes
mesmo de receber as propostas dos constituintes. Muitos constituintes,
com franqueza, nem possuíam propostas. Recebemos da sociedade civil e
até da Comissão Affonso Arinos, da qual eu fiz parte -- comissão que
elaborou uma Constituição toda. O projeto foi entregue ao presidente da
República, o presidente Sarney (PMDB), que teve a gentileza de dizer
``eu não vou enviar isso à Constituinte, porque parece uma intervenção
do Poder Executivo''. Mas os constituintes tinham acesso àquilo. Acho
que alguns leram. Levamos muitas propostas para os constituintes.

\textbf{Em vários momentos da Constituinte, o senhor participou de
algumas mesas, defendendo mais autonomia para as universidades. Como foi
esse momento?}

\textbf{Buarque:} Senti uma boa receptividade. A discussão era sobre o
grau da independência que a universidade teria em relação às políticas
sociais, em relação às estratégias de segurança, do ponto de vista
social, em relação à soberania nacional. À época eu dizia: ``nós temos
que ter autonomia sem ser autistas''. A discussão foi em torno daqueles
que defendiam a autonomia e a aqueles que discutiam a questão do
autismo. A universidade é nossa -- dos professores, dos alunos. A
universidade deve ser administrada com base na sua comunidade, o governo
não tem que se meter no dia a dia das políticas pedagógicas, dos
conteúdos.

Acho que o Governo, representando a população através do Congresso,
também tem direito de dizer qual é o número de profissionais que a
universidade tem que formar. Acho um absurdo a universidade pensar que
não tem nada a ver com a educação de base e não formar professores.
Creio que o Estado Brasileiro deve dizer quantos professores quer,
anualmente, formados em matemática, em química ou biologia. Mas no
conteúdo não nos metemos. Sobre o conteúdo é a universidade quem decide.
A liberdade acadêmica com uma ligação com a sociedade.

Hoje há um grande distanciamento, resultando na falta de professores de
química, matemática, física, português... Porque as universidades não
estão preocupadas em formar esses profissionais e, por outro lado,
porque esses profissionais ganham tão pouco que os estudantes não
escolhem essa carreira. Em parte é também porque a universidade se sente
independente -- não autônoma, mas independente da sociedade. Ela não
pode ser independente. Ela pode ter autonomia na gestão pedagógica -- e
eu digo que defendo a autonomia da gestão financeira, mas prestando
contas com muito rigor. Você pode até gastar o dinheiro conforme a
comunidade acha melhor, mas tem que prestar contas, já que o dinheiro
não é seu, mas do povo.

\textbf{Das propostas que as universidades defenderam, o que foi
garantido de fato na Assembleia Nacional Constituinte?}

\textbf{Buarque:} Praticamente tudo. As universidades foram vitoriosas
na Assembléia Nacional Constituinte. Acho que todos os grupos
organizados foram vitoriosos. O Ulysses dizia que a Constituição é
cidadã. Mas acho que ela está mais para uma constituição corporativa. Os
que tiveram força de se aproximar, conseguiram. Mas os analfabetos não
conseguiram. As crianças não conseguiram. Os universitários conseguiam.
A Constituição não deu um salto no envolvimento do poder público na
educação de base.

Um exemplo é que não foi possível federalizar a educação de base. A
Constituição deveria dizer: ``a responsabilidade da educação das
crianças está nas mãos da União''. Foi o contrário: colocou com a União
as universidades, porque elas tiveram força. Colocou na União as escolas
técnicas, porque as indústrias precisavam das escolas técnicas. Mas o
ensino fundamental não entrou, a pré-escola não entrou -- ficaram
entregues aos municípios e estes não têm dinheiro.

\textbf{Qual a importância das audiências públicas no processo da
elaboração?}

\textbf{Buarque:} Foi muito bom. É uma Constituição para a qual os
constituintes ouviam a opinião pública. Pensei em dizer que ouviram o
povo, mas não foi o povo. O povo não entra no Congresso. Os
constituintes ouviram a opinião pública. A diferença é que o povo é o
conjunto de todos, enquanto a opinião pública é uma parte do povo,
aqueles que fazem opinião, que têm a opinião, são os sindicatos, são as
organizações empresariais, são os jornais. Todos esses participaram das
audiências. Mas o povo descalço, o povo excluído, esses não
participavam. Alguns representavam eles, mas eram poucos. Na verdade a
Constituição ficou muito a favor dos grupos corporativos organizados, os
que puderam entrar e subirem às galerias para aplaudir ou vaiar. O povão
ficou de fora.

\textbf{Foi discutido na Comissão de Educação que esses grupos todos não
estavam defendendo a educação pública. Houve setores defendendo que o
Estado financiasse também as escolas confessionais. Como o senhor viu a
discussão?}

\textbf{Buarque:} Eu, a Constituição e nós todos cometemos o erro de
confundir Estado com Público. São duas coisas diferentes. Sempre que o
dicionário tem duas palavras é porque ele tem dois conceitos diferentes
para elas. Raríssimas palavras são sinônimos absolutos. Toda palavra
traz uma nuance que a diferencia de outra. Público é o que serve ao
público e estatal é o que pertence ao Estado.

Nós, da educação, nos acostumamos a dizer que o que é estatal é público,
mas isso é falso. Nos acostumamos a dizer que o que é particular não é
público, mas também é falso. Acho que o Estado tem que bancar a educação
de todos aqueles cursos cujos profissionais são de interesse público,
mesmo que numa universidade particular. Mas não pode fazer diferença da
confessional para a que é puramente particular.

Numa faculdade em que formam professores de química, física, matemática,
biologia, português... Se for de qualidade, ela é pública. Mas ela é
pública e quem paga é o aluno. O Estado deveria pagar para que a
universidade forme professores. E alguns dizem: ``mas por que não paga
às estatais?''. Porque as estatais não são capazes de absorver a todos.
Temos 600 mil alunos hoje nas públicas -- ou, melhor, nas estatais -- e
temos 3 milhões e 900 mil nas outras. Não dá para pegar os 3 milhões e
colocar dentro das estatais. Precisamos do setor particular. O problema
é que o setor particular está formando pessoas para atender ao interesse
privado dele, não o da empresa que vai contratar. E as estatais também
fazem isso: formam médicos e quantos vão para o SUS (Sistema Único de
Saúde)? Uma universidade estatal que forma um médico cirurgião plástico
de rejuvenescimento para velhos ricos, essa universidade está
trabalhando para o interesse privado.

Mas uma escola particular, convecional ou não, que forme o médico para o
SUS está prestando um serviço público. A Constituição deveria ter
diferenciado o público do estatal. Estatal e particular é conforme a
propriedade, mas público e privado é conforme o produto do aluno.
Esquecemos disso. Foi uma falha. Deveria diferenciar, dizendo: ``o
ensino que cria o profissional de interesse público, esse ensino é
público, portanto será gratuito, bancado pelo estado, seja diretamente,
pagando aos professores na sua folha de pagamento, como as universidades
estatais; seja ofertando bolsas para que o aluno estude na particular''.
Esse deveria ser o conceito, mas não fizemos isso. Tomamos público como
estatal e tomamos particular como privado.

\textbf{Durante todas as etapas nas comissões, subcomissões, depois no
primeiro projeto, foram garantidas algumas vitorias de pautas que o
Fórum defendeu: recursos destinados à educação, salário-educação com
fonte adicional de recursos a serem aplicados apenas no ensino público
fundamental e outras várias conquistas. Isso foi votado em novembro, mês
em que também aconteceu uma virada regimental, com a criação do Centrão.
Isso se deu após várias entidades apresentarem propostas aos deputados
constituintes, que -- como o senhor disse -- não tinham propostas e
passaram a defender as propostas das entidades ligadas à educação. Com a
criação do Centrão as pessoas envolvidas com a questão da educação
ficaram receosas?}

\textbf{Buarque:} Não só as pessoas que defendiam o ensino, mas todas as
pessoas progressistas do País se assustaram. O Centrão significa
``direitização''. O Centrão deu nome ao que na verdade é a direita
conservadora. Preciso dizer: o perfil dos constituintes já era
conservador, mas estávamos tão entusiasmados com a democracia que
considerávamos que todo democrata era de esquerda. Mas há muitos
democratas de direita. Ficamos deslumbrados como se a democracia fosse
sinônimo de progressimo. E quando surge o Centrão dentro de um grupo que
já é conservador, significa que é um grupo à direita da sociedade e dos
interesses da Nação. E o Centrão amarrou a Constituição. Mas, para não
parecer que estavam contra o avanço da democracia, fizeram concessões
corporativas, mas sem permitir reformas sociais. Não há reforma social
de fato na Constituição. Não há reforma agrária verdadeira na
Constituição.

Nem mesmo por parte da comunidade de educação houve propostas
revolucionárias. Foram propostas corporativas. Quem teve força foram as
universidades, não a educação de base. Esta foi relegada na
Constituição, salvo a Emenda Calmon, que garante 18\% da União, 25\% do
Estado e Município para a educação. Mas é uma a visão financista:
percentagem de dinheiro, não uma visão estrutural de como deve ser a
educação. Não houve cobrança sobre os professores na Constituição, mas a
gente não pode ter uma boa educação se o professor não se dedica. Não
houve piso nacional do salário do professor na Constituição -- deveria
ter um piso bom, não um piso qualquer. A Constituição não foi um
instrumento de transformação social, mas um instrumento de regularização
da democracia conservadora que o Brasil tem até hoje.

\textbf{Na Constituição foi elaborada uma proposta conjunta chamada de
``emendão''. Nesse acordo a questão da destinação dos recursos públicos
exclusivamente às escolas públicas foi deixado de lado pelos
parlamentares progressistas, pois a partir da criação do Centrão não
havia forças favoráveis para negociar essa questão.}

\textbf{Buarque:} O problema não é colocar o dinheiro nas estatais. O
problema é colocar o dinheiro com o destino de interesse público. A
gente confundiu o estado com público. Existe uma privatização do
estatal. O Estado foi apropriado. O Estado Brasileiro sempre foi
apropriado pelas classes dominantes: os empresários, os latifundiários.
A diferença é que, com a democracia, as corporações sindicais passaram a
ser também classes dominantes. Os sindicatos dos trabalhadores do Estado
são classes dominantes no País. A Constituição foi a favor das classes
dominantes, sejam empresariais ou sejam trabalhistas. Não foi a favor
das classes oprimidas, excluídas. Tanto que faz 20 anos (da promulgação
da Constituição) e a gente continua com a concentração de renda muito
parecida à daquela época.

Não mudamos muito desde então. A ideia de não permitir dinheiro público
para a educação privada considerava que todo o dinheiro que fosse para a
estatal era público. E nem todo dinheiro que vai para a estatal é
público. E considerava que todo dinheiro que fosse para a universidade
privada era de interesse privado, mas nem sempre é. Pode haver interesse
público numa educação mantida em instituição particular.

\textbf{O que a Constituição de 1988 mudou na prática?}

\textbf{Buarque:} A democracia, o direito pleno à liberdade e a garantia
de alguns direitos corporativos para os grupos sociais organizados. Isso
é o que mudou na prática.

\textbf{Ulysses Guimarães disse em seu discurso que ``a Constituição não
é perfeita, mas ao menos desbravadora''. Em que áreas ela mais avançou?}

\textbf{Buarque:} Nos direitos civis. Mas não avançou nas obrigações dos
indivíduos para/com o Brasil. Por isso digo que não é uma constituição
patriótica. É uma Constituição cidadã, mas não patriótica ainda. Para
ser patriótica temos que ter regras para que haja o mesmo direito de
oportunidade para todos. Não foi colocado na Constituição um projeto de
lei que eu defendo: ``todo filho de parlamentar e políticos eleitos tem
que estudar em escola pública''. Isso poderia estar na Constituição.
Também não está na Constituição outro projeto meu que é ``criar uma
carreira nacional do professor''. Há uma porção de carreiras
profissionais na Constituição, mas não há a carreira do professor. No
fundo eu acho que a Constituição desbravou e consolidou a liberdade
democrática e os direitos civis, mas não consolidou uma Nação. Ainda não
é essa Constituição a argamassa da Nação Brasileira.

\textbf{Essa é uma das razões, o fato de que existem muitas matérias da
Constituição que ainda não foram regulamentadas?}

\textbf{Buarque:} Uma das coisas é essa. Para atender a um grupo
corporativo, colocou-se o artigo, que depois não se regularizou. Um
exemplo é o artigo que diz que teríamos que ``abolir o analfabetismo em
10 anos'', mas não se regulamentou dizendo ``será cassado o Prefeito que
não conseguir erradicar o analfabetismo na sua cidade''. Começaríamos a
cumprir o artigo, mas não regularizamos exatamente para poder fazer de
conta que havia resolvido o problema.

\textbf{Após 20 anos de promulgação da Constituinte ainda temos muito o
que fazer pela educação do país?}

\textbf{Buarque:} Temos tudo para fazer. Fizemos pouquíssimo, quase
nada. Alguns dizem ``a gente fez muito''. Em 20 anos passamos de 80\% de
matriculados no ensino fundamental para 95\%. Já deveríamos ter chegado
nos 100\%. E matriculamos mas o aluno não frequenta, alguns frequentam
mas não assistem e outros assistem mas não permanecem até o fim do
ensino médio. Então é falso. Não universalizamos a educação de base do
Brasil. Mentimos que universalizamos a educação de base só porque quase
universalizamos a matrícula no ensino fundamental. Estamos muito longe
de fazer com que o aluno se matricule, fique o dia inteiro, assista as
aulas, aprenda e permaneça até o final do ensino médio. Falta quase tudo
para isso. Só um terço conclui o ensino médio e, desses, metade tem um
ensino médio de péssima qualidade. Faz 20 anos e não chegamos a 18\% dos
nossos jovens terminando um bom ensino médio.

\textbf{JAIR MENEGUELLI}

*Ferramenteiro, foi eleito presidente do Sindicato dos Metalúrgicos de
São Bernardo do Campo e Diadema em 1981. Fundou a CUT (1983), sendo o
primeiro presidente da entidade (1983-1993). Foi deputado federal pelo
PT (1995-2002) e presidente do Conselho Nacional do SESI (2003-2014).

\textbf{O governo de José Sarney, primeiro presidente civil após o
regime militar, implementou vários planos econômicos e todos
fracassaram. Quem mais sofreu com os fracassos foi a classe
trabalhadora. Como se deu esse processo?}

\textbf{Meneguelli:} Todos os planos e diga-se planos econômicos
fracassados... porque já existia uma tese desde os idos da ditadura, e o
ministro da Fazenda era o Delfim Neto, de que era preciso o País
crescer, o ``bolo'' crescer para, depois, dividí-lo. E esse bolo era
sempre feito roubando o salário dos trabalhadores. Tanto que nós fizemos
greve contra um roubo de 34,1\% dos nossos salários e, depois, os planos
confirmaram essa tese: é preciso primeiro resolver o problema da
economia, resolver o problema da indústria, resolver o problema dos
bancos para, depois, resolver o problema dos trabalhadores. E nós éramos
sempre sacrificados. Tanto que, após alguns anos, ganhávamos na Justiça
o que for a aviltado dos nossos salários durante a execução de cada
plano. Todos os planos acabaram, na verdade, prejudicando enormemente os
trabalhadores. Fomos as maiores vítimas desses planos.

\textbf{Foi justamente nesse aspecto de crise econômica e política que
se instalou a Assembleia Nacional Constituinte. Como foi mobilizar,
através da Central Única dos Trabalhadores -- CUT e das outras centrais,
os trabalhadores para se fazerem presentes no processo da ANC? Como se
deu a articulação?}

\textbf{Meneguelli:} O movimento sindical trabalhou. Participamos de
abaixo-assinados. Lembro-me que entregamos, no salão negro da Câmara, um
grande abaixo-assinado ao Ulysses Guimarães (PMDB-SP), que presidia a
Assembleia Nacional Constituinte. Mas obviamente esperávamos ser mais
contemplados na Constituinte. Evidentemente nós não possuíamos ainda uma
força de esquerda muito grande na ANC. Discutimos, apresentamos
propostas e tivemos alguns pequenos avanços, uma pequena redução da
jornada de trabalho, o direito de associação e o direito de greve dos
funcionários públicos -- com algumas regras para ser estabelecidas numa
lei complementar. Houve alguns avanços. Na reforma agrária não considero
que houve avanço, embora tenha surgido a ideia da desapropriação de
terras improdutivas. Mas foi insuficiente para a ideia que se tem sobre
reforma agrária. Hoje tenho a convicção de que a reforma agrária não
significa distribuir terras e dar propriedade. Acho que as terras devem
ser doadas em comodato: aqueles que quiserem trabalhar continuam na
terra indefinidamente e aqueles que não quiserem trabalhar não podem
vender a terra doada pelo estado, mas devolver. É o conceito que tenho
do plano da reforma agrária, mas acho que ainda estamos longe de
conquistar esse sonho.

\textbf{Em fevereiro organizou-se o regimento interno e estabeleceu-se
as regras do jogo. Os partidos de esquerda tiveram pouca influência,
cabendo ao PMDB, sob a liderança de Mário Covas (PMDB-SP), encaminhar as
propostas progressistas. Foi bastante ideológico o embate entre
progressistas e conservadores. Como foi sentar, como presidente da CUT,
com os parlamentares do PMDB, com o PT e os demais parlamentares para
convencê-los de que era necessário levar conquistas para a classe
trabalhadora, que fora estava bastante aviltada com os planos
econômicos?}

\textbf{Meneguelli:} Foi um momento muito difícil. Tive oportunidade de
frequentar algumas reuniões da preparação do regimento da Constituinte,
com um grupo menor, que trabalhou no próprio plenário da Câmara. Tive
acesso diário a tais reuniões -- na verdade por equívoco da Segurança,
que imaginava que eu fosse deputado. Até que um dia o Ulysses Guimaraes
percebeu que eu frequentava aquelas reuniões e me proibiu, porque eu não
era parlamentar. Embora tenhamos conversado com tantos deputados,
tivemos algumas reuniões enquanto Centrão com o relator, o Bernardo
Cabral (PMDB-AM). Não só o Bernardo Cabral, mas como outros
parlamentares que nos ouviam e não diziam ``sim'' ou ``não''. Saíamos
das reuniões sem saber se seria ``sim'' ou ``não''. Mas,
invariavelmente, todas a nossas propostas levadas foram respondidas com
um não.

\textbf{A Emenda Popular elaborada pelo} DIAP (Departamento
Intersindical de Assessoria Parlamentar) \textbf{foi subescrita por
todos os movimentos sindicais, centrais e confederações, ultrapassando 1
milhão de assinaturas. Como se deu o processo das Emendas Populares na
busca do respaldo da sociedade -- não só dos trabalhadores, mas de toda
a sociedade brasileira -- nas bandeiras de lutas do movimento sindical?}

\textbf{Meneguelli:} Usamos de uma grande mobilização. Era uma
possibilidade dentro da Carta Constitucional e nós a usamos. Conseguimos
esse abaixo-assinado, mas que também não teve efeito, porque possuíamos
uma representação minoritária dentro da Assembleia Nacional
Constituinte. As emendas não eram automáticas por terem obtido um milhão
de assinaturas. Elas ainda tinham de passsar pelo crivo da ANC -- e não
passaram. Não sei dizer se àquela época já existia o PROCON (Programa de
Orientação e Proteção ao Consumidor), mas se existisse nós deveríamos
ter reclamado, porque nos fomos enganados. Houve uma falsa ideia de que
automaticamente incluiríamos as emendas na Carta Magna.

\textbf{O movimento sindical usou de várias estratégias para publicizar
fotos de parlamentares que não estavam votando de acordo com a classe
trabalhadora. Como se deu isso?}

\textbf{Meneguelli:} Fizermos a denúncia e fomos chamados até de
fascistas porque estávamos divulgando isso. Lembro, por exemplo, de um
grande painel que montamos na Praça da Sé, em São Paulo, mostrando voto
por voto dos parlamentares, principalmente sobre as reinvindicações e
emendas que nos interessavam. Fomos xingados de fascistas. Fizermos a
denúncia, mas isso não tinha poder de fazer com que os parlamentares
mudassem as suas opiniões. Fazíamos as denúncias para que, talvez numa
próxima eleição, esses deputados constituintes fossem cobrados. Naquele
momento, na verdade, o efeito foi muito pouco.

\textbf{No âmbito da ANC, quais disputas foram as mais ferrenhas, que
tomaram os corredores da Assembleia?}

\textbf{Meneguelli:} Lembro perfeitamente de uma frase do Almir
Pasianoto, que foi ministro do Trabalho e, depois, Juíz do Tribunal
Superior do Trabalho, que dizia: ``o reajuste salarial éalgo pelo qual
devemos brigar anualmente'', porque havia o reajuste salarial, mas
depois, durante o ano, a inflação corroía aquele reajuste e éramos
obrigados a pedir novo reajuste salarial no ano seguinte. A
reivindicação mais importante para o movimento sindical, entre
reivindicações as quais não perderíamos nunca mais caso saíssemos
vitoriosos naquele momento, foi a redução da jornada de trabalho. Essa
foi a nossa principal reivindicação: a redução da jornada de trabalho
para 40 horas semanais, porque sabíamos que seria definitivo, que o fato
de nós trabalharmos reduzindo a jornada nos abriria mais campo para
outros empregos, porque o quão menos trabalhássemos, mais trabalhadores
estariam produzindo. Não conseguimos a redução para 40 horas, mas
conseguimos para 44 horas e foi um avanço. Até hoje continuamos lutando
pela jornada de 40 horas semanais. Acho que esse foi o ponto máximo,
prioritário, o mais interessante para o movimento sindical.

\textbf{Com as idas e vindas, as dificuldades em travar a luta com os
parlamentares, rejeição de propostas, houve momentos de desânimo no
movimento sindical?}

\textbf{Meneguelli:} Eu não diria desânimo. Evidentemente, quando
apresentamos uma reivindicação e não obtemos sucesso, felizes não
ficamos. Mas sabíamos que a batalha era dura. Sabíamos que a maioria da
Assembleia Nacional Constituinte não morria de amores pelos
trabalhadores ou pelo movimento sindical. Mas sabíamos que seria
difícil, que teríamos alguns pequenos avanços, mas que a luta
continuaria. Naquela ocasião reivindicamos as 40 horas semanais.
Passou-se a ANC, passaram-se vários governos e até hoje estamos falando
sobre a redução da jornada de trabalho para 40 horas semanais.

\textbf{Como se comportou o movimento sindical com o sistema de governo
e o mandato de Sarney? Qual era a visão do movimento sindical?}

\textbf{Meneguelli:} O Sarney foi o tampão. O movimento sindical teve
uma reunião com o Sarney na Igreja do Torto. Fomos 8 ou 10 sindicalistas
para reivindicar 13 pontos. Tivemos um almoço oferecido pelo Sarney, uma
feijoada. O que posso dizer desse encontro é que o único resultado foi
que comemos a melhor feijoada do mundo, a melhor feijoada que eu já comi
na vida. Mas, das 13 reinvindicações que apresentamos, não conseguimos
aprovação de absolutamente nada. Então nós queríamos ver o Sarney bem
distante, se possível a volta dele para o Maranhão e, se possível, para
fora do País. É claro que foi um péssimo presidente para a classe
trabalhadora.

\textbf{A Comissão de Sistematização analisa os trabalhos, consegue-se
manter algumas conquistas e, depois, houve a virada regimental do
Centrão -- que foi um golpe para o movimento sindical. Como se deu a
movimentação do movimento sindical junto aos parlamentares de esquerda,
alguns do PMDB, para manter as conquistas de então?}

\textbf{Meneguelli:} A força era pouca e chegara o momento em que
contabilizávamos os votos. Sabíamos que perderíamos. Nos restava
discutir a possibilidade de não assinarmos a Constituição, o que não
significaria absolutamente nada. Assinasse ou não, a Constituição
prevaleceria. O que sobra é posterior à ANC: emendar e reformar a
Constituição, que continua até hoje. Lamentavelmente temos a maior
constituição do mundo, a mais detalhista. Farei uma comparação: sou
defensor de que nós deixemos de lado a CLT e apresentemos um código
nacional do trabalho, muito mais resumido, mais sintético, porque para
um trabalhador entender a CLT, para um humano normal entender a
Constituição Brasileira, ele precisa carregar um advogado debaixo do
braço. É algo esdrúxulo, mas com a qual continuaremos por mais alguns
anos. E, quem sabe, daqui um tempo haja outra Assembleia Nacional
Constituinte?!

\textbf{O que ficou de exemplo para o movimento sindical brasileiro?}

\textbf{Meneguelli:} O que ficou de exemplo é que temos que nos
organizar. E não aprendemos. O movimento sindical brasileiro é dividido.
Temos diversas centrais, mas todo movimento sindical falava de uma
central única de trabalhadores. Acaba que hoje temos mais de 10
centrais, uma infinidade -- milhares e milhares -- de sindicatos, todos
divididos. E a cada mês o Ministério do Trabalho recebe de 80 a 100
novos pedidos para a formação de novos sindicatos. Não aprendemos o que
é preciso para que o movimento sindical possa tratar da sua unificação.
Na verdade aprendemos bastante, mas não colocamos em prática aquilo que
aprendemos.

\textbf{E sua experiência pessoal?}

\textbf{Meneguelli:} Fui sindicalista, depois deputado federal e acho
que consegui menos como deputado do que como sindicalista para os
trabalhadores. Se eu pudesse voltar ao tempo eu voltaria. E se pudesse
optar onde eu queria parar, seria obviamente na época em que fui
dirigente sindical. Foi um momento gratificante na minha vida. Era dura
a luta. Foi dura. Comecei como dirigente sindical em plena ditadura
militar e fazíamos sindicato escondidos nos porões das igrejas, porque
poderíamos ser presos. Mas acho que essa luta foi parte da conquista da
democracia em nosso País.

EGÍDIO FERREIRA LIMA

*Advogado e juíz, começou a carreira política como vereador da cidade de
Vicência (1951-1954), sendo eleito posteriormente deputado estadual e
federal por Pernambuco, sendo um dos representantes do estado na
Assembleia Nacional Constituinte. Foi relator da Comissão de Organização
de Poderes. Está a frente do Instituto Egídio Ferreira Lima, com sede no
Recife.

\textbf{Na semana em que foi instalada a Assembleia Nacional
Constituinte, em 1º de fevereiro de 1987, o jornal a Folha de S. Paulo
estampou na sua primeira página: ``crise marca a abertura da
Constituinte''. A cobertura afirmou que ``o Congresso Constituinte
elaboraria a nova constituição em meio a um quadro de crises de
indefinição econômica e política''. As crises atrapalharam a Assembleia
Nacional Constituinte?}

\textbf{Egídio:} Querdo destacar inicialmente que essa Constituinte não
foi legítima, absolutamente pura, para se fazer a Constituição. E
concordo que a crise econômica estava séria, sim. A inflação era grande.
A cada dia sua remuneração perdia valor e isso repercutia muito
negativamente no emocional de toda a população do País. Nesse período
houve uma sucessão de projetos econômicos, começando pelo Cruzado até
que se acertou com o Real.

Mas a crise política havia sido superada através da eleição de Tancredo
Neves (PMDB-MG) e pela a união do MDB em torno de Tancredo -- houve
apenas um voto discrepante, que foi a abstenção de Jarbas Vasconcelos
(PMDB-PE), que foi governador. Todos os demais votaram pela a eleição de
Tancredo. Houve uma grande disputa entre Tancredo e Ulysses. Houve
reunião partidária onde se discutiu isso, mas quando a Constituinte se
instaurou a crise política estava fundamentalmente superada, porque o
próprio sistema admitiu que se convocasse a Constituinte e o próprio
sistema elegera Tancredo Neves presidente da República. Não havia uma
crise política, havia uma crise normal do embate político, do conflito,
das sobras do conflito, mas a essa altura ja uma parte do governo, da
Arena, ja tinha aderido a eleição de Tancredo... então a crise política
estava praticamente superada, ela tinha só como objetivo superar a fala
da constituinte e promulgar a constituição. Politicamente o País estava
posto, seguindo o caminho da democracia, em busca dela. E qual era esse
caminho? O que faria a democracia? A reorganização do Estado e a
legitimação do Estado pela vontade popular: por isso houve a convocação
da Constituinte para decidir, gerando uma grande expectativa em toda a
Nação.

\textbf{Outro fator de crise foi a disputa dos dois candidatos à
Presidência da Câmara, que foram o Ulysses Guimarães (PMDB-SP) e o
Fernando Lyra (PMDB-PE).} Fernando Lyra colocava que era muito poder nas
mão de Ulysses Guimarães, que este era presidente da Câmara, presidente
da Assembleia Nacional Constituinte, presidente do PMDB e, ainda, na
ausência de Sarney, vice-presidente do País. \textbf{30 de janeiro,
véspera da abertura, houve uma reunião da bancada do PMDB -- que possuía
306 deputados -- e a maioria decidiu pelo o apoio ao Ulysses Guimarães.
O senhor votou no Ulysses Guimarães. Como eram vistas as críticas do
deputado Fernando Lyra a Ulysses Guimarães e ao conjunto do PMDB?}

\textbf{Egídio:} Respondo a você com uma pergunta: como Ulysses não
poderia ou deveria ser deixado de lado na disputa pela Presidência da
Constituinte? Ulysses organizou o MDB no seu começo, constituído, criado
pela própria ditadura. O Ulysses conseguiu transformar o partido com o
apoio sobretudo dos ``autênticos''. Em 1974 o País surpreendeu o sistema
elegendo 16 senadores pelo o voto de maioria. E depois Ulysses candidato
-- eu diria que foi candidato à Presidência da República, não para um
colégio eleitoral -- como maneira de fazer despertar a sociedade. E hoje
isso está claro. Ele candidato à Presidência e teve como vice Barbosa
Lima, presidente da Associação Brasileira de Imprensa e que foi
governador de Pernambuco. Foi útil para mobilizar a sociedade. Não é que
Ulysses quisesse disputar.

Sobre as críticas de Lyra a Ulysses, digo que sou um democrata e, como
tal, acho que o debate dentro do partido é livre. Acho que o Fernando
Lyra tinha o direito de discordar do Ulysses e disputar. Ele se
movimentou, participou ativamente da escolha de Tancredo e do processo
de chagada dele ao governo. Fernando Lyra teve um papel importante
durante todo o combate à ditadura, isso ninguém pode negar. Essa luta,
essa disposição e esse mérito ele os tem. Mas as críticas ao Ulysses
eram infundadas e injustas. Disse que o Ulysses queria se aproveitar do
cargo, que tinha tal ou qual defeito. Ele tinha os defeitos de qualquer
humano.

Na Constituinte nos colocamos em termos de ideias, sem nenhuma mágoa
para o relacionamento político. Ficamos em posições diferentes. Não se
pode imaginar a Constituinte sem a presidência de Ulysses com todo o
papel que ele teve. Não se pode imaginar a Constituinte sem o Ulysses
pegando o primeiro exemplar, chamando-a de Constituinte Cidadã e
gritando as virtudes, a beleza, a liberdade, a estabilidade do Estado
democrático, de representar a promulgação da Constituinte. Acho que o
Fernando teve o direito de disputar, mas dificilmente ganharia para
Ulysses. E Fernando disputou após sair do Ministério, visto que foi
ministro de Sarney. Acho que não foi justa a posição que Fernando
assumiu. Ele poderia disputar sem aquelas criticas a Ulysses.

\textbf{Instalada a Assembleia Nacional Constituinte em 1º de fevereiro,
na sessão do dia seguinte, ainda presidia pelo ministro José Carlos
Moreira Alves (STF), o deputado Plínio Arruda Sampaio (PT-SP) levantou a
questão da impugnação dos 23 senadores oriundos de 1982. ``Eles não
receberam a delegação do povo para elaborar a nova Constituição. Sendo
assim, eles não devem participar da Assembleia Nacional Constituinte''.
Roberto Freire (PCB-PE) também se insurgiu e colocou. Mas no processo de
votação as lideranças partidárias do PMDB, do PFL, do PDS, do PTB, do
PL, do PDC e do PDT votaram pela participação dos senadores de 1982. O
placar foi de 394 a favor e 124 contra, com 17 abstenções. Por que o
entendimento da maioria desses partidos pela a participação desses
senadores?}

\textbf{Egídio:} A Constituinte não foi convocada para ser apenas
Constituinte, querendo uma maior pureza. Deveria ser assim, mas no
Brasil só fez isso em 1946. E a própria convocação já admitia que o
Congresso seria Constituinte e, terminada a Constituição, ele prosseguia
como Congresso: Câmara e Senado. Àquela altura, quando caminhávamos em
busca do Estado de direito, do término do período autoritário, havia
sentido você abrir uma discussão para excluir os senadores? Não fazia o
menor sentido. Seria pegar um fato menor e tumultuar a Constituinte e
seu papel.

A Constituinte e a Constituição dela resultante são tidas como o melhor
trabalho e a melhor constituição que o País já possuiu. Graças à
Constituição vivemos desde o dia 5 de outubro de 1988, sua data de
promulgação, no regime democrático, livre. Nunca a imprensa foi tão
livre. Nunca a cidadania foi tão livre como agora. Se olharmos todo o
período republicano e pós-revolução de 1930, incluindo a
redemocratização de 1946, percebemos que a história foi tumultuada, com
intervenções, com tentativas de insurreição o tempo todo. Não tinha
sentido abrirmos um conflito com o próprio regime, com o próprio
governo.

\textbf{Sobre a composição do quadro partidário. Com a Constituinte
funcionando, o xadrez político das decisões começavam e as regras do
jogo sendo delineadas. Definiu-se de início a importância dos líderes
partidários, que começaram a ter participação efetiva no processo da
Assembleia Nacional Constituinte. A questão partidária exercia
influência de fato na hora das votações em plenário?}

\textbf{Egídio:} A parte mais objetiva, em relação à elaboração do
Estado e o retorno ao Estado de direito estavam substancialmente com o
PMDB: Affonso Arinos (PFL-RJ), Ulysses Guimarães (PMDB-SP)... Eu fui um
vice-líder com atuação constante mesmo quando o líder estava presente.

\textbf{O PMDB possuía 306 deputados. Se o PMDB quisesse fazer uma
Constituição, ele a faria. O PFL possuía 132, o PDS 38, o PDT 26 e os
outros partidos com menos. O PMDB possuía a maior bancada na Assembleia
Nacional Constituinte, com 306 deputados. Na questão da coesão dos
blocos de esquerda, dos partidos, das disputas, alianças e traições... O
PT e o PCdoB apresentavam mais coesão e disciplina partidária nas horas
das votações. E o PMDB, como se comportava?}

\textbf{Egídio:} O PMDB teve uma postura social e democrática em termos
ideológicos, defendendo a democracia, uma economia que atendesse à
amplitude da população, a atenuação dos níveis sociais entre ricos,
classe média e pobres. Essa foi a linha do PMDB. Durante a Constituinte
a luta, a discussão e o debate foram ideológicos. Nunca no Brasil o
debate foi tão ideológico como na votação da Constituinte, durante os
anos de 1987 e 1988. Ideológico por quê? Porque se organizaria um novo
Estado, uma nova economia, um novo Poder Judiciário, um novo Poder
Legislativo. De um lado ficou a esquerda, que às vezes divergia do
grosso do grupo, mas ficava à esquerda do centro-esquerda
social-democrático; e de outro ficava a direita, cujo representante
maior foi o Robertão (Roberto Cardoso Alves, do PMDB-SP)''.

\textbf{Voltando ao PMDB, fale um pouco sobre as tendências do partido.}

\textbf{Egídio:} Antes da constituinte essas tendências eram muito mais
nítidas e muito mais claras. Se traduziam pelos autênticos e os
moderados. O Ulysses teve o mérito de saber conduzir os autênticos, que
eram os mais inquietos: Fernando Lyra, Jarbas Vasconcelos, Alencar
Furtado, Lisânia Maciel, Francisco Pinto... Todo esse pessoal era mais
exacerbado e Ulysses teve o grande mérito de conter isso.

Mas na Constituinte a distinção se desfez: aqueles mais voltados para a
direita e exacerbados ficaram do outro lado, no Centrão, ao lado de
Roberto Cardoso; os outros ficaram com Ulysses e com uma posição mais
avançada. O PT possuía uma posição voltada para o pensamento sindical,
para as entidades de base, que era a luta deles, e queriam democracia.
Havia também os comunistas, que eram um grupo pequeno e que depois
ficaram no Partido Progressista, presidido pelo Roberto Freire (PCB-PE).
Mas a luta se deu mesmo entre os democratas e a direita. Eram dois
grupos de liberais: os mais para a esquerda, como o Marco Maciel
(PFL-PE) e todo o grupo da Arena, que se juntou à votação de Tancredo.
Nota-se três linhas principais dentro da Constituinte: moderados e
autênticos formando um só grupo; a direita que se transformou no
Centrão, conduzida por Roberto Cardoso; e os liberais como Marco Maciel,
que queriam a democracia, lutavam por ela e que se o uniram ao PMDB,
formando o grupo mais voltado para a democracia.

Foram duas fases muito importantes: a primeira foi a das comissões, com
24 comissões básicas, 8 temáticas e a Comissão de Sistematização. Três
episódios foram muito importantes. Vale dizer que a Constituinte foi
muito hábil ao redigir os estatutos e o regimento interno da Assembleia
Constituinte, porque o regimento elegeu 24 comissões básicas e essas
comissões possuíam uma matéria cada. Após elas vieram as comissões
temáticas, que foram muito importantes, como a de Organização dos
Poderes. Cada comissão temática trabalhou em cima de três comissões
básicas, resultando o trabalho em oito ``fatias'' do projeto. Eessas
``fatias'' foram reunidas pelo Relator Geral, o Bernardo Cabral
(PMDB-AM), e o primeiro anteprojeto foi submetido à Comissão de
Sistematização. A Comissão de Sistematização foi uma comissão muito bem
pensada, correta e muito importante. Ela limpou e deu harmonia às
frações que surgiram das demais comissões.

\textbf{Sobre a questão das 24 subcomissões, oito comissões e que cada
subcomissão teve três questões básicas: o bloco conservador -- PFL, PDS,
PL --, que estava defendendo as suas ideias e seus princípios no
processo da Constituinte, como aconteceram as disputas, as coalisões de
aprovação, as coalisões de veto, as atuações regimentais, as discussões,
as procrastinações... Como aconteceu o jogo nos bastidores?}

\textbf{Egídio:} Foi curioso observar que praticamente não se viu
retardamento ou obstrução das votações. Houve articulações nos
bastidores -- depois coloco um fato muito importante que explica o que
estou dizendo agora. A votação evoluiu com certa rapidez. A Constituição
estava prevista para ser votada em dois anos, mas em setembro já estava
terminada e em 5 de outubro foi promulgada. O que houve de importante: a
Comissão de Sistematização era a ``elite'', eram as lideranças da
Constituinte. A Constituinte possuía quase 600 membros. A Comissão votou
o primeiro anteprojeto vindo da Relatoria Geral de Cabral. Coisas
curiosas: avançou-se muito no poder Judiciário, no Poder Executivo, na
área econômica e social e até o sistema parlamentar -- que passou por
mim na comissão temática e foi aprovada e com folga na Comissão de
Sistematização. Ela teve 93 membros e estavam todos: Fernando Henrique
(PMDB-SP), Robertão, Affonso Arinos (PFL-RJ), as lideranças todas. Era
um pessoal capaz, a elite mesmo!

\textbf{Mas nas subcomissões e nas comissões base havia divisões e
disputas nos bastidores?}

\textbf{Egídio:} Havia. Mas as disputas não eram tão grandes. Sabia-se
que estava no caminho, lutava-se para conquistar posições. Mas essa luta
não era tão acirrada porque olhávamos para frente e havia um longo
caminho a percorrer. Votada a matéria das oito comissões temáticas, que
foi o primeiro anteprojeto geral, ela ficou muito avançada -- no bom
sentido, sem radicalismo. Foi quando a ideologia se pôs: o pessoal mais
à direita, Robertão e seus liderados, eles se insurgiram contra o
projeto aprovado com folga pela Comissão de Sistematização. Decidiram
criar o Centrão, que apresentou um substitutivo a todo o projeto da
Constituição, de tal maneira que nós apresentamos emendas para nos
contrapor às posições do Centrão, visto que resultou no Centrão
estabelecendo livres debates sobre tudo. Mas houve algo curioso: não
havia aquela distinção que havia antes e que há hoje entre o pequeno
clero e o clero. E o baixo clero, que vem da Revolução Francesa, aquele
deputado que só defendia interesses e que nem ideologia possuía. O que
ele queria era obter vantagens. Robertão era uma figura que representava
isso. Ele era o pequeno clero, mas possuía tinha ideologia, já que era
de direita. Não se pode negar ao direitista o direito de participar de
um projeto político, não é democrático. Ele pensava politicamente, era
mais pelo capitalismo e via com restrições as conquistas dos
trabalhadores.

\textbf{Dois temas recorrentes tomaram conta dos bastidores: a questão
do mandato de Sarney e a soberania do Congresso Constituinte paralelo ao
funcionamento do Senado e da Câmara. O deputado José Lourenço (PFL-BA)
afirmava que o funcionamento da Câmara e Senado equivalia a fazer da
Constituinte uma comissão mista. Como foi esse debate da soberania?}

\textbf{Egídio:} Inicialmente admintiu-se nos termos que eu contei: não
se viu maior importância ou prejuízo, inclusive em virtude do modo como
foi convocada a Constituinte. Esse debate morreu logo, porque sabia-se
que aquilo nunca se quietava. Algo muito importante, talvez o fato mais
importante da Constituinte e do seu mérito, do seu papel, da sua
praticabilidade, foi a legitimação dela. Como uma constituinte convocada
depois de um regime autoritário, sem que tenha havido de verdade uma
revolução, como ela se legitima com o regime autoritário resvalando
nela, entrando nela como fez a Arena ao se transformar em PDS. E ainda
assim ela se tornou legítima, aceita pelo povo, reestabeleceu o Estado
de direito e criou tranquilidade institucional de lá até aqui. Por que?
Porque houve grande participação popular, constante, efetiva, rica
durante todo o processo da Constituinte. O edifício do Congresso
Nacional com as duas Casas permeadas pela presença de pessoas de todos
os níveis da população brasileira. Havia o operário, o empresário, os
intelectuais, os pensadores. Esse pessoal, andava e passava por nós por
dentro dos prédios. O Ulysses mandou apurar o que era essa presença. E
eram 13 mil pessoas passando diariamente dentro da Constituinte, se
reunindo em comissões, visitando lideranças e atuando dentro. Toda a
sociedade. Então legitimou clara e nitidamente. Foi como se fosse o povo
todo lá dentro. Nunca houve isso no Brasil ou em parte alguma do mundo.
Então isso fez com que a Constituição que votada tenha sido feita pela a
população e absolvida por toda a sociedade brasileira.

\textbf{Sobre o tempo de mandato para a Presidência da República: houve
reações no Palácio e deputados que defendiam cinco anos. O Jornal do
Commercio, em junho, na coluna de Inaldo Sampaio, publicou uma crítica
de um líder de governo na Constituinte, o deputado Carlos Santanna
(PFL-BA), que disse: "Egídio quer se tornar um novo Rui Barbosa e
escrever a Constituição Brasileira sozinho"?}

\textbf{Egídio:} (Risos). O problema dos cinco anos é que não fazia
sentido. O mandato nunca foi de cinco anos, mas de quatro -- até na
ditadura. Por que ampliar o mandato de Sarney? Mas essa luta foi uma
luta menor. Aguentar Sarney quatro anos ou cinco, desde que o País
estivesse redemocratizado, não fazia muita diferença. Votamos contra os
5 anos, mas passou -- sobretudo por força dos ministros, notadamente por
Jurandir Pires, ministro da Guerra, que foi para a televisão e tudo. Mas
isso não foi um fato importante. Mas o fato de Egídio crescer, acho que
isso revela que eu cumpri com meu dever. Eu estava atento a tudo o que
ocorria, porque eu participei intensamente da Constituinte -- e nunca
quis ser Rui Barbosa! (Risos). Seria uma injúria ao Rui e a mim também.
(Risos).

\textbf{Ainda sobre a questão do mandato de Sarney, que perdurou durante
a Assembleia Constituinte toda. Num sábado 13 de junho a comissão
aprovou por 43 votos a 19 o mandato de cinco anos para Sarney. Em 15 de
novembro, na Comissão de Sistematização, vitória dos que defendiam os 4
anos.}

\textbf{Egídio:} Até a decisão final o governo não apareceu
ostensivamente. Leônidas Pires não se pronunciou e o governo só veio a
se pronunciar pelo seus ministros. O Pires fazia ameaças claras. E havia
o líder Carlos Santanna e sobretudo o Robertão, que teve um papel
político nos bastidores maior do que o de Carlos Santana. Houve isso,
mas pessoalmente não acho isso, se cinco ou quatro anos, um problema
fundamental. Acho que cometi um erro: eu poderia ter negociado com
Sarney, porque estive com ele várias vezes. Cheguei a fechar um acordo
com Paulo, o líder dele, sobre o sistema parlamentar. Deveríamos ter
negociado com o Sarney.

\textbf{Lembro, pelos recortes de jornais, que Sarney disse várias
vezes: ``essa questão do meu mandato de governo já está definido.
Inclusive o Egídio sentou comigo e discutimos isso''.}

\textbf{Egídio:} Mas ele furou tudo. Ele chegou a fazer um acordo da
opção de sistema parlamentar. O acordo era optar pelo sistema
parlamentar e não se falou em mandato. Provavelmente estávamos
autorizados a renunciar o problema do mandato, aceitar os 5 anos, desde
que o sistema fosse parlamentarista e não presidencialista.

\textbf{MARCOS TERENA}

*Nascido no distrito indígena de Taunay e pertencente à etnia Xané,
tornou-se aviador da Força Aérea Brasileira e se graduou em
Administração. Líder indígena e escritor, tomou a frente das
articulações políticas dos povos e nações indígenas brasileiras para
lutar pelo reconhecimento dos seus direitos na Assembleia Nacional
Constituinte. É membro da Comissão Brasileira de Justiça e Paz e
coordenador do Fórum Indígena Internacional sobre Biodiversidade.

\textbf{Nos reportemos a 1987 e 1988. O que foi o período
pré-Constituinte? Como é que os povos indígenas começaram a lidar com a
ideia de uma nova constituição a ser construída a partir de fevereiro de
1987?}

\textbf{Terena:} Como estávamos em processo de organizar os movimentos
indígenas como fundadores, criadores do pensamento e do movimento
político do índio no Brasil, trabalhamos com duas possibilidades de
levar os direitos dos índios para dentro do Congresso. Uma delas seria a
eleição. Eu fui candidato constituinte em Brasília e outros indígenas
também se candidataram noutros estados, no sentido de a gente termos uma
representação de no mínimo 3 indígenas. Mas essa ideia não foi exitosa.
Passamos pelo critérios partidários e políticos, mas os votos não
apareceram e não pudemos ter por esse processo a representação no
Congresso Nacional. Para os povos indígenas, especialmente as primeiras
Nações, foi muito grave essa não participação na Constituinte, não
termos a representação. Foi um erro nosso, porque muitos setores sociais
também não puderam se representar. Mas achamos que os todos os outros
setores sociais haviam eleito seus representantes.

O outro processo para levar os direitos dos índios para a Constituinte
foi o cuidado de fazer aliados e nos aproximarmos de representações
políticas e partidárias, principalmente o pessoal da esquerda, que foi o
grande canal que acreditamos ser importante para esse processo.

Muitos indígenas, muitos movimentos, muitos artistas vieram apoiar a
caminhada dos povos indígenas. Foi um momento muito bonito, uma
verdadeira celebração política para nós. Não possuíamos votos. A
quantidade de eleitores indígenas era e continua sendo muito pequena e o
processo eleitoral é muito massacrante, conduzido de forma que viola o
voto e o direito de ser votado. Não tínhamos chance, mas tivemos apoio
do Chico Buarque -- que só fez dois shows no Brasil: um para o Marcos
Terena em Brasília e outro para o Miguel Arraes lá no Recife.

Carregar essa missão junto com os povos indígenas foi fortalecedor,
muito importante. Nós realmente não tínhamos como participar da
Constituinte e precisávamos realmente mobilizar a sociedade para que
também conseguíssemos chegar dentro do Congresso Nacional, onde havia
muitos representantes conservadores e anti-indígenas eleitos e com
legitimidade para defender os seus pontos de vista diante da nossa luta,
da nossa pretensão também.

\textbf{Os povos indígenas não conseguiram eleger representantes. Mas os
povos se fizeram presentes, tiveram suas vozes ecoadas dentro da
Assembleia Nacional Constituinte. Como foi a chegada dos primeiros
povos, das primeiras tribos, as mais variadas etnias indígenas do Brasil
no Congresso? Como se deu esse primeiro contato com essa estrutura de
poder?}

\textbf{Terena:} É importante salientar que tivemos aliados, dentro dos
debates e das articulações no Congresso Nacional, descobrimos aliados.
Deputados e senadores que defendiam os Direitos Humanos, aliados da
causa indígena e da questão ambiental. Foram oportunidades que esses
setores encontraram.

Quando chegamos com os pajés, os caciques, as mulheres, começamos a dar
uma noção para o Constituinte que o índio não é uma algo único, como era
apregoado até então, e que não havia apenas uma língua indígena
brasileira. Na época tínhamos conhecimento de mais de 100 línguas
indígenas. Hoje sabemos da existência de cerca de 240 línguas vivas,
faladas.

E houve todo o processo de orientar as lideranças indígenas sobre a
importância de um documento como a Carta Magna, porque para o indígena
que mora na aldeia não tem importância alguma os papeis construídos
pelos setores da sociedade brasileira. Mas a Constituinte, para nós
articuladores, era um marco também para mudar essa relação com o Poder
Constituinte, com o Constituído e também com a sociedade como um todo:
quebrar preconceitos, quebrar o racismo e mostrar o direito do índio
quando ele luta pela terra. Era também o direito de bem viver e de
qualidade de vida. E hoje não só o Brasil, mas o mundo todo reconhece
que as terras indígenas é um ponto de equilíbrio na relação ambiental e
é um ponto sagrado da qualidade da vida.

\textbf{Entretanto, no primeiro contato das Nações e povos indígenas com
os parlamentares não havia uma compreensão do que era essa grande Nação
indígena. Os parlamentares tinham uma visão distorcida, uma
incompreensão do que eram os povos indígenas. Vocês tiveram que falar
para eles o que era a Nação indígena brasileira.}

\textbf{Terena:} Montamos uma outra estratégia: os índios políticos, os
que não ganharam a eleição. Possuíamos representações indígenas desde a
cidade. Alguns de centro, outros de esquerda, outros de direita. Na
verdade foi uma oportunidade de ter representações indígenas em partidos
para se fazer representar. Tivemos representantes em partidos como o PT,
o PDT, o PcdoB e tivemos indígenas também do PFL. E pedimos audiência
com o presidente da Assembleia Nacional Constituinte, que foi um aliado
forte nosso, o Ulysses Guimarães (PMDB-SP).

Nós preparamos os indígenas políticos para que tivéssemos uma linguagem
política com o chefe maior político da Constituinte. Costurar essa
conversa não era fácil, porque todos os setores queriam falar com o
Ulysses Guimaraes. Mas os assessores do deputado abriram uma brecha para
que nós fôssemos recebidos e foi uma surpresa, porque o Ulysses
Guimarães pensou que encontraria os índios de cocar, pintados, mas na
verdade lá estavam sete indígenas vestidos iguais a ele: com paletó e
gravata e falando questões políticas e partidárias igual a ele. Esse
primeiro encontro estabeleceu uma linha de ação para mostrar para os
deputados constituintes que não estamos brincando de ser índios, mas que
somos indígenas nativos, primeiras nações e que sabemos o que é o
direito indígena dentro de uma articulação política do Congresso e
exigimos esse capitulo na Carta Magna.

Fazer esse trabalho junto aos partidos foi importante para que nós,
através do partido, trouxéssemos votos dos deputados para garantir esse
nossos direitos. Foi um trabalho muito difícil. Os outros setores
perceberam a nossa articulação. Os conservadores, os latifundiários e os
anti-indígenas perceberam a nossa forma de atuar. Por isso foi
importante conseguirmos apoios de deputados do Movimento Negro, como a
Benedita da Silva (PT-RJ); o apoio dos movimentos de trabalhadores, como
a Central Única de Trabalhadores -- CUT; o apoio dos professores, que
possuíam representação no Congresso Nacional; e das representações
religiosas, evangélicas e católicas.

Foi um trabalho árduo e importante, mas não apenas para mostrar um
discurso. E então passamos à outra fase, que é a da articulação
indígena, que era não ter que explicar muita coisa, mas mostrar para as
pessoas o que é ser um Terena, o que é ser um Caiapó, o que significa
ser um Xavante -- todo aquele aglomerado de sociedades indígenas
íntegras e vivas e que todo mundo pensava que éramos ``os índios''. Mas
já não éramos índios, éramos as etnias, éramos os povos. A gente ainda
não tinha nome adequado para colocar e muita gente dizia que esse
movimento poderia violentar, violar a soberania do País.

Esse é um outro conceito que começamos a desmontar na relação com os
militares. Durante a Segunda Guerra Mundial muitos indígenas foram
defender a paz na Itália. Nós, índios, não tínhamos nada a ver com a
guerra, mas fomos ajudar o Brasil a levantar a sua bandeira, para
mostrar também a soberania brasileira, que estava sendo agredida e nós
tínhamos que ser parte e defender também. A questão da Guerra do
Paraguai, a guerra contra os franceses, holandeses... Sempre defendemos
-- não vou dizer ``a soberania do País'', até porque isso é um conceito
geopolítico -- mas o território brasileiro, a terra do Brasil, o
Pantanal, as aguas do Atlântico. Em todos os processos de guerra os
indígenas participaram para defender esse patrimônio. Disseram que as
terras indígenas demarcadas seriam, posteriormente, entregues para as
multinacionais. Disseram que os indígenas demarcariam seus territórios
na fronteira e virariam países independentes. Nós nunca pensamos nisso.
Todas essas armadilhas tivemos que desmontar pouco a pouco, porque
dentro das Forças Armadas nós também começamos a obter setores que
apoiavam a causa indígena contra esse conceito conservador de soberania
nacional.

\textbf{Como foi o encontro com o Ulysses? Indígenas trajados iguais a
homens urbanos, homens políticos. Como foi a antessala, esse momento de
antes de o Ulysses chegar? E como foi a conversa com o Ulysses na defesa
dos direitos indígenas?}

\textbf{Terena:} É claro que o deputado Ulysses Guimarães nunca havia
visto um indígena na frente dele. Ele esperava ver o índio pintado, que
é um índio que está padronizado na cabeça da sociedade brasileira, o
índio ``selvagem'', ``fortão'', pintado de vermelho etc. E nós entramos
como agentes políticos e partidários. A discussão com ele foi de igual
para igual.

A questão pela qual nós não conseguimos ter representação foi porque uns
estavam no PFL, outros no PT e porque que outros estavam se articulando
com seus partidos para somar votos, para ajudar a cidadania indígena a
ser aprovada nos debates e, claro, o Ulysses Guimarães compreendeu tudo
isso. Mas ele não esperava que os indígenas chegassem dessa forma. Na
relação com o Ulysses Guimarães nós tivemos duas fases de articulação.
Essa foi a primeira, que foi o trabalho político e partidário não com os
partidos, mas com os indígenas desses partidos.

A questão dos povos indígenas já havia começado a ser analisada. Textos
afirmando que o Brasil tem uma série de sociedades indígenas e que era
importante o Brasil reconhecer de forma escrita -- ``num outro nível'',
como sempre dizíamos -- e garantir essa visibilidade aos povos indígenas
como povos diferentes, com organizações sociais distintas, mas
partícipes como primeiros povos do nosso País. E o Ulysses prometeu que,
apesar de os indígenas não terem voz efetiva ou representação na
Assembleia Nacional Constituinte, ele seria um aliado, um defensor dos
direitos indígenas. Em 1988 é que o capítulo dos povos indígenas foi
elaborado -- e foi outra grande luta para não deixar em pedaços aquilo
que construímos -- e depois ficou caracterizado como capítulo dos
Direitos Indígenas.

\textbf{Foram dois anos de Assembleia Nacional Constituinte, em várias
fases. Como se organizaram os povos indígenas para as emendas
populares?}

\textbf{Terena:} É como falei: entendíamos todo o processo, mas não
tínhamos capacidade operacional para levantar essas emendas em tempo
hábil. Mas tentamos costurar isso através dos setores sociais. Naquele
tempo não havia a cara do movimento ecológico, o movimento verde, nossos
principais aliados. Mas havia outros setores: as mulheres, o movimento
que originou o Ministério Público -- que na época não existia como órgão
público -- e juristas. Conseguimos trazer as emendas populares para
dentro da relatoria, mas não tínhamos como entrar, como estar lá
cuidando se tiraria ou se não tiraria. Por isso algumas vezes fizemos
plantão no Congresso Nacional com as lideranças nacionais. Eu sempre
considero a força do índio, a sua cultura e a sua espiritualidade.

Então o grande perigo era a ala conservadora, militar e latifundiária. E
eles eram muito fortes e unidos dentro do Congresso Nacional. Pedimos
aos sábios indígenas tradicionais, aos pajés, para que viessem para
fazer uma oração em cima do Congresso Nacional. Houve um dia em que nós
fizemos isso e, de repente, apareceram vários deputados querendo fazer
parte da roda espiritual indígena, sem entender bem o que era aquilo,
mas participar da celebração, porque a espiritualidade a gente não tem
que entender. Isso é um grande mal das religiões: quererem interpretar
Deus, quererem interpretar a espiritualidade. Temos que celebrar e
exercitar, conviver com outro ser humano que está junto com a gente.

Quando os indígenas, os pajés como o Sapaim, como o Getúlio Kaiowá, como
a Dona Quitéria Pankararu começaram a cantar as suas canções e acenderam
o fogo sagrado em cima do Congresso Nacional foi para nós uma das
principais chaves para romper com o preconceito que havia dentro do
Congresso Nacional, do ponto de vista espiritual. E o canto dos povos
indígenas, a força da canção pelo sol, pelo vento, pela natureza que o
deputado e o senador não têm tempo para usufruir -- a maioria deles são
muito doentes, com problemas no coração, muitos com ponte de safena --,
usufruir da força espiritual não só dos indígenas, mas da natureza. Nós
mostramos esse lado, mostramos na prática. Por isso pouco a pouco a
resistência foi desmontada e apareceram aliados populares de menos
esperávamos, de onde não se supunha que fosse aparecer, apareceram esses
valores.

\textbf{Nos embates no primeiro turno houve várias conquistas dos povos
indígenas. Mário Covas (PMDB-SP) foi uma liderança que conseguiu, junto
com os grupos de esquerda, garantir várias conquistas. Em novembro houve
a virada regimental, a questão do Centrão e várias conquistas sociais
foram colocadas em xeque, tiveram que voltar tudo para a votação do
plenário. Os povos indígenas ficaram receosos que as várias conquistas
que garantidas pela Comissão de Sistematização fossem relegadas a
segundo plano?}

\textbf{Terena:} É claro que havia esse risco. A ala conservadora, como
eu falei, ela era muito mais unida que os setores de esquerda, os
setores mais sociais. E nós, indígenas, percebemos que era preciso usar
aquela força indígena partidária para articular votos do Centrão a favor
dos Direitos Indígenas, mas de forma muito discreta para não expor
aquele possível voto diante dos seus próprios grupos do Centrão. Uma das
pessoas que foi muito importante foi a deputada Sandra Cavalcanti
(PFL-RJ), ligada à Igreja Católica e professora. Ela tinha uma forma de
pensar totalmente anti-indígena, mas liderava um grupo e, através do
grupo dela, conseguimos os votos necessários para garantir que, da parte
deles, ninguém mexeria no Direito Indígena. Dentro do Centrão eles
seriam dentro os nossos aliados.

Quero dizer que essa articulação não foi feita apenas pelos indígenas,
mas também pelos setores da Igreja Católica, a CNBB, da Comissão
Brasileira de Justiça e Paz. Como uma pajé disse: ``esses aí são
deputados, então eles votam para garantir o nosso direito e, se eles
votam, não precisamos saber se é de esquerda ou de direita'' -- essa
pajé não entendia bem o que era esquerda ou direita -- ``e o voto deles
tem o mesmo peso, então vamos buscar essas pessoas''. Um dia entenderão
que o indígena é parte da soberania brasileira, também é parte do mundo
do futuro que precisamos garantir. É claro que nem todos do Centrão
concordavam com isso -- muito pelo contrário! Mas se não fosse o voto
deles, através dessa articulação, certamente teríamos perdido aquilo que
estávamos trabalhando.

\textbf{Vencida essa etapa: os povos indígenas conseguiram na
Constituição um capítulo belíssimo sobre a autodeterminação dos povos.
Quais foram as grandes conquistas dos povos indígenas na Constituição de
1988?}

\textbf{Terena:} Na última etapa dessa articulação -- como falei, houve
dois momentos com o Ulysses Guimarães --, após a articulação para que o
capítulo dos povos indígenas fosse respeitado como estava, não tínhamos
certeza na hora da votação geral. E quem era o comandante da votação
geral era o Ulysses Guimarães. Eu voltei lá na assessoria dele para
perguntar se era possível receber os índios novamente. A assessoria do
Ulysses disse que poderíamos ir lá, mas que teríamos que esperar na
porta do gabinete e, assim que houvesse uma brecha, entraríamos. Marcar
com ele estava impossível, pois todo mundo queria falar com ele. Eu
procurei os chefes indígenas que estavam em Brasília -- e eram chefes
que mandam mesmo, não mais os índios de gravata, mas os índios
guerreiros, o que foi uma estratégia que usamos. No meio dos índios
pintados para a guerra estavam os índios que, na primeira vez, estavam
de gravata. E o Ulysses Guimarães nos deu um chá de cadeira e ficamos
esperando na antessala de seu gabinete.

\textbf{PAULO PAIM}

*Metalúrgico, foi eleito em 1981 presidente do sindicato da classe no
município de Canoas. Passou pela Secretaria-Geral e Vice-Presidência da
CUT (1983-1986). Filia-se ao PT em 1985, sendo eleito logo em seguida
deputado federal (1987-2003), quando se elege senador (2003-2018).

\textbf{Na Assembleia Nacional Constituinte o senhor fez parte da
Subcomissão dos Direitos dos Trabalhadores e Servidores Públicos, da
Comissão da Ordem Social e da Subcomissão dos Estados, dentro da
Comissão da Organização do Estado, correto?}

\textbf{Paim:} Eu tive duas indicações, mas depois o PT me deixou
responsável pelo Capítulo da Ordem Social. Foi numa conversa que
tivemos. O Lula disse: ``Não, Paim tu vai cuidar da Ordem Social''.

\textbf{Já no inicio dos trabalhos da Assembleia Nacional Constituinte,
em fevereiro, houve o debate sobre os procedimentos regimentais.
Decide-se que não haverá anteprojeto, mas textos construídos a partir
dos trabalhos das vinte e quatro Subcomissões, das Comissões Temáticas e
da de Sistematização. Esta foi a melhor forma encontrada para que todos
os Constituintes participassem da elaboração da Constituição?}

\textbf{Paim:} Com certeza. Na época se pensou numa grande minuta e que,
a partir dela, faríamos o debate da construção da nova Constituição.
Essa ideia não prevaleceu. Eu não defendi essa tese e outros também não
a aceitavam. Resolvemos que o trabalho deveria surgir efetivamente de
baixo para cima. A Constituição fala que o poder emana do povo. Para
efetivamente o poder emanar do povo nós entendíamos que devíamos
formular um debate via emenda popular, via iniciativa dos senadores e
dos deputados constituintes nas comissões... E foi o que fizemos.

\textbf{Utilizou-se muita coisa do anteprojeto elaborado pela Comissão
Affonso Arinos?}

\textbf{Paim:} Parte dele veio como uma contribuição ao debate, mas o
eixo do nosso anteprojeto foram as emendas populares e os debates feitos
na sociedade, nos estados, em Brasília e nas próprias comissões. Na
Comissão em que atuei durante quase todo o período, a da Ordem Social, o
nosso eixo foi uma proposta apresentada pelo DIAP e assinada por todas
as confederações, elencando o que nós, do Movimento Sindical,
entendíamos que seria adequado para a Ordem Social.

\textbf{Sendo as maiores bancadas na ANC as do PMDB e do PFL, os
partidos de esquerda na Assembleia teriam pouco poder de agenda e de
direção dos trabalhos, restando apenas estabelecer a aliança com a ala
progressista do PMDB? }

\textbf{Paim:} Não era bem assim, até porque o movimento social cumpriu
papel fundamental na Assembleia Nacional Constituinte. Foi a maior
mobilização que vi nos meus 23 anos de parlamento. A pressão popular foi
muito grande e isso se refletiu nas comissões, na redação, nas propostas
aprovadas inclusive no plenário da Assembleia Nacional Constituinte.
Fizemos uma parceria entre, principalmente, PMDB, PDT, PSB, PTB, PCdoB e
o PT, naturalmente. Esses setores tinham uma parceria com uma área muito
expressiva dentro do PMDB. O Mário Covas (PMDB-SP) na época era nosso
aliado direto, o Bernardo Cabral (PMDB-AM), o Ulysses Guimarães
(PMDB-SP). E não eram dos chamados ``partidos de esquerda''.

Houve muitos momentos em que esses parlamentares defenderam posições
nossas. Por exemplo, no direito de greve, conforme uma redação que eu
construí.

O João Paulo de Moura Vades (PT-MG) e eu fomos ao interior de Minas,
numa fazenda. Lá articulamos a negociação. Fomos e voltamos num
``teco-teco'' e, aqui, sentamos numa mesa com o Lula(PT-SP), Olívio
Dutra (PT-RS), João Paulo, Covas(PMDB-SP) e chegamos ao entendimento de
que quem defenderia o direito de greve não eram os sindicalistas, mas o
Mário Covas. E veja a construção que fizemos: o Mário Covas defendeu por
um lado e, depois, eu fui com uma comissão pedir para o Jarbas
Passarinho (PDS-PA) e ele disse: ``está bem, Paim, com essa redação eu
defendo''. E o direito de greve está aí. Foi uma conquista do movimento
social defendida numa seção histórica pelo Jarbas Passarinho e pelo
Mário Covas. Quero demonstrar, com isso, que não nos restava apenas
seguir a orientação dos partidos maiores. O movimento social é muito
forte, estava muito mobilizado e com certeza influenciou na redação da
nossa Constituição -- que na época era mais progressista do que o
período que o País vivia.

\textbf{E como os Constituintes do campo mais conservador reagiram ao
verem na tribuna o Jarbas Passarinho defendendo a emenda do Direito de
Greve?}

\textbf{Paim:} O senador Jarbas Passarinho, naturalmente, antes de ir à
tribuna defender o direito de greve, estabeleceu uma conversa com o
Centrão e mostrou que aquele texto era adequado para a época e que
depois a lei seria regulamentda, como de fato foi -- pelo menos para a
área privada, porque na área pública até hoje não foi regulamentado o
Direito de Greve. Então o Centrão aceitou sem muita resistência, por ter
visto o Covas defendendo uma posição à esquerda e o Jarbas Passarinho,
que tinha mais identidade com eles, também defendendo aquela posição por
entender que era uma redação equilibrada. O Mário Covas era muito
respeitado. Foi o grande líder da Assembleia Nacional Constituinte. Ele
conseguia interagir com os setores conservadores de forma firme e, ao
mesmo tempo, mostrando abertura ao diálogo.

\textbf{No inicio dos trabalhos o deputado Fernando Lyra (PMDB-PE), que
fora ministro da Justiça do governo Sarney, fez críticas ao doutor
Ulysses, dizendo que seria muito poder nas mãos dele, que seria
presidente do Congresso, da Assembleia Nacional Constituinte, do PMDB e,
na ausência do Sarney, vice-presidente do País. Como eram vistas essas
críticas a Ulysses Guimarães? Eram justas?}

\textbf{Paim:} Na época nós entendíamos que essas críticas tinham
procedência, porque não havia necessidade de o doutor Ulysses ser um
super-presidente. Ele presidia praticamente tudo: o Congresso, a Câmara,
o PMDB, a Constituinte... Mas ele cumpriu um papel histórico. Se eu
tivesse que escolher hoje um presidente da Assembleia Nacional
Constituinte, não teria nenhuma dúvida em escolher o Ulysses Guimarães.

\textbf{A gestação da Constituinte não foi fácil. Foram dois meses para
discutir o regimento interno. No dia 10 de novembro o Centrão apresenta
sua proposta de mudança no regimento, aprovada no dia 5 de Janeiro. O
anteprojeto que foi para Comissão de Sistematização e para o Plenário,
documento que afirmava várias conquistas para os trabalhadores e para os
direitos sociais e coletivos, passa por novas negociações, tensões,
articulações. Por que o Centrão conseguiu aprovar a mudança regimental?}

\textbf{Paim:} Isso é natural num processo como aquele. Tínhamos clareza
que não havia só a esquerda, só as centrais e as confederações sindicais
e o movimento social. O Centrão era uma força articulada e com poder de
criar obstáculos que trariam prejuízo para o texto maior. Por isso
estabelecemos um processo de ampla negociação. O texto final foi uma
ampla negociação que envolveu todos os setores da sociedade. Tudo no seu
tempo. Dentro da realidade e da correlação de forças naquele momento,
apesar de tudo, fizemos uma Carta Magna avançada para o seu tempo.

\textbf{O jornalista Ricardo Noblat publicou na coluna dele no Jornal do
Brasil que uma das propostas do grupo suprapartidário do Centrão era
retirar os direitos dos trabalhadores.}

\textbf{Paim:} Havia uma visão naquela época, foi onde mais atuei, de
que a Constituição não deveria ser tão detalhista. Essa é a questão que
o Centrão pegava. Queria que fosse como a Constituição dos Estados
Unidos e que suprimisse esse debate das 40 horas, do direito de greve,
da licença paternidade. Eu tinha uma visão clara de que uma lei é fácil
de revogar, mas que tudo aquilo que eu colocasse na Constituição só
seria retirado com uma emenda Constitucional. No fim prevaleceu essa
nossa posição. Por isso o título da Ordem Social e os artigos sobre os
direitos do trabalhador, do 6° ao 11º, são uma vitória do bloco à
esquerda, do qual fizemos parte.

\textbf{No dia 23 de fevereiro de 1988 há o inicio das votações dos
direitos sociais e trabalhistas. Como se deu a articulação para aprovar
na Constituição garantias como salário mínimo, estabilidade de emprego,
pagamento de horas extras, férias remuneradas, igualdade de direitos
para trabalhadores rurais e urbanos, licença maternidade etc?}

\textbf{Paim:} Foi uma vitória do movimento social organizado, porque
dentro do Congresso éramos minoria. A mobilização da sociedade
brasileira, dos estudantes, dos sindicalistas e dos trabalhadores,
fazendo paradas nas fábricas no momento da votação, graças à
transparência no debate da Assembleia Nacional Constituinte fizemos esse
capítulo, que, para mim, é um dos mais avançados da nossa Carta Magna: o
dos direitos dos trabalhadores da área pública, da área rural e também o
capítulo correspondente aos aposentados e pensionistas. De lá para cá só
houve retrocesso com as emendas Constitucionais que vieram. Todas
retiraram avanços importantes que havíamos assegurado. Naquele momento
foi decisiva a grande mobilização popular que conseguimos fazer,
inclusive com aquele cartaz ``os traidores do povo''. Nós tínhamos claro
que quem votasse contra aquele texto seria denunciado como traidor do
povo. Os cartazes foram multiplicados aos milhares por todo o Brasil.
Ninguém queria ficar com a cara nos postes, nas paradas de ônibus, nas
ruas com a faixa de ``traidor do povo''.

\textbf{Houve forte tensão na discussão sobre a estabilidade de emprego.
Os partidos de esquerda contestaram o Centrão, que era contra. E o
senhor sobe à tribuna e afirma ``o golpe militar e as multinacionais nos
tiraram a estabilidade. Agora é o próprio plenário da Constituinte que
tira da classe trabalhadora o princípio mínimo da estabilidade, em pleno
processo democrático''. Como foi essa subida à tribuna para dizer essas
palavras?}

\textbf{Paim:} Tínhamos muito clara a importância daquele momento
histórico. Sabíamos que o golpe militar teve como eixo acabar com a
estabilidade e a questão da reforma agrária como sempre defendíamos.
Sabia que aquela defesa era uma defesa difícil -- e de fato não passou e
o trabalhador brasileiro hoje não tem estabilidade no emprego -- mas fiz
a defesa com muito coração, com muita consciência da importância daquele
momento. No Brasil só quem tem estabilidade é o servidor público. O
trabalhador da área privada não tem estabilidade. Conseguimos na
Assembleia Nacional Constituinte que pelo menos os dirigentes sindicais
tivessem estabilidade. Mas hoje, infelizmente, os conselheiros do
sindicato e os suplentes da diretoria estão sendo demitidos.

\textbf{Os direitos trabalhistas aprovados foram suficientes ou o senhor
aprovaria outras mudanças?}

\textbf{Paim:} Foi um avanço, mas claro que queríamos mais. Queríamos a
estabilidade, o salário mínimo autoaplicável e turno de seis horas para
todos. Queríamos que a jornada fosse de no máximo 40 horas semanais, não
44. Sabíamos que o aviso prévio proporcional, da forma que ficou, era
mais uma carta de intenções. Desde a Constituinte, há mais de 20 anos, o
aviso prévio ainda continua sendo de 30 dias e fala-se em aviso prévio
proporcional. Mas mesmo assim o capítulo da ordem social, naquela
correlação de forças, foi avançado para a época, uma vitória dos
trabalhadores.

\textbf{Um dos argumentos de quem se opôs à estabilidade é que
aumentaria o desemprego no país.}

\textbf{Paim:} Não é verdade. Sabemos que os países que acatam a
Convenção da Organização Internacional do Trabalho, proibindo a demissão
imotivada, têm muito menos desempregados que o Brasil. Temos uma luta
até hoje para aprovar a Convenção da OIT que recomenda que todos os
países do mundo adotem a política de demissão só com motivo justo. Esse
argumento do aumento do desemprego é falacioso. Se isso fosse
verdadeiro, não teríamos hoje mais de 70 países no mundo que adotam a
convenção da OIT proibindo a demissão imotivada.

\textbf{Sobre a questão da reforma agrária houve muitos confrontos entre
os trabalhadores rurais e os representantes da UDR. Como foi a sua
participação nas discussões?}

\textbf{Paim:} Sem sombra de dúvida este foi um dos momentos mais
difíceis da Assembleia Nacional Constituinte. Fui um daqueles que na
madrugada da noite da votação me desloquei daqui para o aeroporto para
proteger dois ou três senadores que estavam vindo de seus estados de
origem num avião fretado. Havia todo um circo para que eles não pudessem
vir votar. E fomos uns 20 da esquerda para dar essa segurança. Eles
entraram no carro e trouxemos eles para que pudessem votar. Foi um
debate duro, complicado, difícil e passou meio texto. Não era aquilo que
os ruralistas queriam -- o texto deles era muito mais radical contra a
reforma agrária. Na discussão sobre a terra produtiva, se era suscetível
de reforma agrária, passou um meio termo e depois de muito embate com
microfone voando, empurrões, tapas... Eu estava lá a noite toda no
plenário. O Juarez Antunes (PDT-RJ), um líder sindical, levou um soco no
meio desse debate. Foi um momento de muita tensão e a redação não ficou
o que nós queríamos, mas também não ficou o que o Centrão queria. Quando
o embate vai para uma linha de confronto absoluto, a forma de evitar um
prejuízo maior para os trabalhadores foi tentar amenizar a redação.

\textbf{Porque houve uma visão tão maniqueísta no período da
Constituinte, esta polarização de progressistas e conservadores?}

\textbf{Paim:} É natural. Naquela época saíamos da ditadura. Havia um
confronto estabelecido entre quem queria uma constituição avançadíssima
e os outros que queriam deixar tudo como estava. Foi esse o embate que
se deu. E nós só conseguimos fazer com que ela fosse progressista porque
a população assumiu a pressão para que avançássemos nos pontos
considerados polêmicos.

\textbf{Por que o PT não assinou a Carta?}

\textbf{Paulo:} Essa é uma boa pergunta, que me permite esclarecer ao
Brasil que isso não é verdade, mas uma mentira que a grande imprensa
vendeu para sociedade brasileira. O PT assinou a Carta. Basta ler a
Constituição. Está lá quem assinou e tem o meu nome, o do Lula, o do
Genoíno... Nós assinamos a Constituição. Claro que houve o debate
interno no PT se assinaríamos ou não. Prevaleceu por ampla maioria que
deveríamos assinar. Eu tenho minha foto assinando a Constituição com meu
filho ao lado. O que aconteceu é que o PT queria mais avanços na Carta
Magna e votou contra em diversos momentos, mas nos submetemos
naturalmente ao processo democrático da Assembleia Nacional
Constituinte. Todos nós assinamos. Prevaleceu a visão que nós defendemos
de assinar e lá no debate votar contra. Poderíamos dizer, aí sim seria
verdadeiro, que o PT votou contra na Assembleia Nacional Constituinte.
Mas votar e submeter é uma coisa, outra é assinar e reconhecer que o
texto foi um avanço. O PT assinou a Constituição. É só olhar na última
página: estão lá os nome dos que assinaram.

\textbf{Todos os deputados do PT assinaram?}

\textbf{Paim:} Assinaram. O que acontece é que a assinatura é silenciosa
e o voto contra era público. Então dizíamos ``não, a Constituição
deveria ter avançado mais nisso e naquilo e por este motivo nós votamos
contra isso, isso, isso, aquilo''. Na hora de assinar todos assinamos.

\textbf{Uma outra discussão: a Assembleia Nacional Constituinte
legitimou a Anistia de 1979. Qual a sua opinião hoje sobre a punição aos
torturadores do regime militar?}

\textbf{Paim:} Minha posição é clara e é também a posição também do PT:
a tortura não prescreve. Os torturadores deverão sempre responder pelo
ato que cometeram. Não tem que dar anistia para torturador. Eu como o
presidente durante dois anos da Comissão de Direitos Humanos fiz esse
debate. Tortura é um crime hediondo e por isso não prescreve. Eu não
posso concordar que alguém tortura outra pessoa até a morte e depois
simplesmente está anistiado. Não! A legislação tem que ser firme e dura.
Tortura nunca mais!

\textbf{Quais foram os outros momentos mais tensos nas discussões da
Constituinte?}

\textbf{Paim:} O momento mais tenso foi o da reforma agrária. O pau
comeu, foi guerra total. E os momentos... o mais bonito foi o do Alceni
Guerra (PFL-PR), o da licença paternidade. Quando ele começou a falar,
como licença paternidade lembra logo gravidez, as pessoas começaram a
rir. Quando ele terminou de falar todos aplaudiram de pé e havia muita
gente chorando. Isso foi bonito! E passou por unanimidade. Outro momento
diferente foi aquele do Passarinho defendendo o direito de greve na
tribuna e o Mário Covas falando pela esquerda.

\textbf{O Jarbas Passarinho não fez restrições?}

\textbf{Paim:} Não. O Covas fazia uma mediação correta -- e tinha que
fazer para que as propostas da esquerda avançassem. E o Passarinho foi,
digamos, o mediador para o lado de lá. O Passarinho tinha muita
credibilidade, como o Covas e o Ulysses.

\textbf{De seu relacionamento com o Ulysses e o Covas, algo ficou bem
marcado na sua memória?}

\textbf{Paim:} No Covas o que mais me impressionava era a fala dele na
tribuna. Ele era aquele cara que quando ia para a tribuna você não ouvia
um barulho de uma mosca no ar. Ficava todo o plenário em silêncio. A
direita, o Centrão, pelo respeito que tinha a ele; e nós, naturalmente,
pois nos momentos mais polêmicos defendia as nossas posições. A força
dos discursos, dos argumentos, foi o que mais me deixou numa posição de
muito respeito ao Covas. E ele sabia disso. Acho que eu, o Lula, o
Ulysses, o João Paulo, o Genoíno (PT-SP), cumprimos o papel de aglutinar
os movimentos sociais, de articular, defender nas Comissões. Mas no
plenário o Covas era o nosso porta-voz. E não havia problema algum para
nós.

E o Ulysses foi um mestre. Quando as coisas todas embolavam ele trazia a
frase que ficou na minha cabeça: ``senhores e senhoras constituintes,
votem! A decisão será no voto. Votem, votem e votem!''. Ele tentava
articular e, quando não havia entendimento, era hora de votar. Falava
com aquela autoridade que ele possuía e com certeza muitos dos avanços
que tivemos na Constituinte foi por ajuda dele na mediação, junto com
esses nomes que eu citei. Claro que o Lula cumpria um papel na
articulação. Lula sempre foi um grande articulador lá dentro.

\textbf{Por que o PT fechou na questão do presidencialismo e não do
parlamentarismo?}

\textbf{Paim:} No PT não foi uma opção fechada. Eu, por exemplo, sempre
fui parlamentarista e continuo sendo, tanto que nunca me candidatei em
nenhum cargo para o Executivo. Quando cheguei (a Brasília) era
sindicalista. Cheguei indicado não pelo PT, mas por um Congresso
Estadual de Trabalhadores no Rio Grande do Sul. Queríamos um trabalhador
na Constituinte e de lá saiu o meu nome. Eu me filiei ao PT e me
candidatei a deputado federal. Sou parlamentarista convicto antes mesmo
do Congresso. Mas na Assembleia Nacional Constituinte construiu-se uma
Constituição parlamentarista. E houve o plebiscito, mas passou o
presidencialismo.

\textbf{E existe a questão da medida provisória.}

\textbf{Paim:} É. Colocaram um entulho autoritário absurdo que faz com
que o Executivo tenha mais força com o Legislativo. Na verdade é um
instrumento do parlamentarismo que o Jobim (PMDB-RS) trouxe do exterior
dizendo: ``já que queremos o parlamentarismo, na medida provisória temos
que substituir o decreto-lei.'' Nós achamos correto e a ideia teve nosso
aval. Entendíamos que ganharíamos no plebiscito, mas fomos derrotados.
Passou o presidencialismo.

\textbf{Na Constituinte o parlamentarismo ganhou em todas as etapas, mas
perdeu no Plenário. Os jornais noticiavam que houve muita pressão do
Executivo, do presidente Sarney.}

\textbf{Paulo:} Perdeu em parte porque aprovamos o plebiscito. Quando
percebemos que não passaria o parlamentarismo, criou-se a ideia do
plebiscito e ambos concordaram, tanto os presidencialistas quanto os
parlamentaristas. Achávamos que ganharíamos nas ruas. Nós perdemos foi
nas ruas, essa é a grande verdade.

\textbf{Vinte anos após a Assembleia Nacional Constituinte ainda há
muito o que fazer em termos Constitucionais para o país?}

\textbf{Paim:} Claro. Não existe legislação perfeita em lugar algum do
mundo. Eu tenho uma Emenda Constitucional do FUNDEB (Fundo de Manutenção
e Desenvolvimento da Educação Básica e de valorização dos profissionais
da educação) para criar um fundo de investimento para o ensino técnico
profissionalizante. Outra de suma importância: é um absurdo que, no
Estado democrático de direito, tenhamos na Constituição o direito do
voto secreto para os parlamentares. Como é que a população vai saber
como age cada deputado e senador? O parlamentar pode ter um discurso
bonito para fora e, ao votar, trair os interesses dos trabalhadores e da
própria sociedade. Essa emenda, para mim deveria ser a emenda número um,
que a sociedade deveria se debruçar e cobrar, para acabar com toda e
qualquer possibilidade de voto secreto, seja na Câmara dos Vereadores,
na Assembleia Legislativa ou no Congresso Nacional. Sempre é momento de
debater a Carta Magna ou lei complementar que vá aprimorar a legislação
para buscar mais liberdade, justiça e igualdade e para melhorar a
qualidade de vida do povo brasileiro.

\textbf{VICENTINHO}

*​Metalúrgico, iniciou sua militância no sindicato da classe em São
Bernardo do Campo e Diadema, através do qual coordenou a histórica greve
de 1980. Presidiu o sindicato de 1987 a 1993. É fundador do PT (1981) e
da CUT (1983), tendo presidido a Central de 1994 a 2000. É deputado
federal pelo PT (2003-2018).

\textbf{A Assembleia Nacional Constituinte tem início no dia 1° de
fevereiro de 1987. No mesmo dia, duas horas antes, o movimento sindical
inaugurou, com um ato oficial, o lobby sindical. Como se deu a ação do
movimento sindical na Constituinte?}

\textbf{Vicentinho:} Já havia a Central Única dos Trabalhadores como a
nossa grande referência nacional de mobilização. Nesse período
realizamos algumas vigílias -- da terra, da criança e do adolescente...
-- e participamos, em 1985, da grande greve, chamada ``Operação Vaca
Brava'', que teve como objetivo a redução da jornada de trabalho.
Trabalhávamos 48 horas semanais e queríamos 40 horas. Foi um movimento
que durou 54 dias. Sofremos muito, mas conseguimos reduzir para 44. No
período constitucional direcionamos as nossas baterias para a questão da
terra, que tem grande simbolismo, e a grande campanha pela participação
popular. Também arrecadamos assinaturas para as 40 horas semanais. Esse
era o nosso objetivo. Fizemos muitas mobilizações, plenárias,
assembleias, grandes passeatas, vindas a Brasília... De maneira que
conseguimos avançar em algumas coisas. Por exemplo: o que nós já
havíamos conquistado em 1985 para os metalúrgicos virou lei em 1988 para
toda a classe trabalhadora: a jornada máxima de 44 horas semanais.

\textbf{No mês de maio o movimento sindical encaminhou suas propostas
para a Constituição. Como se estabeleceram os primeiros contatos com os
constituintes e as entidades patronais representadas na Constituinte?}

\textbf{Vicentinho:} A Central Única dos Trabalhadores, liderada pelo
nosso companheiro Jair Meneguelli, teve um papel decisivo nesse período.
Meneguelli teve acesso ao plenário da Câmara para dialogar com os
deputados. Mas quando introduzíamos uma concepção cidadã à Constituição
havia várias contradições.

\textbf{O jornal O Globo noticia no dia 29 de maio de 1987 que o governo
Sarney não pretende negociar a estabilidade para os trabalhadores.}
\textbf{Como era a relação do movimento sindical e o governo Sarney?}

\textbf{Vicentinho:} Era uma relação conflituosa. Nesse período fizemos
manifestações e greves gerais. É bom lembrar que foi ainda no governo
Sarney que ocorreram as ultimas condenações com base na Lei de Segurança
Nacional. Na época eu e Arrudo Escobar a fomos condenados na Lei de
Segurança Nacional na Nova República.

\textbf{Quando o Centrão se articulou para tirar direitos dos
trabalhadores o movimento sindical protestou com cartazes, outdoors.
Como foi feita essa articulação?}

\textbf{Vicentinho:} Naquele período percebemos que o povo precisava
saber quem era quem e percebemos que os meios de comunicação não
informavam devidamente. Fizemos uma campanha nacional mostrando quem
estava ao lado dos trabalhadores e quem estava contra. No período surgiu
o DIAP (Departamento Intersindical de Assessoria Parlamentar), que teve
um papel decisivo, inclusive dando nota sobre o comportamento dos
parlamentares nas questões do nosso interesse.

\textbf{Havia a CUT, a CGT, a CONTAG e vários sindicatos. Na defesa dos
direitos dos trabalhadores havia unidade ou interesses conflitantes?}

\textbf{Vicentinho:} Havia unidade nos interesses, mas havia diferenças
na prática. A Central Única dos Trabalhadores -- CUT foi a grande
baluarte nessa luta por uma Constituição cidadã para os trabalhadores.
As diferenças se davam em torno da liberdade e autonomia sindical e do
fim do imposto sindical. Havia uma estrutura corporativista, por vezes
antidemocrática, que sobrevive até hoje e que não queria modificações.

\textbf{O movimento sindical também lutou para que as eleições diretas
acontecessem. Como foi essa atuação?}

\textbf{Vicentinho:} Foi um grande momento. Começamos a perceber que não
podíamos ficar ligados à luta econômica, corporativa, à relação de
capital e trabalho. Constatamos que muitas consequências boas ou más
dependiam do voto do povo e de quem estaria governando. Nós sempre
lutamos contra a ditadura militar e queríamos a eleição direta, por isso
resolvemos sair do casulo, da visão corporativista, e partir para a
sociedade, junto aos estudantes, trabalhadores rurais, intelectuais e
partidos políticos em busca de mais democracia em nosso País.

\textbf{A campanha que a CUT fez contra os deputados constituintes que
estavam votando contra os direitos dos trabalhadores usou o slogan
``traidores do povo''.}

\textbf{Vicentinho:} Na época nós os considerávamos assim. Votavam
contra nosso povo e a gente batia para valer.

\textbf{Meneguelli era o presidente da CUT e você o presidente do
Sindicato de São Bernardo do Campo e Diadema. Como o senhor trabalhava
junto aos sindicalistas dessa região para repassar as informações e
discussões travadas na Constituinte?}

\textbf{Vicentinho:} Tínhamos uma relação direta com a nossa Central, um
observatório, o jornal da central e fazíamos plenárias. Nas votações
decisivas nós acompanhávamos em telões. A gente acompanhava
permanentemente.

\textbf{Em 23 de fevereiro há o inicio das votações dos direitos sociais
e trabalhistas: salário mínimo, estabilidade no emprego, pagamento de
horas extras, etc.} \textbf{Todas essas conquistas não foram fáceis. Era
possível ter avançado mais?}

\textbf{Vicentinho:} Foi o possível na época, dentro daquela correlação
de forças. A situação se complicou tanto que houve deputados que nem
concordaram com a própria Constituinte. Deputados do meu partido que
queriam muito mais. Todos nós queríamos muito mais. Mas é evidente que
comparando com a nossa história -- e havíamos saído de uma ditadura --
essa Constituição foi a mais avançada de todas.

\textbf{O texto final da Constituição representou uma vitória do
movimento sindical?}

\textbf{Vicentinho:} Não representou a vitória do movimento sindical.
Representou a vitória da cidadania, que deu um grande salto de
qualidade. É uma Constituição em que, pela primeira vez, se coloca os
direitos sociais, assegura a soberania, o direito à individualidade,
atua contra qualquer tipo de tortura, consolida o habeas corpus, avança
na questão de gênero, na questão do preconceito racial. Houve um salto
de qualidade -- mas queríamos mais.

\textbf{A Constituição é de fato cidadã, como falou o Ulysses Guimarães
ao promulgar?}

\textbf{Vicentinho:} Sim. Nós consideramos que ela foi cidadã. Talvez na
época, como a gente queria muito mais, a gente não considerasse assim,
mas hoje eu reconheço que foi um salto de qualidade.

\textbf{Vinte anos após a promulgação da Constituição, há muito pelo que
lutar pela classe dos trabalhadores?}

\textbf{Vicentinho:} Sim. Após 20 anos estamos empenhados pela a redução
da jornada de trabalho de 44 para 40 horas semanais. E sou o relator do
projeto. Já se passou 20 anos, o empresariado ganhou muito ao longo
desse período, um crescimento da sua produtividade de 84\% e é
necessário gerarmos empregos. Será melhor para a família e é importante
para a diminuição de acidentes. E a todo momento apresentamos Propostas
de Emendas à Constituição, como a PEC da Reforma Sindical, da qual eu
sou autor em parceria com meu colega Maurício Rands (PT-PE).

\textbf{Anexos:}

\textbf{DISCURSO DO DEPUTADO ULYSSES GUIMARÃES, PRESIDENTE DA ASSEMBLÉIA
NACIONAL CONSTITUINTE, EM 05 DE OUTUBRO DE 1988, POR OCASIÃO DA
PROMULGAÇÃO DA CONSTITUIÇÃO FEDERAL.}

\textbf{Ulysses Guimarães}

``Exmo. Sr. Presidente da República, José Sarney; Exmo. Sr. Presidente
do Senado Federal, Humberto Lucena; Exmo. Sr. Presidente do Supremo
Tribunal Federal, ministro Rafael Mayer; Srs. Membros da Mesa da
Assembléia Nacional Constituinte; eminente Relator Bernardo Cabral;
preclaros Chefes do Poder Legislativo de nações amigas; insignes
Embaixadores, saudados no decano D. Carlo Furno; Exmos. Srs. Ministros
de Estado; Exmos. Srs. Governadores de Estado; Exmos. Srs. Presidentes
de Assembléias Legislativas; dignos Líderes partidários; autoridades
civis, militares e religiosas, registrando o comparecimento do Cardeal
D. José Freire Falcão, Arcebispo de Brasília, e de D. Luciano Mendes de
Almeida, Presidente da CNBB; prestigiosos Srs. Presidentes de
confederações, Sras. e Srs. Constituintes; minhas senhoras e meus
senhores:

Estatuto do Homem, da Liberdade, da Democracia. Dois de fevereiro de
1987: ``Ecoam nesta sala as reivindicações das ruas. A Nação quer mudar,
a Nação deve mudar, a Nação vai mudar.'' São palavras constantes do
discurso de posse como Presidente da Assembléia Nacional Constituinte.
Hoje, 5 de outubro de 1988, no que tange à Constituição, a Nação mudou.

A Constituição mudou na sua elaboração, mudou na definição dos poderes,
mudou restaurando a Federação, mudou quando quer mudar o homem em
cidadão, e só é cidadão quem ganha justo e suficiente salário, lê e
escreve, mora, tem hospital e remédio, lazer quando descansa. Num país
de 30.401.000 analfabetos, afrontosos 25\% da população, cabe advertir:
a cidadania começa com o alfabeto.

Chegamos! Esperamos a Constituição como o vigia espera a aurora.
Bem-aventurados os que chegam. Não nos desencaminhamos na longa marcha,
não nos desmoralizamos capitulando ante pressões aliciadoras e
comprometedoras, não desertamos, não caímos no caminho. Alguns a
fatalidade derrubou: Virgílio Távora, Alair Ferreira, Fábio
Lucena,Antonio Farias e Norberto Schwantes.

Pronunciamos seus nomes queridos com saudade e orgulho: cumpriram com o
seu dever. A Nação nos mandou executar um serviço. Nós o fizemos com
amor, aplicação e sem medo. A Constituição certamente não é perfeita.
Ela própria o confessa, ao admitir a reforma. Quanto a ela, discordar,
sim.

Divergir, sim. Descumprir, jamais. Afrontá-la, nunca. Traidor da
Constituição é traidor da Pátria. Conhecemos o caminho maldito: rasgar a
Constituição, trancar as portas do Parlamento, garrotear a liberdade,
mandar os patriotas para a cadeia, o exílio, o cemitério.

A persistência da Constituição é a sobrevivência da democracia. Quando,
após tantos anos de lutas e sacrifícios, promulgamos o estatuto do
homem, da liberdade e da democracia, bradamos por imposição de sua
honra: temos ódio à ditadura. Ódio e nojo.

Amaldiçoamos a tirania onde quer que ela desgrace homens e nações,
principalmente na América Latina. Assinalarei algumas marcas da
Constituição que passará a comandar esta grande Nação.

A primeira é a coragem. A coragem é a matéria-prima da civilização. Sem
ela, o dever e as instituições perecem. Sem a coragem, as demais
virtudes sucumbem na hora do perigo. Sem ela, não haveria a cruz, nem os
evangelhos. A Assembléia Nacional Constituinte rompeu contra o
establishment, investiu contra a inércia, desafiou tabus. Não ouviu o
refrão saudosista do velho do Restelo, no genial canto de Camões.

Suportou a ira e perigosa campanha mercenária dos que se atreveram na
tentativa de aviltar legisladores em guardas de suas burras abarrotadas
com o ouro de seus privilégios e especulações.

Foi de audácia inovadora a arquitetura da Constituinte, recusando
anteprojeto forâneo ou de elaboração interna. O enorme esforço é
dimensionado pelas 61.020 emendas, além de 122 emendas populares,
algumas com mais de 1 milhão de assinaturas, que foram apresentadas,
publicadas, distribuídas, relatadas e votadas, no longo trajeto das
subcomissões à redação final.

A participação foi também pela presença, pois diariamente cerca de 10
mil postulantes franquearam, livremente, as 11 entradas do enorme
complexo arquitetônico do Parlamento, na procura dos gabinetes,
comissões, galeria e salões.

Há, portanto, representativo e oxigenado sopro de gente, de rua, de
praça, de favela, de fábrica, de trabalhadores, de cozinheiros, de
menores carentes, de índios, de posseiros, de empresários, de
estudantes, de aposentados, de servidores civis e militares, atestando a
contemporaneidade e autenticidade social do texto que ora passa a
vigorar. Como o caramujo, guardará para sempre o bramido das ondas de
sofrimento, esperança e reivindicações de onde proveio.

A Constituição é caracteristicamente o estatuto do homem. É sua marca de
fábrica. O inimigo mortal do homem é a miséria. O estado de direito,
consectário da igualdade, não pode conviver com estado de miséria. Mais
miserável do que os miseráveis é a sociedade que não acaba com a
miséria.

Tipograficamente é hierarquizada a precedência e a preeminência do
homem, colocando-o no umbral da Constituição e catalogando-lhe o número
não superado, só no art. 5º., de 77 incisos e 104 dispositivos.

Não lhe bastou, porém, defendê-lo contra os abusos originários do Estado
e de outras procedências. Introduziu o homem no Estado, fazendo-o credor
de direitos e serviços, cobráveis inclusive com o mandado de injunção.
Tem substância popular e cristã o título que a consagra: ``a
Constituição cidadã''.

Vivenciados e originários dos Estados e Municípios, os Constituintes
haveriam de ser fiéis à Federação. Exemplarmente o foram.

No Brasil, desde o Império, o Estado ultraja a geografia. Espantoso
despautério: o Estado contra o País, quando o País é a geografia, a base
física da Nação, portanto, do Estado. É elementar: não existe Estado sem
país, nem país sem geografia. Esta antinomia é fator de nosso atraso e
de muitos de nossos problemas, pois somos um arquipélago social,
econômico, ambiental e de costumes, não uma ilha. A civilização e a
grandeza do Brasilpercorreram rotas centrífugas e não centrípetas. Os
bandeirantes não ficaram arranhando o litoral como caranguejos, na
imagem pitoresca mas exata de Frei Vicente do Salvador. Cavalgaram os
rios e marcharam para o oeste e para a História, na conquista de um
continente.

Foi também indômita vocação federativa que inspirou o gênio do
Presidente Juscelino Kubitschek, que plantou Brasília longe do mar, no
coração do sertão, como a capital da interiorização e da integração.

A Federação é a unidade na desigualdade, é a coesão pela autonomia das
províncias. Comprimidas pelo centralismo, há o perigo de serem
empurradas para a secessão. É a irmandade entre as regiões. Para que não
se rompa o elo, as mais prósperas devem colaborar com as menos
desenvolvidas. Enquanto houver Norte e Nordeste fracos, não haverá na
União Estado forte, pois fraco é o Brasil. As necessidades básicas do
homem estão nos Estados e nos Municípios. Neles deve estar o dinheiro
para atendê-las.

A Federação é a governabilidade. A governabilidade da Nação passa pela
governabilidade dos Estados e dos Municípios. O desgoverno, filho da
penúria de recursos, acende a ira popular, que invade primeiro os paços
municipais, arranca as grades dos palácios e acabará chegando à rampa do
Palácio do Planalto.

A Constituição reabilitou a Federação ao alocar recursos ponderáveis às
unidades regionais e locais, bem como ao arbitrar competência tributária
para lastrear-lhes a independência financeira. Democracia é a vontade da
lei, que é plural e igual para todos, não a do príncipe, que é
unipessoal e desigual para os favorecimentos e os privilégios.

Se a democracia é o governo da lei, não só ao elaborá-la, mas também
para cumpri-la, são governo o Executivo e o Legislativo.

O Legislativo brasileiro investiu-se das competências dos Parlamentos
contemporâneos. É axiomático que muitos têm maior probabilidade de
acertar do que um só. O governo associativo e gregário é mais apto do
que o solitário.

Eis outro imperativo de governabilidade: a co-participação e a
co-responsabilidade. Cabe a indagação: instituiu-se no Brasil o
tricameralismo ou fortaleceu-se o unicameralismo, com as numerosas e
fundamentais atribuições cometidas ao Congresso Nacional? A resposta
virá pela boca do tempo. Faço votos para que essa regência trina prove
bem.

Nós, os legisladores, ampliamos nossos deveres. Teremos de honrá-los. A
Nação repudia a preguiça, a negligência, a inépcia. Soma-se à nossa
atividade ordinária, astante dilatada, a edição de 56 leis
complementares e 314 ordinárias. Não esqueçamos que, na ausência de lei
complementar, os cidadãos poderão ter o provimento suplementar pelo
mandado de injunção.

A confiabilidade do Congresso Nacional permite que repita, pois tem
pertinência, o slogan: ``Vamos votar, vamos votar'', que integra o
folclore de nossa prática constituinte, reproduzido até em horas de
diversão e em programas humorísticos.

Tem significado de diagnóstico a Constituição ter alargado o exercício
da democracia, em participativa além de representativa. É o clarim da
soberania popular e direta, tocando no umbral da Constituição, para
ordenar o avanço no campo das necessidades sociais. O povo passou a ter
a iniciativa de leis. Mais do que isso, o povo é o superlegislador,
habilitado a rejeitar, pelo referendo, projetos aprovados pelo
Parlamento. A vida pública brasileira será também fiscalizada pelos
cidadãos. Do Presidente da República ao Prefeito, do senador ao
Vereador.

A moral é o cerne da Pátria. A corrupção é o cupim da República.
República suja pela corrupção impune tomba nas mãos de demagogos, que, a
pretexto de salvá-la, a tiranizam. Não roubar, não deixar roubar, pôr na
cadeia quem roube, eis o primeiro mandamento da moral pública. Pela
Constituição, os cidadãos são poderosos e vigilantes agentes da
fiscalização, através do mandado de segurança coletivo; do direito de
receber informações dos órgãos públicos, da prerrogativa de petição aos
poderes públicos, em defesa de direitos contra ilegalidade ou abuso de
poder; da obtenção de certidões para defesa de direitos; da obtenção de
certidões para defesa de direitos; da ação popular, que pode ser
proposta por qualquer cidadão, para anular ato lesivo ao patrimônio
público, ao meio ambiente e ao patrimônio histórico, isento de custas
judiciais; da fiscalização das contas dos Municípios por parte do
contribuinte; podem peticionar, reclamar, representar ou apresentar
queixas junto às comissões das Casas do Congresso Nacional; qualquer
cidadão, partido político, associação ou sindicato são partes legítimas
e poderão denunciar irregularidades ou ilegalidades perante o Tribunal
de Contas da União, do Estado ou do Município. A gratuidade facilita a
efetividade dessa fiscalização.

A exposição panorâmica da lei fundamental que hoje passa a reger a Nação
permite conceituá-la, sinoticamente, como a Constituição coragem, a
Constituição cidadã, a Constituição federativa, a Constituição
representativa e participativa, a Constituição do Governo síntese
Executivo-Legislativo, a Constituição fiscalizadora. Não é a
Constituição perfeita. Se fosse perfeita, seria irreformável. Ela
própria, com humildade e realismo, admite ser emendada, até por maioria
mais acessível, dentro de 5 anos.

Não é a Constituição perfeita, mas será útil, pioneira, desbravadora.
Será luz, ainda que de lamparina, na noite dos desgraçados. É caminhando
que se abrem os caminhos. Ela vai caminhar e abri-los.

Será redentor o caminho que penetrar nos bolsões sujos, escuros e
ignorados da miséria. Recorde-se, alvissareiramente, que o Brasil é o
quinto país a implantar o instituto moderno da seguridade, com a
integração de ações relativas à saúde, à previdência e à assistência
social, assim como a universalidade dos benefícios para os que
contribuam ou não, além de beneficiar 11 milhões de aposentados,
espoliados em seus proventos.

É consagrador o testemunho da ONU de que nenhuma outra Carta no mundo
tenha dedicado mais espaço ao meio ambiente do que a que vamos
promulgar. Sr. Presidente José Sarney: V.Exa. cumpriu exemplarmente o
compromisso do saudoso, do grande Tancredo Neves, de V.Exa. e da Aliança
Democrática ao convocar a Assembléia Nacional Constituinte.

A Emenda Constitucional nº26 teve origem em mensagem do Governo, de
V.Exa., vinculando V.Exa. à efemeridade que hoje a Nação celebra.

Nossa homenagem ao Presidente do Senado, Humberto Lucena, atuante na
Constituinte pelo seu trabalho, seu talento e pela colaboração fraterna
da Casa que representa. Sr. Ministro Rafael Mayer, Presidente do Supremo
Tribunal Federal, saúdo o Poder Judiciário na pessoa austera e modelar
de V.Exa. O imperativo de ``Muda Brasil'', desafio de nossa geração, não
se processará sem o conseqüente ``Muda Justiça'', que se
instrumentalizou na Carta Magna com a valiosa contribuição do poder
chefiado por V.Exa. Cumprimento o eminente Ministro do Supremo Tribunal
Federal, Moreira Alves, que, em histórica sessão, instalou em 1o de
fevereiro de 1987 a Assembléia Nacional Constituinte.

Registro a homogeneidade e o desempenho admirável e solidário de seus
altos deveres, por parte dos dignos membros da Mesa Diretora, condôminos
imprescindíveis de minha Presidência.

O Relator Bernardo Cabral foi capaz, flexível para o entendimento, mas
irremovível nas posições de defesa dos interesses do País. O louvor da
Nação aplaudirá sua vida pública.

Os Relatores Adjuntos, José Fogaça, Konder Reis e Adolfo Oliveira,
prestaram colaboração unanimemente enaltecida.

Nossa palavra de sincero e profundo louvor ao mestre da língua
portuguesa Prof. Celso Cunha, por sua colaboração para a escorreita
redação do texto.

O Brasil agradece pela minha voz a honrosa presença dos prestigiosos
dignitários do Poder Legislativo do continente americano, de Portugal,
da Espanha, de Angola, Moçambique, Guiné Bissau, Príncipe e Cabo Verde.
As nossas saudações.

Os Srs. Governadores de Estado e Presidentes das Assembléias
Legislativas dão realce singular a esta solenidade histórica. Os Líderes
foram o vestibular da Constituinte. Suas reuniões pela manhã e pela
madrugada, com autores de emendas e interessados, disciplinaram,
agilizaram e qualificaram as decisões do Plenário. Os Anais guardarão
seus nomes e sua benemérita faina.

Cumprimento as autoridades civis, eclesiásticas e militares, integrados
estes com seus chefes, na missão, que cumprem com decisão, de prestigiar
a estabilidade democrática.

Nossas congratulações à imprensa, ao rádio e à televisão. Viram tudo,
ouviram o que quiseram, tiveram acesso desimpedido às dependências e
documentos da Constituinte. Nosso reconhecimento, tanto pela divulgação
como pelas críticas, que documentam a absoluta liberdade de imprensa
neste País. Testemunho a coadjuvação diuturna e esclarecida dos
funcionários e assessores, abraçando-os nas pessoas de seus excepcionais
chefes, Paulo Affonso Martins de Oliveira e Adelmar Sabino. Agora
conversemos pela última vez, companheiras e companheiros constituintes.

A atuação das mulheres nesta Casa foi de tal teor, que, pela edificante
força do exemplo, aumentará a representação feminina nas futuras
eleições.

Agradeço a colaboração dos funcionários do Senado -- da Gráfica e do
Prodasen.

Agradeço aos Constituintes a eleição como seu Presidente e agradeço o
convívio alegre, civilizado e motivador.

Quanto a mim, cumpriu-se o magistério do filósofo: o segredo da
felicidade é fazer do seu dever o seu prazer.Todos os dias, meus amigos
constituintes, quando divisava, na chegada ao Congresso, a concha
côncava da Câmara rogando as bênçãos do céu, e a convexa do Senado
ouvindo as súplicas da terra, a alegria inundava meu coração.

Ver o Congresso era como ver a aurora, o mar, o canto do rio, ouvir os
passarinhos. Sentei-me ininterruptamente 9 mil horas nesta cadeira, em
320 sessões, gerando até interpretações divertidas pela não-saída para
lugares biologicamente exigíveis. Somadas as das sessões, foram 17 horas
diárias de labor, também no gabinete e na residência, incluídos sábados,
domingos e feriados.

Político, sou caçador de nuvens. Já fui caçado por tempestades. Uma
delas, benfazeja, me colocou no topo desta montanha de sonho e de
glória. Tive mais do que pedi, cheguei mais longe do que mereço. Que o
bem que os Constituintes me fizeram frutifique em paz, êxito e alegria
para cada um deles.

Adeus, meus irmãos. É despedida definitiva, sem o desejo de retorno.

Nosso desejo é o da Nação: que este Plenário não abrigue outra
Assembléia Nacional Constituinte. Porque, antes da Constituinte, a
ditadura já teria trancado as portas desta Casa.

Autoridades, Constituintes, senhoras e senhores, A sociedade sempre
acaba vencendo, mesmo ante a inércia ou antagonismo do Estado.

O Estado era Tordesilhas. Rebelada, a sociedade empurrou as fronteiras
do Brasil, criando uma das maiores geografias do Universo.

O Estado, encarnado na metrópole, resignara-se ante a invasão holandesa
no Nordeste. A sociedade restaurou nossaintegridade territorial com a
insurreição nativa de Tabocas e Guararapes, sob a liderança de André
Vidal de Negreiros, Felipe Camarão e João Fernandes Vieira, que cunhou a
frase da preeminência da sociedade sobre o Estado: ``Desobedecer a
El-Rei, para servir a El-Rei''.

O Estado capitulou na entrega do Acre, a sociedade retomou-o com as
foices, os achados e os punhos de Plácido de Castro e dos seus
seringueiros.

O Estado autoritário prendeu e exilou. A sociedade, com Teotônio Vilela,
pela anistia, libertou e repatriou.

A sociedade foi Rubens Paiva, não os facínoras que o mataram.

Foi a sociedade, mobilizada nos colossais comícios das Diretas-já, que,
pela transição e pela mudança, derrotou o Estado usurpador.

Termino com as palavras com que comecei esta fala: a Nação quer mudar.

A Nação deve mudar. A Nação vai mudar.

A Constituição pretende ser a voz, a letra, a vontade política da
sociedade rumo à mudança.

Que a promulgação seja nosso grito:

-- Mudar para vencer! Muda, Brasil!''

\textbf{CRONOLOGIA DA ASSEMBLEIA NACIONAL CONSTITUINTE}

\textbf{-1985-}

• 27 de Novembro: Emenda constitucional nº 26, de 27 de novembro de
1985, determina que os membros da Câmara dos Deputados e do Senado
Federal reunir-se-ão, unicameralmente, em Assembléia Nacional
constituinte, livre e soberana, no dia 1 º de fevereiro de 1987, na sede
do Congresso Nacional.

• 7 e 8 de dezembro: Plenárias do Movimento Pró-Constiuinte.

\textbf{-1986-}

• 15 de novembro: Eleição dos deputados federais e de dois terços dos
senadores que comporão a Assembléia Constituinte (primeira eleição do
congresso Nacional em que o direito de sufrágio se estende aos
analfabetos).

\textbf{-1987-}

• 1º de fevereiro: Instalação da Assembléia Nacional Constituinte.

• 2º de fevereiro: Eleição do Presidente da Assembleia Nacional
Constituinte.

• Delegação do movimento Pró-participação Popular na Constituinte .

• 19 e 20 de fevereiro: Reunião, em Brasília, de plenários, comitês e
movimentos pró-participação popular na Constituinte.

• Fevereiro: Em debate crucial sobre procedimentos, se decide que não
haveria anteprojeto, mas texto construído a partir dos trabalhos das 24
subcomissões.

• 19 de março: Aprovação do Regimento da Assembleia, que determina,
entre outras medidas, o recebimento de sugestões de órgãos legislativos
subnacionais, de entidades associativas e de tribunais, além de das de
parlamentares (art.13,11); a realização de audiências públicas, pelas
subcomissões, para ouvir a sociedade (Art. 14); a apreciação de
``emendas populares com 30 mil assinaturas'' (Art.24); a obrigatoriedade
do voto nominal e matéria constitucional.

• 27 de Março: Eleição da mesa Diretora da ANC.

• 27 de Março a 6 de maio: Milhares de sugestões apresentadas por
constituintes e entidades externas são recebidas para a apreciação
formal da Assembléia.

• 7 de Abril a 25 de Maio: Subcomissões temáticas realizam quase
duzentas audiências públicas, uma verdadeira radiografia no Brasil.

• 1º de abril a 12 de junho: trabalho das comissões temáticas.

• 9 de abril: Instalação da Comissão de Sistematização. Relator e
relatores adjuntos são definidos.

• 25 de maio: Conclusão dos trabalhos das Subcomissões com aprovação dos
24 relatórios parciais.

• 9 a 12 de junho: Seminário Nacional para avaliação dos Trabalhos da
Constituinte, com a presença de várias entidades (CEAC da Unb, INESC,
DIAP, IBASE, CEDAC, FASE, etc) e de centenas de pessoas.

• 15 de Junho: Encaminhamento dos relatórios das Comissões Temáticas
para a Comissão de Sistematização.

• 16 de junho: Lançamento da campanha nacional de apoio as emendas
populares.

• 26 de junho: O relator entrega o anteprojeto com 501 artigos.

• 11 de julho: Marcha sobre o congresso, organizada pela UDR.

• 12 de julho: Encaminhamento ao plenário do projeto de constituição da
comissão de sistematização com 496 artigos.

• 15 de julho: Apresentação de 20.791 emendas ao anteprojeto de
Constituição, entre as quais 122 populares. Início da discussão do
projeto em plenário.

• 17 de julho: Dia Nacional de Mobilização para coleta de assinaturas
das emendas populares.

• 12 de agosto: Ato público, em Brasília, para a entrega das emendas
populares.

• 13 de agosto: Fim do prazo para a apresentação das emendas.

• 23 de agosto: Fim da primeira discussão do anteprojeto em Plenário,
voltando à Comissão de Sistematização.

• 26 de agosto a 4 de setembro: Defesa das emendas populares no plenário
da comissão de sistematização, por representantes da sociedade civil.

• 28 de agosto: Prazo de apresentação de emendas ao substitutivo, com o
recebimento de 14.320 emendas.

• 18 de novembro: Término da votação na comissão de sistematização, com
a consequente transferência de trabalhos para o plenário.

• 24 de novembro: O projeto de constituição com 355 artigos ,aprovado na
Comissão de sistematização, é entregue ao Presidente da assembleia
nacional constituinte.

• 26 de novembro: Início da discussão, em plenário do projeto, aprovado.

• 2 de dezembro: Mudança importante no regimento interno da assembleia
(defendida pelo grupo que ficou conhecido como ``Centrão'').

\textbf{-1988-}

• 5 de janeiro: Aprovação da resolução 03/88 (reforma regimental
apresentada pelo ``Centrão'').

• 3 de fevereiro: Início de votação em primeiro turno.

• 30 de junho: fim da votação em 1 º turno.

• 11 de julho: Fim do prazo de recebimento de emendas.

• 22 de julho: Início da votação em segundo turno.

• 27 de julho: O presidente Ulysses Guimarães defende a Assembleia
Constituinte em pronunciamento na televisão.

• 18 de agosto: Entrega, pelo relator, dos pareceres sobre a emenda.

• 2 de setembro: Término da votação em segundo turno.

• 22 de setembro: Plenário aprova, em votação global de turno único, a
redação final.

• 5 de outubro: Promulgada a constituição da República Federativa do
Brasil.

\textbf{ABREVIAÇÕES}

\textbf{ANC} -- Assembléia Nacional Constituinte

\textbf{BASA} -- Banco da Amazônia

\textbf{BNB} -- Banco do Nordeste do Brasil

\textbf{CD} -- Câmara dos Deputados

\textbf{CEBs} -- Comunidades Eclesiais do Brasil

\textbf{CGT} -- Confederação Geral dos Trabalhadores

\textbf{CLT --} Consolidação das Leis do Trabalho

\textbf{CNBB} -- Conferência Naciona dos Bispos do Brasil

\textbf{CONTAG} -- Confederação Nacional dos Trabalhadores na
Agricultura

\textbf{CUT} -- Confederação Geral dos Trabalhadores

\textbf{CN} -- Congresso Nacional

\textbf{DANC} -- Diário da Assembléia Nacional Constituinte

\textbf{DIAP --} Departamento Intersindical de Assessoria Parlamentar

\textbf{DVS --} Destaque para Votar em Separado

\textbf{FCO --} Fundo Constitucional de Financiamento do Centro-Oeste

\textbf{FNE --} Fundo Constitucional de Financiamento do Nordeste

\textbf{FNO --} Fundo Constitucional de Financiamento do Norte

\textbf{FUNDEB --} Fundo de manutenção e Desenvolvimento da Educação
Básica e de valorização dos profissionais da educação

\textbf{INSS --} Instituto Nacional de Seguridade Social

\textbf{IPI --} Imposto sobre Produto Industrializado

\textbf{IR --} Imposto de Renda

\textbf{PROCON --} Programa de Orientação e Proteção ao Consumidor

\textbf{SF} -- Senado Federal

\textbf{STF --} Supremo Tribunal Federal

\textbf{SUS --} Sistema Único de Saúde

\textbf{TSE --} Tribunal Superior Eleitoral

\textbf{UBES} -- União Brasileira dos Estudantes Secundaristas

\textbf{UDR} -- União Democrática Ruralista

\textbf{UNE} -- União Nacional dos Estudantes

\textbf{\\
}

\textbf{PARTIDOS POLÍTICOS}

\textbf{PCB} -- Partido Comunista Brasileiro

\textbf{PCdoB} -- Partido Comunista do Brasil

\textbf{PFL} -- Partido da Frente Liberal

\textbf{PDC} -- Partido Democrático Cristão

\textbf{PDS} -- Partido Democrático Social

\textbf{PDT} -- Partido Democrático Trabalhista

\textbf{PMDB} -- Partido do Movimento Democrático Brasileiro

\textbf{PT} -- Partido dos Trabalhadores

\textbf{PL} -- Partido Liberal

\textbf{PMB} -- Partido Municipalista Brasileiro

\textbf{PSB} -- Partido Socialista Brasileiro

\textbf{PSC} -- Partido Social Cristão

\textbf{PTB} -- Partido Trabalhista Brasileiro

\textbf{PSDB} --Partido da Social Democracia Brasileira
