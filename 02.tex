\part{Ensaios}

\chapter*{Educação da Constituição Cidadã ao Golpe~de~2016}

\addcontentsline{toc}{chapter}{Educação da Constituição Cidadã ao
Golpe de 2016,\\ \scriptsize{por Heleno Araújo}}
\hedramarkboth{Educação da Constituição Cidadã ao Golpe de 2016}{}

\begin{flushright}
\emph{Heleno Araújo}\footnote{Professor da Educação Básica, presidente da
  Confederação Nacional dos Trabalhadores em Educação (\versal{CNTE}), diretor
  de Assuntos Educacionais do Sindicato dos Trabalhadores em Educação de
  Pernambuco (\versal{SINTEPE}) e coordenador do Fórum Nacional Popular de
  Educação (\versal{FNPE}).}
\end{flushright}

Dentro deste contexto de golpe contra a democracia, de ataques profundos
aos direitos trabalhistas e sociais, quero saudar o Cleonildo Cruz e a
Liana Cirne Lins pela iniciativa de publicar este livro \emph{A constituição traída --
da abertura democrática ao golpe e à prisão de Lula}.

A contribuição que apresento para este debate é sobre a educação, das
conquistas obtidas na Constituição Cidadã aos ataques que sofre, mais uma vez,
através de um golpe contra a democracia, afetando os direitos da
população brasileira.

Tendo como referência a mobilização social dos movimentos organizados
que defendem uma educação pública, gratuita, laica e emancipadora, este
texto parte das conquistas que obtivemos na Constituição Federal de
1988, da participação social nas conferências de educação para todos,
culminando com a assinatura do pacto pela valorização do magistério e
pela qualidade social da educação. Apresento portanto, alguns avanços
importantes para a educação na Lei de Diretrizes e Base de
1996,\footnote{Lei nº 9.394, de 20 de dezembro de 1996, que
  estabelece as diretrizes e bases da educação nacional.} fazendo
também um breve relato dos retrocessos no período do governo neoliberal
de Fernando Henrique Cardoso, e de avanços consideráveis nos governos do
presidente Luiz Inácio Lula da Silva, que tiveram seguimento no governo da
presidenta Dilma Rousseff. Trato ainda das medidas do governo golpista,
ilegítimo e corrupto contra a educação. Finalizo este texto com as ações
do movimento educacional em defesa da democracia e pelo
reestabelecimento da aplicação de políticas educacionais que atendam às
demandas do povo brasileiro ao direito a uma educação pública,
gratuita, laica e emancipadora para todos e todas, que necessariamente
passa pela valorização dos profissionais que nela atuam e a qualidade
social necessária para garantir o acesso e a permanência, contribuindo
para a formação cidadã de crianças, jovens, adultos e idosos.

O movimento educacional brasileiro teve uma forte atuação para inserir
na Constituição Federal de 1988 o capítulo dedicado à educação.\footnote{Constituição
  Federal de 1988, Capítulo \versal{III} -- da Educação, da Cultura e do
  Desporto, Seção I -- da educação, artigos 205 ao 214.} A \versal{IV} Conferência Brasileira de
Educação, que reuniu seis mil pessoas tendo em vista a indicação de
propostas para a nova Constituição Federal através da Carta de Goiânia,
renovou a disposição de luta dos educadores brasileiros, para exigir que
os problemas educacionais fossem tratados de maneira responsável e
coerente com as reais necessidades e interesses da população.
Reivindicaram que fossem consagrados os princípios do direito de todos
os cidadãos e todas as cidadãs à educação básica e superior e do dever
do Estado de promover os meios para garanti"-la. Para concretizar estes
princípios, prometeram organizar as entidades para exigirem compromissos
dos candidatos às Constituintes em nível federal e estadual.\footnote{\mbox{Carta
de Goiânia, 1986. Disponível em \emph{\textless{}https://bit.ly/2DwEhNh\textgreater{}.}}}

Estimulados pelos princípios indicados na carta de Goiânia, a Associação
Nacional de Educação (\versal{ANDE}), a Associação Nacional dos Docentes de
Ensino Superior (\versal{ANDES}), a Associação Nacional de Política e
Administração da Educação (\versal{ANPAE}), a Associação Nacional de
Pós"-Graduação e Pesquisa em Educação (\versal{ANPED}), a Confederação dos
Professores do Brasil (\versal{CPB}), o Centro de Estudos Educação e Sociedade
(\versal{CEDES}), o Comando Geral dos Trabalhadores (\versal{CGT}), a Central Única dos
Trabalhadores (\versal{CUT}), a Federação de Sindicatos de Trabalhadores
Técnico"-Administrativos em Instituições de Ensino Superior Públicas do
Brasil (\versal{FASUBRA}), a Federação Nacional dos Orientadores Educacionais
(\versal{FENOE}), a Ordem dos Advogados do Brasil (\versal{OAB}), a Sociedade Brasileira
para o Progresso da Ciência (\versal{SBPC}), a União Brasileira dos Estudantes
Secundaristas (\versal{UBES}), a União Nacional dos Estudantes (\versal{UNE}), e outras
entidades, instalaram em fevereiro de 1987 o Fórum Nacional em Defesa da
Escola Pública (\versal{FNDEP}), que teve o lançamento oficial no dia 09 de abril
de 1987, por intermédio da Campanha Nacional pela Escola Pública e
Gratuita, defendendo quatro princípios básicos para a Constituição
Federal:

\begin{enumerate}
\item{}Educação é direito de todo o cidadão, sendo dever do Estado
oferecer ensino público, gratuito e laico para todos os níveis;

\item{}O governo federal destinará nunca menos que 13\%, e os governos
dos estados, do Distrito Federal, e dos municípios aplicarão no mínimo
25\% de sua receita tributária na manutenção do desenvolvimento do
ensino público e gratuito;

\item{}As verbas públicas destinam"-se exclusivamente às escolas
públicas, criadas e mantidas pelo governo federal, pelos estados,
Distrito Federal e municípios;

\item{}A democratização da escola em todos os níveis deve ser
assegurada quanto ao acesso, permanência e gestão. (\versal{ANDE}, 1987, p.
67).\footnote{\versal{PINHEIRO}, C. M. \emph{O Fórum Nacional em
  Defesa da Escola Pública e o princípio de gestão democrática na
  Constituição Federal de 1988}. 2015. 234 f. Dissertação (mestrado) --
  Faculdade de Filosofia e Ciências, Universidade Estadual Paulista
  Júlio de Mesquita Filho, São Paulo. Disponível em:
  \emph{\textless{}https://bit.ly/2NLAgWe\textgreater{}.}}
\end{enumerate}

A atuação do Fórum Nacional em Defesa da Escola Pública garantiu, na
Constituição Federal de 1988, a educação como direito de todos e dever
do Estado, devendo ser promovida e incentivada com a colaboração da
sociedade, preparando o cidadão e a cidadã para o exercício pleno da
cidadania. Nos princípios da gratuidade nos estabelecimentos oficiais,
da igualdade de condições de acesso e permanência, da liberdade de
aprender e ensinar, do pluralismo de ideias e concepções de práticas
pedagógicas, da valorização dos profissionais do ensino, da gestão
democrática e de um padrão de qualidade das unidades de ensino.

A vida e a luta para a classe trabalhadora nunca foram fáceis. O \versal{FNDEP}
lutou, apresentou propostas, reivindicou, resistiu, mas, não conseguiu
emplacar na legislação todas as propostas apresentadas. Enfrentou os
tubarões da iniciativa privada que tinham e têm a ambição de transformar
a educação em mercadoria para exploração financeira dos capitalistas
gananciosos e que não estão nem aí para o direito humano e social à
educação.

Mesmo assim, a mobilização foi grande e intensa, conseguindo colocar na
Lei nº 9.394 de 1996, a Lei de Diretrizes e Bases da Educação Nacional
(\versal{LDB}), indicadores importantes e fundamentais para o avanço da educação
brasileira, firmando os princípios que devem ser colocados em prática no
ensino, o de garantir igualdade de
condições para o acesso e permanência na escola,
a liberdade de aprender, ensinar,
pesquisar e divulgar a cultura, o pensamento, a arte e o saber,
o pluralismo de ideias e de concepções
pedagógicas, o respeito à liberdade e
apreço à tolerância, a coexistência de
instituições públicas e privadas de
ensino, a gratuidade do ensino público
em estabelecimentos oficiais, a
valorização do profissional da educação
escolar, a gestão democrática do
ensino público e da legislação dos sistemas de ensino,
a aplicação de um padrão de qualidade,
a valorização da experiência
extraescolar, a vinculação entre a
educação escolar, o trabalho e as práticas sociais
e considerar a diversidade
étnico"-racial. A promoção da autonomia da escola para definir e
encaminhar a gestão administrativa, financeira e pedagógica, alinhada
com as políticas da valorização dos profissionais da educação foram
conquistas importantes do Fórum Nacional em Defesa da Escola
Pública.

A realização da Conferência de Educação para Todos,\footnote{Pacto pela
  valorização do magistério e qualidade da educação -- Conferência
  Nacional: acordo de educação para todos. Compromisso com a qualidade e
  a profissionalização do magistério: por uma escola de cidadãos. Plano
  Decenal de Educação para Todos, 1993--2003. \versal{MEC}, assinado em Outubro de
  1994.} durante o governo Itamar Franco, teve como objeto final a
assinatura do pacto pela valorização do magistério e pela qualidade
social da educação. Nesse pacto surgiram as propostas de um fundo
nacional para financiar a educação básica, a criação do piso salarial
profissional nacional para os profissionais do magistério público, a
distribuição de tarefas entre os entes federados e os trabalhadores e as
trabalhadoras em educação, para reestruturar o currículo da formação
inicial dos e das profissionais da educação, ampliar o investimento da
união para aplicar na infraestrutura das escolas, entre outras medidas
fundamentais, objetivando atender à demanda social ao direito à educação
com qualidade social e a valorização dos e das profissionais do
magistério público da educação básica.

Com a vitória do Partido da Social Democracia Brasileira (\versal{PSDB}, número
45) na eleição para presidente da República em 1994, Fernando Henrique
Cardoso rasgou o pacto pela valorização do magistério e pela qualidade
da educação, em outubro de 1995,\footnote{Primeiro ano de governo do \versal{FHC},
  que assumiu a presidência do Brasil em janeiro de 1995 e governou por
  dois mandatos até o ano de 2002. Na sua primeira gestão, apresentou a
  proposta de reeleição, que foi aprovada pelo congresso nacional com
  fortes suspeitas de compras de votos dos e das parlamentares.} quando
o valor do piso salarial profissional nacional do magistério público da
educação básica deveria ter sido aplicado, no valor R\$ 300,00 para uma
professora com formação normal média e com uma jornada de 40 horas aulas
de trabalho por semana. Além de não aplicar o piso, \versal{FHC} e seu Ministro
da Educação Paulo Renato transformaram o piso salarial em salário médio,
frustrando a expectativa de milhões de professores e professoras, e
reduziram a abrangência do financiamento da educação básica para um fundo
destinado apenas ao ensino fundamental, abandonando os investimentos
necessários à educação infantil, ao ensino médio e as modalidades da
educação básica.

Dez anos depois da Constituinte de 1988, que obrigava garantir o acesso
à escola das pessoas na idade dos sete aos catorze anos, além de
indicar o dever do Estado em ofertar creche, acesso à pré"-escola, ao
ensino médio e atender as pessoas que não concluíram a educação básica
na idade apropriada. O que eles fizeram? Empurraram para mais uma década
a determinação constitucional, promovendo estagnação e retrocessos nas
políticas educacionais do país. Ainda em 1998, com a reforma
administrativa, que alterou a Constituição Federal, o (des)governo de
\versal{FHC} delegou ao setor privado, através das Organizações Sociais (\versal{OS}) e
das Organizações da Sociedade Civil de Interesse Público (\versal{OSCIP}), o
atendimento à educação pública, tirando o preceito constitucional de
exclusividade do Estado para atender essa demanda social.\footnote{Em
  2015, o Supremo Tribunal Federal, considerou constitucional a
  Lei nº 9.637, de 15 de maio de 1998, que dispõe sobre a
  qualificação de entidades como organizações sociais, a criação do
  Programa Nacional de Publicização, a extinção dos órgãos e entidades
  que menciona e a absorção de suas atividades por organizações sociais,
  e dá outras providências. Esta lei autoriza o governante entregar a
  escola pública para ser administrada pelo setor privado, através das
  Organizações Sociais.}

A Constituição Cidadã de 1988 só passou a ser efetivamente colocada em
prática a partir do governo do presidente Luiz Inácio Lula da Silva.
Neste período, sentimos uma maior aproximação do texto Constitucional
com as ações práticas das políticas educacionais. A Confederação
Nacional dos Trabalhadores em Educação (\versal{CNTE}), criada para organizar e
defender o conjunto dos trabalhadores e das trabalhadoras em educação e
lutar pelo direito à educação pública, gratuita, laica, democrática e
emancipadora, contabilizou conquistas importantes nos períodos do
governo Lula e Dilma.

A destinação de recursos para formação continuada dos funcionários e das
funcionárias da educação foi um bom começo, nunca antes o Ministério da
Educação (\versal{MEC}) havia destinado recursos para este segmento da nossa
categoria profissional. A aprovação da 21ª Profissão Técnica de nível
médio de Serviços de Apoio Escolar propiciou a criação do
profuncionário, programa
de formação continuada em cursos técnicos pós"-médio para os porteiros,
merendeiras e administrativos das escolas públicas municipais, estaduais
e distritais, atendendo uma das reivindicações históricas da categoria.
Ações que culminaram com o reconhecimento destes
trabalhadores e trabalhadoras como profissionais da
educação, fruto da mudança constitucional realizada em 2007, através da
Emenda Constitucional nº 53/2006\footnote{Emenda Constitucional
  nº 53, de 19 de dezembro de 2006, que dá nova redação aos artigos 7, 23,
  30, 206, 208, 211 e 212 da Constituição Federal e ao art. 60 do Ato
  das Disposições Constitucionais Transitórias.} e que foi regulamentada
pela Lei nº 12.014 de 2009,\footnote{Lei nº 12.014, de 06 de
  agosto de 2009, que altera o art. 61 da lei nº 9.394, de 20 de dezembro
  de 1996, com a finalidade de discriminar as categorias de
  trabalhadores que se devem considerar profissionais da educação.}
corrigindo uma distorção que perdurou durante doze anos, haja vista a
Lei de Diretrizes e Bases da Educação Nacional ser aprovada
desconsiderando estes profissionais da educação.

O retorno da responsabilidade da União para com a educação básica se deu
no governo Lula com a aprovação da Lei do \versal{FUNDEB},\footnote{Lei nº
  11.494, de 20 de junho de 2007, que regulamenta o Fundo de Manutenção e
  Desenvolvimento da Educação Básica e de Valorização dos Profissionais
  da Educação -- \versal{FUNDEB}, de que trata o art. 60 do Ato das Disposições
  Constitucionais Transitórias; altera a Lei nº
  10.195, de 14 de fevereiro de 2001; revoga dispositivos das Leis
  nº 9.424, de 24 de dezembro de 1996, 10.880,
  de 9 de junho de 2004, e 10.845, de 5 de março de 2004; e dá outras
  providências.} uma lei que resgatou parte do pacto de 1994, criando
um fundo para financiar toda educação básica, da creche ao ensino médio,
passando por todas as modalidades. Essa mudança constitucional elevou a
participação financeira da União de R\$ 400 milhões em 2006,
contemplando apenas dois Estados, com mais de R\$ 12 bilhões em 2016,
para o atendimento a doze Estados brasileiros. Além de preparar o
terreno para alcançar outra reivindicação histórica da nossa categoria,
a \versal{EC} 53/2006 determinou a criação do Piso Salarial Profissional Nacional
para o Magistério Público da Educação Básica, que foi regulamentado pela
Lei nº 11.738 de 2008,\footnote{Lei nº 11.738, de 16 de
  julho de 2008, que regulamenta a alínea \versal{E} do inciso \versal{III} do \emph{caput} do art.
  60 do Ato das Disposições Constitucionais Transitórias, para instituir
  o piso salarial profissional nacional para os profissionais do
  magistério público da educação
  básica.}
resgatando outra medida do pacto de 1994, que havia sido rasgado pelo
governo do \versal{PSDB}.

Outro importante indicador para qualidade social da educação é a gestão
democrática, e no governo Lula tivemos a oportunidade de construir as
propostas para as políticas educacionais de forma democrática, com uma
ampla comissão organizadora da Conferência de Educação, que envolvia
entidades educacionais de trabalhadores e trabalhadoras, dos e das
estudantes, dos pais e das mães dos e das estudantes, conselhos de
educação, gestores públicos e privados, centrais sindicais,
confederações dos empresários e diversas entidades e fóruns dos
movimentos sociais. Realizamos, então, a Conferência Nacional da
Educação Básica (\versal{CONEB} 2008) com as etapas municipais, estaduais e
distrital no ano de 2007 e a etapa nacional em 2008, as Conferências
Nacionais de Educação (\versal{CONAE} 2010 e 2014), envolvendo na última edição
mais de quatro milhões de pessoas pensando, debatendo e apontando o rumo
que deveria seguir a educação no Brasil. Os documentos finais da \versal{CONAE}
2010 e 2014,\footnote{Disponíveis em \emph{\textless{}https://bit.ly/1uJb9T6\textgreater{}.}}
contribuíram de forma incisiva na elaboração do novo Plano Nacional de
Educação,\footnote{Lei nº 13.005, de 25 de junho de 2014, que aprova
  o Plano Nacional de Educação -- \versal{PNE} -- e dá outras providências.} sendo que o
projeto de lei nº 8.035 de 2010 foi o que mais recebeu emendas
na história do Congresso Nacional, fruto da intensa mobilização
democrática em defesa da educação como direito humano e social para
todos e todas. Afirmo que das vinte metas que compõem o planejamento
para educação brasileira de 2014 até 2024, dezessete delas têm as nossas
digitais e as marcas das nossas lutas e reivindicações históricas do
movimento educacional brasileiro.

A conquista do \versal{PNE} 2014--2024, que aprovou ampliar os investimentos em
educação, vinculando 10\% do Produto Interno Bruto (\versal{PIB}) do país, tendo
como base a regulamentação do Custo Aluno Qualidade (\versal{CAQ}), só foi
possível graças a outra mudança na Constituição Federal, realizada pela
Emenda Constitucional nº 59 de 2009.\footnote{Emenda
  Constitucional nº 59, de 11 de novembro de 2009. Acrescenta § 3º ao
  art. 76 do Ato das Disposições Constitucionais Transitórias para
  reduzir, anualmente, a partir do exercício de 2009, o percentual da
  Desvinculação das Receitas da União incidente sobre os recursos
  destinados à manutenção e desenvolvimento do ensino de que trata o
  art. 212 da Constituição Federal, dá nova redação aos incisos \versal{I} e \versal{VII}
  do art. 208, de forma a prever a obrigatoriedade do ensino de quatro a
  dezessete anos e ampliar a abrangência dos programas suplementares
  para todas as etapas da educação básica, e dá nova redação ao § 4º do
  art. 211 e ao § 3º do art. 212 e ao \emph{caput} do art. 214, com a
  inserção neste dispositivo de inciso \versal{VI}.} Essa alteração na
Constituição colocou a educação como política prioritária e fundamental
para o desenvolvimento do Brasil; ela ampliou o direito subjetivo à
educação para as pessoas dos quatro aos dezessete anos de idade,
englobando, assim, o direito à educação da pré"-escola ao ensino médio,
inclusive obrigando a atender as pessoas que não concluíram a educação
básica na idade adequada. Essa emenda restituiu os recursos financeiros
destinados à educação que haviam sido subtraídos pela Desvinculação dos
Recursos da União (\versal{DRU}) do governo entreguista e privatista de \versal{FHC}.
Obrigou a ser vinculado percentual do \versal{PIB} para investimentos
em educação, evitando o que aconteceu com o \versal{PNE} 2001--2010, que \versal{FHC} vetou
a vinculação prevista de 7\% do \versal{PIB} para educação. Devido a essa
alteração na Constituição Federal, a presidenta Dilma assinou sem vetos
o novo \versal{PNE} 2014--2024, mesmo o seu governo tendo enviado ao Congresso
Nacional a proposta de vincular 7\% do \versal{PIB} para a próxima década, o
movimento educacional brasileiro ganhou mais esta batalha no Congresso
Nacional alternado a proposta do \versal{PL} 8.035/2010 para 10\% do \versal{PIB}.

Estas conquistas foram proporcionadas devido a outra medida importante
no final do governo Lula, com a criação do Fórum Nacional de Educação
(\versal{FNE}), inicialmente criado por uma portaria do Ministro da Educação e
posteriormente firmado por lei (Lei nº 13.005/2014), determinando sua
criação no âmbito do \versal{MEC}, com atribuições de participar do processo de
concepção, implementação e avaliação da política nacional de educação;
acompanhar, junto ao Congresso Nacional, a tramitação de projetos
legislativos referentes à política nacional de educação, em especial a
de projetos de leis dos planos decenais de educação, definidos pelo
artigo nº 214 da Constituição Federal de 1988, que teve sua redação
alterada pela Emenda à Constituição nº 59/2009; acompanhar e avaliar os
impactos da implementação do Plano Nacional de Educação"-\versal{PNE}; acompanhar
e avaliar o processo de implementação das deliberações das conferências
nacionais de educação; elaborar seu regimento interno e aprovar \emph{ad
referendum} o regimento interno das Conferências Nacionais de Educação;
oferecer suporte técnico aos Estados, Municípios e Distrito Federal para
a organização de seus fóruns e de suas conferências de educação; zelar
para que os fóruns e as conferências de educação dos Estados, do
Distrito Federal e dos Municípios estejam articulados à Conferência
Nacional de Educação; planejar e coordenar a realização de Conferências
Nacionais de Educação, bem como divulgar as suas deliberações; analisar
e propor políticas públicas para assegurar a implementação das
estratégias e o cumprimento das metas;
analisar e propor a revisão do
percentual de investimento público em educação.

Este Fórum de Educação também foi atacado com o golpe, promovido pelos
corruptos da elite conservadora e concentradora de riqueza, contra a
democracia, os direitos da classe trabalhadora e as conquistas sociais
da população brasileira.

Estava anunciado para o ano de 2016, pelo Instituto de Pesquisa
Econômica Aplicada (\versal{IPEA}), que a nossa população sairia da situação de
miserabilidade. A \versal{EC} 59/2009 determinou a universalização do direito à
educação para todas as pessoas dos quatro aos dezessete anos de idade,
como dever do Estado a ser cumprido no ano de 2016. Nesse ano tão
esperado para se concretizar as conquistas e promover avanços sociais,
fomos surpreendidos pelo golpe parlamentar, jurídico e midiático, por
pessoas corruptas que distorceram o texto constitucional para impor o
impedimento ao mandato da presidenta Dilma Rousseff, eleita com mais de
54 milhões de votos.

As provas da existência do golpe são muitas. A presidenta Dilma foi
afastada do cargo, mas não perdeu seus direitos políticos. Então,
cometeu ou não crime? Todas as evidências técnicas indicam que não.
Logo, o julgamento e a decisão foram
políticos, organizados e manipulados por uma corja de políticos
corruptos e indecentes que com a conivência do judiciário e o apoio dos
grandes meios de comunicação induziram parte significativa da população
a exigir a saída da presidenta Dilma Rousseff. Todo movimento foi
orquestrado pelas pessoas mais ricas do país, que não aceitam distribuir
as riquezas produzidas com todos e todas, não aceitam que os filhos dos
pobres tenham acesso à universidade, não aceitam que as águas do rio São
Francisco ajudem a acabar com a seca no nordeste do país, não aceitam
que a classe trabalhadora frequente restaurante, teatro e possa viajar
de avião.

Incomodamos essa elite podre e nojenta, que só pensa em nos explorar e
concentrar riqueza, no momento em que um operário, sem formação de nível
superior e eleito pelo Partido dos Trabalhadores (\versal{PT} 13) provou que os
trabalhadores e as trabalhadoras sabem como governar este país. Foi um
operário, um trabalhador, um ex"-sindicalista que mais investiu recursos
na educação, que construiu mais escolas de educação profissional,
ampliou vagas no ensino superior público, com cotas para estudantes da
educação básica pública, negros/negras, índios/as e quilombolas e levou
a universidade para o interior do país. Fez muito mais para a educação
que o presidente sociólogo e poliglota.

A aliança golpista entre o \versal{PMDB} e o \versal{PSDB}, infelizmente com o apoio
e votos do \versal{PSB},\footnote{Um dos objetivos do Partido
  Socialista Brasileiro é estimular o desenvolvimento de
  valores morais e comportamentos culturais que contribuam para acelerar
  a abolição dos antagonismos de classes e da exploração entre classes e
  segmentos sociais, bem como de todas as formas que justificam
  ideologicamente a discriminação e a marginalização de indivíduos e
  grupos sociais. Com os votos pelo golpe, a maioria dos e das
  parlamentares do \versal{PSB} rasgou o compromisso estatutário do partido para
  com o povo brasileiro.} em menos de um ano já promoveu uma devassa
na Constituição Federal e continua ameaçando prejudicar a população em
muito mais.

A aprovação da Emenda Constitucional nº 95/2016,\footnote{Emenda
  Constitucional nº 95, de 15 de dezembro de 2016, que altera o Ato das
  Disposições Constitucionais Transitórias, para instituir o Novo Regime
  Fiscal, e dá outras providências.} não só inverteu a ordem numérica
da \versal{EC} 59, mas inverteu toda lógica que estava sendo construída nos
últimos anos. Essa \versal{EC} 95, congelou os investimentos nas áreas sociais
para os próximos vinte anos, o que para a educação significa redução de
recursos até 2036. Atuando na contramão das metas e estratégias
aprovadas na Lei do \versal{PNE} 2014--2024, os vetos na Lei de Diretrizes
Orçamentárias (\versal{LDO}) em todos os itens que indicavam aplicação de recursos
para garantir a implementação do \versal{PNE} são inaceitáveis e temos que reagir
firmemente a mais este golpe.

A alteração na Constituição Federal, alinhada com a aprovação da Medida
Provisória 746/2016,\footnote{Tornando"-se a Lei nº 13.415, de 16
  de fevereiro de 2017, altera as Leis nº. 9.394, de 20 de dezembro de
  1996, que estabelece as diretrizes e bases da educação nacional; e
  11.494, de 20 de junho 2007, que regulamenta o Fundo de Manutenção e
  Desenvolvimento da Educação Básica e de Valorização dos Profissionais
  da Educação; a Consolidação das Leis do Trabalho (\versal{CLT}), aprovada pelo
  Decreto"-Lei nº 5.452, de 1º de maio de 1943; e o Decreto"-Lei nº 236,
  de 28 de fevereiro de 1967; revoga a Lei nº 11.161, de 5 de agosto de
  2005; e institui a Política de Fomento à Implementação de Escolas de
  Ensino Médio em Tempo Integral.} com terceirização
irrestrita\footnote{Lei nº 13.429, de 31 de março de 2017, que
  altera dispositivos da Lei nº 6.019, de 3 de
  janeiro de 1974, que dispõe sobre o trabalho temporário nas empresas
  urbanas e dá outras providências; e dispõe sobre as relações de
  trabalho na empresa de prestação de serviços a terceiros.} e a
retirada dos direitos trabalhistas,\footnote{Lei nº 13.467, de 13
  de julho de 2017, que altera a Consolidação das Leis do Trabalho (\versal{CLT}),
  aprovada pelo Decreto"-Lei nº 5.452, de
  1º~de maio de 1943, e as Leis
  nº 6.019, de 3 de janeiro de 1974, 8.036, de
  11 de maio de 1990, e 8.212, de 24 de julho de 1991, a fim de adequar
  a legislação às novas relações de trabalho.} promoverão grandes
retrocessos nas políticas educacionais, tais como: fim do concurso
público para ingresso na carreira dos e das profissionais da educação;
aumento das terceirizações nas escolas públicas; irregularidades na
sequência do processo de ensino"-aprendizagem, ocasionando fortes
prejuízos aos alunos e alunas; a desprofissionalização da categoria com
a permissão da contratação por ``notório saber'' de pessoas para atuar
na educação profissional; estagnação na política de formação continuada,
prejudicando a aplicação da lei do piso salarial profissional do
magistério público e afastando a possibilidade de cumprimento da meta
dezoito do \versal{PNE} de garantir planos de cargos e carreira para todos/as
os/as profissionais da educação, tendo como referência o piso salarial
profissional nacional para os/as profissionais da educação, conforme
determina o artigo 206 da Constituição Federal. Todos esses desmandos
deterioram ainda mais as condições da infraestrutura das escolas,
incentivam a entrega da escola pública ao setor privado, através das \versal{OS}
e das \versal{OSCIP}. São prejuízos incalculáveis para a atual e as futuras
gerações de estudantes e profissionais da educação.

A resistência para recompor a democracia e restabelecer os direitos
conquistados está na ordem do dia. Precisamos de muita organização,
formação política e mobilização da maioria da população brasileira para
barrar os ataques desta elite exploradora e concentradora de riqueza.
Neste sentido, as entidades do movimento educacional brasileiro, ao
serem golpeadas com o Decreto de 26 de abril de 2017,\footnote{Decreto
  do Executivo Federal, de 26 de abril de 2017, que convoca a
  3ª~Conferência Nacional de Educação.} que alterou as
deliberações do pleno do Fórum Nacional de Educação para a realização da
\versal{CONAE} 2018, revogando o Decreto do Executivo Federal de 09 de maio de
2016\footnote{Convoca a 3ª Conferência Nacional de Educação.} que contemplava
as deliberações do pleno do \versal{FNE}, e com a Portaria do \versal{MEC} nº 577, de 27
de abril de 2017,\footnote{Portaria nº 577, de 27 de abril de
  2017, Ministério da Educação, Diário Oficial da União, de 28/04/2017
  (nº 81, seção 1, pág. 39). Dispõe sobre o fórum nacional de
  educação.}
que promoveu uma recomposição do \versal{FNE}, reduzindo a participação das
entidades da sociedade civil de 42 para 24 representações e aumentando a
representação do governo e do setor privado para ter o controle do
fórum, a maioria das entidades da sociedade civil se retirou do \versal{FNE}
golpista da portaria 577 e instituíram o Fórum Nacional Popular de
Educação (\versal{FNPE}),\footnote{\textless{}\emph{http://www.fnpe.com.br}\textgreater{}}
que terá como instrumento de debate, formação e mobilização social a
realização da Conferência Nacional Popular de Educação (\versal{CONAPE}
2018),\footnote{A etapa municipal e/ou intermunicipal da \versal{CONAPE}
  2018 acontecerá até o mês de outubro de 2017, a etapa
  estadual/distrital até março de 2018 e a etapa Nacional acontecerá nos
  dias 26, 27 e 28 de abril de 2018, em Belo Horizonte/\versal{MG}.} nas etapas
municipais e/ou intermunicipais, estaduais, distrital e nacional. Esta é
mais uma contribuição na luta contra o golpe e pelo restabelecimento da
democracia. Porque somos nós que movimentamos e produzimos as riquezas
deste país, ``este é o nosso País, esta é a nossa bandeira, é por amor a
esta Pátria"-Brasil, que a gente segue em fileira''.\footnote{Música ``Ordem
  e Progresso'', Letra: Zé Pinto.}

Fora todos os Corruptos!

Viva a Democracia!

Sigamos Firmes na Luta por Nenhum Direito a Menos!
